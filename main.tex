% ============================================================================
% MAIN DOCUMENT - Artificial Intelligence for Law and Finance
% ============================================================================
% This is the root document for the complete book.
% Individual chapters can be compiled standalone using subfiles.
%
% Build this file to generate the complete book PDF.
% Build chapter files directly to generate standalone chapter PDFs.
% ============================================================================

\documentclass[11pt,oneside]{book}

% ============================================================================
% PATH HANDLING FOR SUBFILES
% ============================================================================
\usepackage{import}  % Must be loaded before subfiles
\usepackage{subfiles}

% ============================================================================
% SHARED PREAMBLE
% ============================================================================
% All typography, colors, packages, and custom commands
% Use relative path that works when loaded as subfile
\makeatletter
\def\input@path{{./}{../}{../../}}
\makeatother
% ============================================================================
% SHARED PREAMBLE - Artificial Intelligence for Law and Finance
% ============================================================================
% This preamble is used by both:
%   - Individual chapter compilations (standalone mode)
%   - Complete book compilation (via subfiles)
%
% Do NOT include \documentclass or \begin{document} here
% ============================================================================

% ============================================================================
% MODERN TYPOGRAPHY AND LAYOUT
% ============================================================================
\usepackage[margin=1.2in, top=1.3in, bottom=1.3in]{geometry}
\usepackage[utf8]{inputenc}
\usepackage[T1]{fontenc}

% Modern font: Libertinus (elegant, readable serif)
\usepackage{libertinus}
\usepackage{libertinust1math} % Math support for Libertinus

% Typography enhancements
\usepackage{microtype}  % Improved typography and spacing
\usepackage{setspace}   % Line spacing control
\setstretch{1.15}       % Slightly increased line spacing for readability

% Paragraph formatting
\usepackage{parskip}    % Space between paragraphs instead of indentation
\setlength{\parskip}{0.5em}
\setlength{\parindent}{0pt}

% Language and quotation support (must come before biblatex)
\usepackage[english]{babel}
\usepackage{csquotes}

% ============================================================================
% MATHEMATICS AND SYMBOLS
% ============================================================================
\usepackage{amsmath}
\usepackage{amssymb}

% ============================================================================
% GRAPHICS AND FIGURES
% ============================================================================
\usepackage{graphicx}
\usepackage{float}      % Better float control

% ============================================================================
% COLOR SYSTEM - Educational Semantic Palette
% ============================================================================
%
% ARCHITECTURE:
%   This color system uses a four-layer architecture for modularity and clarity:
%
%   Layer 1: PRIMITIVES - Raw RGB values (the actual color palette)
%            Named by visual appearance (slate-900, green-600, etc.)
%            Change these to rebrand the entire document
%
%   Layer 2: SEMANTICS - Educational content types
%            Named by pedagogical purpose (definition, example, key, caution)
%            Maps content types to primitive colors
%
%   Layer 3: COMPONENTS - Specific usage contexts
%            Named by component role (bg-*, border-*, text-*)
%            Makes code self-documenting and explicit
%
%   Layer 4: LEGACY - Backward compatibility aliases
%            Preserves old naming for gradual migration
%
% EDUCATIONAL CONTENT TYPES:
%   - definition: Formal definitions, theoretical concepts (Blue)
%   - example:    Concrete examples, demonstrations, applications (Green)
%   - key:        Important takeaways, essential concepts (Orange/Amber)
%   - caution:    Warnings, pitfalls, common mistakes (Red)
%   - note:       Supplementary info, asides, optional reading (Neutral)
%   - theorem:    Formal statements, proofs, logical arguments (Indigo)
%   - practice:   Exercises, problems, hands-on activities (Teal)
%
% VARIANT PATTERN:
%   Each semantic type has three consistent variants:
%   - [type]-dark:  For text and strong borders (high contrast, readable)
%   - [type]-base:  For medium emphasis (icons, accents, highlights)
%   - [type]-light: For backgrounds (subtle, doesn't overwhelm content)
%
% ============================================================================

\usepackage{xcolor}

% ============================================================================
% LAYER 1: PRIMITIVE COLOR PALETTE
% ============================================================================
% Raw color values named by visual appearance, not purpose.
% These form the foundation - modify these to rebrand the entire document.
% Organized by color family with numeric scale: 900 (darkest) → 100 (lightest)
% ============================================================================

% --- SLATE/BLUE FAMILY: Professional, Formal, Structural ---
\definecolor{slate-900}{RGB}{30,58,95}          % Deep slate blue - maximum depth/authority
\definecolor{slate-700}{RGB}{71,101,135}        % Medium slate - readable structure
\definecolor{slate-100}{RGB}{240,246,252}       % Ice blue - subtle professional backgrounds

% --- GREEN FAMILY: Natural, Practical, Concrete ---
\definecolor{green-900}{RGB}{46,125,50}         % Forest green - maximum depth/seriousness
\definecolor{green-600}{RGB}{67,160,71}         % Vibrant green - clear emphasis
\definecolor{green-100}{RGB}{232,245,233}       % Light green - fresh, approachable backgrounds

% --- AMBER/ORANGE FAMILY: Warm, Important, Attention ---
\definecolor{amber-900}{RGB}{230,124,0}         % Deep amber - strong attention signal
\definecolor{amber-600}{RGB}{251,140,0}         % Medium amber - warm emphasis
\definecolor{amber-100}{RGB}{255,243,224}       % Light amber - gentle highlight backgrounds

% --- RED FAMILY: Alert, Warning, Critical ---
\definecolor{red-900}{RGB}{198,40,40}           % Deep red - serious warnings/errors
\definecolor{red-600}{RGB}{229,57,53}           % Medium red - clear alert signal
\definecolor{red-100}{RGB}{255,235,238}         % Light red - soft warning backgrounds

% --- TEAL FAMILY: Modern, Technical, Alternative ---
\definecolor{teal-700}{RGB}{0,128,128}          % Deep teal - strong technical tone
\definecolor{teal-600}{RGB}{0,137,123}          % Professional teal - modern accent
\definecolor{teal-100}{RGB}{224,242,241}        % Light teal - fresh technical backgrounds

% --- INDIGO/PURPLE FAMILY: Abstract, Theoretical, Formal ---
\definecolor{indigo-700}{RGB}{63,81,181}        % Medium indigo - thoughtful, abstract tone
\definecolor{indigo-600}{RGB}{92,107,192}       % Lighter indigo - approachable formal tone
\definecolor{indigo-100}{RGB}{232,234,246}      % Light indigo - subtle theoretical backgrounds

% --- GRAY FAMILY: Neutral, Text, Structure ---
\definecolor{gray-900}{RGB}{30,32,34}           % Almost black - primary body text
\definecolor{gray-800}{RGB}{52,58,64}           % Very dark gray - strong headings
\definecolor{gray-700}{RGB}{73,80,87}           % Dark gray - secondary text
\definecolor{gray-600}{RGB}{95,100,105}         % Medium-dark gray - muted/de-emphasized text
\definecolor{gray-500}{RGB}{134,142,150}        % Medium gray - borders, dividers
\definecolor{gray-400}{RGB}{173,181,189}        % Medium-light gray - subtle dividers
\definecolor{gray-300}{RGB}{210,215,220}        % Light gray - soft borders
\definecolor{gray-200}{RGB}{233,236,239}        % Very light gray - hover states
\definecolor{gray-100}{RGB}{247,248,250}        % Near white - subtle backgrounds

% --- WARM NEUTRALS ---
\definecolor{cream-100}{RGB}{252,250,246}       % Warm cream - inviting, friendly backgrounds


% ============================================================================
% LAYER 2: SEMANTIC COLOR MAPPINGS (Educational Content Types)
% ============================================================================
% Maps educational content types to primitive colors.
% USE THESE when creating educational content boxes and callouts.
% Each type follows the pattern: [type]-dark, [type]-base, [type]-light
% ============================================================================

% --- DEFINITION: Formal definitions, theoretical concepts, foundational knowledge ---
% Color: Blue (professional, formal, authoritative, structural)
% Use for: Formal term definitions, conceptual frameworks, theoretical foundations
\definecolor{definition-dark}{RGB}{30,58,95}    % slate-900 → text, strong borders
\definecolor{definition-base}{RGB}{71,101,135}  % slate-700 → medium borders, accents
\definecolor{definition-light}{RGB}{240,246,252}% slate-100 → box backgrounds

% --- EXAMPLE: Concrete examples, demonstrations, applications, case studies ---
% Color: Green (practical, real-world, natural, accessible)
% Use for: Code examples, real-world applications, demonstrations, case studies
\definecolor{example-dark}{RGB}{46,125,50}      % green-900 → text, strong borders
\definecolor{example-base}{RGB}{67,160,71}      % green-600 → medium borders, accents
\definecolor{example-light}{RGB}{232,245,233}   % green-100 → box backgrounds

% --- KEY: Important takeaways, essential concepts, memorable points ---
% Color: Orange/Amber (warm, attention-getting, important, energetic)
% Use for: Key takeaways, must-remember concepts, essential points, summaries
\definecolor{key-dark}{RGB}{230,124,0}          % amber-900 → text, strong borders
\definecolor{key-base}{RGB}{251,140,0}          % amber-600 → medium borders, accents
\definecolor{key-light}{RGB}{255,243,224}       % amber-100 → box backgrounds

% --- CAUTION: Warnings, pitfalls, common mistakes, things to avoid ---
% Color: Red (alert, warning, danger, critical attention)
% Use for: Common mistakes, pitfalls to avoid, deprecated patterns, errors
\definecolor{caution-dark}{RGB}{198,40,40}      % red-900 → text, strong borders
\definecolor{caution-base}{RGB}{229,57,53}      % red-600 → medium borders, accents
\definecolor{caution-light}{RGB}{255,235,238}   % red-100 → box backgrounds

% --- NOTE: General notes, asides, supplementary information, optional reading ---
% Color: Neutral/Gray (calm, supportive, non-intrusive, optional)
% Use for: Sidebar notes, historical context, tangential info, further reading
\definecolor{note-dark}{RGB}{73,80,87}          % gray-700 → text, strong borders
\definecolor{note-base}{RGB}{134,142,150}       % gray-500 → medium borders, accents
\definecolor{note-light}{RGB}{252,250,246}      % cream-100 → box backgrounds

% --- THEOREM: Mathematical statements, formal proofs, logical arguments ---
% Color: Indigo/Purple (abstract, theoretical, formal, rigorous)
% Use for: Theorems, lemmas, proofs, formal logical statements
\definecolor{theorem-dark}{RGB}{63,81,181}      % indigo-700 → text, strong borders
\definecolor{theorem-base}{RGB}{92,107,192}     % indigo-600 → medium borders, accents
\definecolor{theorem-light}{RGB}{232,234,246}   % indigo-100 → box backgrounds

% --- PRACTICE: Exercises, problems, hands-on activities, try-it-yourself ---
% Color: Teal (modern, technical, interactive, hands-on)
% Use for: Practice problems, exercises, coding challenges, interactive tasks
\definecolor{practice-dark}{RGB}{0,128,128}     % teal-700 → text, strong borders
\definecolor{practice-base}{RGB}{0,137,123}     % teal-600 → medium borders, accents
\definecolor{practice-light}{RGB}{224,242,241}  % teal-100 → box backgrounds


% ============================================================================
% LAYER 3: COMPONENT-SPECIFIC ALIASES
% ============================================================================
% Explicit naming for specific component usage (backgrounds, borders, text).
% These make code self-documenting and usage intentions clear.
% Use these when building custom components or tcolorbox definitions.
% ============================================================================

% --- General Text Colors (Body Text, Headings, Emphasis) ---
\definecolor{text-primary}{RGB}{30,32,34}       % gray-900 → main body text
\definecolor{text-secondary}{RGB}{73,80,87}     % gray-700 → secondary text, captions
\definecolor{text-muted}{RGB}{95,100,105}       % gray-600 → de-emphasized, metadata

% --- Background Colors for Content Boxes ---
\definecolor{bg-definition}{RGB}{240,246,252}   % definition-light → definition boxes
\definecolor{bg-example}{RGB}{232,245,233}      % example-light → example boxes
\definecolor{bg-key}{RGB}{255,243,224}          % key-light → key takeaway boxes
\definecolor{bg-caution}{RGB}{255,235,238}      % caution-light → warning boxes
\definecolor{bg-note}{RGB}{252,250,246}         % note-light → note/aside boxes
\definecolor{bg-theorem}{RGB}{232,234,246}      % theorem-light → theorem boxes
\definecolor{bg-practice}{RGB}{224,242,241}     % practice-light → practice boxes
\definecolor{bg-neutral}{RGB}{247,248,250}      % gray-100 → neutral backgrounds

% --- Border Colors for Content Boxes ---
\definecolor{border-definition}{RGB}{71,101,135}% definition-base → definition frames
\definecolor{border-example}{RGB}{67,160,71}    % example-base → example frames
\definecolor{border-key}{RGB}{251,140,0}        % key-base → key takeaway frames
\definecolor{border-caution}{RGB}{229,57,53}    % caution-base → warning frames
\definecolor{border-note}{RGB}{210,215,220}     % gray-300 → note frames
\definecolor{border-theorem}{RGB}{92,107,192}   % theorem-base → theorem frames
\definecolor{border-practice}{RGB}{0,137,123}   % practice-base → practice frames
\definecolor{border-neutral}{RGB}{210,215,220}  % gray-300 → subtle neutral borders

% --- Text Colors for Content Boxes ---
\definecolor{text-definition}{RGB}{30,58,95}    % definition-dark → definition text/titles
\definecolor{text-example}{RGB}{46,125,50}      % example-dark → example text/titles
\definecolor{text-key}{RGB}{230,124,0}          % key-dark → key takeaway text/titles
\definecolor{text-caution}{RGB}{198,40,40}      % caution-dark → warning text/titles
\definecolor{text-note}{RGB}{73,80,87}          % note-dark → note text/titles
\definecolor{text-theorem}{RGB}{63,81,181}      % theorem-dark → theorem text/titles
\definecolor{text-practice}{RGB}{0,128,128}     % practice-dark → practice text/titles

% --- Primary Structural Colors (Headings, Links, Main Theme) ---
\definecolor{primary}{RGB}{30,58,95}            % slate-900 → main brand/theme color
\definecolor{primary-light}{RGB}{71,101,135}    % slate-700 → lighter brand variant
\definecolor{accent}{RGB}{251,140,0}            % amber-600 → attention/emphasis accent


% ============================================================================
% LAYER 4: LEGACY COMPATIBILITY ALIASES
% ============================================================================
% Backward-compatible aliases preserving old color names.
% Allows gradual migration to new semantic system without breaking existing code.
% TODO: Gradually replace these throughout the document with semantic names.
% ============================================================================

\definecolor{agentblue}{RGB}{30,58,95}          % → slate-900 / primary
\definecolor{agentlightblue}{RGB}{240,246,252}  % → slate-100 / bg-definition
\definecolor{accentorange}{RGB}{191,97,35}      % Legacy (slightly warmer than amber-900)
\definecolor{highlightgray}{RGB}{252,250,246}   % → cream-100 / bg-note
\definecolor{bordergray}{RGB}{210,215,220}      % → gray-300 / border-neutral
\definecolor{darkgray}{RGB}{95,100,105}         % → gray-600 / text-muted

% Old semantic names from previous color system
\definecolor{primary-slate}{RGB}{30,58,95}      % → slate-900
\definecolor{accent-amber}{RGB}{191,97,35}      % Legacy warm amber
\definecolor{secondary-sage}{RGB}{82,121,111}   % Legacy sage green (no longer in palette)
\definecolor{bg-ice}{RGB}{240,246,252}          % → slate-100
\definecolor{bg-cream}{RGB}{252,250,246}        % → cream-100
\definecolor{bg-amber-light}{RGB}{254,245,237}  % Legacy (slightly different than amber-100)
\definecolor{bg-gray-cool}{RGB}{247,248,250}    % → gray-100
\definecolor{border-slate}{RGB}{71,101,135}     % → slate-700
\definecolor{border-sage}{RGB}{118,145,137}     % Legacy sage border (no longer in palette)

% ============================================================================
% TABLES - Modern styling
% ============================================================================
\usepackage{booktabs}    % Professional quality tables
\usepackage{longtable}   % Tables spanning multiple pages
\usepackage{array}
\usepackage{multirow}
\usepackage{tabularx}    % Better column sizing

% Table row coloring
\usepackage{colortbl}

% Landscape pages
\usepackage{pdflscape}
\renewcommand{\arraystretch}{1.3}  % Better row spacing

% ============================================================================
% BOXES AND VISUAL ELEMENTS
% ============================================================================
\usepackage{tcolorbox}
\tcbuselibrary{breakable,skins,theorems}

% Definition box style - Most prominent, formal definitions
\newtcolorbox{definitionbox}[1][]{
  enhanced,
  colback=bg-definition,
  colframe=border-definition,
  fonttitle=\bfseries\large,
  coltitle=white,
  boxrule=1.5pt,
  arc=3pt,
  left=10pt,
  right=10pt,
  top=10pt,
  bottom=10pt,
  breakable,
  drop shadow={shadow xshift=1pt, shadow yshift=-1pt, opacity=0.08},
  attach boxed title to top left={yshift=-2mm, xshift=4mm},
  boxed title style={
    colback=definition-dark,
    colframe=definition-dark,
    arc=2pt,
    boxrule=0pt
  },
  #1
}

% Highlight box style - Warm, inviting context and notes
\newtcolorbox{highlightbox}[1][]{
  enhanced,
  colback=bg-note,
  colframe=border-note,
  fonttitle=\bfseries,
  coltitle=note-dark,
  boxrule=1pt,
  arc=2.5pt,
  left=10pt,
  right=10pt,
  top=8pt,
  bottom=8pt,
  breakable,
  drop shadow={shadow xshift=0.5pt, shadow yshift=-0.5pt, opacity=0.05},
  #1
}

% Key takeaway box - Stands out for important points
\newtcolorbox{keybox}[1][]{
  enhanced,
  colback=bg-key,
  colframe=key-base,
  fonttitle=\bfseries\large,
  coltitle=white,
  boxrule=2pt,
  arc=3pt,
  left=12pt,
  right=12pt,
  top=10pt,
  bottom=10pt,
  breakable,
  drop shadow={shadow xshift=1pt, shadow yshift=-1pt, opacity=0.1},
  borderline west={3pt}{0pt}{key-base},
  attach boxed title to top left={yshift=-2mm, xshift=4mm},
  boxed title style={
    colback=key-base,
    colframe=key-base,
    arc=2pt,
    boxrule=0pt
  },
  #1
}

% Question box style - Distinct professional style for evaluation questions
\newtcolorbox{questionbox}[1][]{
  enhanced,
  colback=gray-100,
  colframe=gray-600,
  fonttitle=\bfseries,
  coltitle=white,
  boxrule=1.5pt,
  arc=2pt,
  left=10pt,
  right=10pt,
  top=8pt,
  bottom=8pt,
  breakable,
  borderline west={4pt}{0pt}{slate-700},
  attach boxed title to top left={yshift=-2mm, xshift=4mm},
  boxed title style={
    colback=slate-700,
    colframe=slate-700,
    arc=2pt,
    boxrule=0pt
  },
  #1
}

% ============================================================================
% LISTS - Better spacing and appearance
% ============================================================================
\usepackage{enumitem}
\setlist{
  itemsep=0.4em,
  parsep=0.2em,
  topsep=0.5em,
  leftmargin=1.5em
}

% Custom list for evaluation questions
\newlist{evallist}{enumerate}{1}
\setlist[evallist]{
  label=\textbf{Q\arabic*.},
  leftmargin=2em,
  itemsep=0.5em,
  labelsep=0.5em
}

% ============================================================================
% TIKZ for diagrams
% ============================================================================
\usepackage{tikz}
\usetikzlibrary{
  positioning,
  arrows.meta,
  shapes.geometric,
  shapes.misc,
  calc,
  decorations.pathreplacing,
  backgrounds,
  fit,
  shadows.blur
}

% Ensure TikZ uses the document's main font (Fira Sans)
\tikzset{
  every node/.style={font=\normalfont},
  every text node part/.style={font=\normalfont}
}

% ============================================================================
% SECTION STYLING
% ============================================================================
\usepackage{titlesec}
\usepackage{titletoc}

% Section formatting
\titleformat{\section}
  {\Large\bfseries\color{primary}}
  {\thesection}{1em}{}
  [\vspace{0.3em}{\color{primary}\titlerule[1.5pt]}\vspace{0.5em}]

% Subsection formatting
\titleformat{\subsection}
  {\large\bfseries\color{primary}}
  {\thesubsection}{1em}{}
  [\vspace{-0.3em}]

% Subsubsection formatting
\titleformat{\subsubsection}
  {\normalsize\bfseries\color{text-secondary}}
  {\thesubsubsection}{1em}{}

% Paragraph formatting (for \paragraph)
\titleformat{\paragraph}[runin]
  {\normalsize\bfseries\color{primary}}
  {}{0em}{}[.]
\titlespacing*{\paragraph}{0pt}{1em}{0.5em}

% ============================================================================
% TABLE OF CONTENTS STYLING
% ============================================================================
% Customize ToC title
\renewcommand{\contentsname}{%
  {\color{primary}\Large\bfseries Contents}%
}

% Section entries in ToC
\titlecontents{section}
  [1.5em]                          % left margin
  {\vspace{0.5em}\bfseries}        % above code
  {\color{primary}\contentslabel{1.5em}}  % numbered entry format
  {}                               % numberless entry format
  {\titlerule*[0.75pc]{.}\contentspage}  % filler and page

% Subsection entries in ToC
\titlecontents{subsection}
  [3.8em]                          % left margin
  {\vspace{0.2em}}                 % above code
  {\contentslabel{2.3em}}          % numbered entry format
  {}                               % numberless entry format
  {\titlerule*[0.75pc]{.}\contentspage}  % filler and page

% ============================================================================
% CAPTIONS - Cleaner appearance
% ============================================================================
\usepackage[
  font={small},
  labelfont={bf,color=primary},
  format=plain,
  justification=justified,
  skip=8pt
]{caption}

% ============================================================================
% HYPERLINKS AND PDF PROPERTIES
% ============================================================================
\usepackage{hyperref}
\hypersetup{
  colorlinks=true,
  linkcolor=primary,
  citecolor=accent,
  urlcolor=primary,
  bookmarksnumbered=true,
  pdfborder={0 0 0}
}

% Better cross-referencing
\usepackage{cleveref}
\crefname{section}{Section}{Sections}
\crefname{figure}{Figure}{Figures}
\crefname{table}{Table}{Tables}

% ============================================================================
% BIBLIOGRAPHY
% ============================================================================
\usepackage[
  backend=biber,
  style=authoryear,
  maxcitenames=2,
  maxbibnames=99,
  uniquename=false,
  uniquelist=false,
  sorting=nyt,
  dashed=false
]{biblatex}

% Citation formatting
\DeclareFieldFormat{citetitle}{\mkbibquote{#1}}
\DeclareFieldFormat[article]{citetitle}{#1}
\DeclareFieldFormat[inproceedings]{citetitle}{#1}
\DeclareFieldFormat[book]{citetitle}{\mkbibemph{#1}}
\DeclareFieldFormat[techreport]{citetitle}{#1}
\DeclareFieldFormat[misc]{citetitle}{#1}

% ============================================================================
% CUSTOM COMMANDS
% ============================================================================

% Highlighted text for key terms
\newcommand{\keyterm}[1]{\textbf{\color{primary}#1}}

% Three-level hierarchy terms
\newcommand{\levelone}[1]{\textbf{#1}}
\newcommand{\leveltwo}[1]{\textbf{#1}}
\newcommand{\levelthree}[1]{\textbf{#1}}


% ============================================================================
% BIBLIOGRAPHY SETUP
% ============================================================================
% Use consolidated bibliography at root level
\addbibresource{bib/refs.bib}

% ============================================================================
% DOCUMENT METADATA
% ============================================================================
\hypersetup{
  pdftitle={Artificial Intelligence for Law and Finance},
  pdfauthor={},
  pdfsubject={Artificial Intelligence, Law, Finance, Technology},
  pdfkeywords={AI, machine learning, legal technology, fintech, regulation, agentic systems}
}

% ============================================================================
% TITLE AND FRONT MATTER
% ============================================================================
\title{
  \vspace{-1em}
  {\Huge\bfseries\color{primary} Artificial Intelligence}\\[0.5em]
  {\Huge\bfseries\color{primary} for Law and Finance}\\[0.8em]
  {\large\itshape\color{text-secondary} A Comprehensive Textbook for Practitioners and Researchers}
  \vspace{0.5em}
}

\author{
  \vspace{-0.5em}
  {\large\color{text-muted} Draft Version}\\[0.3em]
  {\normalsize\color{text-muted} \today}
}

\date{}

% ============================================================================
% DOCUMENT CONTENT
% ============================================================================
\begin{document}

% Front matter
\frontmatter

\maketitle

% Decorative rule after title block
\vspace{-1em}
\noindent{\color{border-neutral}\rule{\textwidth}{0.5pt}}
\vspace{1.0em}

% Draft notice - professional blue-gray theme
\begin{center}
\begin{tcolorbox}[
    enhanced,
    width=0.85\textwidth,
    colback=bg-note,
    colframe=border-note,
    fonttitle=\bfseries\large\color{note-dark},
    boxrule=1.5pt,
    arc=4pt,
    left=16pt,
    right=16pt,
    top=12pt,
    bottom=12pt,
    breakable,
    drop shadow={shadow xshift=1pt, shadow yshift=-1pt, opacity=0.08},
    borderline west={4pt}{0pt}{note-dark}
  ]

  % Header - centered
  \begin{center}
    {\large\bfseries\color{note-dark} Working Draft Textbook} \\[0.2em]
    {\small\color{text-muted} Version 0.1 --- In Active Development}
  \end{center}

  \vspace{0.25em}

  % Body text - justified
  {\small
    This textbook is currently under active development. Individual chapters are being drafted, reviewed, and refined independently before integration into the complete book. Content, structure, and formatting are subject to change as the project evolves.

    \vspace{0.5em}

    This is a collaborative scholarly work aimed at legal and financial practitioners, regulators, and researchers. We welcome feedback and contributions.

    \vspace{0.5em}

    You can find the most current version of this project here:\\
    \url{https://github.com/mjbommar/ai-law-finance-book/}
  }

\end{tcolorbox}
\end{center}

\vspace{1em}

\tableofcontents
\newpage

% ============================================================================
% VISUAL GUIDE TO BOX TYPES
% ============================================================================
\chapter*{How to Read This Book}
\addcontentsline{toc}{chapter}{How to Read This Book}

\begin{highlightbox}[title={\textbf{Visual Cues: Box Types and Importance}}]
This textbook uses colored boxes to signal different types of content. Each box type has a distinct visual appearance to help you navigate:

\textbf{Box types by color:}
\begin{itemize}[nosep]
  \item \textbf{Definition boxes} (dark blue borders, ice blue backgrounds) — formal definitions and theoretical frameworks
  \item \textbf{Key takeaway boxes} (amber/orange borders, light yellow backgrounds) — essential concepts you should remember
  \item \textbf{Question boxes} (gray borders with dark blue accent) — evaluation rubrics and decision frameworks
  \item \textbf{Theorem boxes} (indigo borders, light purple backgrounds) — formal mathematical statements
  \item \textbf{Note boxes} (neutral gray borders, cream backgrounds) — supplementary context and asides
\end{itemize}

\textbf{Importance levels within boxes:}

To reduce visual overload, some boxes use an \textbf{importance} scale that adjusts border weight and opacity:
\begin{itemize}[nosep]
  \item \textbf{High} (bold borders, drop shadows) — emphatic takeaways and formal results
  \item \textbf{Medium} (standard appearance) — core definitions and primary guidance
  \item \textbf{Low} (lighter borders, subtle backgrounds) — supporting examples, historical milestones, or repeated concepts
\end{itemize}

You can skim efficiently by reading \textbf{high} and \textbf{medium} boxes first, then returning to \textbf{low} boxes when you want deeper context.
\end{highlightbox}

\vspace{1em}

Individual chapters may also include chapter-specific reading guidance to help you navigate according to your goals (quick reference, historical context, or comprehensive understanding).

\newpage

% Main content
\mainmatter

% ============================================================================
% FOUNDATIONS: LLMS AND PROMPTING
% ============================================================================
\part{Foundations: LLMs and Prompting}

% Foundations Chapter A — LLM Primer and Mechanics
\subfile{chapters/01-foundations-llm-primer-mechanics/main}

% Foundations Chapter B — Conversations and Reasoning
\subfile{chapters/02-foundations-conversations-reasoning/main}

% Foundations Chapter C — Structured Outputs and Tool Use
\subfile{chapters/03-foundations-structured-outputs-tools/main}

% Foundations Chapter D — Multimodal Fundamentals
\subfile{chapters/04-foundations-multimodal/main}

% Foundations Chapter E — Prompt Design, Evaluation, and Optimization
\subfile{chapters/05-foundations-prompt-design-eval-optimization/main}

% ============================================================================
% PART I: AGENTS AND AGENTIC SYSTEMS
% ============================================================================
\part{Agents and Agentic Systems}

% Chapter 1: What is an Agent?
\subfile{chapters/06-agents-part-1/main}

% Chapter 2: How to Build an Agent (Part II)
\subfile{chapters/07-agents-part-2/main}

% Chapter 3: How to Govern an Agent (Part III)
\subfile{chapters/08-agents-part-3/main}

% ============================================================================
% KNOWLEDGE GRAPHS & SEMANTIC WEB
% ============================================================================
\part{Knowledge Graphs \& Semantic Web}

% KG Chapter 1: Foundations
\subfile{chapters/09-kg-foundations/main}

% KG Chapter 2: Operations with LLMs
\subfile{chapters/10-kg-operations-llm/main}

% ============================================================================
% FUTURE CHAPTERS (Add as they are developed)
% ============================================================================
% \subfile{chapters/prompting-and-meta-prompting/main}
% \subfile{chapters/llm-foundations/main}
% \subfile{chapters/legal-applications/main}
% \subfile{chapters/financial-applications/main}

% ============================================================================
% BACK MATTER
% ============================================================================
\backmatter

% Unified bibliography for entire book
\printbibliography

\end{document}
