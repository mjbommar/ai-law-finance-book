% ============================================================================
% How to Read — Agents Part III (Govern)
% Audience: CRO, CCO, CFO, GC, CISOs, regulators
% Labels prefix: sec:agents3-*
%
% Executive mnemonic (comments only): APEX‑2D
%   A — Attribution/Provenance
%   P — Privilege/Confidentiality
%   E — Evidence/Audit
%   X — Escalation/eXit
%   Dials — Autonomy/Actuation; Jurisdiction/Localization
% This section should signpost where each APEX item is handled and remind
% readers that the two dials set control strength.
% ============================================================================

\section*{How to Read This Chapter}
\addcontentsline{toc}{section}{How to Read This Chapter}

This chapter is written for executives accountable for risk and compliance. It translates technical agent capabilities (Part I–II) into controls, tests, and deployment practices you can own. It assumes familiarity with Part I’s opening sections that define “agent” (Goal, Perception, Action), introduce the six operational properties for an “agentic system” (adds Iteration, Adaptation, Termination), and outline the four analytical dimensions (Autonomy, Entity Frames, Goal Dynamics, Persistence).

\vspace{0.5em}

\begin{highlightbox}[title={\textbf{Path 1: Executives (CRO/CCO/CFO/GC)}}]
Read Sections~\ref{sec:agents3-intro}--\ref{sec:agents3-deploy}. You’ll get (building directly on Part I’s definitions and evaluation rubric):
\begin{itemize}
  \item A practical map of laws, ethics, and standards
  \item A controls catalogue with COSO/COBIT/ISO/SOC alignment
  \item Evaluation and conformance tests before go‑live
  \item Procurement, SLA, and change‑control checklists
\end{itemize}
\end{highlightbox}

\vspace{0.5em}

\begin{highlightbox}[title={\textbf{Path 2: Risk and Compliance Teams}}]
Read Sections~\ref{sec:agents3-controls}--\ref{sec:agents3-conformance} for control design, evidence requirements, and test procedures.
\end{highlightbox}

\vspace{0.5em}

\begin{highlightbox}[title={\textbf{Path 3: Full Coverage}}]
Read end‑to‑end, including organizational readiness and case‑based patterns for different environments (firm, in‑house, court).
\end{highlightbox}

\vspace{1em}
\noindent\textcolor{border-neutral}{\rule{\textwidth}{1.5pt}}
