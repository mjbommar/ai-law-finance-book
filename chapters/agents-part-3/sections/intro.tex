% ============================================================================
% Introduction and Scope — Agents Part III (Govern)
% Purpose: Define governance scope, roles, oversight modes
% Label: sec:agents3-intro
% ============================================================================

% Executive mnemonic (comments only): APEX‑2D
%   A — Attribution/Provenance (source‑grounding, quote fidelity)
%   P — Privilege/Confidentiality (ethical walls, retention)
%   E — Evidence/Audit (complete, tamper‑evident logs)
%   X — Escalation/eXit (oversight modes, stop/rollback)
%   Dials — Autonomy/Actuation; Jurisdiction/Localization
% This section sets context for all four questions and both dials.

\section{Introduction and Scope}
\label{sec:agents3-intro}

Part III explains how to \textit{govern} agentic systems in practice. We focus on what a chief risk officer, compliance officer, CFO, general counsel, and security leadership need to approve, monitor, and continuously improve deployments.

\begin{keybox}[title={Executive Objectives}]
\begin{itemize}
  \item Map laws, ethics, and standards to concrete, testable controls.
  \item Establish evidence requirements, KPIs/SLAs, and change‑management gates.
  \item Ensure interoperability and auditability via protocol‑level metadata.
\end{itemize}
\end{keybox}

\paragraph{Roles and Oversight.} We distinguish \emph{provider}, \emph{deployer}, and \emph{user}. Oversight modes include human‑in‑the‑loop (pre‑approval), human‑on‑the‑loop (monitor/interrupt), and human‑in‑command (override/stop). Choice depends on task risk.

\vspace{0.5em}
\noindent\textcolor{border-neutral}{\rule{\textwidth}{1.5pt}}

\subsection*{Terminology Bridge from Part I}
\addcontentsline{toc}{subsection}{Terminology Bridge from Part I}

This governance chapter explicitly reuses the terms introduced at the beginning of Part I:

\begin{definitionbox}[title={Core Terms from Part I}]
\begin{itemize}
  \item \textbf{Agent (Level 1)}: possesses \emph{Goal}, \emph{Perception}, and \emph{Action}.
  \item \textbf{Agentic System}: adds \emph{Iteration}, \emph{Adaptation}, and \emph{Termination} to the Level 1 baseline.
  \item \textbf{Analytical Dimensions}: \emph{Autonomy}, \emph{Entity Frames} (human / hybrid / machine / institutional), \emph{Goal Dynamics} (accept, adapt, negotiate), and \emph{Persistence}.
  \item \textbf{6‑Question Evaluation Rubric}: practical test used to decide whether something is agentic in operation.
\end{itemize}
\end{definitionbox}

All governance requirements below map to these foundations. For example, autonomy and actuation scope drive oversight mode and approval gates; entity frame and persistence drive records, retention, and audit design; goal dynamics influence escalation triggers and SLAs.
