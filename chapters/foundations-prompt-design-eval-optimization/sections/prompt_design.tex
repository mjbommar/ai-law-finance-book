% =============================================================================
% Prompt Design — Prompt Design, Evaluation, Optimization
% Purpose: Prompts as specifications with success criteria
% Label: sec:llmD-design
% =============================================================================

\section{Prompt Design as Specification}
\label{sec:llmD-design}

% Outline (comments): roles, constraints, examples, success criteria, anti-patterns
\begin{definitionbox}[title={Specification Template}]
Include role, objective, constraints, inputs/outputs, success criteria, and examples. Add explicit refusal/edge-case guidance.
\end{definitionbox}

\subsection{JSON vs. XML Prompt Templates}
Provide minimal, unambiguous exemplars. Call out required fields and allowed values. Prefer JSON unless an XML interface is mandatory.

\subsection{Grammar/Schema Constraints (Optional)}
Where supported, use JSON schema or grammars to constrain outputs. Validate and retry on mismatch; log failures.

\subsection{Designing Few-Shot Exemplars}
Curate 3--5 demonstrations that reflect real variability. Include at least one hard negative and one boundary case. Keep examples concise to preserve context.

\subsection{Exemplar Libraries and Retrieval}
Organize examples with metadata (domain, jurisdiction, date, doc type). Retrieve relevant exemplars at runtime via semantic similarity; avoid leaking answers into prompts.

\subsection{Anti-Patterns}
Golden-path examples that hide complexity, leaking labels or solutions, and overly long exemplars that crowd out instructions.
