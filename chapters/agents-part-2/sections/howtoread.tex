% ============================================================================
% How to Read This Chapter — Agents Part II (Build)
% Audience paths and scope. Mirrors Part I style.
% Labels use prefix: sec:agents2-*
% Cross-refs use \Cref{} consistently per docs/style-guide.md
% ============================================================================

\section*{How to Read This Chapter}
\addcontentsline{toc}{section}{How to Read This Chapter}

This chapter explains \textit{how to build an agent}: the architectures, protocols, and evaluation methods that turn definitions from Part I into working legal AI systems.

\vspace{0.5em}

\begin{highlightbox}[title={\textbf{Path 1: Architects and Builders}}]
\textbf{Read Sections~\ref{sec:agents2-architecture}--\ref{sec:agents2-protocols}.} You'll get:
\begin{itemize}
  \item A reference architecture for legal agents
  \item Patterns for tools, memory, planning, and control
  \item Protocol choices and safety hooks (MCP/A2A)
\end{itemize}
\end{highlightbox}

\vspace{0.5em}

\begin{highlightbox}[title={\textbf{Path 2: Integrators and Vendors}}]
\textbf{Read Sections~\ref{sec:agents2-industry}--\ref{sec:agents2-eval}} for:
\begin{itemize}
  \item Industry implementation patterns and integration touchpoints
  \item Technical evaluation (retrieval, reasoning, workflows)
  \item Readiness for Part III governance requirements
\end{itemize}
\end{highlightbox}

\vspace{0.5em}

\begin{highlightbox}[title={\textbf{Path 3: Full Coverage}}]
\textbf{Read end-to-end} for:
\begin{itemize}
  \item A capability-centric build framework aligned to Part I
  \item Practical patterns you can adopt without vendor lock-in
  \item A clean handoff to governance in Part III
\end{itemize}
\end{highlightbox}

\vspace{1em}
\noindent\textcolor{border-neutral}{\rule{\textwidth}{1.5pt}}

