% ============================================================================
% Introduction and Scope — Agents Part II (Build)
% Purpose: Position Part II relative to Part I (concepts) and Part III (governance)
% Label prefix: sec:agents2-intro
% ============================================================================

\section{Introduction and Scope}
\label{sec:agents2-intro}

Part II turns definitions from Part I into \textit{running systems}. We present a reference architecture for legal agents, describe protocol choices that enable safe tool use and interoperability, and offer technical evaluation methods for retrieval, reasoning, and workflow completion.

\begin{keybox}[title={Key Objectives for Part II}]
\begin{itemize}
  \item Provide a vendor-neutral reference architecture aligned to the capability taxonomy from Part I.
  \item Specify protocol considerations (capability discovery, pre/postconditions, audit hooks) for agent-tool and agent-agent interactions.
  \item Define technical evaluation layers (retrieval, reasoning, workflow) that validate build choices before governance gates in Part III.
\end{itemize}
\end{keybox}

\paragraph{Relationship to Other Parts.} Part I defines \emph{what} an agent is and offers a capability-centric taxonomy. Part II explains \emph{how} to build those capabilities. Part III explains \emph{how} to govern and deploy them in practice.

