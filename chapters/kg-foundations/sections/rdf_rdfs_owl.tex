% =============================================================================
% RDF, RDFS, OWL — KG Foundations
% Labels: sec:kg-foundations-rdf, sec:kg-foundations-owl
% =============================================================================

\section{RDF Data Model and RDFS/OWL}
\label{sec:kg-foundations-rdf}

RDF 1.1 defines a graph data model of triples (subject, predicate, object), IRIs, literals with datatypes, and blank nodes. It provides a foundation for interoperable data across systems. \autocite{w3c-rdf11-concepts}

RDFS supplies basic vocabulary for classes and properties. OWL~2 adds richer semantics for classes, properties, and individuals; its profiles (EL, QL, RL) target tractable reasoning in different scenarios. \autocite{w3c-rdf-schema, w3c-owl2-profiles}

\begin{highlightbox}[title={Choosing an OWL 2 Profile}]
Use \textbf{EL} for large taxonomies/hierarchies (e.g., healthcare codes), \textbf{QL} for efficient querying over relational backends, and \textbf{RL} for rule-engine implementations. Profiles trade expressivity for predictable performance.
\end{highlightbox}

\paragraph{Note on RDF 1.2.} Quoted triples and related features are tracked in ongoing RDF 1.2 work; adopt when specifications reach Recommendation status.\autocite{w3c-rdf12-status}

\section{Vocabulary Design Basics}
\label{sec:kg-foundations-owl}
Model only what you need. Start with existing terms (e.g., FOAF, PROV, schema.org modules when appropriate) and mint new IRIs in a controlled namespace. Provide labels, definitions, examples, and version IRIs.

