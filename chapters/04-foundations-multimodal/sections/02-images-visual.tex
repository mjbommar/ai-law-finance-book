% =============================================================================
% Images and Visual Content — Multimodal Fundamentals
% Purpose: OCR, handwriting, diagrams, visual understanding
% Label: sec:llmD2-images
% =============================================================================

\section{Images and Visual Content}
\label{sec:llmD2-images}

Beyond structured documents, legal and financial workflows frequently encounter standalone images: scanned forms, photographs of evidence, screenshots of applications, diagrams, and handwritten notes. This section addresses image understanding, OCR for text extraction, and specialized visual content types.

\subsection{Image Understanding in Legal and Financial Context}
\label{sec:llmD2-images-context}

Images in professional settings differ from general-purpose computer vision tasks. The content is often text-heavy, formally structured, and carries legal or evidentiary significance:

\begin{itemize}
  \item \textbf{Evidence photographs}: Crime scenes, property damage, product defects.
  \item \textbf{Scanned historical documents}: Pre-digital records, handwritten ledgers, typed correspondence.
  \item \textbf{Application screenshots}: UI states as evidence in software disputes, compliance monitoring.
  \item \textbf{Whiteboards and notes}: Meeting artifacts, informal agreements, expert sketches.
  \item \textbf{Identity documents}: Passports, driver's licenses, corporate filings with seals.
\end{itemize}

Vision-language models (VLMs) like GPT-5, Claude 4.5, and Gemini 3 can interpret these images directly, providing descriptions, extracting visible text, and answering questions about content. However, for systematic processing at scale, specialized pipelines remain more reliable and cost-effective.

\subsection{OCR: From Scanned Documents to Text}
\label{sec:llmD2-ocr}

Optical Character Recognition (OCR) converts images of text into machine-readable characters. For legal and financial documents, OCR quality directly impacts downstream accuracy.

\paragraph{OCR Engines.}
Modern OCR options span a range of capabilities:

\begin{definitionbox}[title={OCR Engine Comparison}]
\begin{description}
  \item[Tesseract (Open Source)] Mature, widely used, supports 100+ languages. Accuracy varies with image quality; benefits from preprocessing. No cost for processing.
  \item[Google Cloud Vision OCR] High accuracy, handles complex layouts, returns confidence scores. Requires API access and incurs per-page costs.
  \item[Azure AI Document Intelligence] Combines OCR with layout analysis and field extraction. Pretrained models for invoices, receipts, and IDs. Enterprise pricing.
  \item[Amazon Textract] OCR with table and form extraction. Integrates with AWS ecosystem.
  \item[ABBYY FineReader] Commercial solution with high accuracy on complex documents. Often used in legal e-discovery.
\end{description}
\end{definitionbox}

\paragraph{Accuracy Metrics.}
OCR quality is measured by:

\begin{itemize}
  \item \textbf{Character Error Rate (CER)}: Percentage of characters incorrectly recognized.
  \item \textbf{Word Error Rate (WER)}: Percentage of words with any error.
  \item \textbf{Confidence scores}: Per-character or per-word confidence from the OCR engine.
\end{itemize}

For legal documents, even 99\% accuracy means errors in 1 of every 100 words---potentially changing the meaning of a contract clause. High-stakes applications require human review or multiple OCR passes.

\paragraph{Handling Low-Quality Scans.}
Scanned documents often suffer from:

\begin{itemize}
  \item \textbf{Low resolution}: Aim for 300 DPI minimum; 600 DPI for fine print.
  \item \textbf{Skew and rotation}: Apply deskewing algorithms before OCR.
  \item \textbf{Noise and artifacts}: Photocopier marks, fax degradation, coffee stains.
  \item \textbf{Poor contrast}: Faded ink, colored backgrounds, highlighted text.
\end{itemize}

\begin{keybox}[title={OCR Preprocessing Pipeline}]
\begin{enumerate}
  \item \textbf{Deskew}: Correct rotation and perspective distortion.
  \item \textbf{Denoise}: Remove speckles, lines, and artifacts.
  \item \textbf{Binarize}: Convert to black and white with adaptive thresholding.
  \item \textbf{Resize}: Scale to optimal resolution for the OCR engine.
  \item \textbf{Segment}: Identify text regions versus images, tables, and margins.
\end{enumerate}
\end{keybox}

\subsection{Handwriting and Signature Recognition}
\label{sec:llmD2-handwriting}

Handwritten content presents additional challenges beyond printed text.

\paragraph{Handwritten Text Recognition (HTR).}
HTR models are trained specifically on handwriting samples:

\begin{itemize}
  \item \textbf{Google Cloud Vision}: Supports handwriting detection with the \texttt{DOCUMENT\_TEXT\_DETECTION} feature.
  \item \textbf{Microsoft Azure}: Handwriting recognition through Document Intelligence.
  \item \textbf{AWS Textract}: Detects handwriting in form fields.
  \item \textbf{Transkribus}: Specialized platform for historical handwritten documents.
\end{itemize}

Accuracy varies significantly with handwriting legibility. Cursive, stylized, or hurried writing degrades recognition. For legal applications, handwritten amendments to contracts or handwritten wills require careful verification.

\paragraph{Signature Verification.}
Signatures serve as identity verification and consent indicators. Processing involves:

\begin{itemize}
  \item \textbf{Detection}: Locate signature regions within documents.
  \item \textbf{Extraction}: Isolate the signature image for analysis.
  \item \textbf{Comparison}: Match against known signature samples (for verification).
  \item \textbf{Authenticity assessment}: Detect potential forgeries or alterations.
\end{itemize}

\begin{cautionbox}[title={Legal Admissibility of Signature Analysis}]
Automated signature verification is not universally accepted as evidence of authenticity. Courts may require expert testimony to validate AI-based signature analysis. Document the methodology, confidence levels, and limitations when signature verification is legally significant.
\end{cautionbox}

\subsection{Diagrams, Flowcharts, and Schematics}
\label{sec:llmD2-diagrams}

Technical documents often contain diagrams that encode important information:

\paragraph{Common Diagram Types.}
\begin{itemize}
  \item \textbf{Organizational charts}: Corporate structure, reporting relationships.
  \item \textbf{Process flowcharts}: Workflow diagrams, decision trees.
  \item \textbf{Technical schematics}: Engineering drawings, circuit diagrams, architectural plans.
  \item \textbf{Patent figures}: Invention illustrations with labeled components.
  \item \textbf{Network diagrams}: System architecture, data flow.
\end{itemize}

\paragraph{Extraction Approaches.}
Diagram understanding combines visual and structural analysis:

\begin{itemize}
  \item \textbf{VLM description}: Ask a vision-language model to describe the diagram, identify entities, and explain relationships.
  \item \textbf{Object detection}: Identify boxes, arrows, and connectors as distinct elements.
  \item \textbf{Graph extraction}: Convert visual structure to a machine-readable graph (nodes and edges).
  \item \textbf{Label extraction}: OCR the text labels and associate them with visual elements.
\end{itemize}

\begin{highlightbox}[title={Patent Diagram Analysis}]
Patent filings typically include multiple figures with numbered components referenced in the claims. A patent analysis system should:
\begin{enumerate}
  \item Identify figure numbers and their bounding regions.
  \item Extract component labels (``102'', ``104a'', etc.).
  \item Link labels to their positions in the diagram.
  \item Cross-reference with claim text mentioning those components.
\end{enumerate}
\end{highlightbox}

\subsection{Image Classification and Document Routing}
\label{sec:llmD2-image-classification}

Before detailed processing, images must be classified and routed to appropriate pipelines.

\paragraph{Document Type Classification.}
Classify incoming images by document type:

\begin{itemize}
  \item \textbf{Invoices, receipts, purchase orders}: Route to accounts payable extraction.
  \item \textbf{Contracts and agreements}: Route to legal review and clause extraction.
  \item \textbf{Identity documents}: Route to KYC/AML verification.
  \item \textbf{Correspondence}: Route to communication logging.
  \item \textbf{Evidence photographs}: Route to case file with metadata capture.
\end{itemize}

\paragraph{Bates Stamps and Exhibit Labels.}
Legal documents in discovery often carry:

\begin{itemize}
  \item \textbf{Bates numbers}: Sequential identifiers for tracking pages across a case.
  \item \textbf{Exhibit stamps}: Labels indicating exhibit designation (``Plaintiff's Exhibit 1'').
  \item \textbf{Confidentiality markings}: Designations like ``Confidential'' or ``Attorneys' Eyes Only.''
\end{itemize}

Automated detection of these markings enables routing, access control, and citation generation.

\subsection{Screenshot and UI Analysis}
\label{sec:llmD2-screenshots}

Application screenshots serve as evidence in software disputes, compliance monitoring, and user research.

\paragraph{UI Element Detection.}
Specialized models can identify:

\begin{itemize}
  \item \textbf{Buttons, menus, and controls}: What actions were available.
  \item \textbf{Text fields and their content}: What data was displayed or entered.
  \item \textbf{Error messages and notifications}: System state at capture time.
  \item \textbf{Layout and visual hierarchy}: How information was presented.
\end{itemize}

\paragraph{Temporal Context.}
Screenshots capture a moment in time. Metadata should include:

\begin{itemize}
  \item Timestamp of capture (if available from system metadata).
  \item Application and version being captured.
  \item User context (if relevant and permissible).
  \item Screen resolution and display settings.
\end{itemize}

For compliance monitoring (e.g., trading desk surveillance), screenshots may be captured systematically and require automated analysis at scale.

\subsection{Visual PII and Privilege Markers}
\label{sec:llmD2-visual-pii}

Images may contain sensitive information that text-based PII detection misses.

\paragraph{Faces and Biometric Data.}
Photographs containing faces require special handling:

\begin{itemize}
  \item \textbf{Detection}: Identify face regions using face detection models.
  \item \textbf{Blurring/redaction}: Apply blur or solid overlay to protect identity.
  \item \textbf{Consent tracking}: Document whether face capture was consented.
  \item \textbf{Biometric regulations}: BIPA (Illinois), GDPR, and other laws restrict biometric data processing.
\end{itemize}

\paragraph{Signatures and Handwriting.}
Handwritten content may reveal identity:

\begin{itemize}
  \item Signatures are personally identifying and should be redacted in shared datasets.
  \item Handwriting style can be identifying; consider redaction for anonymization.
\end{itemize}

\paragraph{Privilege Stamps and Watermarks.}
Visual markers indicating privilege status:

\begin{itemize}
  \item \textbf{``Privileged and Confidential'' stamps}: Route to secure handling.
  \item \textbf{``Draft'' watermarks}: Indicate non-final status.
  \item \textbf{Firm letterheads}: May indicate attorney-client communication.
\end{itemize}

\begin{keybox}[title={Visual Content Privacy Checklist}]
\begin{itemize}
  \item Detect and handle faces per applicable biometric laws.
  \item Identify and redact signatures when anonymization is required.
  \item Recognize privilege stamps and route to appropriate controls.
  \item Strip or preserve EXIF metadata based on use case.
  \item Apply C2PA content credentials for provenance tracking.
\end{itemize}
\end{keybox}

