% Iterative Investigation with Explainable Adaptation - Clean Vertical Flow

\begin{figure}[htbp]
\centering
\begin{tikzpicture}[
    cycle box/.style={
        rounded corners=6pt,
        draw=#1,
        line width=1.2pt,
        fill=#1!10,
        minimum width=5cm,
        minimum height=1.2cm,
        font=\small,
        align=center
    },
    detail box/.style={
        rounded corners=4pt,
        draw=#1,
        line width=0.6pt,
        fill=#1!5,
        text width=4.5cm,
        inner sep=5pt,
        font=\scriptsize,
        align=left
    },
    decision/.style={
        rounded corners=2pt,
        draw=amber-600,
        line width=1pt,
        fill=amber-100,
        minimum width=1.6cm,
        minimum height=1cm,
        inner sep=3pt,
        font=\scriptsize,
        align=center
    },
    termination/.style={
        rounded corners=4pt,
        draw=red-600,
        line width=1pt,
        fill=red-100,
        minimum width=2.6cm,
        minimum height=1cm,
        inner sep=4pt,
        font=\scriptsize,
        align=center
    },
    lowrisk/.style={
        rounded corners=4pt,
        draw=gray-500,
        line width=0.6pt,
        fill=gray-100,
        minimum width=2.2cm,
        minimum height=0.6cm,
        font=\scriptsize,
        align=center
    },
    arrow/.style={->, thick, gray-700, >=stealth, line width=1pt},
    loop arrow/.style={->, thick, green-900, >=stealth, line width=1.2pt},
    label text/.style={font=\tiny, text=gray-600}
]

% === CYCLE 1 with detail box ===
\node[cycle box=slate-700] (cycle1) at (-2.5,0) {
    \textbf{\textcolor{slate-900}{Cycle 1: Initial Risk Scoring}}
};

\node[detail box=slate-700, anchor=north west] (c1details) at (1.5,0.6) {
    \textbf{Risk Criteria:}\\[2pt]
    High-value (>\$500K): +3\\
    Overdue >90 days: +2\\
    New customer: +1\\
    Prior adjustments: +2\\
    Related-party: +3
};

% === DECISION 1 ===
\node[decision, below=0.8cm of cycle1] (dec1) {
    Score\\$\geq$6?
};

% Low risk exit - to the left
\node[lowrisk, left=1.2cm of dec1] (lowrisk) {
    Standard procedures
};

% === CYCLE 2-N with detail box ===
\node[cycle box=green-900, below=1.5cm of dec1] (cycle2) {
    \textbf{\textcolor{green-900}{Cycles 2--N: Adaptive Investigation}}
};

\node[detail box=green-900, anchor=north west] (c2details) at (1.5,-3.2) {
    \textbf{Actions:}\\
    Request documents\\
    Analyze payment patterns\\
    Cross-reference data\\[3pt]
    \textbf{Adaptation:}\\
    Incomplete $\rightarrow$ request more\\
    Disputes $\rightarrow$ legal review\\
    Fraud signals $\rightarrow$ escalate
};

% === DECISION 2 ===
\node[decision, below=0.8cm of cycle2] (dec2) {
    Continue?
};

% === TERMINATION OUTCOMES ===
\node[termination, below=1cm of dec2, xshift=-3cm] (term1) {
    \textbf{\textcolor{red-900}{Sufficient Evidence}}\\
    \textcolor{gray-600}{(conf.\ >0.85)}
};

\node[termination, below=1cm of dec2] (term2) {
    \textbf{\textcolor{red-900}{Red Flag}}\\
    \textcolor{gray-600}{$\rightarrow$ Escalate}
};

\node[termination, below=1cm of dec2, xshift=3cm] (term3) {
    \textbf{\textcolor{red-900}{Max Cycles}}\\
    \textcolor{gray-600}{(5 reached)}
};

% === ARROWS ===
% Cycle 1 to Decision 1
\draw[arrow] (cycle1.south) -- (dec1.north);

% Decision 1: Yes to Cycle 2
\draw[arrow] (dec1.south) -- node[right, label text] {Yes} (cycle2.north);

% Decision 1: No to low risk (left)
\draw[arrow] (dec1.west) -- node[above, label text] {No} (lowrisk.east);

% Cycle 2 to Decision 2
\draw[arrow] (cycle2.south) -- (dec2.north);

% Decision 2 to terminations
\draw[arrow] (dec2.south) -- ++(0,-0.3) -| (term1.north);
\draw[arrow] (dec2.south) -- (term2.north);
\draw[arrow] (dec2.south) -- ++(0,-0.3) -| (term3.north);

% Loop back arrow - on left side, wide enough to clear the box
\draw[loop arrow, rounded corners=6pt] (dec2.west) -- ++(-4.5,0) node[below, font=\scriptsize, text=green-900] {Next cycle} -- ++(0,2.3) -- (cycle2.west);

\end{tikzpicture}
\caption{Iterative investigation workflow with explainable adaptation. Cycle~1 scores accounts using risk criteria; high-risk accounts (score $\geq$6) proceed to iterative investigation. Cycles~2--N gather evidence and adapt strategy based on findings. Investigation terminates when sufficient evidence is obtained, a red flag requires escalation, or maximum cycles are reached.}
\label{fig:agents3-iterative-investigation}
\end{figure}
