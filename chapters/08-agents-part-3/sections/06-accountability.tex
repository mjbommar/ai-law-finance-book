% ============================================================================
% Accountability and Organizational Structure — Agents Part III
% Purpose: Define roles, organizational models, and liability allocation
% Label: sec:agents3-accountability
% ============================================================================

\section{Accountability and Organizational Structure}
\label{sec:agents3-accountability}

Technical controls alone do not create accountability. Governance requires explicit assignment of roles and responsibilities: who approves deployments, who monitors performance, who investigates incidents, who escalates to regulators? This section presents three organizational governance models, demonstrates role assignment through RACI matrices, defines escalation and reporting structures, and examines liability allocation. The goal is to ensure every governance activity has a clearly accountable owner.

\subsection{Three Organizational Governance Models}
\label{sec:agents3-governance-models}

Organizations structure AI governance in three primary ways, each with advantages and disadvantages depending on size, AI maturity, and regulatory intensity.

\paragraph{Centralized Model: Single AI Governance Office}
A dedicated AI governance office or committee reports to senior leadership (typically the Chief Risk Officer, Chief Compliance Officer, or Chief Technology Officer). This office establishes policies, reviews all proposed AI deployments, conducts risk assessments, and monitors compliance. This model suits small to medium organizations (500-2,000 employees) with limited AI systems (5-20 use cases), high regulatory stakes (financial services, healthcare, legal), or early AI maturity where governance capability is being built.

\smallskip
\noindent
\begin{minipage}[t]{0.48\textwidth}
\vspace{0pt}
\begin{tcolorbox}[
  enhanced,
  colback=bg-example,
  colframe=example-dark,
  fonttitle=\bfseries,
  coltitle=white,
  title=Advantages,
  boxrule=1pt,
  arc=2pt,
  left=6pt, right=6pt, top=4pt, bottom=4pt,
  equal height group=centralized,
  valign=top
]
\small
\textbf{Consistency}: Single office ensures uniform governance standards across all systems.\\[3pt]
\textbf{Expertise concentration}: Governance specialists develop deep knowledge of regulatory requirements and best practices.\\[3pt]
\textbf{Clear accountability}: One office owns all AI governance decisions.\\[3pt]
\textbf{Easier audit}: Regulators and internal auditors interact with a single governance function.
\end{tcolorbox}
\end{minipage}%
\hfill
\begin{minipage}[t]{0.48\textwidth}
\vspace{0pt}
\begin{tcolorbox}[
  enhanced,
  colback=bg-caution,
  colframe=caution-dark,
  fonttitle=\bfseries,
  coltitle=white,
  title=Disadvantages,
  boxrule=1pt,
  arc=2pt,
  left=6pt, right=6pt, top=4pt, bottom=4pt,
  equal height group=centralized,
  valign=top
]
\small
\textbf{Bottleneck risk}: All deployment decisions route through one office, creating delays.\\[3pt]
\textbf{Limited domain expertise}: Central office may lack deep knowledge of domain-specific requirements (e.g., PCAOB audit standards, ECOA fair lending nuances).\\[3pt]
\textbf{Scalability}: As AI adoption grows, central office becomes overwhelmed.
\end{tcolorbox}
\end{minipage}

\medskip

\textbf{Example}: Regional investment advisory firm (500 employees, 10 AI tools) establishes AI Governance Office under Chief Compliance Officer with governance lead, technical specialist, and support staff conducting quarterly system reviews.

\paragraph{Federated Model: Central Coordination with Distributed Expertise}
A central AI governance function establishes enterprise-wide policies and standards, while domain-specific governance teams (e.g., audit practice AI lead, tax practice AI lead, wealth management AI lead) implement and monitor compliance within their areas. The central function coordinates, audits federated teams, and escalates enterprise-wide issues. This model suits large organizations (5,000+ employees) with diverse AI use cases across multiple domains (50+ systems), mature AI adoption, and domain-specific regulatory requirements (audit, legal, banking, securities).

\smallskip
\noindent
\begin{minipage}[t]{0.48\textwidth}
\vspace{0pt}
\begin{tcolorbox}[
  enhanced,
  colback=bg-example,
  colframe=example-dark,
  fonttitle=\bfseries,
  coltitle=white,
  title=Advantages,
  boxrule=1pt,
  arc=2pt,
  left=6pt, right=6pt, top=4pt, bottom=4pt,
  equal height group=federated,
  valign=top
]
\small
\textbf{Domain expertise}: Practice leads understand PCAOB standards, tax regulations, or wealth management suitability rules better than a central office.\\[3pt]
\textbf{Scalability}: Distributed teams prevent central bottlenecks.\\[3pt]
\textbf{Tailored governance}: Each domain calibrates controls to specific regulatory and risk contexts.
\end{tcolorbox}
\end{minipage}%
\hfill
\begin{minipage}[t]{0.48\textwidth}
\vspace{0pt}
\begin{tcolorbox}[
  enhanced,
  colback=bg-caution,
  colframe=caution-dark,
  fonttitle=\bfseries,
  coltitle=white,
  title=Disadvantages,
  boxrule=1pt,
  arc=2pt,
  left=6pt, right=6pt, top=4pt, bottom=4pt,
  equal height group=federated,
  valign=top
]
\small
\textbf{Inconsistency risk}: Different domains may interpret policies differently or adopt varying standards.\\[3pt]
\textbf{Coordination overhead}: Central function must monitor multiple federated teams.\\[3pt]
\textbf{Accountability diffusion}: Harder to pinpoint responsibility when governance is distributed.
\end{tcolorbox}
\end{minipage}

\medskip

\textbf{Example}: Big Four accounting firm (10,000 employees, 50+ AI tools) establishes central AI Governance Committee setting firm-wide policies while each practice (audit, tax, advisory) designates domain-specific AI Leads ensuring compliance with practice-specific regulations (PCAOB, IRS, client confidentiality).

\paragraph{Embedded Model: Governance Within Existing Functions}
AI governance is integrated into existing risk management, compliance, IT governance, and legal functions rather than creating a separate AI-specific structure. Each function applies its existing governance processes to AI systems. This model suits organizations with mature, well-functioning governance (strong ERM, compliance, IT governance), AI systems that extend existing processes (e.g., AI-enhanced fraud detection within existing fraud team), and leadership that prefers integration over new silos.

\smallskip
\noindent
\begin{minipage}[t]{0.48\textwidth}
\vspace{0pt}
\begin{tcolorbox}[
  enhanced,
  colback=bg-example,
  colframe=example-dark,
  fonttitle=\bfseries,
  coltitle=white,
  title=Advantages,
  boxrule=1pt,
  arc=2pt,
  left=6pt, right=6pt, top=4pt, bottom=4pt,
  equal height group=embedded,
  valign=top
]
\small
\textbf{Efficiency}: Leverages existing governance infrastructure.\\[3pt]
\textbf{Avoids silos}: Prevents AI governance from operating in isolation from enterprise risk management.\\[3pt]
\textbf{Cultural fit}: Organizations resistant to new bureaucracy prefer extending existing processes.
\end{tcolorbox}
\end{minipage}%
\hfill
\begin{minipage}[t]{0.48\textwidth}
\vspace{0pt}
\begin{tcolorbox}[
  enhanced,
  colback=bg-caution,
  colframe=caution-dark,
  fonttitle=\bfseries,
  coltitle=white,
  title=Disadvantages,
  boxrule=1pt,
  arc=2pt,
  left=6pt, right=6pt, top=4pt, bottom=4pt,
  equal height group=embedded,
  valign=top
]
\small
\textbf{Expertise gaps}: Existing functions may lack AI-specific knowledge (fairness testing, model validation, adversarial robustness).\\[3pt]
\textbf{Accountability ambiguity}: If AI governance is ``everyone's responsibility,'' it may become no one's priority.\\[3pt]
\textbf{Inconsistent application}: Different functions may apply AI governance unevenly.
\end{tcolorbox}
\end{minipage}

\medskip

This model requires AI-specific training for existing governance personnel and clear assignment of AI oversight responsibilities within each function.

\subsection{RACI Matrix: Operationalizing Accountability}
\label{sec:agents3-raci}

Regardless of governance model, organizations must assign accountability for each governance activity using a RACI framework:

\smallskip
\noindent
\begin{minipage}[t]{0.235\textwidth}
\vspace{0pt}
\begin{tcolorbox}[
  enhanced,
  colback=bg-definition,
  colframe=definition-base,
  fonttitle=\bfseries\Large,
  coltitle=white,
  title=\centering R,
  boxrule=1pt,
  arc=2pt,
  left=4pt, right=4pt, top=4pt, bottom=4pt,
  equal height group=raci,
  valign=top
]
\centering
\small\textbf{Responsible}\\[4pt]
\scriptsize Who does the work?\\[6pt]
\textit{May be multiple people}
\end{tcolorbox}
\end{minipage}%
\hfill
\begin{minipage}[t]{0.235\textwidth}
\vspace{0pt}
\begin{tcolorbox}[
  enhanced,
  colback=bg-definition,
  colframe=definition-base,
  fonttitle=\bfseries\Large,
  coltitle=white,
  title=\centering A,
  boxrule=1pt,
  arc=2pt,
  left=4pt, right=4pt, top=4pt, bottom=4pt,
  equal height group=raci,
  valign=top
]
\centering
\small\textbf{Accountable}\\[4pt]
\scriptsize Who has decision authority and ultimate accountability?\\[6pt]
\textit{\textbf{Only one A per activity}}
\end{tcolorbox}
\end{minipage}%
\hfill
\begin{minipage}[t]{0.235\textwidth}
\vspace{0pt}
\begin{tcolorbox}[
  enhanced,
  colback=bg-definition,
  colframe=definition-base,
  fonttitle=\bfseries\Large,
  coltitle=white,
  title=\centering C,
  boxrule=1pt,
  arc=2pt,
  left=4pt, right=4pt, top=4pt, bottom=4pt,
  equal height group=raci,
  valign=top
]
\centering
\small\textbf{Consulted}\\[4pt]
\scriptsize Who provides input or expertise before decisions?\\[6pt]
\textit{Two-way communication}
\end{tcolorbox}
\end{minipage}%
\hfill
\begin{minipage}[t]{0.235\textwidth}
\vspace{0pt}
\begin{tcolorbox}[
  enhanced,
  colback=bg-definition,
  colframe=definition-base,
  fonttitle=\bfseries\Large,
  coltitle=white,
  title=\centering I,
  boxrule=1pt,
  arc=2pt,
  left=4pt, right=4pt, top=4pt, bottom=4pt,
  equal height group=raci,
  valign=top
]
\centering
\small\textbf{Informed}\\[4pt]
\scriptsize Who is notified after decisions?\\[6pt]
\textit{One-way communication}
\end{tcolorbox}
\end{minipage}

\medskip
\noindent The key principle: \textbf{every governance activity must have exactly one Accountable party}. Diffused accountability (``the team is accountable'') creates gaps where no one takes ownership.

Table~\ref{tab:agents3-raci} provides a sample RACI matrix for AI governance activities.

\begin{table}[!ht]
\centering
\caption{Sample RACI Matrix for AI Governance Activities}
\label{tab:agents3-raci}
\footnotesize
\begin{tabular}{>{\raggedright\arraybackslash}p{4.0cm} >{\centering\arraybackslash}p{1.5cm} >{\centering\arraybackslash}p{1.4cm} >{\centering\arraybackslash}p{1.5cm} >{\centering\arraybackslash}p{1.5cm} >{\centering\arraybackslash}p{2.0cm}}
\toprule
\textbf{Activity} & \textbf{Board / CEO} & \textbf{CRO / CCO} & \textbf{AI Gov. Lead} & \textbf{System Owner} & \textbf{Legal / Compliance} \\
\midrule
Approve enterprise AI governance policy & A & C & R & I & C \\
\addlinespace
Approve low-risk AI deployment & I & I & A & R & C \\
\addlinespace
Approve high-risk AI deployment & A & C & R & R & C \\
\addlinespace
Conduct pre-deployment risk assessment & I & C & A & R & C \\
\addlinespace
Monitor system performance (ongoing) & I & I & C & A, R & I \\
\addlinespace
Investigate fairness violation & I & A & C & R & C \\
\addlinespace
Approve vendor contract (high-risk system) & I & A & C & R & C \\
\addlinespace
Report to board (quarterly AI governance update) & I & A & R & I & C \\
\addlinespace
Respond to regulatory inquiry & C & A & R & R & R \\
\bottomrule
\end{tabular}
\end{table}

\paragraph{Key Observations from the Matrix}:

\begin{itemize}
\item \textbf{Single Accountability}: Each activity has one A. For example, the CRO (Chief Risk Officer) is accountable for fairness violation investigations; the AI Governance Lead is accountable for low-risk deployments.

\item \textbf{Escalation}: High-risk deployments elevate accountability to the Board/CEO, while low-risk deployments can be approved by the AI Governance Lead. This prevents bottlenecks (Board does not review every chatbot deployment) while ensuring senior oversight for consequential systems.

\item \textbf{Multiple Responsible Parties}: Risk assessments may involve both the AI Governance Lead (methodological expertise) and the System Owner (domain knowledge). Both contribute, but only one is Accountable for the final approval.

\item \textbf{Consultation and Information Flow}: Legal and Compliance are Consulted on most activities, ensuring regulatory considerations inform decisions. The Board is Informed of governance activities but not burdened with operational details.
\end{itemize}

Organizations should customize this matrix to their structure, size, and regulatory context. The principle—single accountability per activity—remains universal.

\subsection{Escalation and Reporting}
\label{sec:agents3-escalation-reporting}

Governance requires clear escalation triggers: when must an operational issue be escalated to management, executives, or the board? And what cadence and format should governance reporting follow?

\paragraph{Three-Tier Escalation Model}

% Three-Tier Escalation Model - Traffic Light Visualization

\begin{figure}[htbp]
\centering
\begin{tikzpicture}[scale=0.80, every node/.append style={transform shape},
    remember picture,
    tier circle/.style={circle, minimum size=2.2cm, inner sep=0pt, draw=gray-600, line width=1.5pt, font=\bfseries\small, align=center},
    tier box/.style={rounded corners=4pt, draw=#1, line width=1pt, inner sep=8pt, text width=10cm, font=\small}
]

% Draw text boxes first to establish positions
% Tier 3 description box
\node[tier box=caution-dark, fill=bg-caution, anchor=north west] (box3) at (2.0,0) {%
\textbf{\textcolor{caution-dark}{Material risk requiring immediate executive/board action}}\\[2pt]
\textbf{Examples}: Fairness violation, security breach, regulatory inquiry, systemic failure\\[2pt]
\textbf{Escalate to}: CEO, Board Risk Committee, General Counsel\\[2pt]
\textbf{Timeline}: Immediate (within hours)
};

% Tier 2 description box - positioned directly below box3
\node[tier box=key-dark, fill=bg-key, anchor=north west] (box2) at (box3.south west) {%
\textbf{\textcolor{key-dark}{Issues requiring management review and potential changes}}\\[2pt]
\textbf{Examples}: SLA breach, vendor violation, discrimination complaint, data drift\\[2pt]
\textbf{Escalate to}: CRO, CCO, or AI Governance Committee\\[2pt]
\textbf{Timeline}: 48--72 hours
};

% Tier 1 description box - positioned directly below box2
\node[tier box=example-dark, fill=bg-example, anchor=north west] (box1) at (box2.south west) {%
\textbf{\textcolor{example-dark}{Routine issues managed without escalation}}\\[2pt]
\textbf{Examples}: Low-confidence prediction, minor accuracy fluctuation, user complaint\\[2pt]
\textbf{Handled by}: System owners and AI governance teams\\[2pt]
\textbf{Timeline}: Standard operating procedures
};

% Traffic light housing - fixed width centered at x=0
\fill[gray-800, rounded corners=8pt]
    (-1.3,0 |- box3.north) rectangle (1.3,0 |- box1.south);

% Calculate positions for circles using the center of each box
\node[tier circle, fill=bg-caution, text=caution-dark] (tier3) at (0,0 |- box3.center) {Tier 3\\Critical};
\node[tier circle, fill=bg-key, text=key-dark] (tier2) at (0,0 |- box2.center) {Tier 2\\Significant};
\node[tier circle, fill=bg-example, text=example-dark] (tier1) at (0,0 |- box1.center) {Tier 1\\Operational};

\end{tikzpicture}
\caption{Three-tier escalation model for AI governance issues. Tier 1 (operational) issues are handled routinely; Tier 2 (significant) issues require management review; Tier 3 (critical) issues demand immediate executive or board action. Pre-defining which issues fall into each tier ensures rapid, consistent response.}
\label{fig:agents3-escalation-tiers}
\end{figure}


\paragraph{Reporting Cadence and Audience}

\textbf{Operational Dashboards (Daily/Weekly)}: System owners and AI governance teams monitor real-time or near-real-time dashboards showing performance metrics, error rates, escalation counts, user feedback. These are working tools, not executive reports.

\textbf{Management Reports (Monthly/Quarterly)}: Chief Risk Officer and Chief Compliance Officer receive summary reports: number of systems deployed, risk assessments completed, incidents investigated, SLA compliance, vendor performance, upcoming regulatory developments. Format: 2-5 page executive summary with supporting appendices.

\textbf{Board Presentations (Quarterly/Annual)}: Board receives narrative synthesis: strategic governance posture (are we ahead of or behind regulatory curve?), high-risk system approvals, material incidents and responses, policy changes, budget and resource requests. Format: 10-15 slide deck; focus on risk appetite alignment, not operational details.

\paragraph{Example Escalation: Fairness Violation in Credit Decisioning}
A bank's monthly fairness monitoring detects disparate impact in credit pre-screening (see Section~\ref{sec:agents3-monitoring-incident}). Figure~\ref{fig:agents3-escalation-example} illustrates the Tier 1 → Tier 3 escalation pathway, demonstrating how pre-defined critical issues trigger rapid organizational response with specific time targets at each stage.

% Example Escalation: Fairness Violation - Vertical Timeline

\begin{figure}[htbp]
\centering
\begin{tikzpicture}[
    step box/.style={rounded corners=4pt, draw=#1, line width=1pt, inner sep=8pt, text width=9.5cm, font=\small},
    time marker/.style={font=\small\bfseries, text=gray-600, anchor=east},
    tier marker/.style={font=\footnotesize\bfseries, anchor=west},
    arrow/.style={->, thick, gray-600, >=stealth}
]

% Step 1 - Initial Detection (Tier 1 - Green)
\node[step box=green-900, fill=green-100] (step1) at (0,0) {%
\textbf{\textcolor{green-900}{1. Initial Detection}}\\[2pt]
Compliance analyst (Tier 1 monitoring) identifies 80\% rule violation in credit pre-screening system.
};
\node[time marker] at ([xshift=-0.5cm]step1.west) {T+0};
\node[tier marker, text=green-900] at ([xshift=0.3cm]step1.east) {TIER 1};

% Arrow 1->2
\draw[arrow] (step1.south) -- ++(0,-0.3);

% Step 2 - Immediate Escalation (Tier 3 - Red)
\node[step box=red-900, fill=red-100, below=0.5cm of step1] (step2) {%
\textbf{\textcolor{red-900}{2. Immediate Escalation to Tier 3}}\\[2pt]
Fairness violations are pre-defined as critical (ECOA enforcement risk; regulatory penalties; class action exposure). Analyst notifies Chief Risk Officer.
};
\node[time marker] at ([xshift=-0.5cm]step2.west) {< 2 hrs};
\node[tier marker, text=red-900] at ([xshift=0.3cm]step2.east) {TIER 3};

% Arrow 2->3
\draw[arrow] (step2.south) -- ++(0,-0.3);

% Step 3 - CRO Response (Tier 3 - Red)
\node[step box=red-900, fill=red-100, below=0.5cm of step2] (step3) {%
\textbf{\textcolor{red-900}{3. CRO Response}}\\[2pt]
Halt system immediately (emergency stop protocol); convene incident response team (Legal, Compliance, AI Gov.\ Lead, System Owner); notify CEO and Board Risk Committee.
};
\node[time marker] at ([xshift=-0.5cm]step3.west) {< 24 hrs};

% Arrow 3->4
\draw[arrow] (step3.south) -- ++(0,-0.3);

% Step 4 - Board Involvement (Tier 3 - Red)
\node[step box=red-900, fill=red-100, below=0.5cm of step3] (step4) {%
\textbf{\textcolor{red-900}{4. Board Involvement}}\\[2pt]
Emergency board meeting (if material); approve remediation plan (model retraining, affected applicant notification, CFPB self-report); allocate investigation budget.
};
\node[time marker] at ([xshift=-0.5cm]step4.west) {24--72 hrs};

% Arrow 4->5
\draw[arrow] (step4.south) -- ++(0,-0.3);

% Step 5 - Regulatory Report
\node[step box=slate-700, fill=slate-100, below=0.5cm of step4] (step5) {%
\textbf{\textcolor{slate-700}{5. Regulatory Report}}\\[2pt]
General Counsel files self-report to CFPB within regulatory timeline.
};
\node[time marker] at ([xshift=-0.5cm]step5.west) {Per reg.};

\end{tikzpicture}
\caption{Example escalation pathway for a fairness violation detected in credit decisioning. The pathway demonstrates rapid Tier 1 to Tier 3 escalation for pre-defined critical issues, with specific time targets at each stage ensuring prompt organizational response.}
\label{fig:agents3-escalation-example}
\end{figure}


This escalation pathway ensures the organization responds rapidly to critical risks and maintains board-level visibility into material governance failures.

\subsection{Liability Allocation: Who Bears the Risk?}
\label{sec:agents3-liability}

A foundational reality shapes AI governance: \textbf{liability concentrates on deployers, not vendors or technology}. Understanding this allocation is essential for calibrating governance investments.

\paragraph{Deployers Bear Primary Liability}
When an AI system causes harm—discriminates against a protected class, provides inaccurate advice, breaches confidentiality—the deploying organization faces legal consequences:

\begin{itemize}
\item \textbf{Regulatory penalties}: ECOA violations, GDPR breaches, professional responsibility sanctions.
\item \textbf{Civil liability}: Class actions, individual lawsuits, breach of fiduciary duty claims.
\item \textbf{Reputational harm}: Client defection, loss of trust, negative publicity.
\end{itemize}

The fact that the system was purchased from a reputable vendor, relies on cutting-edge technology, or was approved by experts does not shield the deployer from liability. Professional duties (attorney competence, fiduciary obligations, auditor independence) are non-delegable.

\paragraph{Vendor Liability is Limited by Contract}
Vendor contracts typically shift risk to deployers through:

\begin{itemize}
\item \textbf{Liability caps}: ``Vendor's total liability shall not exceed fees paid in the prior 12 months.'' For a \$50,000/year SaaS subscription, this caps vendor exposure at \$50,000—insufficient to cover a \$5 million ECOA class action settlement or \$10 million GDPR penalty.
\item \textbf{Warranty disclaimers}: ``Vendor makes no warranties regarding accuracy, completeness, or fitness for a particular purpose.'' Deployers cannot recover damages for model hallucinations or bias if the vendor disclaimed such warranties.
\item \textbf{Indemnification limits}: Vendors may indemnify only for certain risks (e.g., IP infringement) but exclude liability for ``deployer's use of the system.''
\end{itemize}

\paragraph{Governance as Primary Defense}
Since deployers bear most liability and cannot fully recover from vendors, \emph{governance becomes the primary defense}:

\begin{itemize}
\item \textbf{Regulatory defense}: Demonstrating reasonable care through documented risk assessments, monitoring, and incident response may reduce penalties or satisfy regulatory expectations.
\item \textbf{Litigation defense}: Evidence of good-faith governance efforts may reduce damages, support summary judgment motions, or enable favorable settlements.
\item \textbf{Insurance}: Insurers may require evidence of governance (policies, audits, controls) as a condition of coverage or premium reduction.
\end{itemize}

Organizations that deploy AI systems without governance face \emph{uninsurable, unmitigated risk}. Conversely, robust governance creates an evidentiary record of due diligence—valuable in regulatory inquiries, litigation, and board oversight.

\paragraph{Example: Credit Decisioning Liability Chain}
A mortgage applicant is denied credit by a bank using an AI underwriting system. The applicant sues under ECOA, alleging disparate impact (the system disproportionately denies applications from Hispanic applicants). The liability chain unfolds:

\begin{enumerate}
\item \textbf{Applicant sues bank}: Under traditional enforcement practice, ECOA liability attaches to the \emph{creditor} (the bank), not the technology vendor. The bank is the defendant, regardless of whether it built the system in-house or purchased it.
\item \textbf{Bank investigates vendor recovery}: The bank's contract with the AI vendor caps liability at \$100,000 (annual subscription fee). The ECOA settlement is \$3 million (class action covering 500 affected applicants). The bank recovers only \$100,000—3\% of total damages.
\item \textbf{Bank disciplines employee}: The bank's AI governance policy required quarterly fairness monitoring. The assigned compliance analyst failed to conduct monitoring for six months. The bank terminates the analyst but remains liable to applicants and regulators (the analyst's failure does not excuse the bank's ECOA violation).
\item \textbf{Regulatory escalation}: The Consumer Financial Protection Bureau (CFPB) investigates and imposes a \$5 million penalty for systemic ECOA violations. The penalty is assessed against the bank, not the vendor or employee.
\end{enumerate}

\textbf{Outcome}: The bank bears \$8 million in total liability (\$3M settlement + \$5M penalty) and recovers \$100K from the vendor. Effective governance—quarterly fairness monitoring, documented risk assessment, incident response protocols—might have detected the bias earlier, limited exposure, and demonstrated good faith to regulators.

\begin{keybox}[title={Liability Reality Check}]
``The AI did it'' is not a legal defense. ``We bought it from a reputable vendor'' does not transfer liability. ``Our employee was supposed to monitor it'' does not excuse organizational failures. Deployers own the risk. Governance is the mechanism for managing it.
\end{keybox}

