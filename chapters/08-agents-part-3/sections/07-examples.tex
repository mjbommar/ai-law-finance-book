% ============================================================================
% Examples in Context — Agents Part III
% Purpose: Demonstrate governance through worked examples in law and accounting
% Label: sec:agents3-examples
% ============================================================================

\section{Examples in Context}
\label{sec:agents3-examples}

This section demonstrates governance principles through worked examples in legal and accounting contexts. (Financial services examples---including credit underwriting, financial planning, and fair lending compliance---are developed throughout Section~\ref{sec:agents3-implementation}.) Each example follows a common governance framework: identify risks, calibrate controls, implement monitoring, and respond to incidents. These examples are illustrative---organizations must tailor governance to their specific regulatory obligations, risk appetite, and operational context---but they demonstrate how the conceptual frameworks from Sections~\ref{sec:agents3-dimensional} through \ref{sec:agents3-accountability} translate into practice.

\subsection{Legal Domain: Professional Responsibility and Incident Management}
\label{sec:agents3-examples-legal}

\paragraph{Example 1: Agentic Legal Research Assistant—Iteration and Verification Controls}
A mid-sized law firm deploys an agentic legal research system that \emph{iteratively} investigates legal questions by formulating search strategies, retrieving cases, analyzing precedential value, cross-referencing citations, adapting its search based on relevance patterns, and terminating when sufficient authority is identified or confidence thresholds require human escalation. \textit{Dimensional profile: HITL + human frame + static goals + stateless.}

Figure~\ref{fig:agents3-incident-report-legal} documents an incident where the system's cross-cycle adaptation introduced citation errors that propagated through subsequent iterations—a failure mode unique to agentic systems. The incident report follows ISO 27001 incident management standards while demonstrating key governance principles: (1) iteration and adaptation compound errors across cycles, making single-point output review insufficient; (2) confidence thresholds must incorporate domain-specific accuracy metrics, not just relevance scores; (3) professional duty under Rule 1.1 requires attorneys to understand iterative system logic, not merely review final outputs.

% Legal Research Incident Report - Styled as Firm Incident Report (ISO 27001 Compliant)

\begin{figure}[htbp]
\centering
\begin{tikzpicture}[
    every node/.style={inner sep=0pt}
]

% Document frame
\node[
    draw=gray-400,
    line width=0.5pt,
    fill=white,
    minimum width=12.5cm,
    minimum height=18.2cm,
    rounded corners=2pt
] (page) at (0,0) {};

% Header bar
\fill[red-900] ([yshift=-0.4cm]page.north west) rectangle ([yshift=-1.4cm]page.north east);

% Firm name and report title
\node[font=\small\bfseries\sffamily, text=white, anchor=west] at ([xshift=0.5cm, yshift=-0.7cm]page.north west) {MORRISON \& STERLING LLP};
\node[font=\footnotesize\sffamily, text=red-100, anchor=west] at ([xshift=0.5cm, yshift=-1.1cm]page.north west) {Risk Management --- Incident Response};

% Report metadata right side
\node[font=\scriptsize\sffamily, text=white, anchor=east] at ([xshift=-0.5cm, yshift=-0.7cm]page.north east) {Incident No. IR-2024-017};
\node[font=\scriptsize\sffamily, text=red-100, anchor=east] at ([xshift=-0.5cm, yshift=-1.1cm]page.north east) {Status: CLOSED};

% Report title
\node[font=\normalsize\bfseries\sffamily, text=slate-900, anchor=north] at ([yshift=-1.8cm]page.north) {AI System Incident Report};
\node[font=\scriptsize\sffamily, text=gray-600, anchor=north] at ([yshift=-2.25cm]page.north) {Agentic Legal Research Assistant --- Cross-Cycle Hallucination};

% Horizontal rule
\draw[gray-300, line width=0.5pt] ([xshift=0.5cm, yshift=-2.5cm]page.north west) -- ([xshift=-0.5cm, yshift=-2.5cm]page.north east);

% Document metadata box
\node[
    draw=gray-300,
    line width=0.4pt,
    fill=gray-100,
    rounded corners=2pt,
    text width=11.3cm,
    anchor=north west,
    inner sep=6pt,
    font=\tiny
] at ([xshift=0.5cm, yshift=-2.7cm]page.north west) {
\begin{tabular}{@{}l@{\hspace{0.2cm}}l@{\hspace{0.6cm}}l@{\hspace{0.2cm}}l@{\hspace{0.6cm}}l@{\hspace{0.2cm}}l@{}}
\textbf{Detected:} Mar 15, 2024 & \textbf{Severity:} High & \textbf{Category:} Output Accuracy \\[2pt]
\textbf{Closed:} Apr 12, 2024 & \textbf{Reporter:} Opposing Counsel & \textbf{Owner:} Chief Risk Officer \\
\end{tabular}
};

% Report content
\node[
    text width=11.5cm,
    anchor=north west,
    font=\tiny,
    align=left
] at ([xshift=0.5cm, yshift=-4.0cm]page.north west) {
\textbf{\textsf{1. INCIDENT DESCRIPTION}}\\[1pt]
Opposing counsel in \textit{Martinez v.\ Consolidated Industries} alerted the firm that a summary judgment motion contained citations that, while identifying real cases, mischaracterized holdings. The motion was prepared using the firm's agentic legal research system, which iteratively investigates legal questions through 2--6 research cycles, adapting search strategies based on relevance patterns.\\[4pt]

\textbf{\textsf{2. IMPACT ASSESSMENT}}\\[1pt]
\textbf{2.1} \textit{Professional Responsibility.} Potential violation of ABA Model Rule 3.3 (Candor Toward the Tribunal) due to submission of inaccurate case characterizations.\\[2pt]
\textbf{2.2} \textit{Client Impact.} Motion credibility compromised; client notified; fee reduction offered.\\[2pt]
\textbf{2.3} \textit{Scope.} Review of 47 prior research sessions identified 6 additional sessions (13\%) with cross-cycle adaptation errors.\\[4pt]

\textbf{\textsf{3. IMMEDIATE RESPONSE}}\\[1pt]
\textbf{3.1} System access suspended firm-wide pending investigation (Mar 15).\\[2pt]
\textbf{3.2} Corrected motion filed with court; candor-to-tribunal explanation submitted per Rule 3.3 (Mar 16).\\[2pt]
\textbf{3.3} Client notified of incident and offered fee reduction for affected matter (Mar 17).\\[2pt]
\textbf{3.4} All pending matters using system flagged for manual citation verification (Mar 17).\\[4pt]

\textbf{\textsf{4. ROOT CAUSE ANALYSIS}}\\[1pt]
\textbf{4.1} \textit{Cycle-Level Audit.} Investigation of iteration logs revealed: Cycles 1--3 correctly identified 8 relevant cases. Cycle 4 detected contradictory authority and attempted to ``harmonize'' holdings through paraphrasing---introducing mischaracterization. Cycles 5--6 propagated the erroneous synthesis without detecting the error.\\[2pt]
\textbf{4.2} \textit{Adaptation Failure.} The system's contradiction-resolution logic created hallucination risk by paraphrasing holdings rather than preserving verbatim quotations.\\[2pt]
\textbf{4.3} \textit{Termination Failure.} System terminated based on confidence threshold (>0.85) despite holding mischaracterization; confidence metric measured legal relevance but did not capture citation accuracy.\\[4pt]

\textbf{\textsf{5. CORRECTIVE ACTIONS}}\\[1pt]
\textbf{5.1} \textit{Adaptation Constraints.} System reconfigured to prohibit paraphrasing of holdings; require verbatim quotations with Bluebook pin cites for all case references.\\[2pt]
\textbf{5.2} \textit{Cross-Cycle Consistency.} Implemented automated flagging when later cycles contradict earlier findings; contradictions now escalate to attorney rather than automated resolution.\\[2pt]
\textbf{5.3} \textit{Termination Revision.} Confidence threshold revised: system terminates only when legal relevance confidence >0.85 AND citation accuracy score >0.95 (verified via database cross-check).\\[2pt]
\textbf{5.4} \textit{HITL Verification.} Attorneys must review cycle-by-cycle logs, not just final output; research workpapers must document validation of each cited case.\\[4pt]

\textbf{\textsf{6. PREVENTIVE ACTIONS}}\\[1pt]
\textbf{6.1} Quarterly iteration audits established: sample 15\% of research sessions; review cross-cycle adaptation patterns for citation accuracy.\\[2pt]
\textbf{6.2} Training updated: all attorneys complete 2-hour module on agentic system risks and cycle-level review requirements.\\[2pt]
\textbf{6.3} Vendor notified of defect; contractual SLA for accuracy monitoring invoked.\\[4pt]

\textbf{\textsf{7. LESSONS LEARNED}}\\[1pt]
\textbf{7.1} Iteration and adaptation compound errors across cycles; single-point output review is insufficient for agentic systems.\\[2pt]
\textbf{7.2} Confidence thresholds must incorporate domain-specific accuracy metrics (citation fidelity), not just relevance scores.\\[2pt]
\textbf{7.3} Professional duty (Rule 1.1 competence) requires attorneys to understand iterative system logic, not merely review outputs.\\[4pt]

\textbf{\textsf{RELATED INCIDENTS:}} None \hfill \textbf{\textsf{REVIEW CYCLE:}} 90 days (Jul 12, 2024)
};

% Footer
\draw[gray-300, line width=0.5pt] ([xshift=0.5cm, yshift=0.7cm]page.south west) -- ([xshift=-0.5cm, yshift=0.7cm]page.south east);
\node[font=\tiny\sffamily, text=gray-500, anchor=south] at ([yshift=0.25cm]page.south) {CONFIDENTIAL --- Page 1 of 1};

\end{tikzpicture}
\caption{Incident report for an agentic legal research system failure. The report follows ISO 27001 incident management standards (classification, root cause analysis, corrective and preventive actions) while documenting an agentic-specific failure mode: cross-cycle error propagation where the system's adaptation logic introduced hallucinations that compounded across subsequent iterations.}
\label{fig:agents3-incident-report-legal}
\end{figure}


\subsection{Accounting Domain: Independence and Professional Skepticism}
\label{sec:agents3-examples-accounting}

\paragraph{Example 2: AI Acceptable Use Policy for Agentic Systems (AICPA Independence)}
A Big Four accounting firm establishes an AI acceptable use policy to operationalize AICPA independence rules and SEC auditor independence requirements for \emph{agentic audit and advisory systems}. \textit{Dimensional profile: Spans HITL, HOTL, and HIC modes across human and institutional frames; policy-level governance rather than a single system.}

Figure~\ref{fig:agents3-ai-acceptable-use-policy} shows an excerpt from the firm's policy. The policy establishes guiding principles (independence, competence, confidentiality), distinguishes permitted uses (research, analytics, documentation assistance) from prohibited uses (management decisions, audit opinions, unauthorized data sharing), and implements safeguards through vendor approval requirements, mandatory professional review, and documentation standards. Training requirements ensure personnel understand both tool capabilities and professional obligations. Incident reporting procedures establish clear escalation pathways when independence concerns or data breaches arise.

% AI Acceptable Use Policy - Styled as Firm Policy Document (ISO 27001 Compliant)

\begin{figure}[htbp]
\centering
\begin{tikzpicture}[
    every node/.style={inner sep=0pt}
]

% Document frame
\node[
    draw=gray-400,
    line width=0.5pt,
    fill=white,
    minimum width=9.8cm,
    minimum height=13.0cm,
    rounded corners=2pt
] (page) at (0,0) {};

% Header bar
\fill[slate-700] ([yshift=-0.4cm]page.north west) rectangle ([yshift=-1.4cm]page.north east);

% Firm name and policy title
\node[font=\small\bfseries\sffamily, text=white, anchor=west] at ([xshift=0.5cm, yshift=-0.7cm]page.north west) {BIG FOUR ACCOUNTING LLP};
\node[font=\footnotesize\sffamily, text=gray-300, anchor=west] at ([xshift=0.5cm, yshift=-1.1cm]page.north west) {National Office---Professional Standards};

% Policy metadata right side
\node[font=\scriptsize\sffamily, text=white, anchor=east] at ([xshift=-0.5cm, yshift=-0.7cm]page.north east) {Policy No. AI-2024-003};
\node[font=\scriptsize\sffamily, text=gray-300, anchor=east] at ([xshift=-0.5cm, yshift=-1.1cm]page.north east) {Version 2.1};

% Policy title
\node[font=\normalsize\bfseries\sffamily, text=slate-900, anchor=north] at ([yshift=-1.8cm]page.north) {Artificial Intelligence Acceptable Use Policy};
\node[font=\scriptsize\sffamily, text=gray-600, anchor=north] at ([yshift=-2.25cm]page.north) {Audit, Tax, and Advisory Services};

% Horizontal rule
\draw[gray-300, line width=0.5pt] ([xshift=0.5cm, yshift=-2.5cm]page.north west) -- ([xshift=-0.5cm, yshift=-2.5cm]page.north east);

% Document metadata box
\node[
    draw=gray-300,
    line width=0.4pt,
    fill=gray-100,
    rounded corners=2pt,
    text width=8.8cm,
    anchor=north west,
    inner sep=6pt,
    font=\tiny\sffamily
] at ([xshift=0.5cm, yshift=-2.7cm]page.north west) {
\begin{tabular}{@{}l@{\hspace{0.3cm}}l@{\hspace{1.2cm}}l@{\hspace{0.3cm}}l@{}}
\textbf{Owner:} Chief Ethics Officer & \textbf{Approved by:} Executive Committee \\[2pt]
\textbf{Effective:} Jan 1, 2025 & \textbf{Review by:} Jan 1, 2026 \\
\end{tabular}
};

% Policy content - using a single node with formatted text
\node[
    text width=8.8cm,
    anchor=north west,
    font=\tiny\sffamily,
    align=left
] at ([xshift=0.5cm, yshift=-4.0cm]page.north west) {
\textbf{\textsf{1. SCOPE}}\\[1pt]
This policy governs the use of artificial intelligence tools, including large language models, agentic systems, and automated analytics platforms, in all audit, tax, and advisory engagements conducted by firm personnel.\\[4pt]

\textbf{\textsf{2. GUIDING PRINCIPLES}}\\[1pt]
\textbf{2.1} \textit{Independence.} AI tools shall not be used in any manner that impairs auditor independence under AICPA Professional Standards or SEC Rule 2-01. Personnel shall not delegate management decisions to AI systems or use AI outputs that create self-review threats.\\[2pt]
\textbf{2.2} \textit{Professional Competence.} Personnel using AI tools must understand the tool's capabilities, limitations, and potential for error. Use of AI does not diminish the professional's responsibility to exercise due care under AT-C Section 105.\\[2pt]
\textbf{2.3} \textit{Confidentiality.} Client data processed by AI tools must be protected consistent with firm confidentiality obligations and applicable data processing agreements.\\[4pt]

\textbf{\textsf{3. PERMITTED USES}}\\[1pt]
AI tools may be used for: (a) research and analysis, including accounting standards research, industry benchmarking, and regulatory guidance review; (b) data analytics, including anomaly detection, transaction testing, and sampling optimization; and (c) documentation assistance, including workpaper drafting and memo summarization, subject to professional review requirements in Section 5.\\[4pt]

\textbf{\textsf{4. PROHIBITED USES}}\\[1pt]
The following uses are prohibited without exception:\\[2pt]
\textbf{4.1} Using AI to make or recommend management decisions for audit clients.\\[2pt]
\textbf{4.2} Issuing or drafting audit opinions based solely on AI analysis without application of professional judgment.\\[2pt]
\textbf{4.3} Submitting client confidential data to AI systems not approved under Section 5.1.\\[4pt]

\textbf{\textsf{5. SAFEGUARDS}}\\[1pt]
\textbf{5.1} \textit{Approved Vendors.} AI tools must be approved by the National Office Technology Committee. Approved vendors must maintain SOC 2 Type II certification and execute firm-standard Data Processing Agreements.\\[2pt]
\textbf{5.2} \textit{Professional Review.} All AI-assisted work product must be reviewed by a manager or partner before inclusion in workpapers or delivery to clients.\\[2pt]
\textbf{5.3} \textit{Workpaper Documentation.} Audit documentation must identify procedures that used AI tools and explain how professional judgment was applied.\\[4pt]

\textbf{\textsf{6. TRAINING AND INCIDENT REPORTING}}\\[1pt]
All audit professionals must complete AI Fundamentals training before using approved AI tools. Personnel must immediately report to the Engagement Partner any suspected independence impairment, data breach, or client complaint involving AI tools.
};

% Footer
\draw[gray-300, line width=0.5pt] ([xshift=0.5cm, yshift=0.7cm]page.south west) -- ([xshift=-0.5cm, yshift=0.7cm]page.south east);
\node[font=\tiny\sffamily, text=gray-500, anchor=south] at ([yshift=0.25cm]page.south) {CONFIDENTIAL---FOR INTERNAL USE ONLY---Page 1 of 1};

\end{tikzpicture}
\caption{AI acceptable use policy excerpt operationalizing AICPA and SEC independence requirements for professional services firms.}
\label{fig:agents3-ai-acceptable-use-policy}
\end{figure}


\textbf{Governance Principles Demonstrated}:
\begin{itemize}
\item \textbf{Domain-specific calibration}: Policy tailored to AICPA and SEC independence rules, not generic AI governance.
\item \textbf{Role-based permissions}: Distinguishes permitted (research, analytics) from prohibited (management decisions, audit opinions) uses.
\item \textbf{Accountability assignment}: Partners responsible for reviewing AI-assisted work; National Office Ethics Group accountable for policy updates.
\end{itemize}

