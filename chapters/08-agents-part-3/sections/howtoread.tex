% ============================================================================
% How to Read This Chapter — Agents Part III
% Purpose: Navigation guide and scope setting
% Label: sec:agents3-howtoread
% ============================================================================

\section*{How to Read This Chapter}
\addcontentsline{toc}{section}{How to Read This Chapter}
\label{sec:agents3-howtoread}

This chapter translates the conceptual foundations from Part I into governance practice. Where Part I asked \emph{what makes a system agentic?}, this chapter addresses \emph{how do we govern agentic systems responsibly?} Our focus is on what chief risk officers, compliance officers, general counsel, and senior leadership need to approve, monitor, and continuously improve deployments in regulated domains.

\paragraph{Reading Paths by Audience}

\begin{itemize}
  \item \textbf{Executives and Board Members}: Focus on Sections~\ref{sec:agents3-intro} (Introduction), \ref{sec:agents3-governance-stack} (Governance Stack), \ref{sec:agents3-accountability} (Accountability), and \ref{sec:agents3-conclusion} (Conclusion). These sections provide strategic framing, regulatory context, organizational structure, and synthesis.

  \item \textbf{Compliance, Risk, and Legal Officers}: Read the entire chapter sequentially. Sections~\ref{sec:agents3-governance-stack} through \ref{sec:agents3-implementation} provide detailed frameworks for regulatory alignment, control calibration, and operational implementation.

  \item \textbf{Practitioners and System Owners}: Emphasize Sections~\ref{sec:agents3-dimensional} (Dimensional Calibration), \ref{sec:agents3-implementation} (Implementation), and \ref{sec:agents3-examples} (Examples). These sections operationalize governance through risk-based control selection, technical architecture, and worked examples.
\end{itemize}

\paragraph{Conceptual Framework}

This chapter builds directly on Part I's analytical framework. Part I established a three-level hierarchy: \emph{agents} (Level 1: minimal agency with Goal, Perception, Action), \emph{agentic systems} (Level 2/3: operational readiness adding Iteration, Adaptation, Termination), and \emph{AI agents} (agentic systems powered by AI/ML). The six properties that define agentic systems—\keyterm{Goal}, \keyterm{Perception}, \keyterm{Action}, \keyterm{Iteration}, \keyterm{Adaptation}, \keyterm{Termination} (GPA+IAT)—map systematically to governance requirements. For example, autonomy and actuation scope determine oversight intensity (human-in-the-loop vs. human-in-command); entity frame and persistence drive audit logging and records retention; goal dynamics influence escalation triggers and revalidation schedules. Part I provided the taxonomy; this chapter provides the governance logic for agentic systems specifically.

\paragraph{What This Chapter Is Not}

This is not a step-by-step compliance manual, nor does it constitute legal advice. Regulatory requirements vary by jurisdiction, sector, and organizational context. Our goal is to equip you with conceptual tools—dimensional calibration, risk-based control selection, organizational accountability structures—that enable you to design governance proportionate to your risk profile. Consult qualified legal, compliance, and technical experts when implementing governance in your organization.

\vspace{0.5em}
\noindent\textcolor{border-neutral}{\rule{\textwidth}{1.5pt}}
\vspace{0.5em}
