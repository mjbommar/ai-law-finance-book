% =============================================================================
% Prompt Fundamentals (First Touch) — LLM Primer & Mechanics
% Purpose: Basic prompt design concepts, zero/few-shot, maturity model
% Label: sec:llm1-prompt-fundamentals
% =============================================================================

\section{Prompt Fundamentals (First Touch)}
\label{sec:llm1-prompt-fundamentals}

Now that we understand how LLMs tokenize text, generate outputs through sampling, and represent meaning through attention, we turn to the practical question: how do you communicate effectively with these systems? The answer lies in \keyterm{prompt design}---the art and engineering of constructing inputs that elicit the desired behavior.

This section provides a first exposure to prompt fundamentals. We introduce core concepts (prompt anatomy, zero-shot versus few-shot, structured inputs and outputs) and present the \keyterm{Prompt Maturity Model}---a five-phase framework that structures the progression from raw text to modular, testable prompt systems. This framework guides the organization of the book itself: Chapters~2 through 5 walk through increasingly sophisticated prompt design patterns.

Think of this section as establishing vocabulary and mental models. Chapter~5 (Prompt Design, Evaluation, and Optimization) provides the comprehensive treatment of prompt engineering as specification, testing, and optimization. Chapters~6 and 7 extend these patterns to autonomous agents that combine prompts with tools, memory, and control flow.

\subsection{Anatomy of a Prompt}
\label{sec:llm1-prompt-anatomy}

A \keyterm{prompt} is the text input you provide to an LLM. While it might seem as simple as ``asking a question,'' effective prompts have internal structure. Modern systems decompose prompts into distinct components, each serving a specific purpose:

\begin{definitionbox}[title={Prompt Components}]
\textbf{System:} Instructions defining the assistant's role, behavior, and constraints. Often set once per session and treated as higher-priority context by the model.

\textbf{Instruction:} The specific task you want performed (``Summarize this contract,'' ``Extract the parties,'' ``Classify this filing'').

\textbf{Context:} Background information the model needs to perform the task (the document to summarize, the contract to analyze, the filing to classify).

\textbf{Input:} The specific query or data point being processed (a clause to interpret, a transaction to evaluate, a question to answer).
\end{definitionbox}

\paragraph{Why Separation Matters.} Separating these components is not just organizational hygiene---it has operational consequences:

\begin{itemize}
  \item \textbf{Control:} System-level instructions can be enforced more strictly than user inputs, reducing prompt injection risk
  \item \textbf{Caching:} Repeated system prompts and static context can be cached to reduce latency and cost
  \item \textbf{Safety:} Role boundaries help prevent user inputs from overriding intended behavior
  \item \textbf{Modularity:} You can swap context or inputs without rewriting instructions
\end{itemize}

\begin{highlightbox}[title={Example: Contract Summarization Prompt}]
\textbf{System:}
\begin{verbatim}
You are a legal document analysis assistant. Your task is
to provide accurate, concise summaries of legal agreements.
Always cite specific sections when making claims. If you
cannot find information, say so explicitly.
\end{verbatim}

\textbf{Instruction:}
\begin{verbatim}
Summarize the key terms of the following merger agreement.
Focus on: purchase price, closing conditions, termination
rights, and governing law.
\end{verbatim}

\textbf{Context:}
\begin{verbatim}
[Full text of merger agreement...]
\end{verbatim}

\textbf{Input:}
\begin{verbatim}
[May be empty for single-document analysis, or could be
a specific question: "What are the conditions precedent?"]
\end{verbatim}
\end{highlightbox}

In practice, some systems merge these categories or represent them differently (role-tagged messages, XML delimiters, structured API fields), but the conceptual separation remains valuable.

\subsection{Zero-Shot Prompting}
\label{sec:llm1-prompt-zeroshot}

\keyterm{Zero-shot prompting} means providing instructions but no examples of the task. You describe what you want; the model must infer how to do it from its pre-training alone.

\paragraph{When Zero-Shot Works.} For tasks the model encountered frequently in training, zero-shot prompts are often sufficient:

\begin{itemize}
  \item Simple text transformations (``Translate this to French,'' ``Summarize in one sentence'')
  \item General knowledge questions (``What is the capital of Delaware?'')
  \item Standard formatting (``Convert this to JSON,'' ``List the main points'')
  \item Classification into well-known categories (``Is this positive or negative?'')
\end{itemize}

\paragraph{When Zero-Shot Fails.} Zero-shot prompts struggle with:

\begin{itemize}
  \item \textbf{Complex reasoning:} Multi-step logic, arithmetic, sophisticated comparisons
  \item \textbf{Domain-specific formats:} Legal citation styles, financial report templates, specialized taxonomies
  \item \textbf{Ambiguous tasks:} ``Analyze this contract'' is too vague without examples of desired analysis depth and focus
  \item \textbf{Novel output structures:} Custom JSON schemas, proprietary data formats
\end{itemize}

\begin{highlightbox}[title={Zero-Shot Example: Document Classification}]
\textbf{Prompt:}
\begin{verbatim}
Classify the following SEC filing as one of: 10-K, 10-Q,
8-K, DEF 14A, S-1.

Filing text: [...]
\end{verbatim}

This works well because SEC filing types are standard and likely well-represented in training data. The model can recognize the format and labels.
\end{highlightbox}

\begin{cautionbox}[title={Zero-Shot Limitations}]
For legal and financial applications:
\begin{itemize}
  \item Zero-shot may produce outputs in the ``general shape'' of what you want but miss critical details
  \item The model's prior training may encode biases or outdated practices
  \item Without examples, you cannot precisely define your output format or style
  \item Consistency across multiple inputs is lower without examples to anchor the model
\end{itemize}
Treat zero-shot as a starting point, not a production-ready approach for high-stakes tasks.
\end{cautionbox}

\subsection{Few-Shot Prompting (Introduction)}
\label{sec:llm1-prompt-fewshot}

\keyterm{Few-shot prompting} provides examples of the desired input-output behavior directly in the prompt. This leverages the model's ability to learn patterns from context---what researchers call \keyterm{in-context learning} \parencite{brown2020fewshot}.

The term ``few-shot'' is inherited from machine learning (few-shot learning means training from limited examples), but in the LLM context, no training occurs. Instead, the model \emph{conditions} its generation on the examples. It's pattern completion: ``You showed me three examples of $X \to Y$; for this new $X$, I will generate a similar $Y$.''

\paragraph{Basic Structure.} Few-shot prompts follow a simple template:

\begin{verbatim}
Task description.

Example 1:
Input: [...]
Output: [...]

Example 2:
Input: [...]
Output: [...]

Example 3:
Input: [...]
Output: [...]

Now you try:
Input: [actual input]
Output:
\end{verbatim}

The model sees this as a pattern to continue. The quality and relevance of the examples directly determine output quality.

\paragraph{Why Few-Shot Is Powerful.} Few-shot prompting allows you to:

\begin{itemize}
  \item \textbf{Define output format:} Show exactly what structure you want (JSON schema, citation style, level of detail)
  \item \textbf{Demonstrate reasoning:} Include intermediate steps, not just final answers
  \item \textbf{Calibrate tone and style:} Examples establish the ``voice'' you expect
  \item \textbf{Handle edge cases:} Show how to handle missing data, ambiguity, or unusual inputs
  \item \textbf{Improve consistency:} Anchoring the model to examples reduces variation across inputs
\end{itemize}

\begin{highlightbox}[title={Few-Shot Example: Legal Citation Extraction}]
\textbf{Prompt:}
\begin{verbatim}
Extract all case citations from the text and return as JSON.

Example 1:
Input: "Pursuant to Brown v. Board of Education, 347 U.S. 483
(1954), segregation is unconstitutional."
Output: {"citations": [{"case": "Brown v. Board of Education",
"reporter": "347 U.S. 483", "year": 1954}]}

Example 2:
Input: "See also Roe v. Wade, 410 U.S. 113 (1973)."
Output: {"citations": [{"case": "Roe v. Wade",
"reporter": "410 U.S. 113", "year": 1973}]}

Example 3:
Input: "No citations in this text."
Output: {"citations": []}

Now extract from this text:
Input: "As held in Marbury v. Madison, 5 U.S. 137 (1803),
judicial review is a cornerstone of constitutional law."
Output:
\end{verbatim}

The examples teach the model (1) the JSON schema, (2) how to parse citation format, and (3) how to handle empty results.
\end{highlightbox}

\begin{keybox}[title={Few-Shot Design Principles}]
\begin{enumerate}
  \item \textbf{Representative examples:} Cover the diversity of inputs you expect, not just easy cases
  \item \textbf{Clear input-output boundaries:} Use consistent delimiters (``Input:'', ``Output:'') to avoid ambiguity
  \item \textbf{Include edge cases:} Show how to handle missing data, errors, ambiguity
  \item \textbf{Quality over quantity:} 3--5 high-quality examples often outperform 20 mediocre ones
  \item \textbf{Order matters:} Place the most relevant example last (recency bias in attention)
  \item \textbf{Reasoning traces:} For complex tasks, show intermediate steps, not just final answers (preview of chain-of-thought in Chapter~2)
\end{enumerate}
\end{keybox}

\paragraph{Forward Reference.} Few-shot prompting is a deep topic. Chapter~2 (Conversations and Reasoning) extends few-shot to chain-of-thought and self-consistency. Chapter~5 (Prompt Design, Evaluation, and Optimization) covers exemplar selection, diversity management, and programmatic exemplar construction for production systems.

\subsection{The Prompt Maturity Model}
\label{sec:llm1-prompt-maturity}

As prompts evolve from exploratory prototypes to production systems, they follow a predictable progression. The \keyterm{Prompt Maturity Model} is a five-phase framework that describes this journey from ad-hoc text inputs to modular, testable, production-grade prompt pipelines.

\begin{table}[htbp]
\centering
\caption{The Five-Phase Prompt Maturity Model}
\label{tab:llm1-prompt-maturity}
\begin{tabular}{@{}clp{4cm}p{3cm}l@{}}
\toprule
\textbf{Phase} & \textbf{Input} & \textbf{Output} & \textbf{Testability} & \textbf{Chapters} \\
\midrule
1: Zero-shot & Freeform text & Freeform text & None (manual review) & 1--2 \\
2: Few-shot & Examples + text & Freeform text & Low (human eval) & 2 \\
3: Structured Input & JSON/XML input & Freeform text & Medium (input validation) & 3, 5 \\
4: Structured I/O & JSON/XML input & JSON/XML output & High (schema validation) & 3, 5 \\
5: Modular & Decomposed pipeline & Structured, composable & Full (unit + integration tests) & 5--7 \\
\bottomrule
\end{tabular}
\end{table}

The progression reflects real-world experience: projects start with Phase~1 (raw text in, raw text out), but frustration with reliability drives adoption of examples (Phase~2). Integration requirements drive structured outputs (Phases~3--4). Scale and complexity drive modular decomposition (Phase~5).

\paragraph{Key Insight: Testability Drives Maturity.} The central theme is testability. Phases~1--2 require human review of every output. Phase~4 (structured I/O) is the minimum standard for production legal and financial applications because outputs can be validated against schemas and evaluated with quantitative metrics. Phase~5 (modular pipelines) enables unit testing of individual prompts, integration testing of workflows, and regression testing across versions.

\begin{keybox}[title={Phase Summary}]
\begin{description}
  \item[Phase 1 (Zero-shot):] Natural language in/out. Use for prototyping only.
  \item[Phase 2 (Few-shot):] Examples improve consistency. Still requires human review.
  \item[Phase 3 (Structured Input):] Programmatic inputs enable validation before LLM calls.
  \item[Phase 4 (Structured I/O):] Production minimum. Schema validation enables automation.
  \item[Phase 5 (Modular):] Enterprise standard. Decomposed, testable, auditable.
\end{description}
\end{keybox}

Chapter~5 (Prompt Design, Evaluation, and Optimization) provides the comprehensive treatment of the Prompt Maturity Model, including detailed guidance on structured I/O design, testing strategies, and modular pipeline architecture. Chapters~6--7 (Agents) demonstrate Phase~5 patterns in autonomous systems.

\subsection{Principles of Effective Prompts}
\label{sec:llm1-prompt-principles}

Across all maturity phases, certain principles consistently improve prompt effectiveness. These are not rigid rules but heuristics refined through empirical observation and research.

\subsubsection{Clarity and Specificity}

\paragraph{Principle:} The model cannot read your mind. Vague instructions yield vague outputs.

\textbf{Ineffective:}
\begin{verbatim}
Analyze this contract.
\end{verbatim}

\textbf{Effective:}
\begin{verbatim}
Extract the following from this merger agreement:
- Purchase price (including any adjustments)
- Closing conditions precedent
- Termination rights and associated fees
- Governing law and dispute resolution
Return as JSON with these exact keys.
\end{verbatim}

The effective version defines scope, format, and specific fields. The model has clear success criteria.

\subsubsection{Appropriate Constraints}

\paragraph{Principle:} Strike a balance between over-constraining (which stifles useful outputs) and under-constraining (which permits hallucination or drift).

\textbf{Over-constrained:}
\begin{verbatim}
Summarize this 50-page contract in exactly 147 words, using
only nouns and verbs, with no more than 12 sentences, each
beginning with a different letter of the alphabet.
\end{verbatim}

\textbf{Under-constrained:}
\begin{verbatim}
Tell me about this contract.
\end{verbatim}

\textbf{Appropriately constrained:}
\begin{verbatim}
Summarize this contract in 200-300 words. Focus on key
business terms, material risks, and any unusual provisions.
\end{verbatim}

The appropriate version gives guidance without imposing arbitrary restrictions.

\subsubsection{Failure Mode Awareness}

\paragraph{Principle:} Design prompts defensively. Anticipate what could go wrong and instruct the model how to handle edge cases.

\begin{itemize}
  \item \textbf{Missing data:} ``If the document does not contain a purchase price, return null for that field.''
  \item \textbf{Ambiguity:} ``If the governing law is unclear, provide the most likely jurisdiction and note your uncertainty.''
  \item \textbf{Out-of-scope inputs:} ``If this is not a merger agreement, return an error indicating document type mismatch.''
  \item \textbf{Hallucination risk:} ``Only cite information explicitly present in the document. Do not infer or assume facts.''
\end{itemize}

Explicitly instructing the model how to fail gracefully reduces catastrophic errors.

\subsubsection{Grounding and Citation}

\paragraph{Principle:} For factual tasks, require the model to ground outputs in sources and cite evidence.

\begin{verbatim}
When making a claim about the contract, cite the specific
section number. Format: "The termination fee is $50M
(Section 8.2)."
\end{verbatim}

This serves dual purposes:
\begin{enumerate}
  \item Makes outputs verifiable (you can check the cited section)
  \item Reduces hallucination (requiring citation discourages fabrication)
\end{enumerate}

For retrieval-augmented generation (RAG), this becomes essential: the model must distinguish between information in the retrieved documents versus information from its pre-training.

\subsubsection{Iterative Refinement}

\paragraph{Principle:} Prompts are software. They require testing, debugging, and version control.

\begin{enumerate}
  \item Start with a simple baseline prompt
  \item Test on representative examples
  \item Identify failure cases
  \item Refine instructions, examples, or constraints
  \item Re-test
  \item Repeat
\end{enumerate}

Document what you tried and why. Version your prompts. Track performance metrics over time. Chapter~5 formalizes this as prompt optimization and evaluation.

\begin{keybox}[title={Prompt Design Checklist}]
Before deploying a prompt to production, verify:
\begin{enumerate}
  \item \textbf{Clear task definition:} Can a human understand exactly what you want?
  \item \textbf{Specified output format:} Have you shown (few-shot) or enforced (schema) the desired structure?
  \item \textbf{Edge case handling:} Does the prompt explain what to do with missing data, ambiguity, or errors?
  \item \textbf{Grounding requirements:} For factual tasks, do you require citations or quotations?
  \item \textbf{Validation strategy:} How will you detect failures (schema validation, human review, automated checks)?
  \item \textbf{Representative testing:} Have you tested on real examples, including difficult cases?
  \item \textbf{Token budget:} Does the prompt + expected output fit comfortably within context limits?
  \item \textbf{Sampling parameters:} Have you set appropriate temperature, top-p, and max tokens?
\end{enumerate}
\end{keybox}

\subsection{Forward References and Chapter Roadmap}
\label{sec:llm1-prompt-forward}

This section introduced prompt fundamentals: the anatomy of prompts, zero-shot versus few-shot approaches, and the Prompt Maturity Model that structures the progression from raw text to modular systems. These concepts recur throughout the book:

\begin{itemize}
  \item \textbf{Chapter~2 (Conversations and Reasoning):} Extends few-shot to chain-of-thought, self-consistency, and reasoning reliability. Covers role-based prompting, memory strategies, and multi-turn conversation design.

  \item \textbf{Chapter~3 (Structured Inputs, Tools, and Function Calling):} Comprehensive treatment of Phase~4 (Structured I/O). Covers JSON schemas, constrained decoding, tool use, and function calling for integration with deterministic systems.

  \item \textbf{Chapter~5 (Prompt Design, Evaluation, and Optimization):} The definitive treatment of prompts as specifications. Covers testing methodologies, exemplar management, automated optimization, A/B testing, and production deployment patterns.

  \item \textbf{Chapters~6--7 (Agents):} Extends prompting to autonomous systems. Agents are, at core, sophisticated prompt orchestration---planning prompts, tool-selection prompts, reflection prompts---composed into goal-directed pipelines. All the principles from this section apply, but at the level of system architecture.
\end{itemize}

Think of this section as establishing a shared language. When we say ``few-shot,'' you understand we mean in-context learning from examples. When we say ``Phase~4,'' you understand we mean structured I/O with schema validation. These concepts are the building blocks for everything that follows.

With prompt fundamentals established, we now examine how LLMs represent meaning numerically---the embedding space that enables semantic search, retrieval-augmented generation, and many advanced techniques central to legal and financial applications.
