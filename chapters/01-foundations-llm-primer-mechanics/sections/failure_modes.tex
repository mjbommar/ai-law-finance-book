% =============================================================================
% Failure Modes — LLM Primer & Mechanics
% Purpose: Common pitfalls to anticipate early
% Label: sec:llmA-fail
% =============================================================================

\section{Common Failure Modes and Why They Happen}
\label{sec:llmA-fail}

% Outline (comments): hallucination, staleness/recency, formatting drift, overlong outputs, misread structure
\begin{cautionbox}[title={Watch Outs}]
\begin{itemize}
  \item \textbf{Ungrounded claims:} require source citations and quotes.
  \item \textbf{Context overflow:} enforce concise prompts and chunk inputs.
  \item \textbf{Formatting drift:} use schemas and validators early.
\end{itemize}
\end{cautionbox}

\subsection{Recency and Cutoff Effects}
Models omit events after their training cutoff. Mitigation: require dated sources and retrieval with freshness controls.

\subsection{Prompt Injection and Jailbreaks (Preview)}
Malicious or untrusted inputs can steer models off-task. Use strict schemas, role separation, and validation; defer to later governance chapters for controls.

\subsection{Domain/Format Shift}
Performance drops when inputs differ from training formats (e.g., complex tables, scanned PDFs). Prefer native text, structure-preserving parsers, and targeted evaluation.

