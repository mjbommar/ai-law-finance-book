% =============================================================================
% Tool Use — Structured Outputs, Tools, Multimodal
% Purpose: Function calling with governance metadata
% Label: sec:llmC-tools
% =============================================================================

\section{Tool Use and Function Calling}
\label{sec:llmC-tools}

% Outline (comments): inputs/outputs; pre/postconditions; idempotency; error classes; governance metadata (purpose, user, privilege, jurisdiction, retention)
\begin{keybox}[title={Invocation Contract Checklist}]
\begin{itemize}
  \item Name, inputs, outputs, and validation rules.
  \item Pre/postconditions and idempotency expectations.
  \item Governance metadata and audit hooks on every call.
\end{itemize}
\end{keybox}

\subsection{Prompted JSON vs. Function Calling}
Prefer function calls for strong typing, validation, and audit logs. Keep prompts minimal by referencing function names and arguments.

\subsection{Governance Metadata on Every Call}
Include purpose, user/actor, privilege flags, jurisdiction, legal basis (where applicable), retention directives, and correlation IDs. This metadata enables downstream audit and Part~III controls.

\subsection{Error Taxonomy, Retries, and Backoff}
Standardize error classes (authn/authz, validation, rate limits, network). Implement retries with exponential backoff and circuit breakers for reliability.

\subsection{Idempotency and Auditing}
Use idempotency keys for externally visible actions; store full request/response payloads with hashes and timestamps for tamper-evident logs.

\subsection{Capability Discovery and Contracts}
Advertise actions with names, versions, input/output schemas, pre/postconditions, and side effects. Keep a registry for change management.

\subsection{Security and Privacy Basics}
Minimize PII; redact before sending to third parties; avoid secrets in prompts; prefer short-lived tokens and scoped credentials.

\subsection{Contract Template (YAML/JSON)}
% Minimal, copyable contract for capability registries
\begin{highlightbox}[title={Minimal Contract (YAML)}]
\small
\begin{verbatim}
name: compute_interest
version: 1.2.0
inputs:
  principal: {type: number, required: true}
  rate: {type: number, required: true}
  period_days: {type: integer, required: true}
preconditions:
  - rate >= 0
postconditions:
  - result >= 0
idempotent: true
governance:
  purpose: finance.calculation
  privilege: client-confidential
  jurisdiction: US
  retention: 7d
\end{verbatim}
\end{highlightbox}

\subsection{Deployment Modes and Metadata}
Indicate whether a call uses on-prem, private cloud, or SaaS endpoints. Add region pins (e.g., EU-only) and DPAs/BAAs as metadata fields.

\subsection{Capacity and Cost Planning}
Document expected rate limits, concurrency, and maximum payload sizes. Track per-call token costs and latency percentiles; define fallbacks.

\subsection{Calculator/Numeracy Guardrail}
For numeric tasks, call a calculator/tool and assert results match within tolerance; do not rely on free-form generation for math.
