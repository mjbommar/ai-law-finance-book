% =============================================================================
% Structured Outputs — Structured Outputs, Tools, Multimodal
% Purpose: JSON/CSV/XML with validation
% Label: sec:llmC-structured
% =============================================================================

\section{Structured Outputs with Validation}
\label{sec:llmC-structured}

% Outline (comments): schema-first prompts; retries; validators; error handling; schema versioning
\begin{definitionbox}[title={Schema-First Prompting}]
\begin{itemize}
  \item Provide explicit field names, types, and allowed values.
  \item Validate and retry on failure; capture errors for audit.
  \item Version schemas and record model/config used.
\end{itemize}
\end{definitionbox}

\subsection{JSON vs. XML vs. CSV: Tradeoffs}
JSON is concise and maps naturally to application objects. XML offers schemas and legacy compatibility. CSV is compact but fragile for nested data. Prefer JSON unless downstream mandates XML/CSV.

\subsection{Prompt Schemas vs. Function Calling}
Prompted JSON/XML can work, but function calling enforces structure at the API level. Use prompts for quick prototypes; migrate to functions for reliability and auditability.

\subsection{Validation, Retries, and Migrations}
Validate outputs against schemas; on failure, retry with error hints. Version schemas; document breaking changes and provide migration notes.

\subsection{Locale, Time, and Numeric Formatting}
Normalize dates (ISO 8601), times (UTC offsets), currencies, and units. Require explicit jurisdiction and effective dates in evidence records.

\subsection{Streaming and Partial Outputs (Optional)}
When streaming, buffer tokens and validate incrementally. Consider end-of-object markers or chunked JSON strategies.

