% How to Read This Chapter - Reading paths for different audiences
% This section appears before the main content to guide readers

\section*{How to Read This Chapter}
\addcontentsline{toc}{section}{How to Read This Chapter}

This chapter answers a deceptively simple question: \textit{What is an agent?}

The term ``agent'' appears everywhere in industry and academia today---from ``AI agents'' to ``agentic workflows''---yet rarely with clear definition. This creates confusion, miscommunication, and inevitably, disappointment.

We hope to help you avoid this disappointment, but \textbf{this chapter is not short and your time is valuable}. You don't need to read everything, especially on first reading. Three reading paths correspond to different goals:

\paragraph{Path 1: Get the working definition} If you need practical clarity fast, read Sections~\ref{sec:intro}--\ref{sec:practical} and stop. You'll get the three-level hierarchy, the six operational properties, and a practical evaluation rubric. This is enough to use the term correctly, evaluate vendor claims, and participate in informed discussions. You can always return for deeper context later.

\paragraph{Path 2: Understand where this came from} If you want to understand why we define agents this way and how the concept evolved, continue through Section~\ref{sec:disciplines}. You'll learn how agency emerged from 1950s philosophy through 1990s distributed systems to today's LLM-powered tools (Section~\ref{sec:history}), and why eight different disciplines---from law to cognitive science---emphasize different aspects of agency (Section~\ref{sec:disciplines}). This historical and disciplinary context explains current debates and grounds our synthesis in established scholarship.

\paragraph{Path 3: Master the analytical framework} If you're writing research, crafting policy, or need comprehensive understanding, read everything. Sections~\ref{sec:dimensions}--\ref{sec:synthesis} provide analytical dimensions (autonomy spectrum, entity frames, goal dynamics), formal specifications, boundary cases, and professional deployment requirements. Section~\ref{sec:furtherlearning} synthesizes key takeaways and connects to subsequent chapters. This path gives you the theoretical foundations needed for rigorous analysis.

\subsection*{What's in Each Section}

The table of contents provides section titles and page numbers. Each section begins with a brief overview of its scope and purpose. Sections~\ref{sec:intro}--\ref{sec:practical} provide definitions and practical tools; Sections~\ref{sec:history}--\ref{sec:synthesis} trace historical and disciplinary evolution; Section~\ref{sec:furtherlearning} concludes with key takeaways and forward references.

\vspace{1em}

While this scope may seem broad, this chapter is not meant to be everything to everyone. We have purposefully focused on the conceptual and educational, not the technical or promotional.

Specifically, this chapter does not include:

\begin{itemize}
\item Marketing claims about ``agentic AI'' products
\item Product reviews or vendor comparisons
\item Implementation guides for specific frameworks
\item Technical tutorials on building agent systems
\end{itemize}

Resources on these topics are widely available elsewhere.

\vspace{0.5em}
\noindent\textcolor{border-neutral}{\rule{\textwidth}{1.5pt}}
