% How to Read This Chapter - Reading paths for different audiences
% This section appears before the main content to guide readers

\section*{How to Read This Chapter}
\addcontentsline{toc}{section}{How to Read This Chapter}

This chapter answers a deceptively simple question: \textit{What is an agent?}

The term ``agent'' appears everywhere in industry and academia today --- from ``AI agents'' to ``agentic workflows'' --- yet rarely with clear definition. This creates confusion, miscommunication, and inevitably, disappointment.

We hope to help you avoid this disappointment, but \textbf{this chapter is not short and your time is valuable}. You don't need to read everything, especially on first reading.  If you're in a hurry, consider which of the three reading paths below best fits your goals.

\vspace{0.5em}

\begin{highlightbox}[colback=bg-definition, colframe=definition-base, boxrule=1.5pt, coltitle=white, title={\textbf{Path 1: Definitions and Concepts}}]
\textbf{Read Sections~\ref{sec:intro}--\ref{sec:practical} only.} You'll get:
\begin{itemize}
\item The definition and three-level hierarchy
\item A rubric with common misconceptions clarified
\item Enough to use the term correctly and evaluate claims
\end{itemize}
\textit{Stop after Section~\ref{sec:practical}—you'll have what you need.}
\end{highlightbox}

\vspace{0.5em}

\begin{highlightbox}[colback=bg-definition, colframe=definition-base, boxrule=1.5pt, coltitle=white, title={\textbf{Path 2: Historical and Disciplinary Context}}]
\textbf{Read Sections~\ref{sec:intro}--\ref{sec:disciplines}.} You'll also learn:
\begin{itemize}
\item How the concept evolved from the 1950s to 2025 (Section~\ref{sec:history})
\item Why different disciplines define agents differently (Section~\ref{sec:disciplines})
\item Historical context for current debates
\item The intellectual foundations of modern definitions
\end{itemize}
\end{highlightbox}

\vspace{0.5em}

\begin{highlightbox}[colback=bg-definition, colframe=definition-base, boxrule=1.5pt, coltitle=white, title={\textbf{Path 3: Understanding and Assessing}}]
\textbf{Read all sections} for:
\begin{itemize}
\item Analytical framework: autonomy, entity frames, goal dynamics (Sections~\ref{sec:dimensions}--\ref{sec:synthesis})
\item Conclusion with key takeaways and reading paths (Section~\ref{sec:furtherlearning})
\item Deep theoretical foundations across disciplines
\end{itemize}
\end{highlightbox}

\subsection*{What's in Each Section}

The table of contents provides section titles and page numbers. Each section begins with a brief overview of its scope and purpose. Sections~\ref{sec:intro}--\ref{sec:practical} provide definitions and practical tools; Sections~\ref{sec:history}--\ref{sec:synthesis} trace historical and disciplinary evolution; Section~\ref{sec:furtherlearning} concludes with key takeaways and forward references.

\vspace{1em}

While this scope may seem broad, this chapter is not meant to be everything to everyone. We have purposefully focused on the conceptual and educational, not the technical or promotional.

Specifically, this chapter does not include:

\begin{itemize}
\item Marketing claims about ``agentic AI'' products
\item Product reviews or vendor comparisons
\item Implementation guides for specific frameworks
\item Technical tutorials on building agent systems
\end{itemize}

Resources on these topics are widely available elsewhere.

\vspace{1em}
\noindent\textcolor{border-neutral}{\rule{\textwidth}{1.5pt}}
