\section{Analytical Framework}
\label{sec:synthesis}

Having traced the historical evolution (\Cref{sec:history}), examined disciplinary perspectives (\Cref{sec:disciplines}), and identified analytical dimensions (\Cref{sec:dimensions}), we now synthesize these insights into a coherent analytical framework. This framework provides both the theoretical foundation for understanding agency and practical tools for evaluating real-world systems.



\subsection{Cross-Cutting Patterns}

Our analysis across disciplines reveals four fundamental patterns that consistently appear when scholars and practitioners identify agency. These patterns transcend individual fields, providing a shared foundation for recognizing agentic behavior regardless of whether we're examining human decision-makers, organizational structures, or computational entities.

\begin{keybox}[title={\textbf{Pattern 1: Goal-Directedness}}]
Agents pursue objectives or performance criteria. Whether we call them intentions in philosophy, rewards in reinforcement learning, mandates in law, or incentives in economics, the common thread remains: agents exhibit behavior oriented toward achieving outcomes rather than merely reacting to stimuli. This distinguishes purposeful action from reflexive response.
\end{keybox}

\begin{keybox}[title={\textbf{Pattern 2: Perception–Action Coupling}}]
Agents operate through iterative cycles of sensing and acting. This pattern appears across theoretical frameworks---from \textcite{russellnorvig2020aima}'s sensor-actuator loops to \textcite{bratman1987intention}'s practical reasoning and \textcite{bandura1989human}'s reciprocal determinism. Entities that transform inputs once without iteration, such as compilers or single-shot classifiers, lack this essential coupling that characterizes agentic systems.
\end{keybox}

\begin{keybox}[title={\textbf{Pattern 3: Selective Autonomy}}]
Agents require some degree of discretion in their choices, as we explored in \Cref{sec:dimensions}. Entities that execute fixed scripts without any decision-making capacity do not qualify as agents. The degree of autonomy varies significantly across domains---from minimal discretion in rule-based entities to substantial independence in modern AI agents---with higher levels of discretion typically raising more complex governance and accountability questions.
\end{keybox}

\begin{keybox}[title={\textbf{Pattern 4: Termination}}]
Purposeful behavior requires boundaries. Agents must have mechanisms for recognizing when to stop, whether through explicit controls (resource budgets, time limits, escalation triggers) or implicit goal satisfaction (achieving target state, exhausting search space). Entities that run indefinitely without clear stopping conditions or decision points fall outside our definition of agentic systems, as they lack the goal-oriented closure that characterizes purposeful action.
\end{keybox}

\subsection{Formal Core}

While the patterns above provide intuitive understanding, precision demands formal specification. We now translate our insights into a mathematical framework that enables rigorous analysis and falsifiable claims about agency.

To formalize our framework, we denote the six essential properties as follows:
\begin{itemize}[leftmargin=1.5em]
\item $G$ = goal or utility function that directs behavior
\item $P$ = perception capability for sensing environment
\item $A$ = action capability for affecting environment
\item $I$ = iteration across multiple steps with state preservation
\item $D$ = discretion or autonomy in decision-making
\item $T$ = termination through explicit or implicit conditions
\end{itemize}

We use the notation $\mathit{has}(x, \cdot)$ to indicate that entity $x$ possesses a given property.

\begin{definitionbox}[title={\textbf{Formal Definitions}}]
We define three nested categories of entities based on their properties:
\begin{align*}
\mathit{Agent}(x) &\iff \mathit{has}(x,G) \land \mathit{has}(x,P) \land \mathit{has}(x,A) \\
&\quad\text{(Minimal agency: goals, perception, action)}\\[0.5em]
\mathit{AgenticSystem}(x) &\iff \mathit{Agent}(x) \land \mathit{has}(x,I) \land \mathit{has}(x,D) \land \mathit{has}(x,T) \\
&\quad\text{(Full operational system: all six properties)}\\[0.5em]
\mathit{AIAgent}(x) &\iff \mathit{AgenticSystem}(x) \land \mathit{implementedWithAI}(x)\\
&\quad\text{(AI-powered implementation of agentic system)}
\end{align*}
\end{definitionbox}

These formal definitions yield several important theoretical results that clarify the relationships between different types of entities:

\begin{theorembox}[title={\textbf{Key Theoretical Results}}]
\textbf{Hierarchical Subsumption.} The categories form a nested hierarchy:
$$\mathit{AIAgent}(x) \Rightarrow \mathit{AgenticSystem}(x) \Rightarrow \mathit{Agent}(x)$$
Every AI agent is necessarily an agentic system, and every agentic system is necessarily an agent, but the converses do not hold.

\textbf{Non-Equivalence.} The categories are strictly distinct:
$$\exists x:\ \mathit{Agent}(x) \land \neg\mathit{AgenticSystem}(x)$$
For example, a single-shot classifier might have goals, perception, and action capabilities, but lacks the iteration, discretion, and termination properties required for an agentic system.

\textbf{Property Attribution.} Individual features can be described as ``agentic'' (such as ``agentic perception'' or ``agentic planning'') without implying the system as a whole qualifies as an $\mathit{AgenticSystem}$. Full agentic system status requires all six properties.
\end{theorembox}

The termination property deserves special attention, as it can manifest in two distinct forms:

\begin{definitionbox}[title={\textbf{Termination Modes}}]
An entity satisfies the termination requirement through either explicit or implicit mechanisms:
$$\mathit{has}(x,T) \iff \text{ExplicitTerm}(x)\ \lor\ \text{ImplicitTerm}(x)$$

\textbf{Explicit termination} involves programmed monitoring of conditions such as:
\begin{itemize}[leftmargin=1.5em,nosep]
\item Resource budgets (computation time, API calls, token limits)
\item Timeout constraints (wall clock time, iteration counts)
\item Escalation triggers (confidence thresholds, error conditions)
\end{itemize}

\textbf{Implicit termination} occurs naturally through:
\begin{itemize}[leftmargin=1.5em,nosep]
\item Goal satisfaction (target state achieved)
\item Quiescent states (no further actions available)
\item Bounded episodes (fixed-duration tasks)
\end{itemize}
\end{definitionbox}

\subsection{Foundations in Context}

Our formal framework draws from multiple theoretical traditions, each contributing essential insights about how agents operate. Understanding these foundations helps practitioners from different backgrounds connect our framework to established concepts in their fields.

\paragraph{Plan-Based Control}
The tradition of \keyterm{plan-based control} emphasizes how agents decompose high-level goals into actionable steps. This approach, central to cognitive science and AI planning systems, treats agency as means-end reasoning---connecting what we want to achieve with how to achieve it. In legal practice, we see this pattern in structured workflows like due diligence, where complex objectives decompose into systematic review tasks. Modern AI agents employ similar hierarchical planning when breaking down user requests into sequences of tool calls and sub-tasks.

\paragraph{Reactive Control}
Not all intelligent behavior requires elaborate planning. The \keyterm{reactive control} paradigm demonstrates that simple perception-action rules can produce sophisticated behavior when properly designed. This insight, pioneered in robotics by researchers like Rodney Brooks, shows that agents can be effective without maintaining complex world models. In fast-changing environments or when sensing is reliable, responsive behavior often outperforms deliberative planning. Contemporary LLM agents balance both approaches---maintaining high-level plans while reacting dynamically to unexpected outputs or errors.

\paragraph{BDI Models}
The \keyterm{Belief-Desire-Intention} (BDI) framework provides a structured account of agent mental states. Agents maintain beliefs about their environment, desires about preferred outcomes, and intentions representing committed plans. This model, influential in agent-oriented programming, explains how agents form and revise commitments as new information arrives. Legal agents exemplify this pattern: maintaining beliefs about applicable law, desires to serve client interests, and intentions manifested as filed documents or negotiation positions.

\paragraph{Reinforcement Learning}
The \keyterm{reinforcement learning} paradigm formalizes how agents learn from experience. Through cycles of action, observation, and reward, agents discover which behaviors achieve their goals. This framework clarifies why iterative interaction outperforms single-shot processing---each cycle provides information that improves future decisions. Modern AI agents increasingly incorporate reinforcement learning, whether through fine-tuning on human feedback or real-time adaptation during task execution.

\paragraph{Intentional Stance}
Following philosopher Daniel Dennett's influential work \parencite{dennett1987intentional}, the \keyterm{intentional stance} treats agency as a pragmatic attribution rather than metaphysical fact. We describe systems as having beliefs, desires, and intentions when such descriptions improve our ability to predict and control their behavior. This perspective liberates us from endless debates about whether AI systems ``really'' have goals or merely simulate goal-directed behavior. What matters is whether treating them as agents yields practical benefits.

\begin{highlightbox}[title={\textbf{Why These Foundations Matter}}]
These theoretical foundations aren't merely academic background---they directly inform how we build and evaluate agentic systems:

\begin{itemize}[leftmargin=1.5em]
\item \textbf{Planning} provides techniques for goal decomposition and task orchestration
\item \textbf{Reactive control} validates simple, robust designs that respond to environmental changes
\item \textbf{BDI models} offer architectures for managing commitments and updating beliefs
\item \textbf{Reinforcement learning} explains how agents improve through experience and feedback
\item \textbf{Intentional stance} justifies treating systems as agents when it enhances prediction and control
\end{itemize}

Together, these traditions ground our six-property framework in established theory while maintaining practical applicability. The result is a definition precise enough for formal verification yet flexible enough to span disciplines from law to computer science.
\end{highlightbox}

\subsection{Boundary Cases and Clarifications}

Testing our framework against boundary cases reveals both its strengths and the subtleties of applying formal definitions to real-world systems. These edge cases aren't merely academic exercises---they illuminate where practitioners disagree about agency and help refine our understanding of what distinguishes agents from non-agents.

\begin{highlightbox}[title={\textbf{Case 1: Thermostats as Minimal Agents}}]
A simple thermostat qualifies as an agent under our minimal definition: it pursues a goal (maintain target temperature), perceives its environment (temperature sensor), acts to affect that environment (heating/cooling control), and iterates continuously. However, its discretion is minimal---essentially binary decisions about heating or cooling. This example shows why minimal agency (Level 1) sets a low bar, and why most interesting questions about governance, liability, and capability emerge only at higher levels of autonomy.
\end{highlightbox}

\begin{highlightbox}[title={\textbf{Case 2: Multi-Agent Collectives and Emergent Behavior}}]
When multiple agents interact, the collective can exhibit properties no single agent possesses. A trading floor, legal team, or swarm of autonomous vehicles may demonstrate emergent behaviors that challenge our individual-focused definitions. Legal systems have long grappled with this through doctrines distinguishing corporate from individual liability. Our framework accommodates both perspectives: we can analyze individual agents within the system or treat the collective itself as an agent with emergent goals and capabilities.
\end{highlightbox}

\begin{highlightbox}[title={\textbf{Case 3: Virtual versus Embodied Agents}}]
Software agents need not have physical bodies to qualify as agents. A contract analysis system operating entirely in cloud infrastructure satisfies all six properties without any physical embodiment. While robotics researchers sometimes insist on embodiment, our framework follows the broader AI tradition of recognizing virtual agents as fully legitimate instances of agency.
\end{highlightbox}

\begin{highlightbox}[importance=low,title={\textbf{Case 4: Single-Shot LLM Interactions}}]
A single ChatGPT response, however sophisticated, typically lacks the iteration property required for agentic systems. It processes input once and generates output without maintaining state across multiple perception-action cycles. The exception would be if we can identify an internal iterative loop within the single interaction---for instance, chain-of-thought reasoning that cycles through multiple reasoning steps before producing output.
\end{highlightbox}

\begin{highlightbox}[importance=low,title={\textbf{Case 5: Tool Use versus Tool-Using Agents}}]
Simply using tools doesn't make an entity agentic. A script that calls an API once isn't an agent. What matters is \textit{purposeful, iterative} tool use directed toward goals. An LLM that repeatedly queries databases, processes results, and decides which tool to invoke next based on accumulated information exhibits agency. The distinction lies in the goal-directed iteration, not the mere presence of tool use.
\end{highlightbox}

\begin{highlightbox}[importance=low,title={\textbf{Case 6: Continuous Control and Implicit Termination}}]
Entities engaged in continuous control (like maintaining server uptime) might seem to violate our termination requirement. However, these entities typically operate in bounded episodes with implicit termination conditions: achieving steady state, exhausting resources, or reaching time limits. Even ``always-on'' entities have mechanisms for recognizing when to stop particular behaviors, satisfying our termination criterion.
\end{highlightbox}

\subsection{Professional Deployment}

Understanding what makes an entity agentic is only the first step. In regulated domains like law and finance, deploying AI agents requires robust governance frameworks that ensure compliance, maintain professional standards, and protect sensitive information. The stakes are particularly high: errors can trigger regulatory sanctions, breach fiduciary duties, or compromise client confidentiality.

Professional deployment demands that we move beyond theoretical properties to operational safeguards. Each of the six properties we've identified creates both opportunities and risks that must be managed through careful entity design and governance controls. The following controls transform theoretical capabilities into trustworthy tools:

\begin{keybox}[title={\textbf{1. Attribution and Provenance}}]
Every factual claim must be traceable to its source. The system should record not just what it concluded but where that information originated---whether from case law, regulatory filings, market data, or expert analysis. This enables independent verification and helps practitioners assess the reliability of agent-generated work product. Modern AI agents should maintain citation chains that would satisfy the standards of legal briefs or investment memoranda.
\end{keybox}

\begin{keybox}[title={\textbf{2. Auditable Reasoning}}]
Professional decisions require transparent rationales. The system must log major reasoning steps, tool invocations, and decision points in forms that preserve privilege while enabling review. This isn't about exposing every neural network activation---it's about capturing the business logic an expert would expect to see. Think litigation hold decisions, trade execution rationales, or compliance determination paths.
\end{keybox}

\begin{keybox}[title={\textbf{3. Bounded Operation}}]
Unbounded iteration violates both practical and regulatory constraints. Systems need explicit limits: computational budgets, time constraints, maximum iteration counts. These bounds prevent runaway processes while ensuring predictable resource consumption. In financial contexts, this might mean limiting the number of market data queries; in legal contexts, capping document review cycles.
\end{keybox}

\begin{keybox}[title={\textbf{4. Escalation Pathways}}]
Agents must know their limits and when to seek help. Clear escalation triggers---low confidence scores, high-stakes decisions, detected conflicts---should route matters to human review. The escalation logic should be transparent and adjustable based on risk tolerance and regulatory requirements. A contract review agent, for instance, might escalate unusual liability provisions or non-standard jurisdiction clauses.
\end{keybox}

\begin{keybox}[title={\textbf{5. Confidentiality Controls}}]
Information barriers are non-negotiable in professional services. The system must enforce ethical walls between matters, implement role-based access controls, and apply appropriate redaction rules. This goes beyond basic security to encompass the nuanced confidentiality obligations of legal privilege, insider trading restrictions, and client confidentiality duties.
\end{keybox}

\begin{keybox}[title={\textbf{6. Accountability Mapping}}]
When things go wrong---and they will---clear lines of responsibility enable rapid response and remediation. Who configured the agent? Who reviewed its outputs? Who authorized its deployment? These questions must have clear answers before the system processes its first client matter. Accountability isn't about blame; it's about maintaining professional standards when humans and AI systems collaborate.
\end{keybox}

\subsection{Common Questions About Agency}

Our framework inevitably raises questions about edge cases and apparent contradictions. Here we address the most common concerns that arise when applying these definitions to real-world systems.

\begin{questionbox}[title={\textbf{Is iteration just repeated action?}}]
Iteration means more than simply repeating the same action multiple times. True iteration involves maintaining state across perception-action cycles, enabling the entity to learn from previous attempts and adjust its approach. An entity that simply retries the same failing action indefinitely isn't iterating in our sense---it's stuck in a loop. Iteration requires the ability to incorporate feedback, correct errors, and make progress toward goals through successive refinements. This is why single-shot entities, no matter how sophisticated, don't qualify as agentic systems under our framework.
\end{questionbox}

\begin{questionbox}[title={\textbf{Do continuous tasks violate the termination requirement?}}]
Entities performing continuous tasks like monitoring or maintenance might seem to run forever, apparently violating our termination requirement. However, these entities still satisfy the termination property through implicit or explicit mechanisms. Implicit termination occurs when the entity reaches a steady state or exhausts available actions. Explicit termination happens through resource budgets, time limits, or escalation triggers. Even an entity that monitors markets continuously operates in bounded episodes---trading sessions, reporting periods, or maintenance windows. The key is that the entity has clear conditions under which it will cease its current behavior pattern, even if it later resumes.
\end{questionbox}

\begin{questionbox}[title={\textbf{If thermostats count as agents, have we trivialized the concept?}}]
Recognizing thermostats as minimal agents doesn't trivialize agency---it establishes a meaningful floor. Many entities fail even this basic test: databases lack goals, compilers lack perception-action loops, and lookup tables lack any form of action. More importantly, while thermostats meet the minimal criteria for Level 1 agency, the interesting questions emerge as we move up the hierarchy. The distinction between a thermostat and an AI legal research assistant isn't whether they're agents, but the degree of autonomy, sophistication, and responsibility they possess. Our framework acknowledges the thermostat's basic agency while reserving ``agentic system'' status for entities with all six properties, and ``AI agent'' designation for those using AI/ML for implementation.
\end{questionbox}

\begin{questionbox}[title={\textbf{Is ``AI agent'' just marketing hype?}}]
In our framework, ``AI agent'' has a precise, falsifiable meaning: an agentic system that uses AI or machine learning (particularly large language models) for planning and orchestration. This excludes single-shot chat completions, no matter how impressive, because they lack iteration. It excludes traditional rule-based systems, even if they meet all six properties, because they don't use AI for implementation. It also excludes simple ML classifiers that lack goal-directedness and autonomous action. When vendors claim their product is an ``AI agent,'' we can test this claim against our six properties and the AI implementation requirement. This transforms marketing language into testable assertions about system architecture and capabilities.
\end{questionbox}
