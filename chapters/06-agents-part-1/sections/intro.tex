\section{Introduction}
\label{sec:intro}

\textbf{Agent.} \textit{Agentic.}

Few terms generate more confusion despite widespread use. While these words appear everywhere—from marketing copy to academic papers—their meanings remain contested and often unclear. Yet despite this definitional chaos, the underlying concepts are deeply intuitive and accessible.

At heart, agents are simply \textbf{``doers'' with a to-do}. As we unpack this accessible starting point, we will discover more explicit conditions for identification. But this four-word formulation captures something essential: agency requires both goals and the capacity to act toward them.

\subsection{Motivation and Approach}

The proliferation of ``agentic AI'' makes definitional clarity urgent. Existing work remains fragmented across purpose and discipline: computer scientists cite Russell and Norvig \parencite{russellnorvig2020aima}, philosophers reference Bratman \parencite{bratman1987intention}, legal scholars consult the Restatement of Agency \parencite{restatement2006agency}, and commercial vendors seem untethered by anything other than sales.

Some of this fragmentation reflects genuinely different perspectives, such as whether we recognize agents by their \textit{internal properties} (mental states, intentions) or \textit{external manifestations} (observable behavior, delegated authority)—a spectrum we explore in Section~\ref{sec:disciplines}.  While theoretical considerations like these can be useful, it is now most critical that we \textbf{establish a practical framework} to guide communication and coordination.

The stakes for getting this right are tangible. In \textit{Mata v. Avianca}, ChatGPT generated plausible but fictitious case citations that an attorney submitted in a court brief \parencite{mata-avianca-2023}. The attorney relied on a single-shot text generator lacking the tools to search legal databases, the iteration to verify citations, and the mechanisms to escalate uncertainty. An agentic legal research system would have used search tools to validate citations, iterated to confirm case holdings, and escalated when unable to locate a source. The failure stemmed not from lack of sophistication but from using a system without the operational properties required for reliable legal research. Distinguishing genuinely agentic systems from sophisticated chatbots is essential for professional practice, regulatory compliance, and client protection.

For legal and financial applications, the six operational properties are \textit{necessary but not sufficient}. Professional deployment demands additional safeguards—attribution to authoritative sources, auditable provenance, escalation protocols, and confidentiality controls—that augment rather than replace the six-property framework. Section~\ref{sec:synthesis} addresses these professional deployment requirements in detail.

To do so, we organize agency into three levels that correspond to answering the following questions:

\textbf{Level 1}: What makes \textit{something}—biological or computational—an agent?

\textbf{Level 2}: What makes computational systems agentic?

\textbf{Level 3}: How do traditional and AI-powered agentic systems differ?

Answering these progressive questions establishes a nested hierarchy with three levels, as illustrated in Figure~\ref{fig:hierarchy}.

% Three-level hierarchy diagram: Agent > Agentic Software > Agentic AI
% Visual representation of subset relationships

\begin{figure}[htbp]
\centering
\begin{tikzpicture}[
    font=\sffamily,
    level1/.style={fill=gray-100, draw=gray-500, line width=1.5pt},
    level2/.style={fill=gray-200, draw=gray-600, line width=1.5pt},
    level3/.style={fill=slate-100, draw=slate-700, line width=2pt},
    label/.style={font=\sffamily\bfseries},
    example/.style={font=\sffamily\small, text=gray-700}
]

% Level 1: Agents (outermost)
\draw[level1] (0,0) circle (4.5cm);
\node[label, align=center] at (0,3.8) {Level 1:\\Agents};
\node[example, align=center] at (0,-3.6) {People, thermostats,\\software systems};

% Level 2: Agentic Software - adjusted label position
\draw[level2] (0,0) circle (3cm);
\node[label, align=center] at (0,2.1) {Level 2:\\Agentic Software};
\node[example, align=center] at (0,-2.1) {Expert systems,\\rule-based agents};

% Level 3: Agentic AI (innermost)
\draw[level3] (0,0) circle (1.5cm);
\node[label, align=center] at (0,0.3) {Level 3:\\Agentic AI};
\node[example, align=center, font=\sffamily\footnotesize] at (0,-0.6) {LLM-based\\agents};

% Annotations for properties - showing cumulative hierarchy
\node[
    align=left,
    font=\sffamily\small,
    text=gray-700,
    inner sep=8pt,
    outer sep=0pt
] at (6.95,0) {
    {\footnotesize\textsc{Properties}}\\[3pt]
    {\footnotesize Level 1 (3):}\\[1pt]
    \quad Goal\\[0.5pt]
    \quad Perception\\[0.5pt]
    \quad Action\\[5pt]
    {\footnotesize Levels 2 \& 3 (+3):}\\[1pt]
    \quad Iteration\\[0.5pt]
    \quad Adaptation\\[0.5pt]
    \quad Termination
};

\end{tikzpicture}
\caption{The three-level hierarchy of agency. Each level is a subset of the one above it: all agentic AI is agentic software, and all agentic software consists of agents.}
\label{fig:hierarchy}
\end{figure}


\subsection{Level 1: Minimal Agency}

We begin with the conceptual foundation. What is the absolute minimum required for something to qualify as an agent—whether human, organizational, or computational? Level 1 establishes this baseline, applicable across all domains and technologies.

\begin{definitionbox}[title={\textbf{Level 1: Agent (Mnemonic: GPA)}}]
	An \keyterm{agent} is any entity that pursues goals through perception and action, with at least minimal discretion over which action to take in response to what it perceives.

	\textcolor{border-neutral}{\rule{\linewidth}{0.4pt}}

	\textit{Examples:} A paralegal researching case law, a thermostat maintaining temperature, an organization pursuing objectives.

	\noindent\textcolor{border-neutral}{\rule{\linewidth}{0.4pt}}\par\noindent
	\textbf{Minimal properties (3):}
	\begin{itemize}[nosep,leftmargin=1.5em]
		\item \textbf{G}oal --- Clear objective or performance criterion
		\item \textbf{P}erception --- Awareness of environment through sensing
		\item \textbf{A}ction --- Capability to affect environment
	\end{itemize}
\end{definitionbox}

\textbf{Property Definitions and Falsification Tests:}

\textbf{Goal (G):} A clear objective, task specification, or performance criterion that directs behavior. Goals may be simple (maintain temperature) or complex (maximize portfolio returns), provided by external principals or internally generated, fixed or dynamic.

\textit{Falsification:} If the entity responds identically regardless of desired outcomes, or transforms inputs mechanically without reference to success criteria, it lacks goals. Examples: compilers (execute predetermined transformations), pure lookup tables (no optimization target).

\textbf{Perception (P):} Awareness of environment through sensing capabilities---sensors, APIs, database access, document reading. The environment need not be physical; abstract spaces (contract negotiations, market data) qualify. \textit{Legal example:} An agent perceives via EDGAR/Westlaw APIs to read regulatory filings and case law, observes which queries return hits, and uses those observations to refine subsequent searches.

\textit{Falsification:} If the entity operates identically regardless of environmental state, or cannot observe consequences of its own actions, it lacks perception. Examples: open-loop controllers (no feedback), write-only systems.

\textbf{Action (A):} Capability to affect environment through actuators---physical forces, variable modifications, tool invocations, API calls, command execution. Requires minimal discretion: selecting among at least two possible actions contingent on perceptions.

\textit{Falsification:} If the entity cannot modify its environment, or executes exactly one predetermined sequence regardless of circumstances, it lacks action. The discretion threshold is ``$\geq$2 policies contingent on perceptions.'' Examples that fail: pure sensors (read-only), fixed scripts with zero conditional logic, fully deterministic single-action transformers that apply one complex transformation regardless of input content (even if the transformation itself is sophisticated).

This trinity forms the conceptual bedrock of agency—equally applicable to humans navigating social contexts, organizations pursuing strategic objectives, biological organisms seeking survival, or computational systems executing tasks. While these three characteristics suffice for theoretical classification, practical deployment demands additional sophistication.

\textbf{Distinguishing agents from non-agents reveals critical boundaries.} Consider systems we encounter daily that, despite their complexity, fail to meet our criteria. A compiler, though it transforms code through sophisticated analysis, operates without autonomous goals—it executes predetermined transformations. Legal databases, while responsive to queries and rich with information, lack independent objectives or environmental adaptation. Even a single ChatGPT response, impressive as it may be, represents a one-shot generation rather than iterative goal pursuit.

This baseline framework illuminates the essence of agency, yet professional applications demand more. The gap between a simple thermostat cycling toward a temperature target and an AI system conducting legal research spans more than technological sophistication—it requires architectural elements that ensure reliability, adaptability, and accountability. Legal research tools, portfolio management systems, and document review platforms operate in environments where stakes are high and errors costly. These operational realities shape our expanded framework for deployable agentic systems.

\subsection{Operational Definition: Agentic Systems}

While Level 1 establishes what makes something an agent, computational systems in production require three additional properties beyond the minimal three. These six properties together define what we call \keyterm{agentic systems}—the operational standard that bridges conceptual agency and real-world deployment. Both traditional software (Level 2) and AI-powered implementations (Level 3) can achieve this operational standard, though they differ fundamentally in how they realize each property.

\begin{definitionbox}[unbreakable,title={\textbf{Operational Definition: Agentic System (GPA+IAT)}}]
	An \keyterm{agentic system} is a goal-directed agent that repeatedly perceives and acts in its environment, adapting from observations until clear termination conditions are met (explicit or implicit).

	\textcolor{border-neutral}{\rule{\linewidth}{0.4pt}}

	\textit{Examples:} Legal research assistants that iteratively refine queries and generate case summaries. Contract analysis systems that identify risks and produce issue reports. Portfolio rebalancing systems that monitor positions and execute trades. Fraud detection systems that analyze transactions and generate alerts.

	\textcolor{border-neutral}{\rule{\linewidth}{0.4pt}}

	\textbf{Additional properties beyond Level 1 (+3):}
	\begin{itemize}[nosep,leftmargin=1.5em]
		\item \textbf{I}teration --- Repeat perceive-act cycles, not single-shot
		\item \textbf{A}daptation --- Adjust strategy based on feedback/results
		\item \textbf{T}ermination --- Clear stopping conditions (explicit or implicit)
	\end{itemize}

	\textit{Total: 6 operational properties (Goal, Perception, Action + Iteration, Adaptation, Termination). For mnemonic convenience we write this as GPA + IAT: Goal (G), Perception (P), Action (A), Iteration (I), Adaptation (A again), Termination (T).}
\end{definitionbox}

These six properties emerged from decades of agent research as commonly recognized operational requirements for reliable computational deployment. While not formally proven as minimum necessary, they consistently appear across deployed systems in domains from robotics to enterprise software, reflecting lessons from fielding real-world implementations. Section~\ref{sec:history} traces how each property became recognized as essential through both theoretical development and practical experience.

\textbf{Additional Property Definitions and Falsification Tests:}

\textbf{Iteration (I):} Multiple perceive-act cycles with state preservation across rounds. The entity repeatedly gathers information, takes action, observes results, and continues---not single-shot processing. Crucially, the system must perceive outcomes of prior actions and update subsequent actions accordingly within the same goal pursuit.

\textit{Falsification:} If the entity processes input once and produces output without maintaining state across cycles, it lacks iteration. Merely repeating the same action without incorporating new observations doesn't qualify. \textit{Batching vs. iteration:} Batched one-pass pipelines processing multiple items sequentially are not iteration unless the system observes outcomes from earlier items and modifies its approach for later items based on those observations. Examples that fail iteration test: single ChatGPT response (one-shot), batch processors applying identical logic to each item without inter-item learning.

\textbf{Adaptation (A):} Strategy modification based on accumulated observations and feedback within a session or task.\footnote{This definition focuses on \textit{session-level adaptation}---modifying behavior within a single task execution based on immediate feedback. This differs from \textit{cross-session learning}, where a system improves through model retraining across multiple tasks (for example, fine-tuning an LLM on user feedback). For our purposes, within-session adaptation is the operational property that distinguishes agentic systems from static tools. In professional contexts (such as procurement or vendor evaluation), be explicit about which type of learning is in scope.} The entity adjusts its approach when initial attempts fail, learns which actions succeed, updates its policy based on results.

\textit{Falsification:} If the entity applies identical logic regardless of outcomes, or cannot modify its approach when initial strategies fail, it lacks adaptation. Fixed rules that never change based on results don't qualify, even if they handle diverse inputs. Examples: basic thermostats (fixed on/off rules), static pattern matchers, rigid workflows that don't adjust to failures.

\textbf{Termination (T):} Clear stopping conditions ensuring bounded operation. Termination may be \textit{implicit} (goal satisfaction, reaching target state, exhausting search space) or \textit{explicit} (resource budgets, time limits, maximum iterations, escalation triggers, confidence thresholds).

\textit{Falsification:} If the entity has no mechanism for recognizing when to stop, or could cycle indefinitely without bounds, it lacks proper termination. Entities requiring external intervention to halt don't meet this criterion. Examples: infinite loops with no exit condition, systems that run until manually killed.

The relationship between levels clarifies important boundaries. Every agentic system qualifies as an agent (possessing the minimal three properties), but the reverse doesn't hold—many agents lack the operational sophistication of agentic systems. A basic mechanical thermostat illustrates this gap: it has five properties (goal: maintain temperature; perception: sensor readings; action: heating/cooling; iteration: continuous monitoring; termination: implicit when target reached), but lacks adaptation---it applies fixed on/off rules without modifying strategy based on outcomes. It responds to temperature changes reactively but doesn't learn patterns or adjust thresholds. In contrast, a \textit{smart} thermostat qualifies as a full agentic system: it learns occupancy patterns, adjusts heating schedules based on observed behavior, and modifies its strategy when energy costs spike. This distinction between reactive control (basic thermostat) and adaptive learning (smart thermostat) clarifies why minimal agency (Level 1, three properties) differs from operational agentic systems (six properties). Similarly, a human paralegal demonstrates all six properties behaviorally but operates through cognitive processes rather than discrete computational cycles.

This operational framework now raises the implementation question: \textit{How} do computational systems realize these six properties? The answer reveals an architectural distinction. Some systems use traditional programming—rules, algorithms, control logic—to manage planning and orchestration. Others employ AI/ML, particularly large language models, for these functions. We distinguish these as Level 2 (traditional) and Level 3 (AI-powered). Critically, this distinction is architectural, not evaluative: neither approach is inherently superior, and the boundary between them remains fluid and context-dependent.

\subsection{Level 2: Traditional Agentic Software}

Level 2 represents the first computational instantiation of agentic systems. These systems achieve all six operational properties through explicit programming—rules, conditional logic, algorithms, and control flow. What defines Level 2 is runtime behavior: decisions flow through programmed logic paths rather than learned models. Whether a system was coded decades ago or yesterday, if its decision-making follows deterministic rules at runtime, it operates at Level 2.

\begin{definitionbox}[unbreakable,title={\textbf{Level 2: Traditional Agentic Software}}]
	Traditional agentic software implements the six operational properties using rules, algorithms, or deterministic logic.

	\textcolor{border-neutral}{\rule{\linewidth}{0.4pt}}

	\textit{Examples:} Conflicts checking systems that scan firm databases and escalate matches. Regulatory compliance monitoring with rule-based alerts. Trading compliance systems that monitor positions and enforce limits. Invoice processing workflows with validation and retry logic.

	\textcolor{border-neutral}{\rule{\linewidth}{0.4pt}}

	\textbf{Additional properties beyond agentic systems (+0):}
	\begin{itemize}[nosep,leftmargin=1.5em]
		\item \textit{No new properties --- same 6 as agentic systems}
	\end{itemize}
\end{definitionbox}

Traditional agentic software uses rules, algorithms, or deterministic logic to implement all six properties. Planning and orchestration are explicitly coded through conditional logic, state machines, or control systems. Level 2 systems can be extremely sophisticated: a conflict checking system might employ graph analysis for relationship detection, fuzzy matching for entity resolution, and adaptive threshold tuning—all implemented through traditional programming techniques. The key architectural characteristic is that decision logic is specified by programmers at design time.

A conflicts checking system exemplifies Level 2: it has goals (identify potential conflicts), perception (scans firm databases for client/matter relationships), action (flags matches and escalates to ethics committee), iteration (continuous monitoring as new matters are opened), adaptation (adjusts matching thresholds based on false positive rates), and termination (stops when matter is cleared or rejected). The entity achieves these properties through explicitly programmed logic—rules for relationship detection, graph traversal algorithms for indirect conflicts, configurable thresholds for fuzzy name matching.

Level 2 systems achieve all six operational properties through traditional software engineering. Their performance depends on design quality, domain expertise, and implementation rigor—not on whether they employ AI/ML techniques. A well-engineered Level 2 system can substantially outperform a poorly designed Level 3 system in reliability, predictability, and effectiveness for its intended domain.

\subsection{Level 3: AI-Powered Agentic Systems}

Level 3 systems use AI/ML—particularly large language models—to manage planning, orchestration, and adaptation. The architectural distinction from Level 2 is straightforward: where Level 2 systems execute explicitly programmed decision logic, Level 3 systems employ neural networks (especially LLMs) or other learned models for these functions. In practice, modern Level 3 systems are typically hybrid: LLMs handle high-level planning and natural language interaction, while traditional code manages structured operations like database queries or API calls. This is what most practitioners mean by ``AI agents.''

\begin{definitionbox}[title={\textbf{Level 3: AI-Powered Agentic Systems}}]
	AI-powered agentic systems implement the six operational properties through strategic integration of artificial intelligence—typically neural network models such as large language models (LLMs) and vision-language models (VLMs)—with traditional computational components.

	\textcolor{border-neutral}{\rule{\linewidth}{0.4pt}}

	\textit{Examples:} AI legal research assistants that iteratively search case law and synthesize findings. AI contract risk analyzers that identify problematic clauses and recommend revisions. AI-powered document review systems that classify content and adapt to feedback. AI trading assistants that analyze market data and adjust strategies.

	\textcolor{border-neutral}{\rule{\linewidth}{0.4pt}}

	\textbf{Additional properties beyond agentic systems (+0):}
	\begin{itemize}[nosep,leftmargin=1.5em]
		\item \textit{No new properties --- same 6 as agentic systems}
	\end{itemize}
\end{definitionbox}

The architectural choice to use AI/ML for planning and orchestration has practical implications. LLMs enable natural language interfaces—users can specify goals conversationally rather than through structured formats. These neural network models handle pattern recognition tasks that would require extensive rule engineering. However, this approach trades some predictability for flexibility: LLM outputs can vary across runs, and behavior may be harder to audit than explicit rule chains. The boundary between Level 2 and Level 3 can blur—is a system using gradient boosting for fraud detection Level 2 or Level 3? Today's practical dividing line focuses on whether LLMs manage high-level planning and orchestration.

Consider an AI contract risk analyzer as a Level 3 exemplar. It possesses all six operational properties: goals (identifies and assesses contract risks), perception (reads contract text and clause context), action (flags problematic provisions, generates risk assessments), iteration (reviews document section by section), adaptation (adjusts risk scoring based on clause combinations and jurisdiction), and termination (stops when complete review is done or high-severity risk triggers immediate escalation). The LLM manages high-level planning—deciding which clauses merit detailed analysis, how to interpret ambiguous language, when to flag issues versus when to request human review—while traditional code handles document parsing, clause extraction, jurisdiction lookup, and risk score calculation. This hybrid architecture is typical of Level 3 systems: AI handles interpretation and strategic decisions, traditional programming handles structured data operations.

Table~\ref{tab:property-levels} maps the progression from minimal agency (3 properties) through agentic systems (6 properties) to implementation paradigms (traditional vs. AI).

\begin{table}[!htb]
	\centering
	\small
	\begin{tabular}{@{}lcccc@{}}
		\toprule
		\textbf{Property} & \textbf{Agent}                                   & \multicolumn{2}{c}{\textbf{Agentic Systems}}     & \textbf{Description}                                                              \\
		                  & \textbf{Level 1}                                 & \textbf{Level 2}                                 & \textbf{Level 3}                                  &                               \\
		                  & \textbf{Conceptual}                              & \textbf{Traditional}                             & \textbf{AI}                                       &                               \\
		\midrule
		Goal              & \tikz\fill[primary] (0,0) circle (0.08cm); & \tikz\fill[primary] (0,0) circle (0.08cm); & \tikz\fill[primary] (0,0) circle (0.08cm);  & Clear objective to pursue     \\
		Perception        & \tikz\fill[primary] (0,0) circle (0.08cm); & \tikz\fill[primary] (0,0) circle (0.08cm); & \tikz\fill[primary] (0,0) circle (0.08cm);  & Sense environment \& results  \\
		Action            & \tikz\fill[primary] (0,0) circle (0.08cm); & \tikz\fill[primary] (0,0) circle (0.08cm); & \tikz\fill[primary] (0,0) circle (0.08cm);  & Affect environment            \\
		\cmidrule{1-5}
		Iteration         & \tikz\draw[primary] (0,0) circle (0.08cm); & \tikz\fill[key-base] (0,0) circle (0.08cm);  & \tikz\fill[key-base] (0,0) circle (0.08cm);   & Repeat perceive-act cycle     \\
		Adaptation        & \tikz\draw[primary] (0,0) circle (0.08cm); & \tikz\fill[key-base] (0,0) circle (0.08cm);  & \tikz\fill[key-base] (0,0) circle (0.08cm);   & Adjust based on feedback      \\
		Termination       & \tikz\draw[primary] (0,0) circle (0.08cm); & \tikz\fill[key-base] (0,0) circle (0.08cm);  & \tikz\fill[key-base] (0,0) circle (0.08cm);   & Stopping condition (explicit or implicit)   \\
		\midrule
		AI-Powered        & \tikz\draw[primary] (0,0) circle (0.08cm); & \tikz\draw[primary] (0,0) circle (0.08cm); & \tikz\fill[example-base] (0,0) circle (0.08cm); & Uses AI/ML for implementation \\
		\bottomrule
	\end{tabular}
	\caption{Property requirements by level. Filled circles indicate required properties: blue (filled) for minimal agency (Level 1), amber (filled) for additional operational properties (Levels 2 \& 3). Empty circles indicate properties not required. Below the horizontal rule: green (filled) indicates AI-powered implementation (Level 3 only). \textit{Accessibility:} Table has 5 columns (Property, Level 1 Agent, Level 2 Traditional, Level 3 AI, Description) and 8 rows. First three rows (Goal, Perception, Action) required at all levels with filled blue circles. Rows 4--6 (Iteration, Adaptation, Termination) required only for Levels 2--3 with filled amber circles, empty for Level 1. Final row (AI-Powered) shows empty circles for Levels 1--2, filled green for Level 3.}
	\label{tab:property-levels}
\end{table}

While Levels 2 and 3 share identical property requirements, they differ fundamentally in \textit{how} each property is implemented. Table~\ref{tab:implementation-contrast} contrasts implementation approaches.

\begin{table}[!htb]
	\centering
	\small
	\rowcolors{2}{white}{bg-note}
	\begin{tabular}{@{}lp{6cm}p{6cm}@{}}
		\toprule
		\textbf{Property} & \textbf{Level 2 (Traditional)} & \textbf{Level 3 (AI-Powered)} \\
		\midrule
		Goal & Hardcoded objectives, config files, explicit targets & Natural language instructions, conversational goal specification \\
		Perception & Structured APIs, SQL queries, regex parsing & LLM document understanding, vision-language models, semantic search \\
		Action & Function calls, database writes, API invocations & Tool orchestration via LLM function calling, generated code execution \\
		Iteration & Control loops (while/for), state machines, workflow engines & LLM manages reasoning loop, decides next tool based on context \\
		Adaptation & Rule adjustments, threshold tuning, A/B test results & Chain-of-thought reasoning, prompt-based strategy revision, in-context learning \\
		Termination & Explicit conditionals (max iterations, timeout timers) & LLM decides task complete based on goal satisfaction, budgets enforced externally \\
		\midrule
		Planning & Decision trees, rule engines, optimization algorithms & LLM generates plans, ReAct-style reasoning traces \\
		Logs & Structured application logs, database audit trails & Natural language reasoning chains, tool invocation histories \\
		\bottomrule
	\end{tabular}
	\caption{Implementation contrast between traditional (Level 2) and AI-powered (Level 3) agentic systems. Both satisfy identical property requirements; the architectural distinction lies in implementation mechanisms. \textit{Accessibility:} Comparison table with 3 columns (Property, Level 2 Traditional, Level 3 AI-Powered) and 8 rows. Rows 1--6 compare the six core properties: Level 2 uses hardcoded objectives, structured APIs, function calls, control loops, rule adjustments, and explicit conditionals; Level 3 uses natural language instructions, LLM understanding, tool orchestration, reasoning loops, chain-of-thought, and external budgets. Rows 7--8 compare planning (decision trees vs. LLM-generated plans) and logs (structured vs. natural language reasoning traces).}
	\label{tab:implementation-contrast}
\end{table}

\subsection{Key Distinctions}

Having traced the progression from minimal agency (3 properties) through agentic systems (6 properties) to implementation paradigms (traditional vs. AI), we can now synthesize what this hierarchy reveals. Level 1 establishes conceptual qualification, the operational definition adds production-readiness requirements, and Levels 2--3 distinguish implementation paradigms. This structure clarifies three critical distinctions that cut through definitional confusion:

The critical distinction lies between \textit{properties} and \textit{implementation}. The major jump in capability occurs between Level 1 (3 properties) and agentic systems (6 properties). Levels 2 and 3, by contrast, have \textit{identical property requirements}---they differ only in how those properties are implemented: through explicit rules and logic (Level 2) or through AI/ML models (Level 3).

This hierarchy enables precise terminology:


\begin{keybox}[title={Terminology Precision}]
	\begin{itemize}[nosep]
		\item \textbf{``Agent'' (noun):} Anything that exhibits Level 1's three minimal properties.
		\item \textbf{``Agentic'' (adjective):} At the system level, describes systems meeting all six operational properties. We may also use it descriptively at the feature/behavior level (e.g., ``agentic behavior/properties''); reserve ``agentic system'' for six-of-six conformance.
		\item \textbf{``AI agent'':} An agentic system specifically powered by AI/ML (Level 3).
	\end{itemize}
\end{keybox}

These definitions enable clear exclusions:

\begin{keybox}[title={What Doesn't Qualify}]
	\begin{itemize}[nosep]
		\item Compilers, databases → lack goals (not Level 1)
		\item Single chatbot responses → lack iteration (not operational)
		\item Traditional ML classifiers → lack iteration and goals (not agentic)
		\item Rule-based expert systems → agentic but not AI-powered (Level 2, not Level 3)
	\end{itemize}
\end{keybox}

This framework provides scaffolding for the historical and theoretical analysis that follows. Section~\ref{sec:practical} provides a practical decision rubric for immediate application. The remaining sections trace where these definitions came from and why they take this particular form, building toward the professional implications explored in Section~\ref{sec:furtherlearning}.
