% Three-level hierarchy diagram: Agent > Agentic Software > Agentic AI
% Visual representation of subset relationships

\begin{figure}[htbp]
\centering
\begin{tikzpicture}[
    font=\sffamily,
    level1/.style={fill=gray-100, draw=gray-500, line width=1.5pt},
    level2/.style={fill=gray-200, draw=gray-600, line width=1.5pt},
    level3/.style={fill=slate-100, draw=slate-700, line width=2pt},
    label/.style={font=\sffamily\bfseries},
    example/.style={font=\sffamily\small, text=gray-700}
]

% Level 1: Agents (outermost)
\draw[level1] (0,0) circle (4.5cm);
\node[label, align=center] at (0,3.8) {Level 1:\\Agents};
\node[example, align=center] at (0,-3.6) {People, thermostats,\\software systems};

% Level 2: Agentic Software - adjusted label position
\draw[level2] (0,0) circle (3cm);
\node[label, align=center] at (0,2.1) {Level 2:\\Agentic Software};
\node[example, align=center] at (0,-2.1) {Expert systems,\\rule-based agents};

% Level 3: Agentic AI (innermost)
\draw[level3] (0,0) circle (1.5cm);
\node[label, align=center] at (0,0.3) {Level 3:\\Agentic AI};
\node[example, align=center, font=\sffamily\footnotesize] at (0,-0.6) {LLM-based\\agents};

% Annotations for properties - showing cumulative hierarchy
\node[
    align=left,
    font=\sffamily\small,
    text=gray-700,
    inner sep=8pt,
    outer sep=0pt
] at (6.95,0) {
    {\footnotesize\textsc{Properties}}\\[3pt]
    {\footnotesize Level 1 (3):}\\[1pt]
    \quad Goal\\[0.5pt]
    \quad Perception\\[0.5pt]
    \quad Action\\[5pt]
    {\footnotesize Levels 2 \& 3 (+3):}\\[1pt]
    \quad Iteration\\[0.5pt]
    \quad Adaptation\\[0.5pt]
    \quad Termination
};

\end{tikzpicture}
\caption{The three-level hierarchy of agency. Each level is a subset of the one above it: all agentic AI is agentic software, and all agentic software consists of agents.}
\label{fig:hierarchy}
\end{figure}
