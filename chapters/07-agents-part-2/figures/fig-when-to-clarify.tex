% When to Clarify: Compact Stakes × Ambiguity Decision Matrix

\begin{figure}[!htb]
\centering
\begin{tikzpicture}[
    cell/.style={
        rounded corners=4pt,
        line width=1.2pt,
        minimum width=5.8cm,
        minimum height=1.8cm,
        align=center,
        inner sep=10pt
    },
    action/.style={
        font=\small\bfseries\sffamily
    },
    example/.style={
        font=\scriptsize\itshape,
        text=gray-700
    },
    axis label/.style={
        font=\scriptsize\bfseries\sffamily,
        text=gray-700
    }
]

% Column headers
\node[axis label] at (-3.1, 2.4) {Low Ambiguity};
\node[axis label] at (3.1, 2.4) {High Ambiguity};

% Row headers
\node[axis label, rotate=90, anchor=south, align=center] at (-6.5, 1.1) {Low\\Stakes};
\node[axis label, rotate=90, anchor=south, align=center] at (-6.5, -1.1) {High\\Stakes};

% Top-left: Low Stakes, Low Ambiguity
\node[cell, draw=green-600, fill=green-100] (tl) at (-3.1, 1.1) {
    \textcolor{green-900}{\textsf{\textbf{PROCEED}}}\\[2pt]
    \textcolor{gray-600}{\scriptsize\itshape ``Apple market cap'' → Apple Inc.}
};

% Top-right: Low Stakes, High Ambiguity
\node[cell, draw=amber-600, fill=amber-100] (tr) at (3.1, 1.1) {
    \textcolor{amber-900}{\textsf{\textbf{CLARIFY BRIEFLY}}}\\[2pt]
    \textcolor{gray-600}{\scriptsize\itshape ``Research SOL'' → which claim type?}
};

% Bottom-left: High Stakes, Low Ambiguity
\node[cell, draw=amber-600, fill=amber-100] (bl) at (-3.1, -1.1) {
    \textcolor{amber-900}{\textsf{\textbf{CONFIRM}}}\\[2pt]
    \textcolor{gray-600}{\scriptsize\itshape ``File this motion'' → verify before acting}
};

% Bottom-right: High Stakes, High Ambiguity
\node[cell, draw=red-600, fill=red-100] (br) at (3.1, -1.1) {
    \textcolor{red-900}{\textsf{\textbf{CLARIFY THOROUGHLY}}}\\[2pt]
    \textcolor{gray-600}{\scriptsize\itshape ``Handle the regulatory response''}
};

\end{tikzpicture}
\caption{Decision matrix for when agents should clarify user intent. Stakes measure consequence of error; ambiguity measures interpretation confidence.}
\label{fig:agents2-when-to-clarify}
\end{figure}
