% When to Clarify: Stakes × Ambiguity Decision Matrix - 2x2 Layout

\begin{figure}[!htb]
\centering
\begin{tikzpicture}[
    base box/.style={
        rounded corners=6pt,
        line width=1.5pt,
        minimum width=6.0cm,
        minimum height=3.0cm,
        align=left,
        inner sep=14pt
    },
    box header/.style={
        font=\small\bfseries\sffamily
    },
    box content/.style={
        font=\scriptsize,
        text=gray-800,
        text width=5.2cm
    },
    axis label/.style={
        font=\scriptsize\bfseries\sffamily,
        text=slate-700
    }
]

% Outer frame
\draw[rounded corners=8pt, draw=slate-700, line width=2pt, fill=slate-100]
    (-7.0,-4.2) rectangle (7.0,4.5);

% Frame title
\node[font=\normalsize\bfseries\sffamily, text=white, fill=slate-700,
      rounded corners=3pt, inner sep=6pt] at (0,4.5) {When to Clarify: Stakes × Ambiguity};

% Axis labels
\node[axis label] at (-5.0,3.6) {AMBIGUITY};
\node[axis label] at (-1.5,3.6) {Low};
\node[axis label] at (4.5,3.6) {High};

\node[axis label, rotate=90] at (-6.5,0) {STAKES};
\node[axis label, rotate=90] at (-6.5,2.2) {Low};
\node[axis label, rotate=90] at (-6.5,-2.2) {High};

% Top-left: Low Stakes, Low Ambiguity - PROCEED
\node[base box, draw=example-dark, fill=example-light] (proceed) at (-3.2,1.65) {};
\node[box header, text=example-dark, anchor=north west] at ([xshift=2pt, yshift=-2pt]proceed.north west) {PROCEED};
\node[box content, anchor=north west] at ([xshift=2pt, yshift=-18pt]proceed.north west) {
    Best interpretation is obvious; cost of error is low\\[6pt]
    \textit{Example:}\\
    ``Apple market cap'' → Apple Inc.
};

% Top-right: Low Stakes, High Ambiguity - CLARIFY BRIEFLY
\node[base box, draw=key-dark, fill=key-light] (brief) at (3.2,1.65) {};
\node[box header, text=key-dark, anchor=north west] at ([xshift=2pt, yshift=-2pt]brief.north west) {CLARIFY BRIEFLY};
\node[box content, anchor=north west] at ([xshift=2pt, yshift=-18pt]brief.north west) {
    Quick question prevents wasted effort\\[6pt]
    \textit{Example:}\\
    ``Research SOL'' → which claim type?
};

% Bottom-left: High Stakes, Low Ambiguity - CONFIRM
\node[base box, draw=key-dark, fill=key-light] (confirm) at (-3.2,-2.05) {};
\node[box header, text=key-dark, anchor=north west] at ([xshift=2pt, yshift=-2pt]confirm.north west) {CONFIRM};
\node[box content, anchor=north west] at ([xshift=2pt, yshift=-18pt]confirm.north west) {
    Clear but consequential; confirmation prevents errors\\[6pt]
    \textit{Example:}\\
    ``File this motion''
};

% Bottom-right: High Stakes, High Ambiguity - CLARIFY THOROUGHLY
\node[base box, draw=caution-dark, fill=caution-light] (thorough) at (3.2,-2.05) {};
\node[box header, text=caution-dark, anchor=north west] at ([xshift=2pt, yshift=-2pt]thorough.north west) {CLARIFY THOROUGHLY};
\node[box content, anchor=north west] at ([xshift=2pt, yshift=-18pt]thorough.north west) {
    Extended dialogue before any action\\[6pt]
    \textit{Example:}\\
    ``Handle the regulatory response''
};

\end{tikzpicture}
\caption{Decision framework for when agents should clarify user intent. Stakes measure consequence of error; ambiguity measures interpretation confidence. High-stakes or high-ambiguity tasks warrant clarification despite potential for slowing response.}
\label{fig:agents2-when-to-clarify}
\end{figure}
