% fig-orchestration-patterns.tex
% Three Multi-Agent Orchestration Patterns
% Part of: Chapter 07 - Agents Part II: How to Build an Agent
% Section: 10-Delegation

\begin{figure}[!htb]
\centering
\resizebox{\textwidth}{!}{%
\begin{tikzpicture}[
    % Card styles
    card/.style={
        rounded corners=6pt,
        line width=1.5pt,
        minimum width=5.0cm,
        minimum height=8.0cm,
        align=center,
        inner sep=0pt
    },
    card header/.style={
        font=\small\bfseries\sffamily,
        text=white,
        minimum height=0.8cm,
        text width=4.6cm,
        align=center
    },
    section label/.style={
        font=\scriptsize\bfseries,
        text=slate-900,
        anchor=west
    },
    section text/.style={
        font=\scriptsize,
        text=gray-800,
        text width=4.2cm,
        align=left
    },
    % Mini-flowchart styles (from agent-loop.tex)
    flow box/.style={
        font=\tiny,
        align=center,
        text width=1.6cm,
        rounded corners=2pt,
        inner sep=3pt,
        minimum height=1.2em,
        line width=0.8pt
    },
    flow arrow/.style={
        -stealth,
        line width=1pt,
        color=border-neutral
    }
]

% ========== LEFT CARD: SEQUENTIAL ==========
\node[card, draw=definition-dark, fill=white] (seq-card) at (-5.5, 0) {};

% Header
\node[card header, fill=definition-dark] at (seq-card.north) [yshift=-0.4cm] {SEQUENTIAL};

% Mini-flowchart (vertical chain)
\node[flow box, fill=bg-definition, draw=border-definition] (seq-coord) at (-5.5, 2.2) {Coord.};
\node[flow box, fill=bg-definition, draw=border-definition] (seq-a) at (-5.5, 1.3) {Agent A};
\node[flow box, fill=bg-definition, draw=border-definition] (seq-b) at (-5.5, 0.4) {Agent B};
\node[flow box, fill=bg-definition, draw=border-definition] (seq-c) at (-5.5, -0.5) {Agent C};
\node[flow box, fill=bg-definition, draw=border-definition] (seq-synth) at (-5.5, -1.4) {Synth\\esize};

\draw[flow arrow] (seq-coord) -- (seq-a);
\draw[flow arrow] (seq-a) -- (seq-b);
\draw[flow arrow] (seq-b) -- (seq-c);
\draw[flow arrow] (seq-c) -- (seq-synth);

% Best for section
\node[section label] at (-7.5, -2.3) {Best for:};
\node[section text, anchor=north west] at (-7.5, -2.6) {
    Tasks with dependencies between steps
};

% Trade-off section
\node[section label] at (-7.5, -3.4) {Trade-off:};
\node[section text, anchor=north west] at (-7.5, -3.7) {
    Slower; blocked by bottlenecks
};

% ========== MIDDLE CARD: PARALLEL ==========
\node[card, draw=key-dark, fill=white] (par-card) at (0, 0) {};

% Header
\node[card header, fill=key-dark] at (par-card.north) [yshift=-0.4cm] {PARALLEL};

% Mini-flowchart (fan-out, fan-in)
\node[flow box, fill=bg-key, draw=border-key] (par-coord) at (0, 2.2) {Coord.};
\node[flow box, fill=bg-key, draw=border-key] (par-a) at (-1.3, 0.9) {Agent A};
\node[flow box, fill=bg-key, draw=border-key] (par-b) at (0, 0.9) {Agent B};
\node[flow box, fill=bg-key, draw=border-key] (par-c) at (1.3, 0.9) {Agent C};
\node[flow box, fill=bg-key, draw=border-key] (par-synth) at (0, -0.4) {Synth\\esize};

\draw[flow arrow] (par-coord) -- (par-a);
\draw[flow arrow] (par-coord) -- (par-b);
\draw[flow arrow] (par-coord) -- (par-c);
\draw[flow arrow] (par-a) -- (par-synth);
\draw[flow arrow] (par-b) -- (par-synth);
\draw[flow arrow] (par-c) -- (par-synth);

% Best for section
\node[section label] at (-2.0, -2.3) {Best for:};
\node[section text, anchor=north west] at (-2.0, -2.6) {
    Independent parallel work
};

% Trade-off section
\node[section label] at (-2.0, -3.4) {Trade-off:};
\node[section text, anchor=north west] at (-2.0, -3.7) {
    Coordination overhead
};

% ========== RIGHT CARD: HIERARCHICAL ==========
\node[card, draw=example-dark, fill=white] (hier-card) at (5.5, 0) {};

% Header
\node[card header, fill=example-dark] at (hier-card.north) [yshift=-0.4cm] {HIERARCHICAL};

% Mini-flowchart (tree with sub-agents)
\node[flow box, fill=bg-example, draw=border-example] (hier-coord) at (5.5, 2.2) {Coord.};
\node[flow box, fill=bg-example, draw=border-example] (hier-a) at (5.0, 1.1) {Agent A};
\node[flow box, fill=bg-example, draw=border-example, font=\tiny, text width=1.2cm] (hier-a1) at (6.3, 1.5) {A1};
\node[flow box, fill=bg-example, draw=border-example, font=\tiny, text width=1.2cm] (hier-a2) at (6.3, 0.8) {A2};
\node[flow box, fill=bg-example, draw=border-example] (hier-b) at (5.0, -0.1) {Agent B};
\node[flow box, fill=bg-example, draw=border-example, font=\tiny, text width=1.0cm] (hier-b1) at (6.5, 0.4) {B1};
\node[flow box, fill=bg-example, draw=border-example, font=\tiny, text width=1.0cm] (hier-b2) at (6.5, -0.1) {B2};
\node[flow box, fill=bg-example, draw=border-example, font=\tiny, text width=1.0cm] (hier-b3) at (6.5, -0.6) {B3};
\node[flow box, fill=bg-example, draw=border-example] (hier-synth) at (5.5, -1.2) {Synth\\esize};

\draw[flow arrow] (hier-coord) -- (hier-a);
\draw[flow arrow] (hier-a) -- (hier-a1);
\draw[flow arrow] (hier-a) -- (hier-a2);
\draw[flow arrow] (hier-coord) -- (hier-b);
\draw[flow arrow] (hier-b) -- (hier-b1);
\draw[flow arrow] (hier-b) -- (hier-b2);
\draw[flow arrow] (hier-b) -- (hier-b3);
\draw[flow arrow] (hier-a) -- (hier-synth);
\draw[flow arrow] (hier-b) -- (hier-synth);

% Best for section
\node[section label] at (3.5, -2.3) {Best for:};
\node[section text, anchor=north west] at (3.5, -2.6) {
    Complex tasks with sub-delegation
};

% Trade-off section
\node[section label] at (3.5, -3.4) {Trade-off:};
\node[section text, anchor=north west] at (3.5, -3.7) {
    Complexity; harder to debug
};

\end{tikzpicture}
}%
\caption{Three multi-agent orchestration patterns. Sequential delegation chains agents in order, ideal for dependent tasks but vulnerable to bottlenecks. Parallel delegation runs agents concurrently, maximizing throughput for independent work but requiring coordination. Hierarchical delegation enables sub-agents to handle specialized sub-tasks, providing flexibility for complex workflows at the cost of debugging complexity.}
\label{fig:agents2-orchestration-patterns}
\end{figure}
