% fig-orchestration-patterns.tex
% Three Multi-Agent Orchestration Patterns
% Part of: Chapter 07 - Agents Part II: How to Build an Agent
% Section: 10-Delegation

\begin{figure}[!htb]
\centering
\resizebox{\textwidth}{!}{%
\begin{tikzpicture}[
    % Card styles
    card/.style={
        rounded corners=6pt,
        line width=1.5pt,
        minimum width=5.2cm,
        minimum height=8.0cm,
        align=center,
        inner sep=0pt
    },
    card header/.style={
        font=\small\bfseries,
        text=white,
        minimum height=0.8cm,
        text width=4.8cm,
        align=center,
        rounded corners=4pt
    },
    section text/.style={
        font=\scriptsize,
        text=gray-800,
        text width=4.4cm,
        align=left,
        anchor=north west
    },
    % Mini-flowchart styles (from agent-loop.tex)
    flow box/.style={
        font=\tiny,
        align=center,
        text width=1.6cm,
        rounded corners=2pt,
        inner sep=3pt,
        minimum height=1.2em,
        line width=0.8pt
    },
    flow arrow/.style={
        -stealth,
        line width=1pt,
        color=border-neutral
    }
]

% ========== LEFT CARD: SEQUENTIAL ==========
\node[card, draw=definition-dark, fill=white] (seq-card) at (-5.5, 0) {};

% Header
\node[card header, fill=definition-dark] at (seq-card.north) [yshift=0.13cm] {\textbf{Sequential}};

% Mini-flowchart (vertical chain)
\node[flow box, fill=bg-definition, draw=border-definition, text width=2.0cm] (seq-coord) at (-5.5, 3.1) {Coordinator};
\node[flow box, fill=bg-definition, draw=border-definition] (seq-a) at (-5.5, 2.2) {Agent A};
\node[flow box, fill=bg-definition, draw=border-definition] (seq-b) at (-5.5, 1.3) {Agent B};
\node[flow box, fill=bg-definition, draw=border-definition] (seq-c) at (-5.5, 0.4) {Agent C};
\node[flow box, fill=bg-definition, draw=border-definition, text width=2.0cm] (seq-synth) at (-5.5, -0.5) {Synthesize};

\draw[flow arrow] (seq-coord) -- (seq-a);
\draw[flow arrow] (seq-a) -- (seq-b);
\draw[flow arrow] (seq-b) -- (seq-c);
\draw[flow arrow] (seq-c) -- (seq-synth);

% Best for and Trade-off
\node[section text] at (-7.7, -1.6) {\textbf{Best for:} Tasks with dependencies between steps\\[0.3em]\textbf{Trade-off:} Slower; blocked by bottlenecks};

% ========== MIDDLE CARD: PARALLEL ==========
\node[card, draw=key-dark, fill=white] (par-card) at (0, 0) {};

% Header
\node[card header, fill=key-dark] at (par-card.north) [yshift=0.13cm] {\textbf{Parallel}};

% Mini-flowchart (fan-out, fan-in)
\node[flow box, fill=bg-key, draw=border-key, text width=2.0cm] (par-coord) at (0, 3.1) {Coordinator};
\node[flow box, fill=bg-key, draw=border-key, text width=1.0cm] (par-a) at (-1.3, 1.8) {Agent\\A};
\node[flow box, fill=bg-key, draw=border-key, text width=1.0cm] (par-b) at (0, 1.8) {Agent\\B};
\node[flow box, fill=bg-key, draw=border-key, text width=1.0cm] (par-c) at (1.3, 1.8) {Agent\\C};
\node[flow box, fill=bg-key, draw=border-key, text width=2.0cm] (par-synth) at (0, 0.5) {Synthesize};

\draw[flow arrow] (par-coord) -- (par-a);
\draw[flow arrow] (par-coord) -- (par-b);
\draw[flow arrow] (par-coord) -- (par-c);
\draw[flow arrow] (par-a) -- (par-synth);
\draw[flow arrow] (par-b) -- (par-synth);
\draw[flow arrow] (par-c) -- (par-synth);

% Best for and Trade-off
\node[section text] at (-2.2, -1.6) {\textbf{Best for:} Independent parallel work\\[0.3em]\textbf{Trade-off:} Coordination overhead};

% ========== RIGHT CARD: HIERARCHICAL ==========
\node[card, draw=example-dark, fill=white] (hier-card) at (5.5, 0) {};

% Header
\node[card header, fill=example-dark] at (hier-card.north) [yshift=0.13cm] {\textbf{Hierarchical}};

% Mini-flowchart (tree with sub-agents)
\node[flow box, fill=bg-example, draw=border-example, text width=2.0cm] (hier-coord) at (5.5, 3.1) {Coordinator};
\node[flow box, fill=bg-example, draw=border-example] (hier-a) at (4.4, 1.9) {Agent A};
\node[flow box, fill=bg-example, draw=border-example, font=\tiny, text width=0.8cm] (hier-a1) at (6.6, 2.2) {A1};
\node[flow box, fill=bg-example, draw=border-example, font=\tiny, text width=0.8cm] (hier-a2) at (6.6, 1.6) {A2};
\node[flow box, fill=bg-example, draw=border-example] (hier-b) at (4.4, 0.5) {Agent B};
\node[flow box, fill=bg-example, draw=border-example, font=\tiny, text width=0.8cm] (hier-b1) at (6.6, 0.9) {B1};
\node[flow box, fill=bg-example, draw=border-example, font=\tiny, text width=0.8cm] (hier-b2) at (6.6, 0.35) {B2};
\node[flow box, fill=bg-example, draw=border-example, font=\tiny, text width=0.8cm] (hier-b3) at (6.6, -0.2) {B3};
\node[flow box, fill=bg-example, draw=border-example, text width=2.0cm] (hier-synth) at (5.5, -0.9) {Synthesize};

\draw[flow arrow] (hier-coord.south west) -- (hier-a.north);
\draw[flow arrow] (hier-a.east) -- (hier-a1.west);
\draw[flow arrow] (hier-a.east) -- (hier-a2.west);
\draw[flow arrow] (hier-coord.south) -- (hier-b.north);
\draw[flow arrow] (hier-b.east) -- (hier-b1.west);
\draw[flow arrow] (hier-b.east) -- (hier-b2.west);
\draw[flow arrow] (hier-b.east) -- (hier-b3.west);
\draw[flow arrow] (hier-a.south) -- (hier-synth.north west);
\draw[flow arrow] (hier-b.south) -- (hier-synth.north);

% Best for and Trade-off
\node[section text] at (3.3, -1.6) {\textbf{Best for:} Complex tasks with sub-delegation\\[0.3em]\textbf{Trade-off:} Complexity; harder to debug};

\end{tikzpicture}
}%
\caption{Three multi-agent orchestration patterns. Sequential delegation chains agents in order, ideal for dependent tasks but vulnerable to bottlenecks. Parallel delegation runs agents concurrently, maximizing throughput for independent work but requiring coordination. Hierarchical delegation enables sub-agents to handle specialized sub-tasks, providing flexibility for complex workflows at the cost of debugging complexity.}
\label{fig:agents2-orchestration-patterns}
\end{figure}
