% fig-rag-pipeline.tex
% RAG Pipeline: Four Steps
% Part of: Chapter 07 - Agents Part II: How to Design an Agent
% Section: 06-memory (RAG)

\begin{figure}[!htb]
\centering
\resizebox{\textwidth}{!}{%
\begin{tikzpicture}[
    % Step box style
    step box/.style={
        rectangle,
        rounded corners=6pt,
        minimum width=3.0cm,
        minimum height=1.8cm,
        align=center,
        line width=1.5pt,
        font=\small\bfseries
    },
    % Input/Output box style
    io box/.style={
        rectangle,
        rounded corners=6pt,
        minimum width=2.4cm,
        minimum height=1.8cm,
        align=center,
        line width=1.5pt,
        font=\small\bfseries,
        fill=bg-key,
        draw=key-dark,
        text=key-dark
    },
    % Description style
    desc/.style={
        font=\scriptsize,
        text width=2.8cm,
        align=center,
        anchor=north
    },
    % Arrow style
    arrow/.style={
        -stealth,
        line width=2pt,
        color=gray-500
    }
]

% Input: Documents
\node[io box] (docs) at (-2.5, 0) {Documents};

% Step 1: Chunking
\node[step box, fill=bg-definition, draw=definition-dark, text=definition-dark] (chunk) at (1.5, 0) {1. Chunking};
\node[desc, text=gray-700] at (chunk.south) [yshift=-0.3cm] {Break documents into semantic units with metadata};

% Step 2: Embedding
\node[step box, fill=bg-definition, draw=definition-dark, text=definition-dark] (embed) at (5.5, 0) {2. Embedding};
\node[desc, text=gray-700] at (embed.south) [yshift=-0.3cm] {Convert each chunk into a vector encoding its meaning};

% Step 3: Retrieval
\node[step box, fill=bg-definition, draw=definition-dark, text=definition-dark] (retrieve) at (9.5, 0) {3. Retrieval};
\node[desc, text=gray-700] at (retrieve.south) [yshift=-0.3cm] {Find chunks similar to the query by comparing vectors};

% Step 4: Generation
\node[step box, fill=bg-definition, draw=definition-dark, text=definition-dark] (generate) at (13.5, 0) {4. Generation};
\node[desc, text=gray-700] at (generate.south) [yshift=-0.3cm] {Inject retrieved content into prompt for LLM response};

% Output: Answer
\node[io box] (answer) at (17.5, 0) {Answer};

% Arrows
\draw[arrow] (docs.east) -- (chunk.west);
\draw[arrow] (chunk.east) -- (embed.west);
\draw[arrow] (embed.east) -- (retrieve.west);
\draw[arrow] (retrieve.east) -- (generate.west);
\draw[arrow] (generate.east) -- (answer.west);

\end{tikzpicture}
}%
\caption{The RAG pipeline transforms documents into retrievable knowledge. Documents are chunked into semantic units, embedded as vectors, retrieved by similarity to queries, and injected into the LLM's context for generation.}
\label{fig:agents2-rag-pipeline}
\end{figure}
