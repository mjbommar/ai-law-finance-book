% ============================================================================
% 01-introduction.tex
% Introduction and Framework
% Part of: Chapter 07 - Agents Part II: How to Build an Agent
% ============================================================================

\section{Introduction}
\label{sec:agents2-intro}

The previous chapter answered: \textit{What is an agent?} This chapter answers: \textit{How do you build one?}

Moving from ``what is an agent?'' to ``how do you build one?'' changes what you need to know. The GPA+IAT framework from the previous chapter---Goals, Perception, Action, Iteration, Adaptation, Termination---gives you the vocabulary to recognize agentic systems and evaluate whether something truly qualifies as an agent \parencite{russell2020artificial,wooldridge1995intelligent}. But recognizing a system is not the same as deploying it, governing it, or knowing when a vendor is overselling it. For that, you need to understand how agents are actually built: what architectural decisions shape capabilities, what tradeoffs those decisions create, and where things go wrong.

This chapter provides that understanding through ten questions---the questions you should ask when evaluating, deploying, or governing any agentic system. Each question maps to a capability every useful agent needs, each capability involves design choices with real consequences, and those choices determine not just what the system can do but how reliably it performs, how it fails, and what controls are possible.

\begin{keybox}[title={The Core Insight}]
\textbf{Agents are not magic; they are architecture.}

Every capability that makes an agent useful---understanding what you asked, finding relevant information, taking action, remembering context, planning complex work, knowing when to stop, asking for help when stuck---corresponds to a concrete design decision. Those decisions have tradeoffs. Those tradeoffs have consequences.

You do not need to build agents yourself. But you need to know what questions to ask---because architectural decisions made at design time determine what the system can do, how reliably it performs, what can go wrong, and what controls are possible.
\end{keybox}

% ----------------------------------------------------------------------------
% The Organizational Analogy
% ----------------------------------------------------------------------------

\subsection{Agents as Professional Teams}
\label{sec:agents2-org-analogy}

The ten questions in this chapter are not arbitrary. They emerge from what agentic systems are designed to do: augment human professionals, automate routine workflows, or eventually replace entire organizational functions. Whatever the ambition, the system must handle the same work those humans and organizations currently perform. A legal research agent must do what a research associate does. A portfolio monitoring agent must do what an analyst does. The capabilities required are not determined by the technology; they are determined by the work. And the work has structure that any system---human or artificial---must accommodate. This yields the design principle that grounds the entire chapter: \textbf{an agentic system requires the same structural capabilities as a professional team}.

Consider how a law firm operates. Work arrives through defined channels: client calls, court filings, internal referrals. Associates must understand what partners actually want, not just what they literally said. Research requires access to the right databases with appropriate search strategies. Actions have consequences---filing a motion, sending a client letter---that require appropriate authorization. Institutional knowledge persists in case files and precedent databases. Complex matters decompose into workstreams with dependencies. Work products have completion criteria. Associates know when to escalate to partners. Teams coordinate across practice groups. And compliance controls ensure the whole operation stays within ethical and regulatory bounds.

A discretionary portfolio management team follows the same pattern. Market data and research flow through defined feeds. Analysts must interpret investment committee mandates. Research requires access to financial databases, company filings, and market intelligence. Trades have real-world consequences requiring compliance checks. Position history and investment theses persist across quarters. Portfolio construction decomposes into sector allocation, security selection, and risk management. Rebalancing has completion criteria. Analysts escalate to portfolio managers when positions approach limits. Teams coordinate across asset classes. And regulatory controls ensure fiduciary compliance.

The structural parallels are not coincidental. Both law firms and investment teams are \textit{cognitive work systems}---organizations that process information, make decisions, and take consequential actions under uncertainty. Agentic systems are also cognitive work systems \parencite{wang2023llmagents,rao1995bdi}. They face the same architectural challenges and require the same structural capabilities.

This mapping has practical implications. When you evaluate an agentic system, you can ask the same questions you would ask about a professional team. When you design governance for an agentic system, you can draw on the same frameworks that govern professional organizations. When you communicate with technical teams, you can use organizational language they will understand.

% ----------------------------------------------------------------------------
% The Ten Questions Framework
% ----------------------------------------------------------------------------

\subsection{The Ten Questions}
\label{sec:agents2-ten-questions}

These organizational parallels yield ten questions that any agentic system must answer---the same questions you would ask when onboarding a new professional or evaluating a team's capabilities. Table~\ref{tab:agents2-framework} maps each question to its corresponding GPA+IAT property, following the order an agent encounters them during execution.

\begin{table}[H]
\centering
\caption{From GPA+IAT properties to architectural questions}
\label{tab:agents2-framework}
\small
\begin{tabular}{p{0.13\textwidth}p{0.24\textwidth}p{0.51\textwidth}}
\toprule
\textbf{Section} & \textbf{GPA+IAT Property} & \textbf{Architectural Question} \\
\midrule
\Cref{sec:agents2-triggers} & Perception & How does the agent know when it has work to do? \\
\Cref{sec:agents2-intent} & Goal & How does the agent understand what is being asked? \\
\Cref{sec:agents2-perception} & Perception & How does the agent find things out? \\
\Cref{sec:agents2-action} & Action & How does the agent make things happen? \\
\Cref{sec:agents2-memory} & Adaptation & How does the agent remember things? \\
\Cref{sec:agents2-planning} & Goal + Iteration & How does the agent break a big job into steps? \\
\Cref{sec:agents2-termination} & Termination & How does the agent know when it is done? \\
\Cref{sec:agents2-escalation} & Termination & How does the agent know when to ask for help? \\
\Cref{sec:agents2-delegation} & Iteration & How does the agent work with other agents? \\
\Cref{sec:agents2-governance} & All & How do we keep the agent safe? \\
\bottomrule
\end{tabular}
\end{table}

Each section addresses one question through organizational analogies, architectural concepts, domain-specific considerations for law and finance, and governance implications. You can read sequentially for cumulative understanding, jump directly to whichever question matters most, or start with \Cref{sec:agents2-synthesis} if you are evaluating vendor claims. We begin with the first question: how does work reach the agent?
