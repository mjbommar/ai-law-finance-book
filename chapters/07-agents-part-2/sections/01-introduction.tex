% ============================================================================
% 01-introduction.tex
% Introduction and Framework
% Part of: Chapter 07 - Agents Part II: How to Build an Agent
% ============================================================================

\section{Introduction}
\label{sec:agents2-intro}

This chapter answers a practical question: \textit{How do you build an agentic system?}

The previous chapter established what agentic systems are through the GPA+IAT framework (Goals, Perception, Action plus Iteration, Adaptation, Termination). This chapter shows how to construct them, translating the six theoretical properties into concrete architectural choices: planning mechanisms, tool integrations, memory systems, escalation protocols, and governance controls. The treatment is conceptual rather than code-focused. You will understand how production systems are architected without needing to implement them yourself.

We organize the chapter around ten fundamental questions that every agentic system must answer. These questions map directly to organizational analogies: how work arrives (inbox and calendar), how instructions are understood (assignment memos), how information is gathered (library access), how actions are taken (filing and execution), how context is preserved (case files), how complex work is decomposed (project plans), how completion is recognized (deliverable criteria), when to escalate (going to the supervisor), how specialists coordinate (co-counsel relationships), and how safety is ensured (compliance and audit). Table~\ref{tab:agents2-framework} provides the complete mapping.

The chapter supports multiple reading strategies depending on your goals. Sequential readers should begin with the organizational analogy and ten-question framework below, then proceed through Questions 1--10 in order before concluding with the synthesis in \Cref{sec:agents2-synthesis}. Random-access readers can jump directly to any question section, as each stands alone with self-contained explanations. Practitioners evaluating vendors should focus on questions most relevant to their procurement criteria; \Cref{sec:agents2-synthesis} provides complete reference architectures with failure mode analysis that maps directly to vendor evaluation checklists.

% ----------------------------------------------------------------------------
% Agentic Systems as Organizations
% ----------------------------------------------------------------------------

\subsection{Agentic Systems as Organizations}
\label{sec:agents2-org-analogy}

An agentic system functions like a professional organization. This observation serves as more than analogy; it is a design principle. Consider how a law firm operates: it has a mission (goals), gathers intelligence through research and client intake (perception), executes work through filing and communication (action), operates in project cycles and matter lifecycles (iteration), learns from experience and adjusts strategy (adaptation), and completes engagements or escalates beyond authority (termination). The six properties that define agentic systems map directly to how professional organizations function.

A discretionary portfolio management team follows the same pattern. A mandate or prospectus sets risk/return objectives (goals); the team watches market data, research, and issuer filings (perception); it rebalances portfolios and places orders with compliance checks (action); it works in daily and quarterly review cycles (iteration); it shifts allocations after performance and risk reviews (adaptation); and it closes or cut positions, escalating to an investment committee when limits or mandates are at risk (termination). The organizational logic is identical even though the domain differs.

This mapping has practical implications. When we design an agentic system, we face the same questions that arise when designing an organization. The system requires specific capabilities, work must be divided among specialists, escalation paths must exist for problems exceeding authority, institutional knowledge must be preserved across matters, and governance controls must ensure safe operation. A partner staffing a complex transaction and an architect designing an agentic system are solving structurally similar problems.

Table~\ref{tab:agents2-framework} makes this correspondence explicit, mapping each GPA+IAT property to the operational questions it generates. The sections that follow address these questions in order, using organizational analogies throughout to ground abstract concepts in familiar professional practice.

\begin{table}[htbp]
\centering
\caption{From GPA+IAT properties to operational questions}
\label{tab:agents2-framework}
\small
\begin{tabular}{lll}
\toprule
\textbf{Section} & \textbf{GPA+IAT Property} & \textbf{Operational Question} \\
\midrule
Triggers & Perception & How does it know when it has work? \\
Intent & Goal & How does it understand what's being asked? \\
Perception & Perception & How does it find things out? \\
Action & Action & How does it make things happen? \\
Memory & Adaptation & How does it remember things? \\
Planning & Goal + Iteration & How does it break work into steps? \\
Termination & Termination & How does it know when it's done? \\
Escalation & Termination & When does it ask for help? \\
Delegation & Iteration & How does it work with other agents? \\
Governance & All & How do we keep it safe? \\
\bottomrule
\end{tabular}
\end{table}
