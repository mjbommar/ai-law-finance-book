% ============================================================================
% 01-introduction.tex
% Introduction and Context
% Part of: Chapter 07 - Agents Part II: How to Build an Agent
% ============================================================================

\section{Introduction}
\label{sec:agents2-intro}

% ----------------------------------------------------------------------------
% Opening and Context
% ----------------------------------------------------------------------------

Part I answered \textit{What is an agent?}---tracing the concept from cybernetics to modern LLM systems and introducing the GPA+IAT framework: Goal, Perception, Action, Iteration, Adaptation, Termination. These six properties distinguish genuine agentic systems from chatbots, simple automation, and marketing hype. An entity that lacks any of these properties---that cannot iterate on feedback, adapt its strategy, or recognize when to stop---is not an agentic system, regardless of what vendors claim.

Part II answers: \textit{How do you build one?}

More precisely: how do you build systems that implement all six properties? The abstract framework becomes concrete architecture. Goal becomes a planning system. Perception becomes tools for reading the environment. Action becomes tools for changing it. Iteration becomes the agent loop. Adaptation becomes memory. Termination becomes guardrails and success criteria. This chapter shows how.

Think about how a law firm or financial institution actually operates. A junior associate or analyst does not work in isolation. They have access to research databases, they maintain case files or deal books that accumulate over time, they break down complex partner assignments or portfolio manager requests into manageable tasks, and they coordinate with other professionals. The associate's effectiveness depends not just on their individual capabilities, but on the infrastructure around them: which systems they can access, what information they can retrieve from past matters, how they decompose ambiguous instructions into concrete work, and how they communicate results back up the chain.

Building an agent system mirrors building a professional organization. You must decide which tools the agent can access---like deciding whether an associate gets a Bloomberg terminal or just basic internet access. You must design memory systems that preserve context across interactions---like maintaining a case file that follows a matter through discovery, motion practice, and trial. You must implement planning mechanisms that break complex goals into steps---like a senior attorney delegating research, drafting, and review tasks. And you must establish protocols for how agents interact with each other and with humans---like the information barriers between practice groups or the escalation procedures when a junior attorney encounters something beyond their authority.

This chapter provides architectural patterns and mental models for designing these systems. We are not teaching you to program agents from scratch. Instead, we are showing you how the components fit together, what trade-offs exist, and how design choices affect capability and risk. Think of this as learning how a firm is organized---which roles exist, what each contributes, and how workflows connect them---even if you are not personally performing every task.

% ----------------------------------------------------------------------------
% Core Concepts: Tools, Memory, Planning
% ----------------------------------------------------------------------------

\begin{definitionbox}[title={\textbf{Tool}}]
A \keyterm{tool} is an interface enabling agents to interact with external systems. Tools are the agent's Westlaw subscription, Bloomberg terminal, EDGAR database access, document management system, and e-filing portal. Without tools, even the most sophisticated model can only reason about what it already knows---like an associate locked in a library with no internet.
\end{definitionbox}

\begin{definitionbox}[title={\textbf{Agent Memory}}]
\keyterm{Agent memory} stores and retrieves information across timescales. Short-term memory is the documents spread across an associate's desk during active work. Long-term memory is the firm's knowledge management system with decades of research memos. Episodic memory is the case file that tracks what happened on this specific matter. Semantic memory is the legal principles every attorney internalizes over their career.
\end{definitionbox}

\begin{definitionbox}[title={\textbf{Planning}}]
\keyterm{Planning} decomposes complex goals into achievable sub-tasks and adapts strategy based on results. When a partner assigns a vague directive like "research our exposure on the employment matter," planning is what transforms that into concrete steps: identify the jurisdiction, search relevant statutes, find case law, synthesize holdings, draft a memo. Without planning, agents thrash between uncoordinated actions like an associate who keeps running searches without a research strategy.
\end{definitionbox}

% ----------------------------------------------------------------------------
% Event-Driven Framing
% ----------------------------------------------------------------------------

\subsection{From Chat to Production: The Event-Driven Agent}
\label{sec:agents2-event-driven}

When a partner asks an associate to ``keep an eye on'' a pending regulatory filing, what does that actually mean? Not sitting at a desk refreshing the SEC's website. It means the associate configures an alert system that monitors EDGAR for the target company's Form 8-K filings and sends a notification when one appears. The associate then reviews the filing, analyzes its implications, updates the case file, and escalates material findings to the partner. That workflow is \keyterm{event-driven}: an external trigger (filing publication) initiates a sequence of perception, analysis, and action steps, bounded by clear termination conditions (analysis complete or escalation required).

Most discussions of AI agents emphasize interactive chat interfaces---the attorney who asks questions and receives answers in a conversation. That mode is valuable for exploration and ideation. But production deployment in legal and financial environments operates differently. These systems respond continuously to external triggers, not human prompts. A contract review system activates when a new draft arrives in the document management system. A portfolio compliance monitor triggers when end-of-day positions exceed mandate limits. A docket tracking system fires when a court clerk uploads a new filing. The architectural patterns, tool requirements, and evaluation methods for event-driven agents differ substantially from chat-based interaction.

\paragraph{Events as Triggers in Legal and Financial Workflows}

Consider the information flows that drive professional work. In legal practice, regulatory filings (SEC EDGAR publications, Federal Register notices) trigger compliance analysis; court docket updates trigger deadline calculations and strategy adjustments; document events (deal room uploads, contract redlines) trigger review workflows; and calendar deadlines trigger resource allocation. Financial institutions operate on similar patterns: market data events (price thresholds, volatility spikes) trigger portfolio rebalancing; position events (reconciliation failures, mandate breaches) trigger compliance reviews; regulatory deadlines trigger reporting workflows; and risk threshold exceedances trigger escalation protocols.

This operational cadence defines professional practice. Agent systems that cannot respond autonomously to such triggers remain research prototypes. The architectural components we examine in Section~\ref{sec:agents2-architecture} must support event-driven activation, and Section~\ref{sec:agents2-triggers} examines trigger channels in detail.

% ----------------------------------------------------------------------------
% From Framework to Components
% ----------------------------------------------------------------------------

\subsection{From Framework to Components}
\label{sec:agents2-framework-mapping}

Part I introduced the GPA+IAT framework as an abstract characterization of agency---six properties that distinguish agents from simple question-answering systems. Part II grounds that framework in concrete technical components. Here is how conceptual properties map to buildable architecture:

\textbf{Goal} becomes the \keyterm{planning system}. Imagine a litigation partner who assigns a junior associate the goal "prepare for summary judgment." The associate cannot simply execute that instruction directly---they must decompose it into achievable steps: review the complaint, identify each claim, research the legal standard for each, find supporting evidence in discovery, draft argument sections, cite check, format the brief. The planning system does the same work for an agent: it takes an abstract objective and produces a sequence of concrete actions.

\textbf{Perception} becomes \textit{read-only tools and retrieval}. Think of a paralegal gathering documents for a filing. They need access to the court's electronic filing system to check formatting requirements, the firm's document management system to retrieve prior filings, public databases to verify party information. Each system is a tool. The paralegal's ability to perceive the current state of the case---what has been filed, what is missing, what deadlines are approaching---depends entirely on which tools they can access and how effectively they can query them.

\textbf{Action} becomes \textit{write and execute tools}. A junior analyst at an investment bank might prepare a trading recommendation, but only a senior trader can actually execute the trade. Similarly, a portfolio manager at a hedge fund can rebalance positions, but the compliance officer must approve any trades that exceed position limits. That distinction---between analysis and execution---is the difference between read-only tools (perception) and write tools (action). Agents with write access can change the state of the world: file documents with courts, send emails to clients, execute trades, modify databases. Each write operation introduces risk and requires corresponding authorization controls.

\textbf{Iteration} becomes the \keyterm{agent loop}. Consider how discovery actually works: you serve requests, the other side produces documents, you review them and identify gaps, you serve follow-up requests, they produce more documents, you revise your theory of the case. This is iteration---a cycle of perceiving new information, reasoning about what it means, and taking the next action. Agents implement the same loop: invoke a tool to gather information, process the results, decide what to do next, invoke another tool, process those results, continue until the goal is satisfied or a termination condition is met.

\textbf{Adaptation} becomes \textit{memory systems}. A third-year associate handling their tenth employment discrimination case does not start from scratch each time. They remember which defenses succeeded in prior cases, which judges care about particular procedural details, which expert witnesses are credible. This accumulated experience makes them more effective over time. Agent memory systems serve the same function: storing past interactions, successful strategies, and domain knowledge so the agent improves with experience rather than treating every task as novel.

\textbf{Termination} becomes \keyterm{guardrails} and \keyterm{success criteria}. Imagine an associate assigned to "research the issue thoroughly" with no other guidance. When do they stop? After reading ten cases? Fifty? After three hours or three days? Without termination conditions, the associate could research indefinitely. Agents face the same problem, but worse---they can burn through API budgets or wander into irrelevant tangents far faster than humans. Termination logic specifies when to stop: goal achieved, budget exhausted, time limit reached, confidence threshold crossed, or unrecoverable error detected.

Each dimension represents a design decision. You must specify how goals decompose (planning algorithms), which tools the agent accesses (tool inventory and permissions), how memory persists (storage architecture), and what triggers termination (success criteria and guardrails). Building an agent means making explicit design choices for each component, not just connecting a language model to an API. The remaining sections detail these components, their interactions, and deployment patterns for legal and financial environments.

% ----------------------------------------------------------------------------
% Why Architecture Matters
% ----------------------------------------------------------------------------

\subsection{Why Architecture Matters}
\label{sec:agents2-why-arch}

Building an agent requires architectural thinking analogous to staffing a major matter. When a law firm takes on complex multi-district litigation, partners assemble teams with differentiated roles: senior associates for substantive strategy, junior associates for research and drafting, paralegals for document management, contract attorneys for discovery review, and outside specialists under protective orders. Each role has different access permissions, shared case files maintain common ground, and the partner sets termination conditions (settlement, budget exhaustion, or changed risk calculus).

Agent systems mirror this structure. They execute multi-step tasks requiring tool integration across systems. They maintain context through memory so each interaction builds on prior work. They adapt when initial approaches fail. They collaborate with other agents or escalate to humans when encountering issues beyond their authority. And they operate at varying autonomy levels calibrated to risk: reading public documents can be autonomous, while filing court documents or executing trades typically requires human approval.

Each capability introduces corresponding requirements. Tools need authentication and audit logging. Memory must respect privilege boundaries and information barriers. Planning requires termination conditions to prevent runaway costs. Protocols must authenticate participants and protect confidential communications.

For legal and financial applications, agents handle privileged material, material non-public information, and personally identifiable data. Security, isolation, and auditability must be designed in from the start. Retrofitting privilege protections after a confidentiality breach, or adding audit logging after a regulatory inquiry, is like building a trading platform and then trying to add wash sale controls. The cost of architectural failure in regulated environments exceeds the cost of proper initial design.

\begin{highlightbox}[title={Architecture Enables Governance}]
Part III will examine how to govern agentic systems---establishing controls, assigning accountability, and ensuring compliance. But governance presupposes architecture. You cannot audit what you did not log. You cannot enforce privilege boundaries that were never implemented. You cannot demonstrate bounded operation without termination mechanisms.

The architectural choices in this chapter are not merely technical decisions. They are the \textit{infrastructure} that makes governance possible. When a regulator asks how the compliance agent detected a breach, when opposing counsel demands production of the agent's reasoning, when a client questions why the agent recommended a particular strategy---architecture determines whether you can answer.

Professional duties are non-delegable: attorneys remain liable for AI-assisted work product, fiduciaries remain accountable for AI-informed recommendations. Part III addresses those obligations. This chapter gives you the architecture to meet them.
\end{highlightbox}

The remaining sections examine triggers and channels for how work enters the system (Section~\ref{sec:agents2-triggers}), tools, memory, and planning (Section~\ref{sec:agents2-architecture}), communication protocols (Section~\ref{sec:agents2-protocols}), evaluation methods (Section~\ref{sec:agents2-evaluation}), and integrated reference architectures (Section~\ref{sec:agents2-synthesis}).

% TODO: Add timeline figure showing November 2024-2025 developments (MCP, A2A, commercial agent frameworks, etc.)
% TODO: Add motivating example of legal AI agent workflow
