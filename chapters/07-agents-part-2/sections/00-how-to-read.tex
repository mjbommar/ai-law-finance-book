% ============================================================================
% 00-how-to-read.tex
% How to Read This Chapter
% Part of: Chapter 07 - Agents Part II: How to Build an Agent
% ============================================================================

\section*{How to Read This Chapter}
\addcontentsline{toc}{section}{How to Read This Chapter}
\label{sec:agents2-howtoread}

This chapter answers a practical question: \textit{How do you build an agentic system?}

Part I established what agentic systems are through the GPA+IAT framework (Goals, Perception, Action plus Iteration, Adaptation, Termination). This chapter shows how to construct them, translating the six theoretical properties into concrete architectural choices: planning mechanisms, tool integrations, memory systems, escalation protocols, and governance controls. The treatment is conceptual rather than code-focused. You will understand how production systems are architected without needing to implement them yourself.

We organize the chapter around ten fundamental questions that every agentic system must answer. These questions map directly to organizational analogies: how work arrives (inbox and calendar), how instructions are understood (assignment memos), how information is gathered (library access), how actions are taken (filing and execution), how context is preserved (case files), how complex work is decomposed (project plans), how completion is recognized (deliverable criteria), when to escalate (going to the supervisor), how specialists coordinate (co-counsel relationships), and how safety is ensured (compliance and audit). Table~\ref{tab:agents2-framework} in \Cref{sec:agents2-intro} provides the complete mapping.

\subsection*{Reading Paths}

The chapter supports multiple reading strategies depending on your goals. Sequential readers should begin with \Cref{sec:agents2-intro}, which establishes the organizational analogy and ten-question framework, then proceed through Questions 1--10 in order before concluding with the synthesis in \Cref{sec:agents2-synthesis}. Random-access readers can jump directly to any question section, as each stands alone with self-contained explanations. Practitioners evaluating vendors should read \Cref{sec:agents2-intro} for the framework, then focus on questions most relevant to their procurement criteria; \Cref{sec:agents2-synthesis} provides complete reference architectures with failure mode analysis that maps directly to vendor evaluation checklists.

\vspace{1em}
\noindent\textit{This chapter is Part II of three. Part I (\Cref{ch:agents-part-1}) defined what agents are. Part III (\Cref{ch:governance}) examines governance, accountability, and compliance controls.}
