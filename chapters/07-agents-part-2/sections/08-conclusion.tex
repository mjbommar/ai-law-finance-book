% ============================================================================
% 08-conclusion.tex
% Conclusion
% Part of: Chapter 07 - Agents Part II: How to Build an Agent
% ============================================================================

\section{Conclusion}
\label{sec:agents2-conclusion}

We began with a simple claim: AI agents are organized like professional teams. The associate reviewing a credit agreement needs tools (Westlaw, the precedent database), memory (prior deals, client context), planning (decompose the review into sections), protocols (how to communicate findings), and evaluation (partner review of work product). The portfolio manager monitoring client mandates needs the same components in a different domain: tools (Bloomberg, the risk engine), memory (investment research, client history), planning (coordinate monitoring, rebalancing, compliance, and risk agents), protocols (how agents share tasks and artifacts), and evaluation (continuous performance monitoring).

The case studies in Section~\ref{sec:agents2-synthesis} made this concrete. The credit facility review agent used ReAct planning to work through document sections, MCP tools to access precedent databases and legal research, episodic memory to track findings within the transaction, and three-layer evaluation to measure retrieval accuracy, analysis quality, and workflow completion. The portfolio management system used hierarchical planning to coordinate specialist agents, A2A protocol for task delegation and artifact exchange, and continuous evaluation to ensure mandate compliance.

These are not toy examples. They represent the kind of work legal and financial professionals do every day---work that can be accelerated by agents when the architecture is right.

\paragraph{What You Now Understand}

You understand that \textbf{tools} give agents the ability to interact with systems---accessing databases, running calculations, generating documents. Without tools, an agent is just a chatbot. With tools, it becomes capable of real work.

You understand that \textbf{memory} enables agents to maintain context and learn from experience---case files, precedent databases, client histories, investment research. Without memory, every interaction starts from scratch.

You understand that \textbf{planning} allows agents to decompose complex goals into manageable steps---ReAct for exploration, Plan-Execute for systematic coverage, hierarchical coordination for multi-agent workflows.

You understand that \textbf{protocols} govern how agents access tools and collaborate---MCP for standardized tool integration, A2A for agent-to-agent coordination. Protocol choice determines interoperability and audit capability.

You understand that \textbf{evaluation} measures whether agents perform at professional standards---retrieval accuracy, reasoning quality, workflow completion, security compliance. Generic benchmarks are insufficient; you need domain-specific metrics and expert review.

\paragraph{What This Lets You Do}

You can evaluate vendor claims critically. Is their ``agentic AI'' really autonomous, or just prompt engineering? Does the architecture support your workflows? What tools, memory, and planning capabilities does it actually have?

You can participate in procurement by asking the right questions. You can design governance that maps controls to architectural components. You can communicate with technical teams because you understand the system architecture. You can design deployment strategies that match your organization's risk tolerance.

\begin{keybox}[title={Architecture Enables Governance}]
You cannot govern what you do not understand. Now that you understand how agents work---their tools, memory, planning, protocols, and evaluation---you are ready to establish controls, set policies, assign accountability, and ensure compliance.
\end{keybox}

\paragraph{From Building to Governing}

Understanding how to build agent systems does not tell you how to govern them. You also need reporting structures, approval workflows, quality standards, compliance requirements, and accountability mechanisms.

Part III translates architectural understanding into governance practice. The tools become objects of approval controls. The memory systems become objects of retention policies. The planning components become objects of oversight mechanisms. The protocols become governance infrastructure. The evaluation frameworks become assurance processes.

Just as you cannot manage a law firm or investment bank without understanding how professionals work, you cannot govern AI agents without understanding their architecture. Part II gave you that understanding. Part III translates it into practice.

