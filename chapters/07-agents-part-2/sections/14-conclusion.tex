% ============================================================================
% 14-conclusion.tex
% Conclusion: From Architecture to Governance
% Part of: Chapter 07 - Agents Part II: How to Build an Agent
% ============================================================================

\section{Conclusion: From Architecture to Governance}
\label{sec:agents2-conclusion}

% ----------------------------------------------------------------------------
% The Framework Complete
% ----------------------------------------------------------------------------

This chapter opened with a claim: \textbf{agents are not magic; they are architecture}. Ten sections later, that claim should feel concrete.

You now know how work reaches an agent (triggers and channels), how instructions become goals (intent extraction), how agents gather information (perception tools) and take action (action tools with approval gates), how context persists across sessions (memory architecture), how complex work decomposes into manageable steps (planning patterns), how agents recognize completion (termination criteria), when they hand off to humans (escalation logic), how they coordinate with other agents (delegation protocols), and what controls keep them safe (governance infrastructure).

Each of these capabilities corresponds to design decisions with real tradeoffs. Richer memory improves context but increases cost and latency. More aggressive escalation improves safety but reduces autonomy. Tighter approval gates reduce risk but slow execution. There are no free lunches in agent architecture, only tradeoffs that must be calibrated to your specific context, risk tolerance, and regulatory environment.

The organizational analogy that structured this chapter is not merely pedagogical. Law firms and investment teams are cognitive work systems that have evolved sophisticated infrastructure for exactly the challenges agents face: distributing cognitive work, maintaining context, coordinating specialists, ensuring quality, and enabling oversight. When you evaluate an agentic system, you can ask the same questions you would ask about a professional team. When you design governance, you can draw on the same frameworks.

% ----------------------------------------------------------------------------
% What You Can Now Do
% ----------------------------------------------------------------------------

\subsection{What This Understanding Enables}
\label{sec:agents2-conclusion-enables}

With architectural literacy, you can:

\textbf{Evaluate vendor claims critically.} When a vendor says their agent ``handles legal research,'' you know to ask: What triggers initiate research? How does it understand the research question? What databases does it access? How does it know when research is complete? What happens when confidence is low? The ten questions provide a systematic evaluation framework.

\textbf{Participate meaningfully in procurement.} You can assess whether a proposed system meets your organization's requirements. Does it enforce matter isolation? Does it maintain audit trails? Does it integrate with your approval workflows? Can it escalate appropriately? You have vocabulary to specify requirements in terms developers understand.

\textbf{Design governance that maps to architecture.} You understand that governance is not separate from architecture---it is enabled by architecture. If you want audit trails, the system must log its reasoning. If you want approval gates, the system must pause before consequential actions. If you want confidentiality, the system must isolate matter context. You can design systems where governance is built in, not bolted on.

\textbf{Communicate with technical teams.} You can describe requirements precisely: ``I need perception tools for these three databases, action tools behind approval gates for these two operations, escalation triggers when confidence drops below threshold, and memory isolation between client matters.'' Shared vocabulary enables collaboration.

% ----------------------------------------------------------------------------
% Current Limitations
% ----------------------------------------------------------------------------

\subsection{Honest Assessment of Current Capabilities}
\label{sec:agents2-conclusion-limitations}

Architectural understanding should include honest acknowledgment of current limitations.

\textbf{The reliability cliff.} As discussed in \Cref{sec:agents2-reliability}, agents exhibit near-perfect success on tasks under four minutes but under 10\% success on tasks exceeding four hours. This is not a minor limitation---it fundamentally shapes what agents can reliably do today. Design systems that decompose work aggressively, that checkpoint progress frequently, and that escalate before reliability degrades.

\textbf{Judgment limitations.} Agents excel at retrieval, systematic execution, and pattern matching. They struggle with nuanced judgment, novel situations, and tasks requiring deep domain expertise. The most effective deployments pair agent capabilities with human judgment rather than attempting to replace it.

\textbf{Brittleness.} Production systems fail unpredictably due to API changes, authentication expiration, format variations, and edge cases that deterministic systems would handle gracefully. Build monitoring, alerting, and graceful degradation into every production deployment.

\textbf{Compounding errors.} Multi-step workflows compound error probabilities at each step. A workflow with ten steps, each 95\% reliable, succeeds only 60\% of the time end-to-end. Shorter workflows with human checkpoints outperform longer autonomous chains.

These limitations are not permanent, but they are real today. Design systems that deliver value despite limitations, not systems that assume limitations away.

% ----------------------------------------------------------------------------
% Essential Resources
% ----------------------------------------------------------------------------

\subsection{Essential Resources}
\label{sec:agents2-conclusion-resources}

For practitioners moving from concepts to deployment:

\textbf{Security fundamentals.} Implement the five controls from \Cref{sec:agents2-security-essentials} before any production deployment: input separation, output validation, least privilege, audit logging, and matter/client isolation. The OWASP LLM Top 10 provides vulnerability taxonomy; the NIST AI Risk Management Framework offers lifecycle guidance.

\textbf{Protocols and standards.} The Model Context Protocol (MCP) standardizes agent-to-tool communication and is production-ready with thousands of available servers. The Agent-to-Agent Protocol (A2A) standardizes multi-agent coordination and is maturing under the Linux Foundation.

\textbf{Research foundations.} For theoretical grounding: Xi et al.\ on agent architecture \parencite{xi2023rise}, Yao et al.\ on ReAct \parencite{yao2022react}, Park et al.\ on memory \parencite{park2023generative}. For evaluation: LegalBench for legal reasoning \parencite{guha2023legalbench}, VLAIR for legal AI performance \parencite{bommarito2025vlair}.

\textbf{Regulatory guidance.} Legal practitioners should monitor ABA ethics opinions, particularly Formal Opinion 512 on supervision \parencite{aba-formal-opinion-512}. Financial practitioners should monitor SEC guidance, FINRA communications, and prudential regulators on model risk management.

\begin{highlightbox}[title={Temporal Warning}]
Resources accurate as of late 2025 may not reflect subsequent developments. Protocol specifications evolve, regulatory frameworks develop, security vulnerabilities emerge. Verify currency before relying on any reference for production decisions.
\end{highlightbox}

% ----------------------------------------------------------------------------
% Bridge to Next Chapter
% ----------------------------------------------------------------------------

\subsection{From Architecture to Governance}
\label{sec:agents2-conclusion-bridge}

This chapter answered: \textit{How do you build an agent?}

The next chapter answers: \textit{How do you govern one?}

These questions are deeply connected. Every architectural decision in this chapter has governance implications:

\begin{itemize}[nosep]
\item Trigger logging enables audit of what initiated agent action
\item Intent extraction enables review of what the agent understood
\item Perception controls enable data governance and access restrictions
\item Action gates enable approval workflows and human oversight
\item Memory isolation enables confidentiality and privilege protection
\item Planning budgets enable bounded, predictable operation
\item Termination criteria enable completion verification
\item Escalation paths enable human intervention when needed
\item Delegation contracts enable accountability across agent teams
\end{itemize}

The relationship is not incidental. \textbf{Architecture enables governance.} If you did not architect for logging, you cannot audit. If you did not architect for approval gates, you cannot require sign-off. If you did not architect for isolation, you cannot enforce confidentiality boundaries. Governance is not bolted onto architecture after the fact; it emerges from architectural decisions made at design time.

The next chapter builds on this foundation. Where this chapter focused on \textit{capability}---what agents can do and how they do it---the next focuses on \textit{control}---ensuring agents do what they should, only what they should, and nothing they should not. The governance frameworks, oversight mechanisms, and compliance strategies in that chapter assume the architectural understanding developed here.

\vspace{1.5em}
\begin{center}
\textit{You cannot govern what you do not understand.}

\textit{You cannot control what you did not architect.}

\textit{Now you can do both.}
\end{center}
