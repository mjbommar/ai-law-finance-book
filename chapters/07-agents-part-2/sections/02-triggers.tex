% ============================================================================
% 02-triggers.tex
% Q1: How Does an Agent Know When It Has Work to Do?
% Part of: Chapter 07 - Agents Part II: How to Build an Agent
% ============================================================================

\section{How Does an Agent Know When It Has Work to Do?}
\label{sec:agents2-triggers}

% ----------------------------------------------------------------------------
% Opening: Q1 Framing and Organizational Analogy
% ----------------------------------------------------------------------------

Consider how work reaches a professional. A client calls with an urgent question, the court docket updates with a new filing, the calendar reminds you that a motion is due tomorrow, and a junior associate realizes an issue exceeds their expertise and brings it to your office. These four channels define how work enters your day: the phone, the inbox, the calendar, and escalation from colleagues.

Agentic systems operate in the same way. A system with tools, memory, and planning capabilities remains idle until work arrives; the architectural question is how tasks enter the system and what events trigger execution.

\begin{definitionbox}[title={Triggers}]
\keyterm{Triggers} are the events that start agent execution. In practice, a trigger might be a docket alert, a price crossing a threshold, a calendar deadline coming due, or an internal ``I can't proceed safely'' signal from the agent itself. Without a trigger, even a highly capable system sits idle.
\end{definitionbox}

\begin{definitionbox}[title={Channels}]
\keyterm{Channels} are how triggers reach the agent. In professional practice, four channels cover almost all work intake:

\textbf{External feeds}: The world pushes work to you (court filings, market data, regulatory updates).

\textbf{Human prompts}: People request work directly (chat, email, collaboration platforms).

\textbf{Scheduled jobs}: Time itself triggers execution (deadlines, periodic checks, end-of-day).

\textbf{Escalation events}: Internal signals that ask for human help (budget exhaustion, low confidence).
\end{definitionbox}

Before an agent can reason or act, it must first notice that work exists. Channels are the sensory apparatus of the system: the ways it becomes aware of its environment and the tasks it must accomplish—just as a lawyer cannot respond to a motion they never received.

% Four Channel Types for Agent Triggers - 2x2 Grid Visualization
% Uses semantic color palette with neutral backgrounds and colored accents

\begin{figure}[htbp]
\centering
\begin{tikzpicture}[scale=0.85, every node/.style={scale=0.85},
    % Top row card style
    channel top/.style={
        rectangle,
        minimum width=5.4cm,
        minimum height=4.8cm,
        fill=white,
        draw=gray-300,
        line width=0.5pt,
        rounded corners=3pt,
        inner sep=0pt
    },
    % Bottom row card style
    channel bottom/.style={
        rectangle,
        minimum width=5.4cm,
        minimum height=4.8cm,
        fill=white,
        draw=gray-300,
        line width=0.5pt,
        rounded corners=3pt,
        inner sep=0pt
    },
    % Header text style - larger and bold
    header/.style={
        font=\sffamily\bfseries,
        anchor=north west
    },
    % Tagline style
    tagline/.style={
        font=\sffamily\scriptsize,
        text=gray-700,
        anchor=north west
    },
    % Example text style - smaller
    examples/.style={
        font=\sffamily,
        text=gray-600,
        anchor=north west,
        text width=4.4cm,
        scale=0.85,
        transform shape
    }
]

% Define positions - center cards symmetrically around origin
\def\hgap{2.9}   % Half the horizontal gap between card centers
\def\vgap{5.3}   % Vertical gap between cards

% ============================================================
% TOP LEFT: External Feeds (Green accent)
% ============================================================
\node[channel top] (feeds) at (-\hgap, 0) {};
\fill[example-base, rounded corners=2pt]
    ([xshift=0.15cm, yshift=-0.15cm]feeds.north west)
    rectangle ([xshift=0.45cm, yshift=0.15cm]feeds.south west);
\node[header, text=example-dark] at ([xshift=0.7cm, yshift=-0.4cm]feeds.north west)
    {\textbf{External Feeds}};
\node[tagline] at ([xshift=0.7cm, yshift=-0.9cm]feeds.north west)
    {The world pushes work to you};
\node[examples] at ([xshift=0.7cm, yshift=-1.4cm]feeds.north west) {
    \textbullet~Court filings\\
    \quad\textcolor{gray-400}{CM/ECF, PACER}\\[0.5ex]
    \textbullet~Market data\\
    \quad\textcolor{gray-400}{Bloomberg, Reuters}\\[0.5ex]
    \textbullet~Regulatory updates\\
    \quad\textcolor{gray-400}{EDGAR, Federal Register}\\[0.5ex]
    \textbullet~Research alerts\\
    \quad\textcolor{gray-400}{Westlaw, Lexis}
};

% ============================================================
% TOP RIGHT: Human Prompts (Blue/Slate accent)
% ============================================================
\node[channel top] (human) at (\hgap, 0) {};  % Right card at +4.4
\fill[definition-base, rounded corners=2pt]
    ([xshift=0.15cm, yshift=-0.15cm]human.north west)
    rectangle ([xshift=0.45cm, yshift=0.15cm]human.south west);
\node[header, text=definition-dark] at ([xshift=0.7cm, yshift=-0.4cm]human.north west)
    {\textbf{Human Prompts}};
\node[tagline] at ([xshift=0.7cm, yshift=-0.9cm]human.north west)
    {People request work directly};
\node[examples] at ([xshift=0.7cm, yshift=-1.4cm]human.north west) {
    \textbullet~Chat interfaces\\
    \quad\textcolor{gray-400}{Direct queries}\\[0.5ex]
    \textbullet~Email routing\\
    \quad\textcolor{gray-400}{Forwarded questions}\\[0.5ex]
    \textbullet~Collaboration tools\\
    \quad\textcolor{gray-400}{Slack, Teams}\\[0.5ex]
    \textbullet~Voice interfaces\\
    \quad\textcolor{gray-400}{Transcribed speech}
};

% ============================================================
% BOTTOM LEFT: Scheduled Jobs (Amber accent)
% ============================================================
\node[channel bottom] (sched) at (-\hgap, -\vgap) {};
\fill[key-base, rounded corners=2pt]
    ([xshift=0.15cm, yshift=-0.15cm]sched.north west)
    rectangle ([xshift=0.45cm, yshift=0.15cm]sched.south west);
\node[header, text=key-dark] at ([xshift=0.7cm, yshift=-0.4cm]sched.north west)
    {\textbf{Scheduled Jobs}};
\node[tagline] at ([xshift=0.7cm, yshift=-0.9cm]sched.north west)
    {Time itself triggers execution};
\node[examples] at ([xshift=0.7cm, yshift=-1.4cm]sched.north west) {
    \textbullet~Calendar deadlines\\
    \quad\textcolor{gray-400}{Motion due dates}\\[0.5ex]
    \textbullet~Compliance checks\\
    \quad\textcolor{gray-400}{Nightly monitoring}\\[0.5ex]
    \textbullet~End-of-day workflows\\
    \quad\textcolor{gray-400}{P\&L, reconciliation}\\[0.5ex]
    \textbullet~Recurring reports\\
    \quad\textcolor{gray-400}{Monthly, quarterly}
};

% ============================================================
% BOTTOM RIGHT: Escalation Events (Red accent)
% ============================================================
\node[channel bottom] (esc) at (\hgap, -\vgap) {};
\fill[caution-base, rounded corners=2pt]
    ([xshift=0.15cm, yshift=-0.15cm]esc.north west)
    rectangle ([xshift=0.45cm, yshift=0.15cm]esc.south west);
\node[header, text=caution-dark] at ([xshift=0.7cm, yshift=-0.4cm]esc.north west)
    {\textbf{Escalation Events}};
\node[tagline] at ([xshift=0.7cm, yshift=-0.9cm]esc.north west)
    {Internal signals requiring intervention};
\node[examples] at ([xshift=0.7cm, yshift=-1.4cm]esc.north west) {
    \textbullet~Budget exhaustion\\
    \quad\textcolor{gray-400}{Token or cost limits}\\[0.5ex]
    \textbullet~Low confidence\\
    \quad\textcolor{gray-400}{Uncertain results}\\[0.5ex]
    \textbullet~Approval gates\\
    \quad\textcolor{gray-400}{Filings, trades}\\[0.5ex]
    \textbullet~Errors and anomalies\\
    \quad\textcolor{gray-400}{Tool failures}
};

\end{tikzpicture}
\caption{Four channel types through which work reaches agentic systems. External feeds push events from outside systems; human prompts arrive through interactive interfaces; scheduled jobs trigger on time-based conditions; and escalation events signal internal limits requiring human intervention. All channels converge on the agentic system's event router.}
\label{fig:agents2-trigger-channels}
\end{figure}


% ----------------------------------------------------------------------------
% External Feeds
% ----------------------------------------------------------------------------

\subsection{External Feeds: The World Pushes Work to You}
\label{sec:agents2-external-feeds}

External feeds deliver events from systems outside the agent's direct control. The external system pushes notifications when events occur, much like receiving service of process rather than checking the courthouse daily to see if you have been sued.

\subsubsection{Legal and Regulatory Feeds}

Court docket systems like CM/ECF send email notifications when documents are filed. An agent monitoring litigation can receive these alerts, retrieve filed documents via PACER, analyze contents, and trigger appropriate responses. When opposing counsel files a motion, the agent alerts the litigation team, extracts key arguments, searches for responsive authority, and calculates response deadlines.

The SEC's EDGAR system publishes corporate filings with API access for programmatic retrieval. For corporate counsel, EDGAR feeds trigger review workflows: when a competitor files a 10-K, an agent retrieves the filing, extracts financial data and risk factors, compares them to your company's disclosures, and flags material differences. Regulatory agencies publish rules and guidance through the Federal Register and agency websites, enabling agents to monitor for changes that affect compliance obligations.

Legal research platforms like Westlaw and Lexis offer citator alerts when monitored cases are cited, distinguished, or overruled. An agent can subscribe to these alerts, retrieve and analyze citing opinions, and notify attorneys of developments affecting their arguments.

\subsubsection{Financial Market Feeds}

Financial institutions receive real-time market data through providers like Bloomberg and Reuters. These feeds push price updates and market events continuously. A portfolio management agent can subscribe to price alerts, receive notifications when thresholds are crossed, evaluate rebalancing rules, and either execute trades within risk limits or escalate to a portfolio manager.

Position and P\&L updates cascade through financial systems: when trades execute, position systems update holdings, risk agents recalculate exposure, compliance agents check concentration limits, and reporting agents update dashboards. This architecture treats agents as event processors that consume upstream events, reason about implications, and produce downstream events.

News feeds deliver breaking headlines, earnings announcements, and sentiment analytics in machine-readable format. When material news hits, agents can retrieve the content, assess sentiment, compare to historical patterns, and alert portfolio managers if the news appears significant.

\begin{highlightbox}[title={Speed vs. Reasoning: A Critical Distinction}]
Market data arrives at millisecond granularity. LLM-based reasoning operates at second-to-minute timescales. This fundamental mismatch determines where agents add value in financial workflows.

\textbf{Agents are not suited for:} High-frequency trading, market-making, latency-sensitive execution. These domains require deterministic algorithms operating at microsecond latencies. An LLM reasoning loop, even a fast one, cannot compete.

\textbf{Agents are suited for:} Strategic portfolio decisions, investment thesis development, rebalancing analysis, compliance monitoring, research synthesis. These tasks operate on timescales of minutes to hours, where reasoning quality matters more than latency.

\textbf{The architecture pattern:} Fast deterministic systems handle real-time data capture and threshold detection. When thresholds trigger (position approaching limit, price target hit, anomaly detected), they generate events that LLM agents process. The agent's role is strategic reasoning and recommendation, not execution speed. This complements latency-sensitive pipelines: keep the microsecond path deterministic, hand off to the agent only once an alert is raised.

Match agent capabilities to task requirements. Speed-critical tasks need traditional algorithms; reasoning-critical tasks need agents.
\end{highlightbox}

\subsubsection{Integration Patterns}

External feeds reach agents through \keyterm{webhooks} (HTTP callbacks for immediate notification) or \keyterm{message queues} (durable event streams with delivery guarantees). Webhooks work well for low-volume, time-sensitive events where immediate delivery matters and occasional missed events are acceptable. Message queues provide ordering, durability, and replay capabilities essential for regulated applications requiring audit trails. In practice, many systems use both: a portfolio management system might use webhooks to receive immediate notification when a stock price crosses a stop-loss threshold, while using message queues to process daily trade confirmations that require guaranteed delivery and audit logging.

% ----------------------------------------------------------------------------
% Human Prompts as Events
% ----------------------------------------------------------------------------

\subsection{Human Prompts as Events}
\label{sec:agents2-human-events}

Human prompts feel different from external feeds because they are interactive and synchronous. However, at the architectural level a human prompt is still just another event type: the user generates an event, the agent receives it through a channel, processes it, and responds. Treating prompts this way simplifies design, because all events can flow through common routing and prioritization logic rather than requiring separate code paths for ``chat'' versus ``background'' work.

\textbf{Chat interfaces} are the most direct channel. The associate types ``Find Fifth Circuit authority on personal jurisdiction for e-commerce defendants,'' the agent searches and presents summaries, and the associate follows up with refinements. The analyst asks for revenue growth comparisons across portfolio companies, receives a table, and requests additional filtering. Chat enables iterative clarification while maintaining architectural consistency: each message is simply an event processed through the standard agent loop, with tighter latency expectations than background tasks.

\textbf{Email routing} enables agents to process work arriving through existing communication channels. A general counsel forwards a business unit's compliance question to an agent mailbox; the agent extracts the question, searches relevant guidance, and emails back an assessment. The challenge is intent classification: email bodies are unstructured and may include forwarded threads with multiple topics.

\textbf{Collaboration platforms} like Slack and Teams allow agents to appear as team members. Users @mention the agent in channels, send direct messages, or use slash commands. The litigation team discussing strategy can invoke research directly in their coordination channel. Security requires authorization checks at the agent layer, since collaboration platforms may log responses and channels may include unauthorized viewers.

\textbf{Voice interfaces} work best for short, urgent requests where typing is impractical. They introduce transcription errors (legal jargon like ``Chevron deference'' may transcribe incorrectly) and authentication challenges. High-stakes voice requests should require explicit confirmation before execution.

% ----------------------------------------------------------------------------
% Scheduled Jobs
% ----------------------------------------------------------------------------

\subsection{Scheduled Jobs: Time as Trigger}
\label{sec:agents2-scheduled}

Some work follows predictable schedules rather than arriving from external events or human prompts: end-of-day reconciliation, monthly compliance reporting, quarterly reviews, annual filings. For these recurring tasks, time itself triggers execution.

\textbf{Calendar-driven deadlines} govern legal practice. Answer the complaint within 21 days. File motions 30 days before hearings. Respond to discovery within 30 days. Agents can monitor litigation calendars, calculate deadlines accounting for court holidays, schedule reminders as deadlines approach, and escalate if work remains incomplete. Sophisticated deadline agents go further, retrieving the complaint, extracting claims, generating draft answers with standard defenses, and presenting drafts for attorney review before filing. Financial institutions face similar deadline-driven work, from SEC reporting deadlines to tax filings to contractual obligations to lenders.

\textbf{Periodic compliance checks} run even when no external event triggers review. An investment compliance agent runs nightly to check portfolios against client guidelines and flag violations. A law firm conflicts agent retrieves new docket entries, extracts party names, and checks them against the conflicts database. These scheduled checks enable continuous monitoring that would be impractical manually across thousands of matters or client accounts.

\textbf{End-of-day workflows} in financial institutions reconcile trades, calculate valuations at market close, generate P\&L reports, and prepare risk reports for the next morning. At market close, an EOD agent retrieves final prices, marks positions to market, calculates P\&L, and identifies unexplained variances. The agent then distributes reports to stakeholders. If any step fails, the agent escalates rather than proceeding with incomplete data. Law firms run similar periodic workflows, reminding attorneys to enter time, generating draft invoices at month-end, and flagging anomalies for partner review.

% ----------------------------------------------------------------------------
% Escalation Events (Brief - Full Treatment in Section 09)
% ----------------------------------------------------------------------------

\subsection{Escalation Events: When Agents Reach Their Limits}
\label{sec:agents2-escalation-brief}

The previous three channel types bring work into the agent system from outside. Escalation events operate internally: the agent generates an event signaling it has reached a limit and requires human intervention, transferring control to human decision-makers when the agent cannot proceed autonomously.

Four escalation triggers appear most frequently:

\textbf{Budget exhaustion}: The agent approaches resource limits (token consumption, iteration counts, time limits, or cost caps) and must decide whether to stop or request additional budget.

\textbf{Low confidence}: Uncertainty is too high for autonomous action. Conflicting authority, novel situations, or results that seem implausible warrant human review.

\textbf{Approval requirements}: Certain actions require explicit human authorization regardless of the agent's confidence: filing court documents, sending client communications, executing large trades.

\textbf{Errors and anomalies}: Tools fail repeatedly, data is inconsistent, or the agent detects red flags that require human investigation.

\Cref{sec:agents2-escalation} provides comprehensive treatment of when and how agents should escalate to humans.

% ----------------------------------------------------------------------------
% Event Routing and Prioritization
% ----------------------------------------------------------------------------

\subsection{Event Routing and Prioritization}
\label{sec:agents2-routing}

With events arriving from multiple channels, agents need routing and prioritization logic. A law firm routes work similarly. Client calls go to appropriate attorneys, court filings route to the litigation coordinator, research requests go to assigned associates. Agent systems implement the same pattern through a central router that receives events, examines metadata, applies routing rules, and dispatches to appropriate handlers.

\textbf{Routing rules} map event attributes to handlers. Court filing notifications for Matter 12345 route to that matter's litigation agent. SEC filings by portfolio companies route to the monitoring agent. Routing can be static (predefined rules) or dynamic (classifiers that analyze content and identify topics). For multi-agent architectures, routing determines delegation: an orchestrator receives high-level tasks, classifies them, and routes to specialist agents.

\textbf{Priority queues} implement tiered processing. Urgent events (emergency motions, margin calls) enter the high-priority queue and are processed immediately, potentially interrupting lower-priority work. Routine tasks enter standard queues. Background work (database updates, model retraining) runs when resources are idle. Priority can be rule-based (certain event types always urgent) or adaptive (priority escalates as deadlines approach).

\textbf{Temporal constraints} require processing within specific windows. Court filings have deadlines, trading must occur during market hours, EOD reports must complete before the next morning. Agents track these constraints, calculate time remaining, and escalate priority as deadlines approach.

\textbf{Overload management} prevents cascading failures when events arrive faster than processing capacity. Rate limiting caps how many events agents accept per minute, protecting downstream APIs. Backpressure signals upstream systems to slow down. Load shedding drops low-priority work to preserve capacity for critical tasks during peak demand. During a market crash, trade execution and risk calculations take precedence; routine reporting can wait.

% ----------------------------------------------------------------------------
% Diagram: Event Routing Architecture
% ----------------------------------------------------------------------------

\begin{figure}[htbp]
\centering
\begin{tikzpicture}[
  node distance=1.5cm and 2.5cm,
  box/.style={rectangle, draw=border-definition, fill=bg-definition, thick, minimum width=2.8cm, minimum height=1cm, align=center, rounded corners=2pt, font=\small},
  source/.style={rectangle, draw=border-example, fill=bg-example, thick, minimum width=2.5cm, minimum height=0.9cm, align=center, rounded corners=2pt, font=\small},
  handler/.style={rectangle, draw=border-key, fill=bg-key, thick, minimum width=2.5cm, minimum height=0.9cm, align=center, rounded corners=2pt, font=\small},
  arrow/.style={-Stealth, thick, draw=primary},
  label/.style={font=\scriptsize\itshape, text=text-secondary}
]

% Event Sources (Left Column)
\node[source] (feed) {External Feeds\\{\scriptsize Court, SEC, Markets}};
\node[source, below=0.8cm of feed] (human) {Human Prompts\\{\scriptsize Chat, Email, Slack}};
\node[source, below=0.8cm of human] (sched) {Scheduled Jobs\\{\scriptsize EOD, Deadlines}};
\node[source, below=0.8cm of sched] (esc) {Escalations\\{\scriptsize Budget, Confidence}};

% Router/Dispatcher (Center)
\node[box, right=of human, yshift=-0.9cm, minimum height=3.5cm] (router) {
  \textbf{Event Router}\\[0.3em]
  {\scriptsize Classify}\\
  {\scriptsize Route}\\
  {\scriptsize Prioritize}\\
  {\scriptsize Queue}
};

% Agent Handlers (Right Column)
\node[handler, right=of router, yshift=1.5cm] (research) {Research Agent\\{\scriptsize Legal/Financial}};
\node[handler, below=0.8cm of research] (compliance) {Compliance Agent\\{\scriptsize Checks, Alerts}};
\node[handler, below=0.8cm of compliance] (report) {Reporting Agent\\{\scriptsize EOD, Dashboards}};
\node[handler, below=0.8cm of report] (human-handler) {Human Review\\{\scriptsize High-Stakes}};

% Arrows from sources to router
\draw[arrow] (feed) -- (router) node[midway, above, label] {events};
\draw[arrow] (human) -- (router);
\draw[arrow] (sched) -- (router);
\draw[arrow] (esc) -- (router);

% Arrows from router to handlers
\draw[arrow] (router.east) -- ++(0.5,0) |- (research.west) node[near start, above, label] {high priority};
\draw[arrow] (router.east) -- ++(0.5,0) |- (compliance.west) node[near start, above, label] {normal};
\draw[arrow] (router.east) -- ++(0.5,0) |- (report.west) node[near start, above, label] {low priority};
\draw[arrow] (router.east) -- ++(0.5,0) |- (human-handler.west) node[near start, above, label] {escalation};

\end{tikzpicture}
\caption{Event routing architecture showing how events from multiple channels flow through a central router that classifies, prioritizes, and dispatches to appropriate handlers.}
\label{fig:agents2-event-routing}
\end{figure}

% ----------------------------------------------------------------------------
% Surfaces (Brief)
% ----------------------------------------------------------------------------

\subsection{Surfaces: How Users Experience Agent Systems}
\label{sec:agents2-surfaces}

The same underlying architecture can manifest through different user interfaces, or \textit{surfaces}. Understanding surfaces matters because the appropriate surface depends on the task, the user's expertise, and how the output will be used. Three primary surfaces serve different purposes. Chat surfaces suit interactive exploration, where the partner thinking through case strategy or the analyst exploring market conditions remains actively engaged, refining direction through dialogue. Automation surfaces suit continuous monitoring such as portfolio surveillance, docket tracking, and compliance alerts, where the agent works in the background and users receive outputs only when relevant. Document surfaces suit defined deliverables like research memos, due diligence reports, and client presentations, where the agent produces work products for human review and editing before distribution.

Most deployments combine these surfaces in practice: chat for ad hoc queries and exploratory thinking, automation for continuous monitoring and alerting, and document generation for formal deliverables that must be filed, sent to clients, or presented to committees. The underlying agent architecture supports all three; the surface simply determines how users encounter the system in their day-to-day work.

% ----------------------------------------------------------------------------
% Evaluating Trigger Systems
% ----------------------------------------------------------------------------

\subsection{Evaluating Trigger Systems}
\label{sec:agents2-evaluating-triggers}

When evaluating agent systems, whether building or buying, assess trigger capabilities against five criteria:

\textbf{Coverage}: Does the system receive events from all relevant sources? A litigation agent that monitors CM/ECF but not state court dockets has incomplete coverage.

\textbf{Latency}: How quickly do events reach the agent? Real-time market data requires sub-second delivery; docket alerts can tolerate minutes.

\textbf{Reliability}: What happens when feeds fail? Systems need retry logic, fallback sources, and alerting when data goes stale.

\textbf{Priority mechanisms}: Can the system distinguish urgent from routine? During a market crash or litigation crisis, the right events must reach the right handlers immediately.

\textbf{Auditability}: Is every trigger logged? When a regulator asks why the agent took action, you need a complete record of the triggering event.

% ----------------------------------------------------------------------------
% Connection to Other Questions
% ----------------------------------------------------------------------------

\subsection{From Triggers to Action}
\label{sec:agents2-triggers-to-action}

Triggers answer how work reaches the agent, but triggering is only the beginning. Once an event arrives, the agent must:

\begin{itemize}[nosep]
\item \textbf{Understand intent} (Q2, \Cref{sec:agents2-intent}): What is being asked?
\item \textbf{Perceive information} (Q3, \Cref{sec:agents2-perception}): What does the agent need to know?
\item \textbf{Take action} (Q4, \Cref{sec:agents2-action}): What should the agent do?
\item \textbf{Remember context} (Q5, \Cref{sec:agents2-memory}): What should persist across sessions?
\item \textbf{Plan execution} (Q6, \Cref{sec:agents2-planning}): How should work be decomposed?
\item \textbf{Recognize completion} (Q7, \Cref{sec:agents2-termination}): When is the task done?
\item \textbf{Escalate when needed} (Q8, \Cref{sec:agents2-escalation}): When should humans intervene?
\end{itemize}

The connection between questions is direct. An external feed delivers a court filing notification. The router classifies it as urgent litigation work and dispatches to the litigation agent. The agent retrieves case context from memory, downloads the filed document through PACER, analyzes content, searches for responsive authority, generates deadline calculations, and drafts a response strategy. At each step, the agent might escalate: low confidence in legal analysis triggers escalation to a senior litigator; filing a responsive document requires approval; approaching budget limits prompts a status update.

\Cref{sec:agents2-intent} examines the next question: once work arrives, how does the agent understand what's being asked?
