% ============================================================================
% 11-governance.tex
% Q10: How Do We Keep the Agent Safe?
% Part of: Chapter 07 - Agents Part II: How to Build an Agent
% ============================================================================

\section{How Do We Keep the Agent Safe?}
\label{sec:agents2-governance}

% ----------------------------------------------------------------------------
% Opening: Q10 Framing and Organizational Analogy
% ----------------------------------------------------------------------------

Every professional organization relies on infrastructure to ensure safety. A law firm does not rely solely on an associate's ethics; it has conflicts committees, billing review systems, and opinion committees. A bank does not rely solely on a trader's judgment; it has risk management engines, compliance monitoring, and internal audit functions. These structures do not do the work; they ensure the work is done safely.

Agentic systems require the same infrastructure. Governance is not a single question but a lens through which all other questions must be viewed. Every capability creates a governance requirement. Every architectural choice enables or constrains oversight. This section synthesizes the governance implications of the ten questions.

\begin{definitionbox}[title={Agentic System Governance}]
\keyterm{Agentic system governance} encompasses the policies, controls, and oversight mechanisms that ensure agents operate safely, ethically, and in compliance with applicable requirements.

Governance is not optional for regulated services \parencite{aba-formal-opinion-512,finra-notice-24-09}. Professional duties are non-delegable: attorneys remain liable for AI-assisted work product, and fiduciaries remain accountable for AI-informed recommendations.
\end{definitionbox}

% ----------------------------------------------------------------------------
% Architecture Enables Governance
% ----------------------------------------------------------------------------

\subsection{Architecture Enables Governance}
\label{sec:agents2-arch-enables-gov}

The architectural choices in this chapter are not merely technical decisions; they are the \textit{infrastructure} that makes governance possible.

You cannot audit what you did not log. You cannot enforce privilege boundaries that were never implemented. You cannot demonstrate bounded operation without termination mechanisms. When a regulator asks how the compliance agent detected a breach, or when opposing counsel demands the agent's reasoning, architecture determines whether you can answer.

This chapter provided the technical architecture. The following chapter details the policy obligations. \Cref{sec:agents2-conclusion-enables} maps these choices to their governance implications.

% ----------------------------------------------------------------------------
% Ten-Question Governance Mapping
% ----------------------------------------------------------------------------

\subsection{Governance Requirements by Question}
\label{sec:agents2-governance-mapping}

Each of the ten questions creates specific governance requirements:

\begin{table}[htbp]
\centering
\caption{Ten-question governance mapping}
\label{tab:agents2-governance-mapping}
\small
\begin{tabular}{clp{6cm}}
\toprule
\textbf{Q} & \textbf{Question} & \textbf{Governance Requirement} \\
\midrule
1 & Triggers & Event authorization, audit logging of all triggers \\
2 & Intent & Purpose limitation, goal alignment verification \\
3 & Perception & Data governance, access controls, provenance tracking \\
4 & Action & Actuation controls, approval gates, rollback capability \\
5 & Memory & State integrity, retention policies, isolation enforcement \\
6 & Planning & Bounded operation, resource budgets, plan validation \\
7 & Termination & Exit protocols, success criteria, graceful degradation \\
8 & Escalation & Human oversight, escalation triggers, response tracking \\
9 & Delegation & Agent identity, delegation contracts, barrier enforcement \\
10 & Governance & Meta-governance, audit architecture, compliance monitoring \\
\bottomrule
\end{tabular}
\end{table}

% ----------------------------------------------------------------------------
% Security Essentials
% ----------------------------------------------------------------------------

\subsection{Security Essentials}
\label{sec:agents2-security-essentials}

Five security controls are essential for regulated agentic systems \parencite{owasp-llm-top10,iso-iec-42001}:

\begin{keybox}[title={Security Controls for Regulated Practice}]
\begin{enumerate}[nosep]
\item \textbf{Input Separation:} Isolate user inputs from system prompts to prevent prompt injection attacks.
\item \textbf{Output Validation:} Verify agent outputs before execution to detect hallucinations and policy violations.
\item \textbf{Least Privilege:} Grant minimum necessary tool access to limit the blast radius of failures.
\item \textbf{Audit Logging:} Maintain comprehensive action logs for accountability and investigation.
\item \textbf{Matter/Client Isolation:} Enforce confidentiality boundaries to protect privileged information.
\end{enumerate}
\end{keybox}

These controls map directly to the framework: Input Separation protects Intent (Q2). Output Validation governs Action (Q4). Least Privilege limits Perception (Q3) and Action (Q4). Audit Logging enables review of Termination (Q7) and Escalation (Q8). Isolation enforces Memory (Q5) boundaries.

% ----------------------------------------------------------------------------
% Transparency and Explainability
% ----------------------------------------------------------------------------

\subsection{Transparency and Explainability}
\label{sec:agents2-transparency-preview}

Regulators and clients require explanations for agent decisions \parencite{zhong2024xai-auditing}. We define four transparency levels serving different audiences:

\textbf{Level 0 (Output Only):} The answer alone (``Suspicious transaction flagged''). Sufficient for low-stakes, high-trust contexts.

\textbf{Level 1 (Summary + Sources):} Conclusion with citations (``Flagged transaction \#45921; exceeds Rule 203(b)(1)''). Enables verification.

\textbf{Level 2 (Reasoning Outline):} Key analytical steps (``Flagged because: (1) \$150K > \$100K threshold, (2) counterparty on watchlist''). Appropriate for substantive work product.

\textbf{Level 3 (Execution Trace):} Full structured record of tool calls, retrieval, and reasoning. Essential for audit and debugging.

The architecture must capture Level 3 traces for all operations, then generate audience-appropriate summaries (Levels 0--2) on demand.

% ----------------------------------------------------------------------------
% Auditability vs. Retention
% ----------------------------------------------------------------------------

\subsection{Auditability vs. Retention}
\label{sec:agents2-retention-preview}

Tension exists between comprehensive logging (audit) and data minimization (privacy). The solution is not ``log everything forever.'' We apply four practices:

\textbf{Structured Logging} captures decisions rather than raw chain-of-thought. This enables selective retention of decision points while discarding ephemeral reasoning.

\textbf{Tiered Retention} implements purpose-specific periods:
\begin{itemize}[nosep]
    \item \textbf{Short-term (days):} Full operational logs for debugging.
    \item \textbf{Medium-term (months):} Structured decision logs for audit.
    \item \textbf{Long-term (years):} Minimal compliance archives as required by regulation.
\end{itemize}

\textbf{Redaction at Capture} applies privacy filters before logging. Once sensitive data enters logs, systematic removal is difficult.

\textbf{Legal Hold Integration} ensures retention schedules yield to preservation obligations when litigation is anticipated.

% ----------------------------------------------------------------------------
% Forward to Chapter 8
% ----------------------------------------------------------------------------

\subsection{Forward to Chapter 8}
\label{sec:agents2-forward-ch8}

This chapter answered \textit{how to build an agentic system}. The ten questions provided the architectural foundations; each created governance requirements the system must support.

The following chapter answers: \textit{how do we govern these systems?} It examines the five-layer governance stack (legal, model, system, process, culture), dimensional controls, and compliance frameworks---including Federal Reserve guidance \parencite{fed-sr11-7} and the NIST AI RMF \parencite{nist-ai-rmf}.

Architecture provides the foundation. Governance provides the controls. Together, they enable responsible deployment.