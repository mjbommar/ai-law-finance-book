% ============================================================================
% 11-governance.tex
% Governance: How Do We Keep the Agent Safe?
% Part of: Chapter 07 - Agents Part II: How to Build an Agent
% ============================================================================

\section{How Do We Keep the Agent Safe?}
\label{sec:agents2-governance}

% ----------------------------------------------------------------------------
% Opening: Q10 Framing and Organizational Analogy
% ----------------------------------------------------------------------------

Every professional organization relies on infrastructure to ensure safety. A law firm does not rely solely on an associate's ethics; it maintains conflicts committees, billing review systems, and opinion committees. A bank does not rely solely on a trader's judgment; it deploys risk management engines, compliance monitoring, and internal audit functions. These structures do not perform the work---they ensure the work is done safely.

Agentic systems require equivalent infrastructure. Governance is not a single question but a lens through which every capability must be viewed. Each architectural choice either enables or constrains oversight. This section synthesizes the governance implications across all ten framework questions.

\begin{definitionbox}[title={Agentic System Governance}]
\keyterm{Agentic system governance} encompasses the policies, controls, and oversight mechanisms that ensure agents operate safely, ethically, and in compliance with applicable requirements.

Governance is not optional for regulated services \parencite{aba-formal-opinion-512,finra-notice-24-09}. Professional duties are non-delegable: attorneys remain liable for AI-assisted work product, and fiduciaries remain accountable for AI-informed recommendations.
\end{definitionbox}

% ----------------------------------------------------------------------------
% Architecture Enables Governance
% ----------------------------------------------------------------------------

\subsection{Architecture Enables Governance}
\label{sec:agents2-arch-enables-gov}

The architectural choices in this chapter are not merely technical decisions; they are the \textit{infrastructure} that makes governance possible.

You cannot audit what you did not log. You cannot enforce privilege boundaries that were never implemented. You cannot demonstrate bounded operation without termination mechanisms. When a regulator asks how the compliance agent detected a breach, or when opposing counsel demands the agent's reasoning, architecture determines whether you can answer.

This chapter provided the technical architecture. The following chapter details the policy obligations. \Cref{sec:agents2-conclusion-enables} maps these choices to their governance implications.

% ----------------------------------------------------------------------------
% Ten-Question Governance Mapping
% ----------------------------------------------------------------------------

\subsection{Governance Requirements by Question}
\label{sec:agents2-governance-mapping}

Each of the ten questions creates specific governance requirements:

\begin{table}[htbp]
\centering
\caption{Ten-question governance mapping}
\label{tab:agents2-governance-mapping}
\small
\begin{tabular}{clp{6cm}}
\toprule
\textbf{Q} & \textbf{Question} & \textbf{Governance Requirement} \\
\midrule
1 & Triggers & Event authorization, audit logging of all triggers \\
2 & Intent & Purpose limitation, goal alignment verification \\
3 & Perception & Data governance, access controls, provenance tracking \\
4 & Action & Actuation controls, approval gates, rollback capability \\
5 & Memory & State integrity, retention policies, isolation enforcement \\
6 & Planning & Bounded operation, resource budgets, plan validation \\
7 & Termination & Exit protocols, success criteria, graceful degradation \\
8 & Escalation & Human oversight, escalation triggers, response tracking \\
9 & Delegation & Agent identity, delegation contracts, barrier enforcement \\
10 & Governance & Meta-governance, audit architecture, compliance monitoring \\
\bottomrule
\end{tabular}
\end{table}

% ----------------------------------------------------------------------------
% Security Essentials
% ----------------------------------------------------------------------------

\subsection{Security Essentials}
\label{sec:agents2-security-essentials}

Five security controls are essential for regulated agentic systems \parencite{owasp-llm-top10,iso-iec-42001}: (1) \textbf{input separation} to prevent prompt injection, (2) \textbf{output validation} to detect hallucinations and policy violations, (3) \textbf{least privilege} to limit blast radius, (4) \textbf{audit logging} for accountability, and (5) \textbf{matter/client isolation} to protect privileged information. \href{https://papers.ssrn.com/abstract=5911464}{\textit{Governing Agents}} details implementation of each control within the five-layer governance framework.

% ----------------------------------------------------------------------------
% Transparency and Explainability
% ----------------------------------------------------------------------------

\subsection{Transparency and Explainability}
\label{sec:agents2-transparency-preview}

Regulators and clients require explanations for agent decisions \parencite{zhong2024xai-auditing}. Four transparency levels serve different audiences: Level 0 (output only), Level 1 (output plus evidence), Level 2 (reasoning outline), and Level 3 (complete execution trace). Architecture must capture Level 3 for all operations, generating audience-appropriate summaries on demand. \href{https://papers.ssrn.com/abstract=5911464}{\textit{Governing Agents}} details transparency requirements.

% ----------------------------------------------------------------------------
% Auditability vs. Retention
% ----------------------------------------------------------------------------

\subsection{Auditability vs. Retention}
\label{sec:agents2-retention-preview}

Tension exists between comprehensive logging and data minimization. The solution involves structured logging (decisions rather than raw reasoning), tiered retention (days/months/years based on purpose), redaction at capture (privacy filters before logging), and legal hold integration (suspension of destruction when litigation is anticipated). \href{https://papers.ssrn.com/abstract=5911464}{\textit{Governing Agents}} provides detailed retention guidance.