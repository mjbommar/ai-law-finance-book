% ============================================================================
% 08-conclusion.tex
% Conclusion
% Part of: Chapter 07 - Agents Part II: How to Build an Agent
% ============================================================================

\section{Conclusion}
\label{sec:agents2-conclusion}

We began with a simple claim: AI agents are organized like professional teams. The associate reviewing a credit agreement needs tools (Westlaw, the precedent database), memory (prior deals, client context), planning (decompose the review into sections), protocols (how to communicate findings), and evaluation (partner review of work product). The portfolio manager monitoring client mandates needs the same components in a different domain: tools (Bloomberg, the risk engine), memory (investment research, client history), planning (coordinate monitoring, rebalancing, compliance, and risk agents), protocols (how agents share tasks and artifacts), and evaluation (continuous performance monitoring).

The reference architectures in Section~\ref{sec:agents2-synthesis} made this concrete. The credit facility review agent used ReAct planning to work through document sections, MCP tools to access precedent databases and legal research, episodic memory to track findings within the transaction, and three-layer evaluation to measure retrieval accuracy, analysis quality, and workflow completion. The portfolio management system used hierarchical planning to coordinate specialist agents, A2A protocol for task delegation and artifact exchange, and continuous evaluation to ensure mandate compliance.

But we were also honest about limitations. Current agents achieve under 10\% success on tasks exceeding four hours. Compounding errors, hallucinations in agentic loops, and brittleness at integration boundaries mean these reference architectures represent target states, not current reality. Production deployment requires human oversight, decomposition into shorter tasks, and acceptance that agents accelerate work rather than replace professional judgment.

\paragraph{What You Now Understand}

You understand that \textbf{tools} give agents the ability to interact with systems: accessing databases, running calculations, generating documents. Without tools, an agent is just a chatbot. With tools, it becomes capable of real work.

You understand that \textbf{memory} enables agents to maintain context and learn from experience: case files, precedent databases, client histories, investment research. Without memory, every interaction starts from scratch.

You understand that \textbf{planning} allows agents to decompose complex goals into manageable steps: ReAct for exploration, Plan-Execute for systematic coverage, hierarchical coordination for multi-agent workflows.

You understand that \textbf{protocols} govern how agents access tools and collaborate: MCP for standardized tool integration, A2A for agent-to-agent coordination. Protocol choice determines interoperability and audit capability.

You understand that \textbf{evaluation} measures whether agents perform at professional standards: retrieval accuracy, reasoning quality, workflow completion, security compliance. Generic benchmarks are insufficient; you need domain-specific metrics and expert review.

\paragraph{What This Lets You Do}

You can evaluate vendor claims critically. Is their ``agentic AI'' really autonomous, or just prompt engineering? Does the architecture support your workflows? What capabilities does it actually provide for perception, reasoning, and action?

You can participate in procurement by asking the right questions. You can design governance that maps controls to architectural components. You can communicate with technical teams because you understand the system architecture. You can design deployment strategies that match your organization's risk tolerance.

\begin{keybox}[title={Architecture Enables Governance}]
You cannot govern what you do not understand. Now that you understand how agents work (their components, capabilities, and failure modes), you are ready to establish controls, set policies, assign accountability, and ensure compliance.
\end{keybox}

% ----------------------------------------------------------------------------
% Governance Bridge
% ----------------------------------------------------------------------------

\subsection{From Architecture to Governance}
\label{sec:agents2-governance-bridge}

The architectural components you now understand become governance objects. When an SEC examiner reviews a compliance agent or opposing counsel demands production of an agent's reasoning, the questions they ask map directly to architecture: tool invocation logs, memory access records, planning decision points, and authorization checkpoints.

Chapter 08 takes these architectural components and turns them into enforceable controls, mapping each layer to audit trails, approval gates, risk tiers, and deployment policies. ``The AI did it'' is not a defense; governance makes accountability explicit. The architecture here makes governance possible; Chapter 08 operationalizes it.
