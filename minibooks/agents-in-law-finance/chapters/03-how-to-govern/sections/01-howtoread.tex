% ============================================================================
% How to Read This Chapter --- Chapter 3 - How to Govern an Agent
% Purpose: Navigation guide and scope setting
% Label: sec:agents3-howtoread
% ============================================================================

\section*{How to Read This Chapter}
\addcontentsline{toc}{section}{How to Read This Chapter}
\label{sec:agents3-howtoread}

This chapter translates the conceptual foundations from Chapter~1 and the tactical information from Chapter~2 into governance practice. Where Chapter~1 asked \emph{what makes a system agentic?} and Chapter~2 highlighted \emph{how to construct an agent?}, this chapter addresses \emph{how do we govern agentic systems responsibly?} Our focus is on what chief risk officers, compliance officers, general counsel, and senior leadership need to approve, monitor, and continuously improve deployments in regulated domains.

\paragraph{Conceptual Framework}

This chapter builds directly on Chapter~1's analytical framework. Chapter~1 established a three-level hierarchy: \emph{agents} (Level 1: minimal agency), \emph{agentic systems} (Level 2/3: operational readiness), and \emph{AI agents} (agentic systems powered by AI/ML). The six GPA+IAT properties defined in Chapter~1 map systematically to governance requirements. For example, autonomy and actuation scope determine oversight intensity (human-in-the-loop vs. human-in-command); entity frame and persistence drive audit logging and records retention; goal dynamics influence escalation triggers and revalidation schedules. Chapter~1 provided the taxonomy; this chapter provides the governance logic for agentic systems specifically.

\paragraph{What This Chapter Is Not}

This is not a step-by-step compliance manual, nor does it constitute legal advice. Regulatory requirements vary by jurisdiction, sector, and organizational context. Our goal is to equip you with conceptual tools (dimensional calibration, risk-based control selection, organizational accountability structures) that enable you to design governance proportionate to your risk profile. Consult qualified legal, compliance, and technical experts when implementing governance in your organization.

