% ============================================================================
% The Governance Stack---Chapter 3 - How to Govern an Agent
% Purpose: Synthesize regulatory landscape into five-layer framework
% Label: sec:agents3-governance-stack
% ============================================================================

\section{The Governance Stack}
\label{sec:agents3-governance-stack}

With dimensional calibration principles established in Section~\ref{sec:agents3-dimensional}, we now map the overlapping obligations in the regulatory landscape those controls must satisfy. Where Chapter~1 established what agents are and Chapter~2 addressed how to design them, this chapter examines how to govern them in practice. Organizations deploying agentic systems in regulated domains face a complex, overlapping web of legal and professional obligations. There is no single ``AI governance law'' that comprehensively addresses all requirements. Instead, governance emerges from the interaction of five layers: foundational law, professional and ethical obligations, sector-specific regulation, AI-specific regulation, and voluntary governance frameworks. Understanding this layered structure (and why no single layer suffices) is essential for designing governance proportionate to organizational risk.

\subsection{Five-Layer Framework}
\label{sec:agents3-five-layers}

We organize governance obligations into five layers, each building on the foundation below:

\begin{enumerate}
\item \textbf{Foundational Law}: Broadly applicable legal obligations governing data protection, discrimination, consumer protection, and contracts. This layer encompasses not only statutes (such as the General Data Protection Regulation (GDPR), Equal Credit Opportunity Act (ECOA), and state consumer protection statutes) but also tort law as well as other common law principles. Common law doctrines provide protections that predate and operate independently of statutory schemes and include:
\begin{itemize}
\item Defamation law constrains AI-generated content that makes false statements of fact about identifiable individuals.
\item Negligence principles may impose duties of care when deploying systems that foreseeably cause harm.
\item Privacy torts (intrusion upon seclusion, public disclosure of private facts) create liability independent of data protection statutes.
\end{itemize}
These legal foundations establish baselines that apply regardless of whether AI is involved.

\item \textbf{Professional and Ethical Obligations}: Duties imposed on licensed professionals (attorneys, investment advisers, auditors, accountants) by their governing bodies, bar associations, or regulatory agencies. These obligations are often more stringent than general law and impose fiduciary duties, confidentiality requirements, and competence standards.

\item \textbf{Sector-Specific Regulation}: Rules tailored to particular industries (banking, securities, insurance, healthcare) that address operational risks, supervision, and consumer protection in those domains. Examples include Federal Reserve guidance on model risk management (SR 11-7), Financial Industry Regulatory Authority (FINRA) rules on automated systems, and Public Company Accounting Oversight Board (PCAOB) auditing standards.

\item \textbf{AI-Specific Regulation}: Laws and regulations explicitly targeting artificial intelligence systems. The European Union's AI Act is the most comprehensive example, establishing risk-based requirements for high-risk AI systems. U.S. states (Colorado, California, New York City) are enacting their own AI-specific rules addressing bias, transparency, and impact assessments.

\item \textbf{Voluntary Governance Frameworks}: Standards, best practices, and certification schemes developed by standards bodies, industry groups, or government agencies. Examples include the NIST AI Risk Management Framework, ISO/IEC 42001 (AI Management Systems), COBIT for IT governance, and SOC 2 for vendor assurance. These frameworks are typically voluntary unless incorporated by reference into contracts or regulatory requirements.
\end{enumerate}

% Float placement: Change [t!] to [b!] (bottom), [h!] (here), or [htbp] (flexible)
\begin{keybox}[title={Why Layering is Necessary}]
No single framework fully satisfies all governance requirements. The EU AI Act establishes high-level risk categories but does not specify how to comply with ECOA's ``principal reasons'' standard for adverse credit decisions. NIST AI RMF provides flexible risk management guidance but does not address attorney-client privilege or auditor independence. Organizations must layer multiple frameworks, augment them with domain-specific controls, and continuously monitor regulatory developments across jurisdictions.
\end{keybox}

\begin{table}[htbp]
\centering
\small
\begin{tabular}{>{\raggedright\arraybackslash}p{4.0cm} >{\raggedright\arraybackslash}p{3.0cm} >{\raggedright\arraybackslash}p{3.8cm}}
\toprule
\textbf{Layer} & \textbf{Source} & \textbf{Key Examples} \\
\midrule
1. Foundational Law & Statutes, regulations, common law & GDPR, ECOA, tort law \\
\addlinespace
2. Professional Obligations & Bar associations, regulatory bodies & ABA Rules, fiduciary duty, AICPA \\
\addlinespace
3. Sector-Specific Regulation & Industry regulators & SR 11-7, FINRA, PCAOB \\
\addlinespace
4. AI-Specific Regulation & AI-targeted laws & EU AI Act, Colorado AI Act \\
\addlinespace
5. Voluntary Frameworks & Standards bodies & NIST AI RMF, ISO 42001, SOC 2 \\
\bottomrule
\end{tabular}
\caption{Summary of the Five-Layer Governance Framework}
\label{tab:agents3-five-layers}
\end{table}

% Five-Layer Governance Framework - 3D Layer Cake Visualization

\begin{figure}[htbp]
\centering
\begin{tikzpicture}[
    label arrow/.style={thick, black!60, ->, >=stealth},
    label text/.style={align=left, anchor=west}
]

% ================= LOCAL COLORS =================
\colorlet{cFoundation}{slate-700}
\colorlet{cProf}{green-900}
\colorlet{cSector}{teal-700}
\colorlet{cAI}{indigo-700}
\colorlet{cVoluntary}{amber-600}
\colorlet{cFrosting}{gray-800}
\colorlet{cFilling}{gray-200}

% ================= CONFIGURATION =================
\def\Rx{3.2}
\def\Ry{1.0}
\def\LayerH{0.7}
\def\FillH{0.12}
\def\CutAngA{-20}
\def\CutAngB{40}

\pgfmathsetmacro{\TotalH}{5*\LayerH + 4*\FillH}

% Pre-calculate edge points on base ellipse
\pgfmathsetmacro{\EdgeRx}{\Rx*cos(\CutAngB)}
\pgfmathsetmacro{\EdgeRy}{\Ry*sin(\CutAngB)}
\pgfmathsetmacro{\EdgeLx}{\Rx*cos(\CutAngA)}
\pgfmathsetmacro{\EdgeLy}{\Ry*sin(\CutAngA)}

% ================= DRAWING =================

% --- 1. BACK SHELL (drawn first, behind everything) ---
% Left wall (from cut edge going back around)
\draw[draw=none, fill=cFrosting]
    (\EdgeLx, \EdgeLy) -- (\EdgeLx, \EdgeLy + \TotalH)
    arc[start angle=\CutAngA, end angle=\CutAngB-360, x radius=\Rx, y radius=\Ry]
    -- (\EdgeRx, \EdgeRy)
    arc[start angle=\CutAngB, end angle=\CutAngA, x radius=\Rx, y radius=\Ry]
    -- cycle;

% --- 2. LAYER 0: Foundational Law ---
\def\yA{0}
% Cut faces (left first, then right - back to front)
\draw[draw=white!50, thin, fill=cFoundation!80!black]
    (0,\yA) -- (\EdgeLx, \EdgeLy + \yA) -- (\EdgeLx, \EdgeLy + \yA + \LayerH) -- (0, \yA + \LayerH) -- cycle;
\draw[draw=white!50, thin, fill=cFoundation]
    (0,\yA) -- (\EdgeRx, \EdgeRy + \yA) -- (\EdgeRx, \EdgeRy + \yA + \LayerH) -- (0, \yA + \LayerH) -- cycle;
% Layer top (ellipse segment)
\pgfmathsetmacro{\yAtop}{\yA + \LayerH}
\draw[draw=white!50, thin, fill=cFoundation!70!white]
    (0, \yAtop) -- (\EdgeRx, \EdgeRy + \yAtop)
    arc[start angle=\CutAngB, end angle=\CutAngA, x radius=\Rx, y radius=\Ry]
    -- cycle;
% Filling layer
\pgfmathsetmacro{\yAf}{\yA + \LayerH}
\draw[draw=none, fill=cFilling!90!gray]
    (0,\yAf) -- (\EdgeLx, \EdgeLy + \yAf) -- (\EdgeLx, \EdgeLy + \yAf + \FillH) -- (0, \yAf + \FillH) -- cycle;
\draw[draw=none, fill=cFilling]
    (0,\yAf) -- (\EdgeRx, \EdgeRy + \yAf) -- (\EdgeRx, \EdgeRy + \yAf + \FillH) -- (0, \yAf + \FillH) -- cycle;
% Filling top
\pgfmathsetmacro{\yAftop}{\yAf + \FillH}
\draw[draw=none, fill=cFilling!80!white]
    (0, \yAftop) -- (\EdgeRx, \EdgeRy + \yAftop)
    arc[start angle=\CutAngB, end angle=\CutAngA, x radius=\Rx, y radius=\Ry]
    -- cycle;
\draw[label arrow] (\EdgeRx, \EdgeRy + \yA + \LayerH/2) -- ++(1.0,0) node[label text] {\footnotesize\textbf{Foundational Law}\\[-2pt]{\scriptsize\textit{GDPR, ECOA, defamation}}};

% --- 3. LAYER 1: Professional & Ethical ---
\pgfmathsetmacro{\yB}{\LayerH + \FillH}
% Cut faces
\draw[draw=white!50, thin, fill=cProf!80!black]
    (0,\yB) -- (\EdgeLx, \EdgeLy + \yB) -- (\EdgeLx, \EdgeLy + \yB + \LayerH) -- (0, \yB + \LayerH) -- cycle;
\draw[draw=white!50, thin, fill=cProf]
    (0,\yB) -- (\EdgeRx, \EdgeRy + \yB) -- (\EdgeRx, \EdgeRy + \yB + \LayerH) -- (0, \yB + \LayerH) -- cycle;
% Layer top
\pgfmathsetmacro{\yBtop}{\yB + \LayerH}
\draw[draw=white!50, thin, fill=cProf!70!white]
    (0, \yBtop) -- (\EdgeRx, \EdgeRy + \yBtop)
    arc[start angle=\CutAngB, end angle=\CutAngA, x radius=\Rx, y radius=\Ry]
    -- cycle;
% Filling layer
\pgfmathsetmacro{\yBf}{\yB + \LayerH}
\draw[draw=none, fill=cFilling!90!gray]
    (0,\yBf) -- (\EdgeLx, \EdgeLy + \yBf) -- (\EdgeLx, \EdgeLy + \yBf + \FillH) -- (0, \yBf + \FillH) -- cycle;
\draw[draw=none, fill=cFilling]
    (0,\yBf) -- (\EdgeRx, \EdgeRy + \yBf) -- (\EdgeRx, \EdgeRy + \yBf + \FillH) -- (0, \yBf + \FillH) -- cycle;
% Filling top
\pgfmathsetmacro{\yBftop}{\yBf + \FillH}
\draw[draw=none, fill=cFilling!80!white]
    (0, \yBftop) -- (\EdgeRx, \EdgeRy + \yBftop)
    arc[start angle=\CutAngB, end angle=\CutAngA, x radius=\Rx, y radius=\Ry]
    -- cycle;
\draw[label arrow] (\EdgeRx, \EdgeRy + \yB + \LayerH/2) -- ++(1.0,0) node[label text] {\footnotesize\textbf{Professional \& Ethical}\\[-2pt]{\scriptsize\textit{Fiduciary duties, PCAOB}}};

% --- 4. LAYER 2: Sector-Specific ---
\pgfmathsetmacro{\yC}{2*(\LayerH + \FillH)}
% Cut faces
\draw[draw=white!50, thin, fill=cSector!80!black]
    (0,\yC) -- (\EdgeLx, \EdgeLy + \yC) -- (\EdgeLx, \EdgeLy + \yC + \LayerH) -- (0, \yC + \LayerH) -- cycle;
\draw[draw=white!50, thin, fill=cSector]
    (0,\yC) -- (\EdgeRx, \EdgeRy + \yC) -- (\EdgeRx, \EdgeRy + \yC + \LayerH) -- (0, \yC + \LayerH) -- cycle;
% Layer top
\pgfmathsetmacro{\yCtop}{\yC + \LayerH}
\draw[draw=white!50, thin, fill=cSector!70!white]
    (0, \yCtop) -- (\EdgeRx, \EdgeRy + \yCtop)
    arc[start angle=\CutAngB, end angle=\CutAngA, x radius=\Rx, y radius=\Ry]
    -- cycle;
% Filling layer
\pgfmathsetmacro{\yCf}{\yC + \LayerH}
\draw[draw=none, fill=cFilling!90!gray]
    (0,\yCf) -- (\EdgeLx, \EdgeLy + \yCf) -- (\EdgeLx, \EdgeLy + \yCf + \FillH) -- (0, \yCf + \FillH) -- cycle;
\draw[draw=none, fill=cFilling]
    (0,\yCf) -- (\EdgeRx, \EdgeRy + \yCf) -- (\EdgeRx, \EdgeRy + \yCf + \FillH) -- (0, \yCf + \FillH) -- cycle;
% Filling top
\pgfmathsetmacro{\yCftop}{\yCf + \FillH}
\draw[draw=none, fill=cFilling!80!white]
    (0, \yCftop) -- (\EdgeRx, \EdgeRy + \yCftop)
    arc[start angle=\CutAngB, end angle=\CutAngA, x radius=\Rx, y radius=\Ry]
    -- cycle;
\draw[label arrow] (\EdgeRx, \EdgeRy + \yC + \LayerH/2) -- ++(1.0,0) node[label text] {\footnotesize\textbf{Sector-Specific}\\[-2pt]{\scriptsize\textit{SR 11-7, FINRA rules}}};

% --- 5. LAYER 3: AI-Specific ---
\pgfmathsetmacro{\yD}{3*(\LayerH + \FillH)}
% Cut faces
\draw[draw=white!50, thin, fill=cAI!80!black]
    (0,\yD) -- (\EdgeLx, \EdgeLy + \yD) -- (\EdgeLx, \EdgeLy + \yD + \LayerH) -- (0, \yD + \LayerH) -- cycle;
\draw[draw=white!50, thin, fill=cAI]
    (0,\yD) -- (\EdgeRx, \EdgeRy + \yD) -- (\EdgeRx, \EdgeRy + \yD + \LayerH) -- (0, \yD + \LayerH) -- cycle;
% Layer top
\pgfmathsetmacro{\yDtop}{\yD + \LayerH}
\draw[draw=white!50, thin, fill=cAI!70!white]
    (0, \yDtop) -- (\EdgeRx, \EdgeRy + \yDtop)
    arc[start angle=\CutAngB, end angle=\CutAngA, x radius=\Rx, y radius=\Ry]
    -- cycle;
% Filling layer
\pgfmathsetmacro{\yDf}{\yD + \LayerH}
\draw[draw=none, fill=cFilling!90!gray]
    (0,\yDf) -- (\EdgeLx, \EdgeLy + \yDf) -- (\EdgeLx, \EdgeLy + \yDf + \FillH) -- (0, \yDf + \FillH) -- cycle;
\draw[draw=none, fill=cFilling]
    (0,\yDf) -- (\EdgeRx, \EdgeRy + \yDf) -- (\EdgeRx, \EdgeRy + \yDf + \FillH) -- (0, \yDf + \FillH) -- cycle;
% Filling top
\pgfmathsetmacro{\yDftop}{\yDf + \FillH}
\draw[draw=none, fill=cFilling!80!white]
    (0, \yDftop) -- (\EdgeRx, \EdgeRy + \yDftop)
    arc[start angle=\CutAngB, end angle=\CutAngA, x radius=\Rx, y radius=\Ry]
    -- cycle;
\draw[label arrow] (\EdgeRx, \EdgeRy + \yD + \LayerH/2) -- ++(1.0,0) node[label text] {\footnotesize\textbf{AI-Specific}\\[-2pt]{\scriptsize\textit{EU AI Act, NYC LL 144}}};

% --- 6. LAYER 4: Voluntary Frameworks ---
\pgfmathsetmacro{\yE}{4*(\LayerH + \FillH)}
% Cut faces
\draw[draw=white!50, thin, fill=cVoluntary!80!black]
    (0,\yE) -- (\EdgeLx, \EdgeLy + \yE) -- (\EdgeLx, \EdgeLy + \yE + \LayerH) -- (0, \yE + \LayerH) -- cycle;
\draw[draw=white!50, thin, fill=cVoluntary]
    (0,\yE) -- (\EdgeRx, \EdgeRy + \yE) -- (\EdgeRx, \EdgeRy + \yE + \LayerH) -- (0, \yE + \LayerH) -- cycle;
% Layer top (this becomes the lid, drawn later)
\draw[label arrow] (\EdgeRx, \EdgeRy + \yE + \LayerH/2) -- ++(1.0,0) node[label text] {\footnotesize\textbf{Voluntary Frameworks}\\[-2pt]{\scriptsize\textit{NIST AI RMF, SOC 2}}};

% --- 7. TOP LID (shifted up to top of cake) ---
\begin{scope}[yshift=\TotalH cm]
\draw[draw=none, fill=cFrosting!90!white]
    (0,0) -- (\EdgeRx, \EdgeRy)
    arc[start angle=\CutAngB, end angle=\CutAngA-360, x radius=\Rx, y radius=\Ry]
    -- cycle;
\end{scope}

% --- 8. FINISHES ---
\draw[thick, cFrosting!50!black] (0,0) -- (0,\TotalH);

\shade[inner color=black!30, outer color=white, opacity=0.3]
    (0,-0.1) ellipse ({\Rx+0.2} and {\Ry+0.2});

\draw[->, thick, black!60] ({-\Rx-0.5}, 0) -- ({-\Rx-0.5}, \TotalH)
    node[midway, left, rotate=90] {\scriptsize Increasing specificity};

\end{tikzpicture}
\caption{The five-layer governance framework visualized as stacked obligations. Each layer builds on the foundation below: foundational law (statutes and common law) provides the broadest baseline; professional duties add licensed-practice requirements; sector-specific rules address industry contexts; AI-specific regulation targets algorithmic systems directly; and voluntary frameworks offer certification and best-practice guidance. Organizations must address all applicable layers---no single layer provides complete coverage.}
\label{fig:agents3-five-layer-framework}
\end{figure}


The remainder of this section examines each layer in detail, identifying key requirements and illustrating how obligations interact.

\subsection{Layer 1: Foundational Law}
\label{sec:agents3-foundational-law}

Foundational law provides the baseline for governance, applicable to all organizations regardless of industry or use case. Three domains are especially relevant:

\paragraph{Data Protection and Privacy: GDPR Article 22 and Stateful Agentic Systems}
The GDPR establishes rights and obligations for processing personal data of EU residents. Article 22 addresses automated decision-making: individuals have the right not to be subject to decisions based solely on automated processing that produce legal or similarly significant effects \parencite{gdpr-article-22}. While not an absolute prohibition (automated decisions are permitted with explicit consent, contractual necessity, or legal authorization), Article 22 requires organizations to implement ``suitable measures'' to safeguard the data subject's rights, including the right to obtain human intervention and contest the decision.

\textbf{Agentic-Specific Challenge: Stateful Decision Accumulation}: Article 22's human intervention requirement becomes complex for stateful agentic systems that accumulate context across multiple cycles. Generic AI guidance suggests ``add a button for human review,'' but this is insufficient for agentic systems. Building on Chapter~1's analysis of iteration and adaptation, meaningful human intervention requires access to the system's accumulated internal state: how the agent's understanding evolved across iterations, what adaptations occurred, and what termination logic triggered the final decision. Without comprehensive state logging (capturing perception, action, and rationale at each cycle), human reviewers cannot meaningfully intervene or contest decisions because they lack visibility into how the agent reached its conclusion.

\emph{Governance Implication}: For agentic systems subject to GDPR Article 22, human intervention controls must be paired with state logging requirements (cross-cycle audit trails). Organizations cannot satisfy Article 22 by providing post-hoc review without the ability to reconstruct the agent's iterative decision process. This links directly to the State Logging control discussed in Section~\ref{sec:agents3-audit-logging}.

Article 32 requires appropriate technical and organizational security measures, including encryption, pseudonymization, and resilience against unauthorized processing. Articles 33-34 mandate breach notification to supervisory authorities (within 72 hours) and affected individuals (without undue delay) when breaches pose risks to rights and freedoms.

For organizations operating globally, GDPR compliance often sets the de facto standard. Even organizations without EU operations may face GDPR obligations if they offer services to EU residents or monitor their behavior.

\paragraph{Anti-Discrimination and Fair Lending: ECOA and Process-Based Discrimination}
The Equal Credit Opportunity Act (ECOA) prohibits credit discrimination based on race, color, religion, national origin, sex, marital status, age, or receipt of public assistance \parencite{ecoa}. Regulation B, which implements ECOA, requires creditors to provide ``principal reasons'' for adverse credit decisions \parencite{ecoa-reg-b}. This explainability requirement is more specific than generic AI transparency guidance; applicants must receive concrete, understandable reasons tied to the factors that most significantly influenced the decision.

ECOA applies regardless of whether the decision was made by a human, an algorithm, or a hybrid system. Courts have applied \emph{disparate impact} theory: even facially neutral criteria can violate ECOA if they disproportionately harm protected classes without adequate business justification.

\textbf{Agentic-Specific Challenge: Process-Based Discrimination}: Traditional fairness testing focuses on \emph{outcome parity} (do protected classes receive approvals at comparable rates?). For agentic credit underwriting systems that iteratively investigate applications across multiple cycles, discrimination can emerge through the \emph{investigation process} itself, not just final decisions. An agentic system might:
\begin{itemize}
\item Adapt to request more verification cycles from applicants with characteristics correlated with protected classes (e.g., shorter U.S. employment tenure as a proxy for national origin).
\item Impose higher process burdens (more documentation requests, longer investigation timelines) on protected groups, causing application abandonment even if the system would ultimately approve.
\item Learn patterns that create disparate \emph{iteration counts} across demographic groups, violating ECOA even when final approval rates satisfy the 80\% rule commonly used in disparate impact analysis.
\end{itemize}

\emph{Governance Implication}: Agentic credit systems require \emph{process parity monitoring} in addition to outcome fairness testing. Organizations must audit:
\begin{itemize}
\item Average investigation cycles by protected class (flag if deviation >20\%).
\item Termination reasons by demographic group (ensure similar rates of confidence-based vs. timeout-based termination).
\item Application abandonment rates during multi-cycle investigation (ensure verification burdens do not disproportionately affect protected classes).
\end{itemize}

Systems require controls ensuring that dynamic investigation strategies do not introduce prohibited discrimination, a challenge that does not arise with single-shot credit scoring models. See Section~\ref{sec:agents3-monitoring-incident} for a worked example of process-based discrimination detected in agentic underwriting.

\paragraph{Consumer Protection}
State consumer protection statutes (e.g., Unfair and Deceptive Acts and Practices laws) prohibit misleading representations and unfair business practices. If an AI chatbot misrepresents its capabilities, provides inaccurate information, or fails to disclose material limitations, the organization may face consumer protection enforcement regardless of whether the misrepresentation was intentional or the result of a model hallucination.

\subsection{Layer 2: Professional and Ethical Obligations}
\label{sec:agents3-professional-obligations}

Licensed professionals face heightened obligations that governance systems must operationalize. We examine three domains:

\paragraph{Legal Practice: ABA Model Rules}
Most U.S. jurisdictions have adopted versions of the American Bar Association's Model Rules of Professional Conduct. Three rules are especially salient for AI governance:

\begin{itemize}
\item \textbf{Rule 1.1 (Competence)}: An attorney must provide competent representation, requiring ``the legal knowledge, skill, thoroughness and preparation reasonably necessary for the representation.'' ABA Formal Opinion 512 (July 2024) clarifies that competence includes understanding AI tools' capabilities and limitations \parencite{aba-formal-opinion-512}; attorneys cannot delegate legal analysis to AI without independent verification.

\item \textbf{Rule 1.6 (Confidentiality)}: An attorney must not reveal information relating to representation of a client unless the client gives informed consent. Using AI tools that transmit client data to third-party vendors, train models on client information, or store data insecurely may violate confidentiality obligations. Organizations must assess vendor data handling practices and obtain client consent where necessary.

\item \textbf{Rule 3.3 (Candor Toward the Tribunal)}: An attorney must not knowingly make false statements of fact or law to a tribunal. Submitting AI-generated legal research without verification can result in fabricated citations or misrepresented holdings, which violates this duty. Courts have imposed sanctions ranging from \$1,000 to \$15,000 for such violations, with some attorneys facing suspension or mandatory disclosure requirements on future filings.
\end{itemize}

Rule 5.3 addresses supervision of nonlawyer assistants, a framework some jurisdictions apply to AI tools. The attorney remains responsible for ensuring that AI-assisted work product meets professional standards.

\paragraph{Financial Services: Fiduciary Duty and Suitability}
\glsadd{fiduciary-duty}Investment advisers owe fiduciary duties to clients under the Investment Advisers Act of 1940 and SEC interpretations. This duty has two components:

\begin{itemize}
\item \textbf{Duty of Care}: Providing advice that is suitable for the client's financial situation, investment objectives, and risk tolerance. An AI-generated portfolio recommendation must be validated against these client-specific factors.

\item \textbf{Duty of Loyalty}: Acting in the client's best interest, including full disclosure of conflicts of interest. If an AI tool is provided by an affiliate, receives compensation from recommended products, or prioritizes firm profitability over client outcomes, the adviser must disclose these conflicts and ensure recommendations remain in the client's best interest.
\end{itemize}

FINRA (Financial Industry Regulatory Authority) Rule 2111 imposes a similar suitability obligation on broker-dealers. Rule 3110 requires firms to supervise associated persons and establish procedures to ensure compliance. For AI systems that generate investment recommendations or execute trades, governance must include supervisory review procedures, monitoring for suitability violations, and escalation protocols.

\paragraph{Audit and Accounting: Independence and Skepticism}
The AICPA Code of Professional Conduct and PCAOB auditing standards impose strict independence and competence requirements on auditors:

\begin{itemize}
\item \textbf{Independence}: Auditors must maintain both independence in fact and independence in appearance. If an AI tool is provided by the audit client, an affiliate, or a vendor with financial ties to the client, independence may be impaired; the SEC and PCAOB closely scrutinize auditor-provided tools that could create management decision-making or self-review threats.

\item \textbf{Professional Skepticism}: PCAOB Auditing Standard 1015 requires auditors to exercise professional skepticism: a questioning mind and critical assessment of audit evidence. Auditors cannot accept AI outputs uncritically; they must understand the tool's methodology, validate its logic, and assess whether results are consistent with other evidence.

\item \textbf{Documentation}: AS 1215 (Audit Documentation) requires auditors to document the nature, timing, extent, and results of audit procedures. If AI is used for sampling, risk assessment, or analytical procedures, the workpapers must explain the tool's logic, parameters, and the auditor's rationale for relying on its output \parencite{pcaob-as1215}.
\end{itemize}

These professional obligations are non-delegable. Governance systems must operationalize competence, confidentiality, independence, and documentation requirements through technical controls and organizational processes.

\subsection{Layer 3: Sector-Specific Regulation}
\label{sec:agents3-sector-regulation}

Sector regulators impose industry-tailored requirements that general frameworks do not address:

\paragraph{Banking: Model Risk Management}
The Federal Reserve, Office of the Comptroller of the Currency (OCC), and Federal Deposit Insurance Corporation (FDIC) issued SR 11-7 on model risk management \parencite{fed-sr11-7}. SR 11-7 applies broadly to model risk management in banking. Key requirements include:

\begin{itemize}
\item \textbf{Model Inventory}: Maintain a comprehensive inventory of models, classified by risk.
\item \textbf{Independent Validation}: Models must be validated by a function independent of the model's development and use. Validation includes conceptual soundness review, ongoing monitoring, and outcomes analysis.
\item \textbf{Model Governance}: Establish board and senior management oversight, clear roles and responsibilities, and policies for model development, implementation, and use.
\item \textbf{Documentation}: Maintain complete documentation of model logic, data sources, assumptions, limitations, and validation.
\end{itemize}

\textbf{Application to Agentic Systems}: For agentic systems deployed in banking (e.g., iterative credit underwriting agents that gather information across multiple cycles, adapt criteria based on discovered patterns, and escalate edge cases), SR 11-7 requires documentation of all six operational properties. Unlike traditional one-shot credit models that execute fixed logic, agentic systems must document iteration logic (when does the system gather more data?), adaptation mechanisms (how do criteria evolve?), and termination conditions (when does it escalate to humans?).

\paragraph{Securities: FINRA Supervision and Algorithmic Trading}
FINRA Rule 3110 requires broker-dealers to establish supervisory systems reasonably designed to achieve compliance with applicable laws and regulations. For firms using algorithmic trading systems or AI-driven investment recommendations, this means:

\begin{itemize}
\item \textbf{Pre-Deployment Testing}: Validate algorithms in a controlled environment before production use.
\item \textbf{Ongoing Monitoring}: Continuously monitor for erroneous or manipulative behavior.
\item \textbf{Risk Controls}: Implement automated controls (e.g., price collars, volume limits) to prevent runaway algorithms.
\item \textbf{Supervisory Review}: Designate supervisors responsible for algorithm oversight and establish escalation procedures.
\end{itemize}

\paragraph{Audit: PCAOB Standards on Audit Evidence and Sampling}
The PCAOB has not issued AI-specific guidance, but existing auditing standards apply. AS 1105 (Audit Evidence) establishes that the auditor is responsible for all audit evidence, regardless of source.

\textbf{Application to Agentic Systems}: For agentic audit systems (e.g., agents that iteratively refine sampling strategies based on discovered anomalies, adapt risk assessments as they review documentation, and terminate when coverage objectives are met), PCAOB standards require:

\begin{itemize}
\item Documentation of iteration logic: How does the system refine its sampling or analysis strategy across cycles?
\item Documentation of adaptation mechanisms: When and why does the system adjust risk assessments or expand sample sizes?
\item Documentation of termination criteria: What triggers the system to conclude its work or escalate to human auditors?
\item Understanding the tool's methodology and assumptions across all six GPA+IAT properties.
\item Professional skepticism maintained throughout; auditors cannot delegate professional judgment to autonomous systems.
\end{itemize}

AS 2315 (Audit Sampling) requires auditors to design samples that provide a reasonable basis for conclusions. Agentic sampling systems must document how iteration and adaptation enhance (rather than compromise) statistical validity.

\subsection{Layer 4: AI-Specific Regulation}
\label{sec:agents3-ai-regulation}

AI-specific regulation is emerging rapidly. We focus on the most comprehensive framework and notable U.S. developments:

\paragraph{EU AI Act: Risk-Based Tiering}
The EU AI Act, which entered into force in August 2024, establishes a risk-based regulatory framework \parencite{eu-ai-act-2024}:

\begin{itemize}
\item \textbf{Prohibited Practices}: AI systems that pose unacceptable risks (e.g., social scoring by governments, real-time biometric identification in public spaces except narrow law enforcement exceptions, manipulative or harmful systems) are banned.

\item \textbf{High-Risk Systems}: AI systems used in employment, education, credit assessment, law enforcement, critical infrastructure, and biometric identification are classified as high-risk. These systems must satisfy stringent requirements:
  \begin{itemize}
  \item \textbf{Risk Management} (Article 9): Establish and maintain a risk management system throughout the AI system's lifecycle.
  \item \textbf{Data Governance} (Article 10): Training, validation, and testing datasets must be relevant, representative, and free from bias to the extent possible.
  \item \textbf{Logging} (Article 12): Maintain automatic recording of events (logs) to enable traceability.
  \item \textbf{Transparency} (Article 13): Provide clear instructions for use, including capabilities, limitations, and expected performance.
  \item \textbf{Human Oversight} (Article 14): Design systems to enable effective oversight, including the ability to override or interrupt the system.
  \item \textbf{Accuracy, Robustness, Cybersecurity} (Article 15): Achieve appropriate levels of accuracy and resilience against errors, faults, and cyberattacks.
  \item \textbf{Conformity Assessment} (Article 43): High-risk systems must undergo third-party conformity assessment before market placement (for certain categories) or internal assessment (for others).
  \end{itemize}

\item \textbf{Limited-Risk and Minimal-Risk Systems}: Lower-risk systems face transparency obligations (e.g., chatbots must disclose they are AI) but not the full high-risk requirements.
\end{itemize}

Penalties for non-compliance are severe: up to €35~million or 7\% of global annual turnover for prohibited practices, and up to €15~million or 3\% for high-risk system violations. Organizations operating in or serving EU markets must assess whether their agentic systems fall within high-risk categories and implement Article 9--15 requirements.

\paragraph{U.S. State and Local AI Laws}
In the absence of comprehensive federal AI legislation, U.S. states and cities are enacting targeted rules:

\begin{itemize}
\item \textbf{Colorado AI Act (SB 24-205)}: Effective January 1, 2026, Colorado's law prohibits algorithmic discrimination: deployment of high-risk AI systems that result in unlawful differential treatment or impact based on protected classifications \parencite{colorado-ai-act}. Deployers must conduct impact assessments documenting the system's purpose, data sources, intended benefits, known limitations, and measures to mitigate discrimination. A rebuttable presumption of compliance applies if deployers complete a reasonable impact assessment in good faith.

\item \textbf{New York City Local Law 144 (Automated Employment Decision Tools)}: Effective since July 2023, NYC requires employers using AI for hiring or promotion to conduct annual bias audits, publish summary results, and notify candidates that an automated tool is in use. Employers must also allow candidates to request alternative evaluation processes.

\item \textbf{California Privacy Rights Act (CPRA) and Proposed AI Legislation}: California has enacted data protection laws that indirectly regulate AI (e.g., CPRA's provisions on automated decision-making) and is considering comprehensive AI legislation addressing high-risk uses.
\end{itemize}

These patchwork requirements mean organizations must track regulatory developments across jurisdictions and tailor governance to the most stringent applicable standard.

\subsection{Layer 5: Voluntary Governance Frameworks}
\label{sec:agents3-frameworks}

Voluntary frameworks provide structured approaches to AI governance. Organizations often adopt multiple frameworks to address different audiences and objectives:

\paragraph{NIST AI Risk Management Framework (AI RMF 1.0)}
Published in January 2023, the NIST AI RMF is a flexible, voluntary framework for managing AI risks \parencite{nist-ai-rmf}. It organizes activities into four functions:

\begin{itemize}
\item \textbf{Govern}: Establish organizational structures, policies, and accountability for AI risk management.
\item \textbf{Map}: Identify context, stakeholders, and potential impacts of AI systems.
\item \textbf{Measure}: Assess and benchmark AI system performance, including trustworthiness characteristics (fairness, transparency, accountability, safety, privacy, security).
\item \textbf{Manage}: Allocate resources, implement risk treatments, and monitor effectiveness.
\end{itemize}

NIST AI RMF emphasizes trustworthiness characteristics and provides flexibility for organizations of different sizes and sectors. It is widely referenced by federal agencies, state regulators, and private-sector organizations as a baseline governance framework.

\paragraph{ISO/IEC 42001:2023 (AI Management Systems)}
ISO/IEC 42001 is an international standard for AI management systems, providing a certifiable framework \parencite{iso-iec-42001}. It establishes requirements for establishing, implementing, maintaining, and continually improving an AI management system. Annex A provides 40+ AI-specific controls organized by category (data management, model development, deployment, monitoring).

ISO/IEC 42001 is especially relevant for organizations:
\begin{itemize}
\item Operating in EU markets (the standard is recognized as supporting EU AI Act compliance).
\item Seeking third-party certification to demonstrate governance maturity.
\item Requiring international recognition (ISO standards are globally accepted).
\end{itemize}

Certification typically costs \$50,000--\$150,000 and requires 3--6 months, depending on organizational size and maturity.

\paragraph{COBIT (IT Governance Framework)}
COBIT, developed by ISACA, is a comprehensive IT governance framework widely used by enterprises. COBIT 2019 includes guidance on emerging technologies, including AI. Organizations with mature IT governance often extend COBIT to cover AI systems rather than creating parallel structures.

COBIT is best suited for organizations seeking to integrate AI governance into existing enterprise IT governance rather than treating AI as a standalone domain.

\paragraph{SOC 2 Type II (Vendor Assurance)}
SOC 2 (Service Organization Control) is an auditing framework for service providers, especially SaaS vendors. SOC 2 Type II reports assess controls over security, availability, processing integrity, confidentiality, and privacy over a period of time (typically 6--12 months).

For organizations procuring AI tools from vendors, a SOC 2 Type II report provides independent assurance that the vendor has implemented and operated controls effectively. Many enterprises require SOC 2 reports as a condition of vendor contracts.

\paragraph{Framework Selection Logic}
Organizations often layer frameworks:

\begin{itemize}
\item \textbf{Start with NIST AI RMF} for flexible internal governance (free, widely recognized, no certification requirement).
\item \textbf{Add ISO/IEC 42001} if seeking certification, operating in EU markets, or facing customer demands for third-party assurance.
\item \textbf{Integrate with COBIT} if mature IT governance structures exist.
\item \textbf{Require SOC 2} from third-party AI vendors to validate their controls.
\end{itemize}

No single framework addresses all requirements. Layering enables organizations to satisfy general governance needs (NIST), achieve certification (ISO), integrate with enterprise governance (COBIT), and validate vendor controls (SOC 2).

\subsection{Seven Common Controls Across Frameworks}
\label{sec:agents3-common-controls}

Despite structural differences, all governance frameworks converge on seven common controls:

\begin{enumerate}
\item \textbf{Risk Assessment and Management}: Identify, assess, prioritize, and mitigate AI-related risks throughout the system lifecycle. Risk assessment is the foundation for all subsequent governance activities.

\item \textbf{Human Oversight}: Implement oversight mechanisms proportionate to system autonomy and risk: human-in-the-loop (pre-approval for high-stakes decisions), human-on-the-loop (monitoring with intervention capability), or human-in-command (strategic oversight with emergency stop authority).

\item \textbf{Audit Logging and Traceability}: Maintain tamper-evident logs that capture inputs, outputs, decisions, and human interventions. Logs must enable reconstruction of decisions for audit, investigation, and regulatory review.

\item \textbf{Explainability and Transparency}: Provide stakeholders (users, auditors, regulators) with understandable information about how the system operates, what factors influence decisions, and what limitations exist. Explainability techniques must be validated for faithfulness (reflects actual model logic), completeness (material factors included), and usefulness (enables informed decisions).

\item \textbf{Vendor Management}: Assess, monitor, and manage third-party AI vendors. Vendor due diligence, contract negotiation, ongoing monitoring, and escalation procedures are essential because vendor risks cascade into organizational liability.

\item \textbf{Incident Response and Remediation}: Detect, triage, contain, investigate, remediate, and learn from AI system failures. Incident response must be rapid (fairness violations and safety failures require immediate action) and systematic (root cause analysis, notification, continuous improvement).

\item \textbf{Documentation and Record-Keeping}: Maintain comprehensive documentation of system purpose, design, data sources, validation results, deployment decisions, monitoring outputs, and incidents. Documentation supports audits, regulatory inquiries, and continuous improvement.
\end{enumerate}

While frameworks differ in emphasis and structure, these seven controls represent governance universals. Section~\ref{sec:agents3-dimensional} (presented earlier) established how to calibrate control intensity based on system properties. Section~\ref{sec:agents3-implementation} operationalizes these calibrated controls through technical architecture and organizational processes.

