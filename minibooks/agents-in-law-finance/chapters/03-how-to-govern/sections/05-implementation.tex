% ============================================================================
% Implementation: Building Governance Systems — Chapter 3 - How to Govern an Agent
% Purpose: Operationalize governance through controls and processes
% Label: sec:agents3-implementation
% ============================================================================

\section{Implementation}
\label{sec:agents3-implementation}

Section~\ref{sec:agents3-dimensional} established principles for calibrating control intensity. We now focus on building governance systems by operationalizing those principles. We now turn to operationalizing those principles: how to design and implement risk assessment, audit logging, explainability, human oversight, vendor management, performance monitoring, and incident response. Where Chapter~2 established the technical infrastructure for logging, escalation, and action controls, we now address the governance policies and organizational processes that make those capabilities enforceable. We focus on actionable guidance—what practitioners and governance teams actually build—illustrated through examples from legal, financial, and audit domains.

\subsection{Risk Assessment}
\label{sec:agents3-risk-assessment}

All governance begins with risk assessment. Before deploying an agentic system, organizations must systematically identify harm scenarios, assess their likelihood and impact, document mitigations, and define reassessment triggers.

\paragraph{Risk Assessment Methodology}
Effective risk assessment addresses six categories of AI-related harms:

\begin{itemize}
\item \textbf{Bias and Fairness}: Does the system produce discriminatory outcomes? Are protected classes disproportionately harmed?
\item \textbf{Accuracy and Reliability}: Does the system produce correct outputs? What is the error rate? What are the consequences of errors?
\item \textbf{Security}: Can adversaries manipulate inputs (prompt injection), poison training data, or exfiltrate sensitive information?
\item \textbf{Privacy}: Does the system access, process, or disclose personal or confidential information inappropriately?
\item \textbf{Safety}: Can system failures cause physical harm, financial loss, or operational disruption?
\item \textbf{Compliance}: Does deployment violate laws, regulations, or professional obligations?
\end{itemize}

For each risk category, assess \emph{likelihood} (how probable is this harm?), \emph{impact} (if it occurs, how severe are the consequences?), \emph{affected stakeholders} (who is harmed?), and \emph{mitigations} (what controls reduce risk?). Document \emph{residual risk} after mitigations and obtain approval from appropriate governance authority (e.g., risk committee, general counsel, board for high-risk systems).

Define \emph{reassessment triggers}: When must the risk assessment be updated? Common triggers include model updates, policy changes, regulatory developments, incident discoveries, and significant drift in performance or fairness metrics.

\paragraph{Example: Agentic Financial Planning Assistant Risk Assessment}
\glsadd{escalation}\glsadd{planning}
A registered investment adviser deploys an agentic financial planning system that \emph{iteratively} analyzes client portfolios, adapts recommendations based on market conditions and client feedback, and determines when to escalate to human advisers.

The system iterates through analysis-recommendation-feedback loops over days or weeks, adapts its strategy based on client responses and market changes, and terminates when confidence thresholds are met or escalation is required. As Chapter~2 established, these iteration, adaptation, and termination capabilities create governance requirements beyond simple AI tools. \textit{Dimensional profile: HITL + hybrid frame + adaptive goals + stateful.}

The risk assessment identifies five primary concerns, summarized in Table~\ref{tab:agents3-financial-planning-risk}.

\emph{Compliance risk} ranks highest: the system may recommend unsuitable investments, violating Advisers Act fiduciary duty. Unlike a simple Q\&A chatbot, iteration and adaptation create compounding risk—a flawed recommendation in cycle one shapes subsequent analysis.

\glsadd{hallucination}\emph{Accuracy risk} stems from potential hallucination of market data or misinterpretation of client constraints. Across iterative cycles, these errors compound rather than self-correct.

\emph{Adaptation risk} arises when the system drifts from regulatory compliance—as an instance, learning to recommend higher-fee products based on firm incentives rather than client best interest.

\emph{Iteration risk} manifests as either excessive iteration (analysis paralysis) or premature termination (incomplete analysis).

\emph{Security risk}, though less likely, carries the highest impact: prompt injection across iterative cycles could manipulate accumulated state, potentially disclosing other clients' information.

\begin{table}[htbp]
\centering
\caption[Risk Assessment: Agentic Financial Planning Assistant]{Risk Assessment: Agentic Financial Planning Assistant. Risk levels: \protect\tikz[baseline=-0.5ex]\protect\draw[fill=red-900,draw=none] (0,0) circle (0.35em); Critical, \protect\tikz[baseline=-0.5ex]\protect\draw[fill=red-600,draw=none] (0,0) circle (0.35em); High, \protect\tikz[baseline=-0.5ex]\protect\draw[fill=amber-600,draw=none] (0,0) circle (0.35em); Moderate, \protect\tikz[baseline=-0.5ex]\protect\draw[fill=green-600,draw=none] (0,0) circle (0.35em); Low.}
\label{tab:agents3-financial-planning-risk}
\footnotesize
\begin{tabular}{@{}>{\sffamily\raggedright\arraybackslash}p{1.6cm} >{\centering\arraybackslash}p{0.9cm} >{\centering\arraybackslash}p{0.9cm} >{\raggedright\arraybackslash}p{5.2cm} >{\centering\arraybackslash}p{1.0cm}@{}}
\toprule
\textbf{Risk} & \textbf{Likely} & \textbf{Impact} & \textbf{Key Mitigations} & \textbf{Residual} \\
\midrule
Compliance & \tikz[baseline=-0.5ex]\draw[fill=red-600,draw=none] (0,0) circle (0.35em); & \tikz[baseline=-0.5ex]\draw[fill=red-600,draw=none] (0,0) circle (0.35em); & HITL approval; monthly compliance; quarterly fiduciary review & \tikz[baseline=-0.5ex]\draw[fill=amber-600,draw=none] (0,0) circle (0.35em); \\
\addlinespace
Accuracy & \tikz[baseline=-0.5ex]\draw[fill=amber-600,draw=none] (0,0) circle (0.35em); & \tikz[baseline=-0.5ex]\draw[fill=red-600,draw=none] (0,0) circle (0.35em); & Verified data; cross-cycle checks; human final review & \tikz[baseline=-0.5ex]\draw[fill=amber-600,draw=none] (0,0) circle (0.35em); \\
\addlinespace
Adaptation & \tikz[baseline=-0.5ex]\draw[fill=amber-600,draw=none] (0,0) circle (0.35em); & \tikz[baseline=-0.5ex]\draw[fill=red-600,draw=none] (0,0) circle (0.35em); & Limited to methods; criteria fixed; quarterly audit & \tikz[baseline=-0.5ex]\draw[fill=green-600,draw=none] (0,0) circle (0.35em); \\
\addlinespace
Iteration & \tikz[baseline=-0.5ex]\draw[fill=green-600,draw=none] (0,0) circle (0.35em); & \tikz[baseline=-0.5ex]\draw[fill=red-600,draw=none] (0,0) circle (0.35em); & Termination (5 cycles, conf.\ >0.85, 14-day timeout) & \tikz[baseline=-0.5ex]\draw[fill=green-600,draw=none] (0,0) circle (0.35em); \\
\addlinespace
Security & \tikz[baseline=-0.5ex]\draw[fill=green-600,draw=none] (0,0) circle (0.35em); & \tikz[baseline=-0.5ex]\draw[fill=red-900,draw=none] (0,0) circle (0.35em); & Input sanitization; state validation; data isolation & \tikz[baseline=-0.5ex]\draw[fill=green-600,draw=none] (0,0) circle (0.35em); \\
\bottomrule
\end{tabular}
\end{table}

\textbf{Monitoring}: Daily compliance review, weekly adaptation log review, monthly accuracy and termination analysis, continuous client feedback.

This risk assessment illustrates how agentic properties (iteration, adaptation, autonomous termination) create governance requirements beyond simple AI tools. The system's ability to iterate and adapt demands \emph{cross-cycle consistency checks}, \emph{adaptation audits}, and \emph{termination condition validation}—controls unnecessary for non-agentic systems.

\subsection{Audit Logging}
\label{sec:agents3-audit-logging}

Audit logging allows organizations to reconstruct decisions, investigate incidents, satisfy regulatory inquiries, and demonstrate accountability. Logging requirements scale with autonomy: high-autonomy systems (HIC) call for more detailed logs than low-autonomy systems (HITL, where human review serves as primary control).

\paragraph{Logging Architecture Requirements}
Effective audit logging captures:

\begin{itemize}
\item \textbf{Inputs}: What data did the system perceive? Include user queries, retrieved documents, API responses, sensor readings—whatever the system used to make decisions.
\item \textbf{Outputs}: What did the system produce? Include recommendations, actions taken, messages sent, decisions rendered.
\item \textbf{Decision Rationale}: Why did the system produce this output? For high-autonomy or high-consequence systems, log intermediate reasoning steps, confidence scores, alternative options considered.
\item \textbf{Human Interventions}: When did humans approve, reject, or modify system outputs? Who made the decision? What was their rationale?
\item \textbf{System State}: For stateful systems, log state changes to support reconstruction of how the system's understanding evolved.
\end{itemize}

Logs must be stored in \emph{tamper-evident} formats (e.g., append-only databases, cryptographic hashing) with access controls limiting who can read or delete logs. Retention periods must satisfy regulatory requirements: 7-10 years for financial services, 25 months minimum for ECOA adverse action records, potentially longer for litigation hold purposes.

\paragraph{Example: Agentic Credit Underwriting Audit Logging (ECOA Compliance)}
A bank deploys an agentic mortgage underwriting system that \emph{iteratively} investigates applications by requesting additional documentation, querying third-party data sources (employment verification, asset verification), and analyzing trends across multiple applicants. The system adapts its investigation strategy based on discovered risk patterns and terminates when sufficient information is gathered or escalation is required.

\textit{Dimensional profile: HIC + institutional frame + adaptive goals + stateful.}

The system's agentic properties shape its governance requirements. Its \textbf{goal} is to approve qualified applicants while managing credit risk and satisfying ECOA requirements. Its \textbf{perception} draws on application data, third-party verification responses, and historical default patterns. Its \textbf{actions} include requesting documents, querying APIs, and generating preliminary assessments. \textbf{Iteration} operates across 3--7 investigation cycles over 5--15 days. \textbf{Adaptation} adjusts investigation depth based on risk indicators and application complexity. \textbf{Termination} occurs under explicit conditions: confidence exceeds 0.90, the maximum of 7 cycles is reached, or red flags trigger escalation to a senior underwriter.

Equal Credit Opportunity Act Regulation B mandates that lenders provide ``principal reasons'' for adverse credit decisions \parencite{ecoa-reg-b}. For agentic systems that iterate across multiple cycles and adapt their investigation strategy, the logging architecture must capture \emph{cross-cycle decision evolution} to permit reconstruction of how the system's assessment changed over time. The technical logging capabilities from Chapter~2's governance surface now require policy decisions about what must be logged, how long to retain it, and who can access it.

\begin{listingbox}[title={Listing: Agentic Underwriting Audit Log (Simplified JSON)}]
\small
\begin{verbatim}
{
  "application_id": "APP-2024-00123",
  "session_start": "2024-11-20T14:32:15Z",
  "session_end": "2024-11-28T09:15:42Z",
  "model_version": "agentic-underwriting-v2.1",
  "total_cycles": 4,
  "termination_reason": "confidence_threshold_met",
  "cycles": [
    {
      "cycle": 1,
      "timestamp": "2024-11-20T14:32:15Z",
      "perception": ["application_form", "credit_report"],
      "action": "request_employment_verification",
      "preliminary_assessment": "UNCERTAIN",
      "confidence": 0.62,
      "rationale": "Initial DTI borderline; need employment stability confirmation"
    },
    {
      "cycle": 2,
      "timestamp": "2024-11-22T10:18:33Z",
      "perception": ["employment_verification_response"],
      "action": "request_asset_documentation",
      "preliminary_assessment": "UNCERTAIN",
      "confidence": 0.71,
      "rationale": "Employment stable; need asset verification for down payment"
    },
    {
      "cycle": 3,
      "timestamp": "2024-11-25T15:42:09Z",
      "perception": ["bank_statements", "investment_accounts"],
      "action": "analyze_comparable_approvals",
      "preliminary_assessment": "LIKELY_APPROVE",
      "confidence": 0.84,
      "rationale": "Assets verified; comparable risk profile to approved cases"
    },
    {
      "cycle": 4,
      "timestamp": "2024-11-28T09:15:42Z",
      "perception": ["market_conditions", "portfolio_concentration_analysis"],
      "action": "generate_final_recommendation",
      "final_decision": "APPROVE",
      "confidence": 0.92,
      "recommendation": "Approve with standard terms"
    }
  ],
  "final_decision_factors": [
    {"factor": "verified_employment_stability", "weight": 0.35},
    {"factor": "sufficient_liquid_assets", "weight": 0.30},
    {"factor": "comparable_risk_profile", "weight": 0.25},
    {"factor": "credit_score_within_guidelines", "weight": 0.10}
  ]
}
\end{verbatim}
\end{listingbox}

\textbf{Retention}: 25 months (ECOA requirement) + 7 years (standard banking litigation hold).

\textbf{Security}: Logs are encrypted at rest, with access restricted to compliance officers, auditors, and authorized investigators. The system uses append-only storage with cryptographic integrity verification (tamper-evident) and per-cycle hash chains to detect any alteration.

\textbf{Retrievability}: Indexed by application ID, applicant (hashed identifier to protect PII), decision date, termination reason, and number of cycles. Permits compliance officers to query: ``Show all adverse decisions where the system terminated due to timeout instead of confidence'' or ``Identify applications where preliminary assessment changed from \texttt{LIKELY\_APPROVE} to \texttt{ADVERSE} between cycles.''

\textbf{Validation}: Quarterly audit sampling verifies logs support reconstruction of iterative decision evolution; test whether system's cross-cycle adaptations comply with fair lending principles; validate termination conditions are consistently applied.

This logging architecture satisfies ECOA's explainability requirement while addressing agentic-specific concerns: it captures \emph{how} the system's understanding evolved across cycles, \emph{what} triggered adaptation, and \emph{why} the system terminated. Without cross-cycle logging, the bank cannot reconstruct agentic decision-making or show that adaptation did not introduce prohibited discrimination.

\subsection{Explainability}
\label{sec:agents3-explainability}

Explainability translates system behavior into understandable information for stakeholders—users, auditors, regulators, affected individuals. Regulatory requirements vary: ECOA requires ``principal reasons,'' GDPR requires ``meaningful information about the logic involved,'' PCAOB requires auditors to document the rationale for audit procedures. Explainability techniques must be selected based on regulatory requirements and validated for \emph{faithfulness} (reflects actual model logic), \emph{completeness} (material factors included), and \emph{usefulness} (enables informed decisions).

\paragraph{Example: Agentic Audit Investigation System (PCAOB Compliance)}
A Big Four accounting firm develops an agentic audit assistant that \emph{iteratively} investigates high-risk accounts receivable. The system analyzes transactions, requests documentation, cross-references third-party data, adapts its strategy based on discovered anomalies, and escalates to senior auditors when material issues arise. \textit{Dimensional profile: HOTL + institutional frame + adaptive goals + stateful.}

PCAOB Auditing Standards require auditors to document the rationale for procedures in workpapers \parencite{pcaob-as1215,pcaob-as2315}. For agentic systems, explainability must capture \emph{why} the system escalated certain accounts, \emph{how} its strategy evolved across cycles, and \emph{what} evidence supported termination decisions. \Cref{fig:agents3-iterative-investigation} illustrates the workflow. Each cycle generates explanations logged to audit workpapers.

% Iterative Investigation with Explainable Adaptation - Clean Vertical Flow

\begin{figure}[htbp]
\centering
\begin{tikzpicture}[
    cycle box/.style={
        rounded corners=6pt,
        draw=#1,
        line width=1.2pt,
        fill=#1!10,
        minimum width=5cm,
        minimum height=1.2cm,
        font=\small,
        align=center
    },
    detail box/.style={
        rounded corners=4pt,
        draw=#1,
        line width=0.6pt,
        fill=#1!5,
        text width=4.5cm,
        inner sep=5pt,
        font=\scriptsize,
        align=left
    },
    decision/.style={
        rounded corners=2pt,
        draw=amber-600,
        line width=1pt,
        fill=amber-100,
        minimum width=1.6cm,
        minimum height=1cm,
        inner sep=3pt,
        font=\scriptsize,
        align=center
    },
    termination/.style={
        rounded corners=4pt,
        draw=red-600,
        line width=1pt,
        fill=red-100,
        minimum width=2.6cm,
        minimum height=1cm,
        inner sep=4pt,
        font=\scriptsize,
        align=center
    },
    lowrisk/.style={
        rounded corners=4pt,
        draw=gray-500,
        line width=0.6pt,
        fill=gray-100,
        minimum width=2.2cm,
        minimum height=0.6cm,
        font=\scriptsize,
        align=center
    },
    arrow/.style={->, thick, gray-700, >=stealth, line width=1pt},
    loop arrow/.style={->, thick, green-900, >=stealth, line width=1.2pt},
    label text/.style={font=\tiny, text=gray-600}
]

% === CYCLE 1 with detail box ===
\node[cycle box=slate-700] (cycle1) at (-2.5,0) {
    \textbf{\textcolor{slate-900}{Cycle 1: Initial Risk Scoring}}
};

\node[detail box=slate-700, anchor=north west] (c1details) at (1.5,0.6) {
    \textbf{Risk Criteria:}\\[2pt]
    High-value (>\$500K): +3\\
    Overdue >90 days: +2\\
    New customer: +1\\
    Prior adjustments: +2\\
    Related-party: +3
};

% === DECISION 1 ===
\node[decision, below=0.8cm of cycle1] (dec1) {
    Score\\$\geq$6?
};

% Low risk exit - to the left
\node[lowrisk, left=1.2cm of dec1] (lowrisk) {
    Standard procedures
};

% === CYCLE 2-N with detail box ===
\node[cycle box=green-900, below=1.5cm of dec1] (cycle2) {
    \textbf{\textcolor{green-900}{Cycles 2--N: Adaptive Investigation}}
};

\node[detail box=green-900, anchor=north west] (c2details) at (1.5,-3.2) {
    \textbf{Actions:}\\
    Request documents\\
    Analyze payment patterns\\
    Cross-reference data\\[3pt]
    \textbf{Adaptation:}\\
    Incomplete $\rightarrow$ request more\\
    Disputes $\rightarrow$ legal review\\
    Fraud signals $\rightarrow$ escalate
};

% === DECISION 2 ===
\node[decision, below=0.8cm of cycle2] (dec2) {
    Continue?
};

% === TERMINATION OUTCOMES ===
\node[termination, below=1cm of dec2, xshift=-3cm] (term1) {
    \textbf{\textcolor{red-900}{Sufficient Evidence}}\\
    \textcolor{gray-600}{(conf.\ >0.85)}
};

\node[termination, below=1cm of dec2] (term2) {
    \textbf{\textcolor{red-900}{Red Flag}}\\
    \textcolor{gray-600}{$\rightarrow$ Escalate}
};

\node[termination, below=1cm of dec2, xshift=3cm] (term3) {
    \textbf{\textcolor{red-900}{Max Cycles}}\\
    \textcolor{gray-600}{(5 reached)}
};

% === ARROWS ===
% Cycle 1 to Decision 1
\draw[arrow] (cycle1.south) -- (dec1.north);

% Decision 1: Yes to Cycle 2
\draw[arrow] (dec1.south) -- node[right, label text] {Yes} (cycle2.north);

% Decision 1: No to low risk (left)
\draw[arrow] (dec1.west) -- node[above, label text] {No} (lowrisk.east);

% Cycle 2 to Decision 2
\draw[arrow] (cycle2.south) -- (dec2.north);

% Decision 2 to terminations
\draw[arrow] (dec2.south) -- ++(0,-0.3) -| (term1.north);
\draw[arrow] (dec2.south) -- (term2.north);
\draw[arrow] (dec2.south) -- ++(0,-0.3) -| (term3.north);

% Loop back arrow - on left side, wide enough to clear the box
\draw[loop arrow, rounded corners=6pt] (dec2.west) -- ++(-4.5,0) node[below, font=\scriptsize, text=green-900] {Next cycle} -- ++(0,2.3) -- (cycle2.west);

\end{tikzpicture}
\caption{Iterative investigation workflow with explainable adaptation. Cycle~1 scores accounts using risk criteria; high-risk accounts (score $\geq$6) proceed to iterative investigation. Cycles~2--N gather evidence and adapt strategy based on findings. Investigation terminates when sufficient evidence is obtained, a red flag requires escalation, or maximum cycles are reached.}
\label{fig:agents3-iterative-investigation}
\end{figure}


\textbf{Explainability Validation}:
\begin{itemize}
\item \textbf{Faithfulness}: Verify explanations match actual investigation logic by reviewing audit logs (do logged perceptions and actions align with explanations?).
\item \textbf{Completeness}: Confirm all material risk indicators that triggered escalation appear in explanations.
\item \textbf{Usefulness}: Senior auditor reviews cycle-level explanations and confirms they support professional judgment (``Does the system's escalation rationale justify senior auditor involvement?'').
\end{itemize}

\textbf{Workpaper Documentation}: The audit workpaper includes:
\begin{itemize}
\item Initial risk scoring methodology (Cycle 1 criteria).
\item Cycle-by-cycle investigation narrative (what the system perceived, what actions it took, why it adapted).
\item Escalation rationale (why this account required human review).
\item Senior auditor's assessment: ``We deployed an agentic audit assistant to investigate 47 high-risk receivables. The system iteratively gathered evidence across 2-5 cycles per account, adapting its strategy based on discovered documentation quality and anomaly patterns. It escalated 8 accounts for senior review due to identified red flags (revenue recognition concerns, collectability doubts). We reviewed the system's investigation logs, assessed the escalated accounts, and obtained sufficient appropriate audit evidence to support our conclusions.''
\end{itemize}

This agentic design satisfies PCAOB's requirement that auditors understand their methodology while demonstrating how iteration and adaptation improve audit effectiveness. The system's ability to \emph{learn} during investigation (adapting strategy based on discovered evidence) and \emph{escalate appropriately} (terminating when human judgment is required) exemplifies agentic governance in practice.

\subsection{Human Oversight Workflows}
\label{sec:agents3-human-oversight}
\glsadd{human-in-the-loop}
\glsadd{human-on-the-loop}
\glsadd{human-in-command}

Section~\ref{sec:agents3-autonomy-calibration} defined three oversight modes. This section operationalizes them through workflows, notification mechanisms, intervention interfaces, and escalation procedures.

\paragraph{HITL (Human-in-the-Loop): Approval Workflows}
HITL systems require human pre-approval before executing high-consequence actions. Implementation requires:

\begin{itemize}
\item \textbf{Approval Queue}: System generates a recommendation and adds it to a queue visible to authorized reviewers.
\item \textbf{Notification}: Alert the reviewer (email, dashboard notification, SMS for time-sensitive actions).
\item \textbf{Review Interface}: Present the recommendation, supporting evidence, system confidence, and options (approve, reject, modify, request more information).
\item \textbf{Accountability}: Log who approved, when, and any modifications made.
\item \textbf{Automation Bias Mitigation}: To prevent rubber stamping, randomize the presentation order of recommendations, periodically inject known-incorrect recommendations as controls, and track approval/rejection rates per reviewer (flag reviewers with suspiciously high approval rates).
\end{itemize}

\paragraph{HOTL (Human-on-the-Loop): Monitoring and Intervention}
HOTL systems operate autonomously but humans monitor and can intervene. Implementation requires:

\begin{itemize}
\item \textbf{Monitoring Dashboard}: Real-time or near-real-time display of system activity (actions taken, error rates, escalation triggers, user feedback).
\item \textbf{Escalation Triggers}: Define conditions requiring human review (e.g., low-confidence decisions <0.7, user complaints, outcomes near policy boundaries, anomalies detected).
\item \textbf{Intervention Protocol}: How does the human halt the system, override a decision, or modify parameters? Must be accessible in real-time.
\item \textbf{Escalation Pathway}: If the monitoring human cannot resolve an issue, to whom do they escalate? (Senior supervisor, compliance officer, emergency stop authority.)
\end{itemize}

\textbf{Example: Agentic Credit Underwriting HOTL Monitoring.}
A mortgage lender's agentic underwriting system (described in Section~\ref{sec:agents3-audit-logging}) operates in HOTL mode, iteratively investigating applications across multiple cycles. Senior underwriters monitor aggregate system performance through a dashboard (\Cref{fig:agents3-hotl-dashboard}) displaying agentic-specific metrics and escalation triggers. When escalation frequency spikes or average cycles increase significantly, supervisors investigate root causes such as data quality degradation, overly conservative termination thresholds, or emerging risk patterns requiring strategy adjustment.

% HOTL Monitoring Dashboard - Agentic Credit Underwriting

\begin{figure}[htbp]
\centering
\begin{tikzpicture}[
    metric box/.style={
        rounded corners=4pt,
        draw=gray-500,
        line width=0.75pt,
        fill=white,
        minimum width=2.2cm,
        minimum height=1.5cm,
        font=\small,
        align=center
    },
    metric value/.style={font=\large\bfseries},
    metric label/.style={font=\scriptsize, text=gray-600},
    alert box/.style={
        rounded corners=3pt,
        draw=#1,
        line width=1pt,
        fill=#1!8,
        text width=9.5cm,
        inner sep=6pt,
        font=\footnotesize
    },
    section header/.style={
        font=\small\bfseries\sffamily,
        text=slate-700,
        anchor=west
    },
    status indicator/.style={
        circle,
        minimum size=0.3cm,
        inner sep=0pt
    },
    dashboard frame/.style={
        rounded corners=8pt,
        draw=gray-300,
        line width=1pt,
        fill=bg-neutral,
        inner sep=10pt
    },
    pill filter/.style={
        rounded corners=3pt,
        draw=gray-300,
        fill=gray-100,
        inner sep=2pt,
        font=\tiny\sffamily,
        text=gray-700
    },
    pill filter active/.style={
        pill filter,
        draw=primary,
        text=primary,
        fill=definition-light
    },
    meta label/.style={
        font=\tiny\sffamily,
        text=gray-600
    }
]

% Main dashboard frame
\node[dashboard frame, minimum width=11.5cm, minimum height=12.8cm] (frame) at (0,-4.6) {};

% Application chrome (top navigation bar)
\node[
    rounded corners=6pt,
    draw=gray-300,
    line width=0.6pt,
    fill=gray-100,
    minimum width=10.8cm,
    minimum height=0.6cm
] (topbar) at (0,1.5) {};
\node[meta label, anchor=east] at ([xshift=-0.3cm]topbar.east) {Last refreshed: 2026-07-15 09:42 ET};

% Dashboard view title and filters
\node[font=\small\bfseries\sffamily, text=slate-900, anchor=west] (viewtitle) at (-5.0,0.7) {HOTL Monitoring --- Agentic Credit Underwriting};
\node[meta label, anchor=west] at ([yshift=-0.3cm]viewtitle.south west) {Portfolio: Retail mortgage \textbullet\ Region: US \textbullet\ Channel: Online};
\node[pill filter active, anchor=west] at ([xshift=4.5cm,yshift=-0.3cm]viewtitle.south west) {Today};
\node[pill filter, anchor=west] at ([xshift=5.8cm,yshift=-0.3cm]viewtitle.south west) {7d};
\node[pill filter, anchor=west] at ([xshift=6.8cm,yshift=-0.3cm]viewtitle.south west) {30d};

% === METRICS ROW ===
\node[section header] at (-5.0,-0.4) {Daily Metrics};

% Metric 1: Avg Cycles
\node[metric box] (m1) at (-3.8,-1.5) {};
\node[metric value, text=slate-900] at ([yshift=0.15cm]m1.center) {3.2};
\node[metric label] at ([yshift=-0.4cm]m1.center) {Avg Cycles/App};

% Metric 2: Confidence Exits
\node[metric box] (m2) at (-1.3,-1.5) {};
\node[metric value, text=example-dark] at ([yshift=0.15cm]m2.center) {847};
\node[metric label] at ([yshift=-0.4cm]m2.center) {Confidence Exits};

% Metric 3: Timeout Exits
\node[metric box] (m3) at (1.2,-1.5) {};
\node[metric value, text=key-base] at ([yshift=0.15cm]m3.center) {23};
\node[metric label] at ([yshift=-0.4cm]m3.center) {Timeout Exits};

% Metric 4: Consistency Score
\node[metric box] (m4) at (3.7,-1.5) {};
\node[metric value, text=slate-900] at ([yshift=0.15cm]m4.center) {94\%};
\node[metric label] at ([yshift=-0.4cm]m4.center) {Consistency Score};

% === ESCALATION TRIGGERS ===
\node[section header] at (-5.0,-2.8) {Active Escalation Triggers};

% Alert 1 - Green (normal)
\node[alert box=example-dark] (a1) at (0.3,-3.7) {
    \textbf{Low Confidence After Max Cycles}~~\textnormal{(Conf.\ <0.70 after 7 cycles)}\\[1pt]
    \textcolor{gray-600}{3 applications escalated to senior underwriter for manual completion}
};
\node[status indicator, fill=example-base] at ([xshift=-0.3cm]a1.west) {};

% Alert 2 - Green (normal)
\node[alert box=example-dark, below=0.3cm of a1] (a2) {
    \textbf{Timeout Termination Review}~~\textnormal{(System timeout vs.\ confidence exit)}\\[1pt]
    \textcolor{gray-600}{2 applications flagged for investigation adequacy review}
};
\node[status indicator, fill=example-base] at ([xshift=-0.3cm]a2.west) {};

% Alert 3 - Amber (elevated)
\node[alert box=key-base, below=0.3cm of a2] (a3) {
    \textbf{Applicant Dispute}~~\textnormal{(Preliminary assessment challenged)}\\[1pt]
    \textcolor{gray-600}{1 dispute --- underwriter reviewing cycle-by-cycle investigation log}
};
\node[status indicator, fill=key-base] at ([xshift=-0.3cm]a3.west) {};

% Alert 4 - Green (normal)
\node[alert box=example-dark, below=0.3cm of a3] (a4) {
    \textbf{Cross-Cycle Inconsistency}~~\textnormal{(Contradictory assessments detected)}\\[1pt]
    \textcolor{gray-600}{0 applications flagged --- consistency score within normal range}
};
\node[status indicator, fill=example-base] at ([xshift=-0.3cm]a4.west) {};

% Alert 5 - Scheduled (gray)
\node[alert box=gray-500, below=0.3cm of a4] (a5) {
    \textbf{Fairness Metric Violation}~~\textnormal{(Disparate impact >20\%)}\\[1pt]
    \textcolor{gray-600}{Monthly review scheduled --- current DI ratio: 8.3\% (within threshold)}
};
\node[status indicator, fill=gray-500] at ([xshift=-0.3cm]a5.west) {};

% Legend
\node[font=\scriptsize, text=gray-600, anchor=north west] at (-5.0,-10.3) {
    \tikz\node[status indicator, fill=example-base] {}; Normal~~
    \tikz\node[status indicator, fill=key-base] {}; Elevated~~
    \tikz\node[status indicator, fill=caution-base] {}; Critical~~
    \tikz\node[status indicator, fill=gray-500] {}; Scheduled
};

\end{tikzpicture}
\caption{HOTL monitoring dashboard for agentic credit underwriting. The dashboard displays daily operational metrics (average investigation cycles, termination reasons, cross-cycle consistency) and active escalation triggers with status indicators. Senior underwriters monitor aggregate performance and intervene when triggers fire.}
\label{fig:agents3-hotl-dashboard}
\end{figure}


\paragraph{HIC (Human-in-Command): Strategic Oversight and Emergency Stop}
HIC systems operate with high autonomy. Humans set goals and constraints, monitor aggregate performance, and retain emergency stop authority. Implementation requires:

\begin{itemize}
\item \textbf{Strategic Goal-Setting}: Executives define objectives, risk appetite, and constraints (e.g., ``Fraud detection system must achieve 95\% precision, maintain false positive rate <1\%, and satisfy GDPR Article 22 requirements'').
\item \textbf{Aggregate Monitoring}: Statistical dashboards (daily/weekly/monthly) showing performance trends, fairness metrics, error rates, drift indicators. Not individual-decision review.
\item \textbf{Emergency Stop}: Accessible to authorized personnel (CTO, Chief Risk Officer, compliance head); tested quarterly; documented procedures for graceful shutdown (complete in-progress transactions, notify affected users, preserve state).
\item \textbf{Revalidation Triggers}: Define when the system must be revalidated before continuing operation (e.g., fairness violation detected, accuracy below SLA, regulatory policy change).
\end{itemize}

\subsection{Vendor Management}
\label{sec:agents3-vendor-management}

\glsadd{hallucination}Most organizations procure AI systems from vendors rather than building in-house. Vendor risk cascades into organizational liability: if the vendor's model hallucinates, is biased, or breaches confidentiality, the deploying organization faces regulatory penalties and reputational harm. Governance must include vendor due diligence, contract negotiation, and ongoing monitoring.

\paragraph{Vendor Due Diligence Framework (Three Phases)}
The framework proceeds through three phases. \textbf{Phase 1 (Assessment)} uses questionnaires to gather information about the vendor's data sources, model architecture, security practices, performance benchmarks, and fairness testing methodology. \textbf{Phase 2 (Document Review)} examines supporting documentation to verify vendor claims. This includes SOC 2 reports, data processing agreements, model validation reports, and security certifications. \textbf{Phase 3 (Validation)} confirms claims through reference checks with existing clients in similar domains and pilot testing with representative data to validate accuracy, explainability, and performance before full deployment.

% Three-Phase Vendor Due Diligence - Horizontal Flow

\begin{figure}[htbp]
\centering
\begin{tikzpicture}[
    phase box/.style={
        rounded corners=6pt,
        draw=slate-700,
        line width=1.5pt,
        fill=slate-100,
        minimum width=3.2cm,
        minimum height=2.2cm,
        font=\small,
        align=center
    },
    phase number/.style={
        circle,
        fill=slate-700,
        text=white,
        font=\bfseries\small,
        minimum size=0.7cm,
        inner sep=0pt
    },
    arrow/.style={->, thick, slate-700, >=stealth, line width=1.5pt}
]

% Phase 1
\node[phase box] (phase1) at (0,0) {
    \textbf{Assessment}\\[4pt]
    Questionnaire on\\
    data, model,\\
    security, fairness
};
\node[phase number] at ([yshift=0.3cm]phase1.north) {1};

% Phase 2
\node[phase box] (phase2) at (5,0) {
    \textbf{Document Review}\\[4pt]
    SOC 2, DPA,\\
    validation reports,\\
    certifications
};
\node[phase number] at ([yshift=0.3cm]phase2.north) {2};

% Phase 3
\node[phase box] (phase3) at (10,0) {
    \textbf{Validation}\\[4pt]
    Reference checks,\\
    pilot testing,\\
    performance verification
};
\node[phase number] at ([yshift=0.3cm]phase3.north) {3};

% Arrows
\draw[arrow] (phase1.east) -- (phase2.west);
\draw[arrow] (phase2.east) -- (phase3.west);

\end{tikzpicture}
\caption{Three-phase vendor due diligence framework. Phase 1 gathers information through questionnaires; Phase 2 reviews supporting documentation and certifications; Phase 3 validates claims through reference checks and pilot testing with representative data.}
\label{fig:agents3-vendor-diligence}
\end{figure}


\paragraph{Contract Negotiation: Shifting Risk to Vendors Where Possible}
Negotiate contract terms that allocate risk appropriately:

\begin{table}[htbp]
\centering
\small
\begin{tabular}{>{\raggedright\arraybackslash}p{2.8cm} >{\raggedright\arraybackslash}p{7.8cm}}
\toprule
\textbf{Contract Term} & \textbf{Negotiation Focus} \\
\midrule
Liability Caps & Higher/uncapped liability for confidentiality breaches and negligence in high-risk cases \\
\addlinespace
Update Notification & 30-60 days advance notice before material model updates \\
\addlinespace
Audit Rights & Annual audits or upon incident discovery \\
\addlinespace
Data Handling & Prohibit training on customer data; deletion upon termination \\
\addlinespace
SLAs & Performance thresholds with specified remedies \\
\bottomrule
\end{tabular}
\caption{Key contract terms for allocating risk in AI vendor agreements.}
\label{tab:agents3-contract-terms}
\end{table}

\paragraph{Agentic-Specific Risk: Adaptation Opacity}
Agentic systems that learn and adapt create a unique vendor risk that traditional AI contracts do not address: \textbf{adaptation opacity}—the vendor's model silently updates its decision-making strategy in the background without formal version changes, invalidating continuous validation requirements and creating regulatory exposure.

\textbf{The Problem}: Regulatory frameworks like SR 11-7 (Federal Reserve model risk management) require ongoing validation of models used by banking institutions \parencite{fed-sr11-7}. Organizations validate ``Model v2.1'' and deploy it. If the vendor's agentic system \emph{adapts}—adjusting feature weights, refining decision criteria, or modifying iteration logic—the deployed system may behave materially differently from the validated version, yet the vendor does not issue a new version number or notify the customer. The organization continues operating under the assumption it is using validated ``v2.1,'' but the system's actual behavior has drifted. This breaks continuous validation, exposes the organization to regulatory penalties (``You deployed an unvalidated model''), and creates fairness risk (adaptation may introduce prohibited discrimination).

\textbf{Why Traditional Contracts Fail}: Standard AI vendor contracts address \emph{formal version updates} (``Vendor will notify Customer of material updates''). But agentic systems' adaptation mechanisms operate \emph{within} a version, not across versions. The vendor's position: ``We did not update the model—v2.1 is still v2.1. The system is designed to adapt; that is a feature, not a bug.'' The customer's regulatory obligation: ``We must validate material changes to model behavior, regardless of version numbering.''

\textbf{Contractual Mitigation—Adaptation Transparency Clauses}: For agentic vendor systems, negotiate contractual provisions that address adaptation opacity:

\begin{table}[htbp]
\centering
\small
\begin{tabular}{>{\raggedright\arraybackslash}p{3.2cm} >{\raggedright\arraybackslash}p{7.4cm}}
\toprule
\textbf{Clause Type} & \textbf{Purpose} \\
\midrule
Adaptation Disclosure & Vendor identifies all adaptation mechanisms; specifies static vs. adaptive components \\
\addlinespace
Change Log Access & API/dashboard access to logs showing changes, timing, and rationale \\
\addlinespace
Material Change Thresholds & Triggers requiring notification and revalidation rights \\
\addlinespace
Audit Rights & Periodic behavioral validation; vendor cooperation with audits \\
\addlinespace
Adaptation Freeze & Option to disable learning during exams or investigations \\
\bottomrule
\end{tabular}
\caption{Adaptation transparency clauses for agentic vendor contracts.}
\label{tab:agents3-adaptation-clauses}
\end{table}

\begin{tcolorbox}[
  enhanced,
  colback=bg-example,
  colframe=example-base,
  coltitle=white,
  fonttitle=\bfseries,
  title={Example Contractual Language: Adaptation Transparency},
  boxrule=1pt,
  left=3mm,
  right=3mm,
  top=2mm,
  bottom=2mm,
  breakable
]
\small
\textbf{Section X: Adaptation Transparency and Change Control}

\textbf{X.1 Adaptation Disclosure.} Vendor has disclosed in Exhibit C all mechanisms by which the System adapts its decision-making logic, including feature weight updates, threshold adjustments, and strategy refinements. Vendor represents that Exhibit C is complete and accurate as of the Effective Date.

\textbf{X.2 Change Logs.} Vendor shall maintain detailed change logs documenting all adaptation events, including timestamp, changed parameters, magnitude of change, and triggering feedback. Customer shall have API access to change logs with daily refresh.

\textbf{X.3 Material Change Notification.} If any of the following thresholds are met, Vendor shall notify Customer within five (5) business days and provide root cause analysis: (a) any feature weight changes by more than ten percent (10\%) absolute within thirty (30) days; (b) decision threshold changes by more than five percent (5\%) within thirty (30) days; (c) accuracy degrades by more than five percent (5\%) on validation dataset; or (d) disparate impact ratio for any protected class changes by more than ten percent (10\%).

\textbf{X.4 Revalidation Rights.} Upon Material Change notification, Customer may elect to: (a) require Vendor to revert System to last validated configuration (at no cost to Customer); (b) conduct revalidation testing (Vendor shall cooperate and bear reasonable costs); or (c) pause System operation pending resolution.

\textbf{X.5 Adaptation Freeze.} Upon forty-eight (48) hours' notice, Customer may require Vendor to disable all adaptation mechanisms, causing the System to operate with static parameters. Vendor shall maintain freeze mode for up to ninety (90) days per Calendar Year at no additional cost.
\end{tcolorbox}

\textbf{Governance Benefit}: These contractual provisions operationalize continuous validation requirements for adaptive agentic systems. Without adaptation transparency, organizations deploying vendor agentic systems face a compliance gap: regulatory obligations demand ongoing validation, but vendor opacity prevents detection of material changes. Adaptation transparency clauses shift this burden back to vendors and provide customers with the visibility necessary to satisfy SR 11-7, ECOA, and similar frameworks.

\paragraph{Ongoing Monitoring}
Vendor due diligence does not end at contract signature. Implement:
\begin{itemize}
\item \textbf{Performance Monitoring}: Track accuracy, error rates, user complaints. Compare vendor claims to observed performance.
\item \textbf{Security Monitoring}: Review vendor security incident reports; conduct annual security assessments.
\item \textbf{Accuracy Audits}: Quarterly or semi-annual testing of vendor outputs against ground truth.
\item \textbf{Escalation Procedures}: Define error rate thresholds triggering vendor review (e.g., ``If hallucination rate exceeds 5\%, escalate to General Counsel; consider vendor termination'').
\end{itemize}

\paragraph{Example: Law Firm Foundation Model Vetting}
A law firm evaluates a foundation model vendor for legal research assistance. Due diligence identifies five risk categories:

\begin{itemize}
\item \textbf{Confidentiality}: Vendor uses multi-tenant architecture; customer queries may be logged for training. \emph{Mitigation}: Negotiate zero-retention DPA; require vendor to delete all firm data within 30 days of session termination; annual audit rights.
\item \textbf{Conflicts}: Vendor serves competing law firms; could create conflicts if data is shared. \emph{Mitigation}: Vendor affirms data isolation per client; third-party audit confirms isolation controls.
\item \textbf{Accuracy}: Vendor claims 95\% citation accuracy but provides no independent validation. \emph{Mitigation}: Firm conducts pilot testing with 200 known cases; achieves 60\% accuracy (below acceptable threshold). Vendor contract includes accuracy SLA (90\%); quarterly accuracy audits; right to terminate if SLA violated for two consecutive quarters.
\item \textbf{Hallucination}: Model occasionally fabricates case law. \emph{Mitigation}: HITL verification (attorney must independently verify all citations before filing); firm maintains hallucination log; if hallucination rate >5\%, escalate to General Counsel.
\item \textbf{Regulatory Compliance}: ABA Rule 1.6 confidentiality obligations. \emph{Mitigation}: Vendor contract includes uncapped liability for confidentiality breaches; cyber insurance confirmation.
\end{itemize}

Firm approves vendor with conditions: HITL verification mandatory, quarterly accuracy audits, annual security review, zero-retention DPA. This risk-calibrated approach enables use while protecting against residual vendor risks.

\subsection{Performance Monitoring and Incident Response}
\label{sec:agents3-monitoring-incident}

Governance is not a one-time validation but a continuous cycle. Systems must be monitored for performance degradation, fairness violations, data drift, and security incidents. When failures occur, organizations must detect, contain, investigate, remediate, and learn. This shift from inspection-based to continuous monitoring mirrors the evolution of Statistical Process Control in manufacturing---a historical parallel we examined in Chapter~2.

% Incident Response Cycle - Vertical Flow with Full Descriptions
% Redesigned for US Trade 6x9 format (text width ~11.75cm)

\begin{figure}[htbp]
\centering
\begin{tikzpicture}[
    stage box/.style={
        draw=slate-700,
        line width=1pt,
        fill=white,
        rounded corners=3pt,
        minimum width=10.5cm,
        minimum height=1.5cm,
        inner sep=6pt
    },
    stage num/.style={
        circle,
        draw=slate-700,
        line width=1.5pt,
        fill=amber-600,
        text=white,
        font=\small\bfseries\sffamily,
        minimum size=0.65cm,
        inner sep=0pt
    },
    stage title/.style={
        font=\small\bfseries\sffamily,
        text=slate-700,
        anchor=west,
        text width=1.6cm
    },
    stage desc/.style={
        font=\footnotesize\sffamily,
        text=gray-700,
        text width=6.5cm,
        align=left
    },
    arrow/.style={->, thick, slate-700, >=stealth}
]

% Vertical spacing
\def\vstep{1.85cm}

% Stage 1 - Detect
\node[stage box] (s1) at (0, 0) {};
\node[stage num] at ([xshift=0.5cm]s1.west) {1};
\node[stage title] at ([xshift=0.95cm]s1.west) {Detect};
\node[stage desc, anchor=west] at ([xshift=3.0cm]s1.west) {Monitoring alerts, user complaints, audit findings, or external reports identify a potential issue.};

% Stage 2 - Triage
\node[stage box] (s2) at (0, -\vstep) {};
\node[stage num] at ([xshift=0.5cm]s2.west) {2};
\node[stage title] at ([xshift=0.95cm]s2.west) {Triage};
\node[stage desc, anchor=west] at ([xshift=3.0cm]s2.west) {Assess severity. Fairness violation $\rightarrow$ escalate/halt. Accuracy issue $\rightarrow$ investigate. Security breach $\rightarrow$ activate response team.};

% Stage 3 - Contain
\node[stage box] (s3) at (0, -2*\vstep) {};
\node[stage num] at ([xshift=0.5cm]s3.west) {3};
\node[stage title] at ([xshift=0.95cm]s3.west) {Contain};
\node[stage desc, anchor=west] at ([xshift=3.0cm]s3.west) {Limit harm. Safety failures $\rightarrow$ halt immediately. Accuracy degradation $\rightarrow$ revert model. Security breach $\rightarrow$ isolate systems.};

% Stage 4 - Investigate
\node[stage box] (s4) at (0, -3*\vstep) {};
\node[stage num] at ([xshift=0.5cm]s4.west) {4};
\node[stage title] at ([xshift=0.95cm]s4.west) {Investigate};
\node[stage desc, anchor=west] at ([xshift=3.0cm]s4.west) {Root cause analysis. Data quality? Model drift? Adversarial attack? Process breakdown?};

% Stage 5 - Remediate
\node[stage box] (s5) at (0, -4*\vstep) {};
\node[stage num] at ([xshift=0.5cm]s5.west) {5};
\node[stage title] at ([xshift=0.95cm]s5.west) {Remediate};
\node[stage desc, anchor=west] at ([xshift=3.0cm]s5.west) {Fix root cause. Retrain model, update data sources, patch vulnerability, revise process.};

% Stage 6 - Notify
\node[stage box] (s6) at (0, -5*\vstep) {};
\node[stage num] at ([xshift=0.5cm]s6.west) {6};
\node[stage title] at ([xshift=0.95cm]s6.west) {Notify};
\node[stage desc, anchor=west] at ([xshift=3.0cm]s6.west) {Regulatory notification (if required), affected individuals (if harm), internal stakeholders.};

% Stage 7 - Post-Incident Review
\node[stage box] (s7) at (0, -6*\vstep) {};
\node[stage num] at ([xshift=0.5cm]s7.west) {7};
\node[stage title] at ([xshift=0.95cm]s7.west) {Review};
\node[stage desc, anchor=west] at ([xshift=3.0cm]s7.west) {Document lessons learned, update risk assessment, revise controls to prevent recurrence.};

% Arrows between stages
\draw[arrow] (s1.south) -- (s2.north);
\draw[arrow] (s2.south) -- (s3.north);
\draw[arrow] (s3.south) -- (s4.north);
\draw[arrow] (s4.south) -- (s5.north);
\draw[arrow] (s5.south) -- (s6.north);
\draw[arrow] (s6.south) -- (s7.north);

% Feedback loop arrow from Review back to Detect
\draw[arrow, rounded corners=6pt, dashed]
    ([xshift=0.2cm]s7.east) -- ([xshift=0.5cm]s7.east) -- ([xshift=0.5cm]s1.east) -- ([xshift=0.2cm]s1.east);
\node[font=\scriptsize\sffamily, text=slate-700, rotate=90, anchor=south] at ([xshift=0.95cm, yshift=-5.55cm]s1.east) {Continuous Improvement};

\end{tikzpicture}
\caption{The seven-stage incident response cycle for AI system failures. The dashed feedback loop emphasizes that post-incident review improves detection capabilities, creating continuous improvement in governance controls.}
\label{fig:agents3-incident-response-cycle}
\end{figure}


\paragraph{Performance Monitoring: Four Dimensions}
Monitor continuously across four dimensions:

\begin{enumerate}
\item \textbf{Performance Metrics}: Accuracy, precision, recall, F1 score, latency—whatever aligns with business objectives. Establish SLAs and alert when performance degrades below thresholds.
\item \textbf{Data Drift}: Are input distributions changing? If the system was trained on 2020-2022 mortgage applications and is now seeing 2024 applications with different characteristics (higher interest rates, different applicant demographics), performance may degrade.
\item \textbf{Concept Drift}: Are input-output relationships changing? In practice, fraud patterns evolve; a fraud detection model trained on 2022 patterns may miss 2024 attack vectors.
\item \textbf{Fairness Metrics}: For systems affecting protected classes, monitor approval rates, error rates, and disparate impact ratios by demographic group. Regulatory expectations and enforcement practice under ECOA effectively require lenders to monitor for disparate impact as part of fair lending compliance. Similarly, GDPR Article 22 requires ongoing assessment of automated decision-making.
\end{enumerate}

\paragraph{Example: Disparate Impact in Agentic Credit Underwriting}
A regional bank deploys an agentic mortgage underwriting system that iteratively investigates applicants: it performs initial risk scoring, requests documentation based on risk indicators, adapts its investigation strategy based on applicant responses, and terminates when sufficient information is gathered or confidence thresholds trigger human escalation. \textit{Dimensional profile: HIC + institutional frame + adaptive goals + stateful.}

During routine monthly fairness monitoring, the compliance team detects a significant disparity: Hispanic applicants have a 65\% approval rate compared to 82\% for white applicants. This represents a clear violation of the 80\% rule commonly used in disparate impact analysis (65/82 = 79.3\%). Following the Tier 3 escalation pathway for critical issues (see \Cref{fig:agents3-escalation-example}), the compliance analyst immediately notifies the Chief Risk Officer, who halts the system and escalates to the CEO within hours of detection (\Cref{fig:agents3-incident-email-discovery}).

% Credit Underwriting Incident - Discovery Email from CRO to CEO

\begin{figure}[htbp]
\centering
\begin{tikzpicture}[
    every node/.style={inner sep=0pt}
]

% Document frame
\node[
    draw=gray-400,
    line width=0.5pt,
    fill=white,
    minimum width=13.5cm,
    minimum height=14.0cm,
    rounded corners=2pt
] (page) at (0,0) {};

% Email header fields
\node[font=\scriptsize\sffamily, text=gray-600, anchor=west] at ([xshift=0.5cm, yshift=-0.4cm]page.north west) {\textbf{From:}};
\node[font=\scriptsize\sffamily, text=gray-900, anchor=west] at ([xshift=1.6cm, yshift=-0.4cm]page.north west) {Sarah Chen, Chief Risk Officer <s.chen@firstnational.com>};

\node[font=\scriptsize\sffamily, text=gray-600, anchor=west] at ([xshift=0.5cm, yshift=-0.75cm]page.north west) {\textbf{To:}};
\node[font=\scriptsize\sffamily, text=gray-900, anchor=west] at ([xshift=1.6cm, yshift=-0.75cm]page.north west) {Michael Torres, Chief Executive Officer <m.torres@firstnational.com>};

\node[font=\scriptsize\sffamily, text=gray-600, anchor=west] at ([xshift=0.5cm, yshift=-1.1cm]page.north west) {\textbf{Cc:}};
\node[font=\scriptsize\sffamily, text=gray-900, anchor=west] at ([xshift=1.6cm, yshift=-1.1cm]page.north west) {Jennifer Walsh, General Counsel; Board Risk Committee Chair};

\node[font=\scriptsize\sffamily, text=gray-600, anchor=west] at ([xshift=0.5cm, yshift=-1.45cm]page.north west) {\textbf{Date:}};
\node[font=\scriptsize\sffamily, text=gray-900, anchor=west] at ([xshift=1.6cm, yshift=-1.45cm]page.north west) {June 15, 2024, 3:12 PM};

\node[font=\scriptsize\sffamily, text=gray-600, anchor=west] at ([xshift=0.5cm, yshift=-1.8cm]page.north west) {\textbf{Subject:}};
\node[font=\scriptsize\bfseries\sffamily, text=gray-900, anchor=west] at ([xshift=1.6cm, yshift=-1.8cm]page.north west) {URGENT: Fair Lending Violation Detected -- Agentic Underwriting System Halted};

% Priority indicator
\node[font=\tiny\sffamily, text=white, fill=red-900, rounded corners=2pt, inner sep=3pt, anchor=east] at ([xshift=-0.5cm, yshift=-0.4cm]page.north east) {URGENT};

% Horizontal rule
\draw[gray-300, line width=0.5pt] ([xshift=0.5cm, yshift=-2.15cm]page.north west) -- ([xshift=-0.5cm, yshift=-2.15cm]page.north east);

% Email body
\node[
    text width=12.5cm,
    anchor=north west,
    font=\scriptsize,
    align=left
] at ([xshift=0.5cm, yshift=-2.4cm]page.north west) {
Michael,\\[6pt]
I'm escalating a critical issue that requires your immediate awareness. Our compliance team's monthly fairness review identified a significant disparity in the agentic mortgage underwriting system this morning: Hispanic applicants have a 65\% approval rate compared to 82\% for white applicants. This violates the 80\% rule commonly used in disparate impact analysis (65/82 = 79.3\%) and constitutes a potential ECOA fair lending violation.\\[6pt]
\textbf{Actions taken in the last two hours:}\\[3pt]
\begin{tabular}{@{\hspace{1em}\textbullet~}p{11.5cm}@{}}
Halted the agentic underwriting system immediately per emergency stop protocol\\[1pt]
Reverted all mortgage applications to manual underwriting\\[1pt]
Preserved 12 months of cycle-level system logs in read-only archive\\[1pt]
Convened incident response team: Jennifer Walsh (GC) is leading legal strategy, Marcus Thompson (AI Governance), and Rita Gonzalez (Mortgage Systems Lead)\\
\end{tabular}\\[6pt]
I just got off the phone with Jennifer. As you know, ECOA violations create strict liability---intent doesn't matter. She's already reviewing our regulatory notification obligations and will have a timeline for CFPB self-reporting by tomorrow. We face potential enforcement action, class action exposure, and reputational risk.\\[6pt]
I'm requesting an emergency board briefing within 72 hours to approve the investigation scope and remediation budget. I'll send a detailed findings report once the investigation is complete.\\[8pt]
Sarah\\[6pt]
\textcolor{gray-400}{\rule{3cm}{0.5pt}}\\[3pt]
\textbf{Sarah Chen}~\textbar~Chief Risk Officer\\
First National Bank~\textbar~Tel: (555) 234-5678\\
\textcolor{slate-700}{s.chen@firstnational.com}\\[12pt]
\textit{\textcolor{gray-500}{\scriptsize This email and any attachments are confidential and intended solely for the addressee. If you have received this message in error, please notify the sender immediately and delete this message. Unauthorized use, disclosure, or copying is prohibited.}}
};

\end{tikzpicture}
\caption{Initial escalation email from the Chief Risk Officer to CEO upon detection of a fair lending violation. The email documents immediate containment actions taken within two hours of detection, consistent with the Tier 3 escalation pathway for critical issues.}
\label{fig:agents3-incident-email-discovery}
\end{figure}


Ten weeks later, with the investigation complete and remediation implemented, the CRO sends a closure report (\Cref{fig:agents3-incident-email-closure}). The investigation reveals a governance challenge unique to agentic systems: the discrimination did not originate in the scoring model itself, which was facially neutral and passed traditional fairness testing. Instead, bias emerged through \emph{how the system investigated applicants across cycles}---Hispanic applicants triggered more verification cycles, leading to higher abandonment rates before final decisions were rendered. The system was discriminating in its \emph{process}, not its \emph{decisions}.

% Credit Underwriting Incident - Closure Email from CRO to CEO

\begin{figure}[htbp]
\centering
\begin{tikzpicture}[
    every node/.style={inner sep=0pt}
]

% Document frame
\node[
    draw=gray-400,
    line width=0.5pt,
    fill=white,
    minimum width=13.5cm,
    minimum height=20.2cm,
    rounded corners=2pt
] (page) at (0,0) {};

% Email header fields
\node[font=\scriptsize\sffamily, text=gray-600, anchor=west] at ([xshift=0.5cm, yshift=-0.4cm]page.north west) {\textbf{From:}};
\node[font=\scriptsize\sffamily, text=gray-900, anchor=west] at ([xshift=1.6cm, yshift=-0.4cm]page.north west) {Sarah Chen, Chief Risk Officer <s.chen@firstnational.com>};

\node[font=\scriptsize\sffamily, text=gray-600, anchor=west] at ([xshift=0.5cm, yshift=-0.75cm]page.north west) {\textbf{To:}};
\node[font=\scriptsize\sffamily, text=gray-900, anchor=west] at ([xshift=1.6cm, yshift=-0.75cm]page.north west) {Michael Torres, Chief Executive Officer <m.torres@firstnational.com>};

\node[font=\scriptsize\sffamily, text=gray-600, anchor=west] at ([xshift=0.5cm, yshift=-1.1cm]page.north west) {\textbf{Cc:}};
\node[font=\scriptsize\sffamily, text=gray-900, anchor=west] at ([xshift=1.6cm, yshift=-1.1cm]page.north west) {Jennifer Walsh, General Counsel; Board Risk Committee Chair};

\node[font=\scriptsize\sffamily, text=gray-600, anchor=west] at ([xshift=0.5cm, yshift=-1.45cm]page.north west) {\textbf{Date:}};
\node[font=\scriptsize\sffamily, text=gray-900, anchor=west] at ([xshift=1.6cm, yshift=-1.45cm]page.north west) {August 30, 2024, 4:47 PM};

\node[font=\scriptsize\sffamily, text=gray-600, anchor=west] at ([xshift=0.5cm, yshift=-1.8cm]page.north west) {\textbf{Subject:}};
\node[font=\scriptsize\bfseries\sffamily, text=gray-900, anchor=west, text width=10cm] at ([xshift=1.6cm, yshift=-1.8cm]page.north west) {CLOSED: Fair Lending Incident FL-2024-003 -- Findings and Remediation};

% Horizontal rule
\draw[gray-300, line width=0.5pt] ([xshift=0.5cm, yshift=-2.15cm]page.north west) -- ([xshift=-0.5cm, yshift=-2.15cm]page.north east);

% Email body
\node[
    text width=12.5cm,
    anchor=north west,
    font=\scriptsize,
    align=left
] at ([xshift=0.5cm, yshift=-2.4cm]page.north west) {
Michael,\\[6pt]
I'm pleased to confirm that Incident FL-2024-003 is now closed. The system has been remediated and redeployed. Here's what we found and what we've changed.\\[6pt]
\textbf{Root Cause: Process-Based Discrimination.} The discrimination wasn't in the scoring model---Cycle 1 risk assessment was facially neutral and passed our traditional fairness tests. The problem was \textit{how the system investigated applicants across subsequent cycles}:\\[3pt]
\begin{tabular}{@{\hspace{1em}\textbullet~}p{11.5cm}@{}}
Hispanic applicants triggered more verification cycles (5.2 avg vs. 3.8 for white applicants)\\[1pt]
The system had learned to flag shorter U.S. employment tenure for extra scrutiny---a proxy for national origin\\[1pt]
Extended verification led to higher abandonment: 28\% of Hispanic applicants withdrew vs. 12\% of white applicants\\
\end{tabular}\\[3pt]
The system wasn't discriminating in its \textit{decisions}---it was discriminating in its \textit{process}. Applicants who would have been approved were dropping out because we were investigating them more aggressively.\\[6pt]
\textbf{Remediation Implemented.} Rita Gonzalez and Marcus Thompson led the technical remediation. Before redeploying, we made four changes: (1) prohibited employment tenure from influencing investigation depth; (2) added cycle-count parity monitoring that flags deviations >20\% from demographic medians; (3) implemented abandonment tracking by protected class; and (4) retrained and revalidated the system for both outcome \textit{and} process fairness.\\[6pt]
\textbf{Regulatory and Applicant Notification.} Jennifer filed our self-report with the CFPB on July 12th, within the 30-day window the board approved. Her team has also sent notification letters to all 847 affected Hispanic applicants from the past 12 months, offering expedited re-review with human oversight, waived fees, and priority processing. So far, 312 have requested re-review.\\[6pt]
\textbf{What We're Changing Going Forward.} This one caught us off guard. Our fairness testing was focused on approval rates---we weren't looking at how the system \textit{investigated} people differently. I talked with Marcus yesterday, and he's already updating our AI Governance framework to treat ``iteration bias'' as a distinct risk category. We're revising our validation protocols to require fairness testing across the full investigation process, not just final decisions. We're also adding this to the quarterly model risk review.\\[6pt]
I'll present the full post-incident review at next month's board meeting, including proposed policy updates for all agentic systems.\\[8pt]
Sarah\\[6pt]
\textcolor{gray-400}{\rule{3cm}{0.5pt}}\\[3pt]
\textbf{Sarah Chen}~\textbar~Chief Risk Officer\\
First National Bank~\textbar~Tel: (555) 234-5678\\
\textcolor{slate-700}{s.chen@firstnational.com}\\[12pt]
\textit{\textcolor{gray-500}{\scriptsize This email and any attachments are confidential and intended solely for the addressee. If you have received this message in error, please notify the sender immediately and delete this message. Unauthorized use, disclosure, or copying is prohibited.}}
};

\end{tikzpicture}
\caption{Closure email summarizing investigation findings and remediation actions. The root cause---process-based discrimination through unequal verification burdens---represents a failure mode unique to agentic systems that traditional outcome-focused fairness testing would not detect.}
\label{fig:agents3-incident-email-closure}
\end{figure}


\textbf{Governance Principles Illustrated}:
\begin{itemize}
\item \textbf{Agentic fairness risk}: Discrimination emerges through \emph{how} the system iterates (process), not just final outcomes.
\item \textbf{Cross-cycle accountability}: Traditional fairness testing (outcome parity) is insufficient; organizations must audit investigation process across cycles.
\item \textbf{Adaptation constraints}: System learning must be constrained to prevent adaptation from introducing prohibited proxies.
\item \textbf{Termination parity}: Cycle-count monitoring ensures investigation burdens are distributed fairly across demographic groups.
\end{itemize}

These technical and operational controls require clear organizational ownership and accountability structures---the subject we turn to next.

