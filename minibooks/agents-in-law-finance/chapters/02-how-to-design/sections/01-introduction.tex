% ============================================================================
% 01-introduction.tex
% Introduction and Framework
% Part of: Chapter 2 - How to Design an Agent
% ============================================================================

\section{Introduction}
\label{sec:agents2-intro}

\glsadd{agentic-system}
\glsadd{gpa}
\glsadd{iat}
\glsadd{agent}
The previous chapter introduced the GPA+IAT framework: a way to recognize agentic systems through Goals, Perception, Action, Iteration, Adaptation, and Termination. Now we turn to the practical: how do you design one?

The answer starts with a simple observation: agentic systems are meant to do real work, whether augmenting human professionals, automating routine workflows, or even replacing entire organizational functions. Whatever the ambition, the system must handle the same work those humans and organizations currently perform. A contract lifecycle management agent must do what a legal operations team does; a portfolio monitoring agent must ``act like'' an analyst; a due diligence agent must perform the same tasks as junior associates reviewing acquisition targets. This means an agentic system requires the same structural capabilities as a professional team: receiving and understanding work, coordinating action, and operating under governance controls.

Notably, these structural capabilities don't depend on implementation choices. Building agents inevitably requires choosing between Python and C\#, between GPT and Claude, between libraries like LangChain or custom implementations---but the design decisions that matter transcend these details. Many commercial providers and open source libraries obscure the key architectural questions, and you will learn best by building your own conceptual foundation before adopting any particular vendor's framework.

\begin{keybox}[title={Agents are not magic; they are architecture}]
	Behind each architectural question lies a design decision with real tradeoffs---capabilities that make agents useful require concrete choices about how systems should work, and those choices determine not just what the system can do but how reliably it performs, how it fails, and what controls remain available when things go wrong.
\end{keybox}

This chapter organizes those architectural decisions into ten questions: the kind you should be asking whether you're building, evaluating, deploying, or governing an agentic system.

% ----------------------------------------------------------------------------
% The Ten Questions Framework
% ----------------------------------------------------------------------------

\subsection{The Ten Questions}
\label{sec:agents2-ten-questions}

Designing an agent means answering these ten questions. The capabilities an agent needs are not determined by the technology; they are determined by the work. And work has structure that any system, human or artificial, must accommodate.

Examine how a law firm operates. Work arrives through defined channels: a client calls, court filings appear on the docket, or an originating attorney refers a matter to a specialist. That work typically arrives as instructions that need clarification, and associates quickly learn to read between the lines, developing heuristics for what partners actually want rather than what they literally said.

Research demands access to the right databases (Westlaw, Lexis, PACER, internal matter management systems) and thoughtful search strategies. Actions like filing motions or sending client letters have real consequences and require appropriate authorization and privilege review. Institutional knowledge accumulates in case files, precedent databases, and practice group work product repositories.

Complex matters break down into workstreams with dependencies: discovery must complete before summary judgment motions, corporate due diligence must finish before closing, compliance audits must conclude before regulatory filings. Each work product has clear completion criteria and quality standards.

Associates know when to escalate to partners: when legal issues exceed their experience, when client relationship implications arise, or when matter budgets risk overruns. Teams coordinate across practice groups, with corporate attorneys working alongside litigators on acquisition disputes, or securities lawyers collaborating with compliance specialists on disclosure obligations.

Throughout, ethics rules and professional responsibility standards keep the whole operation within bounds: conflicts checks before engagement, confidentiality protections during representation, and privilege safeguards in every communication.

A discretionary portfolio management team follows the same pattern. Market data and research flow through defined feeds, and analysts must interpret investment committee mandates that leave room for judgment. Research requires access to financial databases, company filings, and market intelligence. Trades have real-world consequences that demand compliance checks before execution. Position history and investment theses persist across quarters, informing future decisions. Portfolio construction breaks down into sector allocation, security selection, and risk management, each with its own completion criteria. Analysts escalate to portfolio managers when positions approach limits, teams coordinate across asset classes, and regulatory controls ensure fiduciary compliance throughout.

These structural parallels are not coincidental. Law firms and investment teams are both \textit{cognitive work systems}: organizations that process information, make decisions, and take consequential actions under uncertainty \parencite{hollnagel2005joint,hollan2000distributed}. Agentic systems are cognitive work systems too \parencite{wang2024llmagents,rao1995bdi}, which means they face the same architectural challenges and require the same structural capabilities.

This mapping has practical implications. When you evaluate a contract review agent, you can ask the same questions you would ask about a junior associate: How does it know which clauses require partner review? How does it maintain client confidentiality across concurrent matters? When you design governance for a regulatory compliance monitoring agent, you can draw on the same frameworks that govern compliance departments: approval workflows, audit trails, exception handling procedures. When you communicate with technical teams building a litigation support agent or a portfolio optimization agent, you can use organizational language they will understand: escalation paths, authorization levels, quality control checkpoints.

Table~\ref{tab:agents2-framework} lists these ten questions in the order an agent encounters them during execution.

\begin{table}[htbp]
	\centering
	\caption{Architectural questions for agentic systems}
	\label{tab:agents2-framework}
	\small
	\begin{tabular}{p{0.18\textwidth}p{0.70\textwidth}}
		\toprule
		\textbf{Section}                                & \textbf{Architectural Question}                    \\
		\midrule
		\hyperref[sec:agents2-triggers]{Triggers}       & How does the agent know when it has work to do?    \\
		\hyperref[sec:agents2-intent]{Intent}           & How does the agent understand what is being asked? \\
		\hyperref[sec:agents2-perception]{Perception}   & How does the agent find things out?                \\
		\hyperref[sec:agents2-action]{Action}           & How does the agent make things happen?             \\
		\hyperref[sec:agents2-memory]{Memory}           & How does the agent remember things?                \\
		\hyperref[sec:agents2-planning]{Planning}       & How does the agent break a big job into steps?     \\
		\hyperref[sec:agents2-termination]{Termination} & How does the agent know when it is done?           \\
		\hyperref[sec:agents2-escalation]{Escalation}   & How does the agent know when to ask for help?      \\
		\hyperref[sec:agents2-delegation]{Delegation}   & How does the agent work with other agents?         \\
		\hyperref[sec:agents2-governance]{Governance}   & How do we design systems that can be governed?     \\
		\bottomrule
	\end{tabular}
\end{table}
\glsadd{trigger}\glsadd{intent}\glsadd{perception}\glsadd{tools}\glsadd{memory}\glsadd{planning}\glsadd{escalation}\glsadd{delegation}

Each section addresses one question through organizational analogies, architectural concepts, domain-specific considerations for law and finance, and governance implications. You can read sequentially for cumulative understanding, jump directly to whichever question matters most, or skip to the end for synthesis.

Lastly, remember that, like other human processes or software systems, agents require governance. Chapter~3 (\textit{How to Govern an Agent}) addresses compliance frameworks and controls in detail.
