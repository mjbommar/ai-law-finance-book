% ============================================================================
% CHAPTER 2: How to Design an Agent
% Architectures, Protocols, and Technical Evaluation
% ============================================================================
% This file serves as a subfile wrapper for Chapter 2 content.
% When compiled standalone, it produces a complete document.
% When included via \subfile, it integrates into the main book.
% ============================================================================

\documentclass[../../main.tex]{subfiles}

\begin{document}

% ============================================================================
% CHAPTER INTRODUCTION
% ============================================================================

\begin{highlightbox}[title={Chapter Overview}]
This chapter translates the conceptual foundations of Part I into architectural decisions. We organize agent design around \textbf{ten fundamental questions} that mirror the flow of work through any agent system:

\begin{enumerate}
  \item How does the agent know it has work? (Triggers)
  \item How does the agent understand what is being asked? (Intent)
  \item How does the agent find what it needs? (Perception)
  \item How does the agent make things happen? (Action)
  \item How does the agent remember? (Memory)
  \item How does the agent break jobs into steps? (Planning)
  \item How does the agent know when it is done? (Termination)
  \item When does the agent ask for help? (Escalation)
  \item How does the agent work with other agents? (Delegation)
  \item How do we design systems that can be governed? (Governance Surface)
\end{enumerate}

Each question maps directly to architectural components and implementation choices.
\end{highlightbox}

% ============================================================================
% CONTENT SECTIONS
% ============================================================================

\section{Introduction}
\label{sec:intro}

\textbf{Agent.} \textit{Agentic.}

Few terms generate more confusion despite widespread use. While these words appear everywhere, from marketing copy to academic papers, their meanings remain contested and often unclear. Yet despite this definitional chaos, the underlying concepts are deeply intuitive and accessible.

At heart, agents are simply \textbf{``doers'' with a to-do}. As we unpack this accessible starting point, we will discover more explicit conditions for identification. But this four-word formulation captures something essential: agency requires both goals and the capacity to act toward them.

\subsection{Motivation and Approach}

The proliferation of ``agentic AI'' makes definitional clarity urgent. Existing work remains fragmented across purpose and discipline: computer scientists cite Russell and Norvig \parencite{russellnorvig2020aima}, philosophers reference Bratman \parencite{bratman1987intention}\glsadd{bdi-architecture}, legal scholars consult the Restatement of Agency \parencite{restatement2006agency}\glsadd{agency-relationship}, and commercial vendors seem untethered by anything other than sales.

Some of this fragmentation reflects genuinely different perspectives, such as whether we recognize agents by their \textit{internal properties} (mental states, intentions) or \textit{external manifestations} (observable behavior, delegated authority); a spectrum we explore in Section~\ref{sec:disciplines}. While theoretical considerations like these can be useful, it is now most critical that we \textbf{establish a practical framework} to guide communication and coordination.

The stakes for getting this right are tangible. By late 2025, courts worldwide had identified over 400 cases involving AI-generated hallucinations\glsadd{hallucination} in legal filings including fabricated citations, fictitious holdings, and nonexistent cases submitted to tribunals. The ABA's Formal Opinion 512 (July 2024) established that attorneys bear full responsibility for verifying AI-generated content. The opinion noted that leading legal AI systems ``hallucinate between 17\% and 33\% of the time'' \parencite{aba-formal-opinion-512}.

These failures share a common pattern: attorneys treated single-shot text generators as if they were research tools, when those systems lacked the ability to search authoritative databases, iterate to verify citations, or escalate uncertainty. An agentic legal research system (one exhibiting all six operational properties we introduce below) would validate citations through tool access, confirm holdings through iteration, and flag unverifiable sources through escalation. The distinction between genuinely agentic systems and sophisticated chatbots is not academic; it is essential for professional practice, regulatory compliance, and client protection.

For legal and financial applications, these six operational properties are \textit{necessary but not sufficient}. Professional deployment demands additional safeguards such as attribution to authoritative sources, auditable provenance, escalation protocols, and confidentiality controls that augment rather than replace the core framework. Section~\ref{sec:synthesis} addresses these professional deployment requirements in detail.

Building toward these operational requirements, we organize agency into three levels that correspond to the following questions:

\textbf{Level 1}: What makes \textit{something}, biological or otherwise, an agent?

\textbf{Level 2}: What makes computational systems agentic?

\textbf{Level 3}: How do traditional and AI-powered agentic systems differ?

Answering these progressive questions establishes a nested hierarchy with three levels, as illustrated in \Cref{fig:hierarchy}.\glsadd{three-level-hierarchy}

% Three-level hierarchy diagram: Agent > Agentic Software > Agentic AI
% Visual representation of subset relationships

\begin{figure}[htbp]
\centering
\begin{tikzpicture}[
    font=\sffamily,
    level1/.style={fill=gray-100, draw=gray-500, line width=1.5pt},
    level2/.style={fill=gray-200, draw=gray-600, line width=1.5pt},
    level3/.style={fill=slate-100, draw=slate-700, line width=2pt},
    label/.style={font=\sffamily\bfseries},
    example/.style={font=\sffamily\small, text=gray-700}
]

% Level 1: Agents (outermost)
\draw[level1] (0,0) circle (4.5cm);
\node[label, align=center] at (0,3.8) {Level 1:\\Agents};
\node[example, align=center] at (0,-3.6) {People, thermostats,\\software systems};

% Level 2: Agentic Software - adjusted label position
\draw[level2] (0,0) circle (3cm);
\node[label, align=center] at (0,2.1) {Level 2:\\Agentic Software};
\node[example, align=center] at (0,-2.1) {Expert systems,\\rule-based agents};

% Level 3: Agentic AI (innermost)
\draw[level3] (0,0) circle (1.5cm);
\node[label, align=center] at (0,0.3) {Level 3:\\Agentic AI};
\node[example, align=center, font=\sffamily\footnotesize] at (0,-0.6) {LLM-based\\agents};

% Annotations for properties - showing cumulative hierarchy
\node[
    align=left,
    font=\sffamily\small,
    text=gray-700,
    inner sep=8pt,
    outer sep=0pt
] at (6.95,0) {
    {\footnotesize\textsc{Properties}}\\[3pt]
    {\footnotesize Level 1 (3):}\\[1pt]
    \quad Goal\\[0.5pt]
    \quad Perception\\[0.5pt]
    \quad Action\\[5pt]
    {\footnotesize Levels 2 \& 3 (+3):}\\[1pt]
    \quad Iteration\\[0.5pt]
    \quad Adaptation\\[0.5pt]
    \quad Termination
};

\end{tikzpicture}
\caption{The three-level hierarchy of agency. Each level is a subset of the one above it: all agentic AI is agentic software, and all agentic software consists of agents.}
\label{fig:hierarchy}
\end{figure}


\subsection{Level 1: Minimal Agency}

We begin with the conceptual foundation. What is the absolute minimum required for something to qualify as an agent, whether human, organizational, or computational? Level 1 establishes this baseline, applicable across all domains and technologies.

\begin{definitionbox}[unbreakable,title={\textbf{Level 1: Agent (Mnemonic: GPA)}}]
\glsadd{agent}
\glsadd{gpa}
	An \keyterm{agent} is any entity that pursues goals through perception and action, with at least minimal discretion over which action to take in response to what it perceives.

	\textcolor{border-neutral}{\rule{\linewidth}{0.4pt}}

	\textbf{Minimal properties:}
	\begin{itemize}[nosep,leftmargin=1.5em]
		\item \textbf{G}oal: Clear objective or performance criterion
		\item \textbf{P}erception: Awareness of environment through sensing
		\item \textbf{A}ction: Capability to affect environment
	\end{itemize}
\end{definitionbox}

\textbf{Property Definitions and Falsification Tests:}

\textbf{Goal (G):}\glsadd{goal} A clear objective, task specification, or performance criterion that directs behavior. Goals may be simple (maintain temperature) or complex (maximize portfolio returns), provided by external principals or internally generated, fixed or dynamic.

\textit{Falsification:} If the entity responds identically regardless of desired outcomes, or transforms inputs mechanically without reference to success criteria, it lacks goals. Examples: compilers (execute predetermined transformations), pure lookup tables (no optimization target).

\textbf{Perception (P):}\glsadd{perception} Awareness of environment through sensing capabilities such as sensors, APIs, database access, or document reading. The environment need not be physical; abstract spaces (contract negotiations, market data) qualify. \textit{Legal example:} An agent perceives via EDGAR/Westlaw APIs to read regulatory filings and case law, observes which queries return hits, and uses those observations to refine subsequent searches.

\textit{Falsification:} If the entity operates identically regardless of environmental state, or cannot observe consequences of its own actions, it lacks perception. Examples: open-loop controllers (no feedback), write-only systems.

\textbf{Action (A):}\glsadd{action} Capability to affect environment through actuators such as physical forces, variable modifications, tool invocations, API calls, or command execution. Requires minimal discretion: selecting among at least two possible actions contingent on perceptions.

\textit{Falsification:} If the entity cannot modify its environment, or executes exactly one predetermined sequence regardless of circumstances, it lacks action. The discretion threshold is ``$\geq$2 policies contingent on perceptions.'' Examples that fail: pure sensors (read-only), fixed scripts with zero conditional logic, and single-shot transformers that apply one transformation regardless of input. Sophistication alone does not confer agency; a complex translation model still lacks discretion if it cannot choose among alternative actions based on what it observes.

This trinity forms the conceptual bedrock of agency, equally applicable to humans navigating social contexts, organizations pursuing strategic objectives, biological organisms seeking survival, or computational systems executing tasks. While these three characteristics suffice for theoretical classification, practical deployment demands more.

\begin{examplebox}[title={Level 1 Agents in Practice}]
\begin{itemize}[nosep,leftmargin=1.5em]
\item \textbf{Paralegal}: Goals (comprehensive research), perceives (documents), acts (retrieves and organizes)
\item \textbf{Thermostat}: Goals (target temperature), perceives (sensor readings), acts (heating/cooling)
\item \textbf{Organization}: Goals (market position), perceives (competitive dynamics), acts (coordinated initiatives)
\end{itemize}
\end{examplebox}

Readers from philosophy, psychology, or law may find our inclusion of thermostats jarring. Traditional definitions often require more: philosophers may demand intentional mental states or consciousness; psychologists emphasize self-regulation and reflective awareness; legal scholars define agency through consensual fiduciary relationships; economists presuppose self-interested preference orderings. A thermostat satisfies none of these richer criteria.

\begin{keybox}[title={\textbf{A Note on Definitional Breadth}}]
We adopt a property-based definition deliberately. Following \textcite{dennett1987intentional}'s intentional stance, we treat agency as a predictive attribution: when it is \textit{useful} to describe an entity as pursuing goals, perceiving its environment, and taking action, we call it an agent.
\end{keybox}

Rather than adjudicating centuries of debate about what agency ``really'' requires, we identify the observable properties that consistently appear where entities exhibit goal-directed behavior and that matter for governance. This low definitional floor is not an oversight; it allows the gradient from minimal agents through agentic systems to AI agents to do meaningful analytical work. The thermostat and the AI legal research assistant are both agents under our framework, but they differ profoundly in autonomy, adaptation, and governance requirements. Those differences, not the shared label, are what matter for professional practice.

These examples span vastly different domains, yet each satisfies the same three-property test. \textbf{Distinguishing agents from non-agents reveals equally critical boundaries.} Consider tools we encounter daily that, despite their utility, fail to meet our criteria. A calculator, whether handheld or embedded in a spreadsheet, transforms inputs into outputs but pursues no objectives of its own; it waits passively for instructions. A legal research database like Westlaw or EDGAR contains vast information and responds to queries, yet it lacks independent goals or the capacity to act on its own initiative. Even a single ChatGPT response, however sophisticated, represents a one-shot generation rather than iterative goal pursuit: the system produces output and stops, without perceiving whether that output succeeded or adapting its approach.

This baseline framework illuminates the essence of agency, yet professional applications demand more. The gap between a simple thermostat cycling toward a temperature target and an AI system conducting legal research spans more than technological sophistication: it requires architectural elements that ensure reliability, adaptability, and accountability. Legal research tools, portfolio management systems, and document review platforms operate in environments where stakes are high and errors costly. These operational realities shape our expanded framework for deployable agentic systems.

\subsection{Operational Definition: Agentic Systems}

While Level 1 establishes what makes something an agent, computational systems in production require three additional properties beyond the minimal three. These six properties together define what we call \keyterm{agentic systems}: the operational standard that bridges conceptual agency and real-world deployment. Both traditional software (Level 2) and AI-powered implementations (Level 3) can achieve this operational standard, though they differ fundamentally in how they realize each property.

\begin{definitionbox}[unbreakable,title={\textbf{Operational Definition: Agentic System (GPA+IAT)}}]
\glsadd{agentic-system}
\glsadd{iat}
	An \keyterm{agentic system} is a goal-directed agent that repeatedly perceives and acts in its environment, adapting from observations until clear termination conditions are met (explicit or implicit).

	\textcolor{border-neutral}{\rule{\linewidth}{0.4pt}}

	\textbf{Additional properties beyond Level 1:}
	\begin{itemize}[nosep,leftmargin=1.5em]
		\item \textbf{I}teration: Repeat perceive-act cycles, not single-shot
		\item \textbf{A}daptation: Adjust strategy based on feedback/results
		\item \textbf{T}ermination: Clear stopping conditions (explicit or implicit)
	\end{itemize}
\end{definitionbox}

Together with the three Level 1 properties, agentic systems exhibit six operational properties: Goal, Perception, Action, Iteration, Adaptation, and Termination. For mnemonic convenience, we write this as \textbf{GPA + IAT}, representing the foundational three (Goal, Perception, Action) plus the three that distinguish operational systems (Iteration, Adaptation, Termination).

These six properties emerged from decades of agent research as commonly recognized operational requirements for reliable computational deployment. While not formally proven as minimum necessary, they consistently appear across deployed systems in domains from robotics to enterprise software, reflecting lessons from fielding real-world implementations. Section~\ref{sec:history} traces how each property became recognized as essential through both theoretical development and practical experience.

\textbf{Additional Property Definitions and Falsification Tests:}

\textbf{Iteration (I):}\glsadd{iteration} Multiple perceive-act cycles with state preservation across rounds. The entity repeatedly gathers information, takes action, observes results, and continues (not single-shot processing). Crucially, the system must perceive outcomes of prior actions and update subsequent actions accordingly within the same goal pursuit.

\textit{Falsification:} If the entity processes input once and produces output without maintaining state across cycles, it lacks iteration. Merely repeating the same action without incorporating new observations does not qualify. \textit{Batching vs. iteration:} Batched one-pass pipelines processing multiple items sequentially are not iteration unless the system observes outcomes from earlier items and modifies its approach for later items based on those observations. Examples that fail iteration test: single ChatGPT response (one-shot), batch processors applying identical logic to each item without inter-item learning.

\textbf{Adaptation (A):}\glsadd{adaptation} Strategy modification based on accumulated observations and feedback within a session or task. The entity adjusts its approach when initial attempts fail, learns which actions succeed, updates its policy based on results.

This definition focuses on \textit{session-level adaptation}: modifying behavior within a single task execution based on immediate feedback. This differs from \textit{cross-session learning}, where a system improves through model retraining across multiple tasks (for example, fine-tuning an LLM on user feedback). For our purposes, within-session adaptation is the operational property that distinguishes agentic systems from static tools. In professional contexts (such as procurement or vendor evaluation), be explicit about which type of learning is in scope.

\textit{Falsification:} If the entity applies identical logic regardless of outcomes, or cannot modify its approach when initial strategies fail, it lacks adaptation. Fixed rules that never change based on results do not qualify, even if they handle diverse inputs. Examples: basic thermostats (fixed on/off rules), static pattern matchers, rigid workflows that do not adjust to failures.

\textbf{Termination (T):}\glsadd{termination} Clear stopping conditions ensuring bounded operation. Termination may be \textit{implicit} (goal satisfaction, reaching target state, exhausting search space) or \textit{explicit} (resource budgets, time limits, maximum iterations, escalation triggers, confidence thresholds).

\textit{Falsification:} If the entity has no mechanism for recognizing when to stop, or could cycle indefinitely without bounds, it lacks proper termination. Entities requiring external intervention to halt do not meet this criterion. Examples: infinite loops with no exit condition, systems that run until manually killed.

With all six properties defined, we can identify agentic systems across professional domains:

\begin{examplebox}[title={Agentic Systems in Professional Practice}, breakable=false]
\begin{itemize}[nosep,leftmargin=1.5em]
\item \textbf{Legal research assistant}: Iterates through queries, adapts based on results, terminates when scope exhausted
\item \textbf{Contract analysis}: Iterates clause-by-clause, adapts to document type, terminates with risk report
\item \textbf{Portfolio rebalancing}: Iterates on positions, adapts to market conditions, terminates at target allocation
\item \textbf{Fraud detection}: Iterates on transactions, adapts thresholds to new patterns, terminates with alerts
\end{itemize}
\end{examplebox}

The relationship between levels clarifies important boundaries. Every agentic system qualifies as an agent (possessing the minimal three properties), but the reverse does not hold. Many agents lack the operational sophistication of agentic systems. A basic mechanical thermostat illustrates this gap: it has five properties (goal: maintain temperature; perception: sensor readings; action: heating/cooling; iteration: continuous monitoring; termination: implicit when target reached), but lacks adaptation, applying fixed on/off rules without modifying strategy based on outcomes. It responds to temperature changes reactively but does not learn patterns or adjust thresholds. In contrast, a \textit{smart} thermostat qualifies as a full agentic system: it learns occupancy patterns, adjusts heating schedules based on observed behavior, and modifies its strategy when energy costs spike. This distinction between reactive control (basic thermostat) and adaptive learning (smart thermostat) clarifies why minimal agency (Level 1, three properties) differs from operational agentic systems (six properties). Similarly, a human paralegal demonstrates all six properties behaviorally but operates through cognitive processes rather than discrete computational cycles.

This operational framework now raises the implementation question: \textit{How} do computational systems realize these six properties? The answer reveals an architectural distinction. Some systems use traditional programming (rules, algorithms, control logic) to manage planning and orchestration. Others employ AI/ML, particularly large language models, for these functions. We distinguish these as Level 2 (traditional) and Level 3 (AI-powered). Critically, this distinction is architectural, not evaluative: neither approach is inherently superior, and the boundary between them remains fluid and context-dependent.

\subsection{Level 2: Traditional Agentic Software}

Level 2 represents the first computational instantiation of agentic systems. These systems achieve all six operational properties through explicit programming: rules, conditional logic, algorithms, and control flow. What defines Level 2 is runtime behavior: decisions flow through programmed logic paths rather than learned models. Whether a system was coded decades ago or yesterday, if its decision-making follows deterministic rules at runtime, it operates at Level 2.

\begin{definitionbox}[unbreakable,title={\textbf{Level 2: Traditional Agentic Software}}]
\glsadd{agentic-system}
	Traditional agentic software implements the six operational properties using rules, algorithms, or deterministic logic.

	\textcolor{border-neutral}{\rule{\linewidth}{0.4pt}}

	\textbf{Additional properties beyond agentic systems:}
	\begin{itemize}[nosep,leftmargin=1.5em]
		\item \textit{No new properties (same 6 as agentic systems)}
	\end{itemize}
\end{definitionbox}

Traditional agentic software uses rules, algorithms, or deterministic logic to implement all six properties. Planning and orchestration are explicitly coded through conditional logic, state machines, or control systems. Level 2 systems can be extremely sophisticated: a conflict checking system might employ graph analysis for relationship detection, fuzzy matching for entity resolution, and adaptive threshold tuning, all implemented through traditional programming techniques. The key architectural characteristic is that decision logic is specified by programmers at design time.

To illustrate, a conflicts checking system exemplifies Level 2: it has goals (identify potential conflicts), perception (scans firm databases for client/matter relationships), action (flags matches and escalates to ethics committee), iteration (continuous monitoring as new matters are opened), adaptation (adjusts matching thresholds based on false positive rates), and termination (stops when matter is cleared or rejected). The entity achieves these properties through explicitly programmed logic like rules for relationship detection, graph traversal algorithms for indirect conflicts, and configurable thresholds for fuzzy name matching.

Level 2 systems achieve all six operational properties through traditional software engineering. Their performance depends on design quality, domain expertise, and implementation rigor, not on whether they employ AI/ML techniques. A well-engineered Level 2 system can substantially outperform a poorly designed Level 3 system in reliability, predictability, and effectiveness for its intended domain. Beyond conflicts checking, traditional agentic software appears throughout professional practice:

\begin{examplebox}[title={Traditional Agentic Software Examples}]
\begin{itemize}[nosep,leftmargin=1.5em]
\item \textbf{Legal}: Conflicts checking with graph traversal and fuzzy matching; docketing systems with deadline tracking
\item \textbf{Financial}: Trading compliance monitoring; regulatory threshold alerts; invoice validation workflows
\end{itemize}
\end{examplebox}

\subsection{Level 3: AI-Powered Agentic Systems}

\glsadd{llm}Level 3 systems use AI/ML, particularly large language models, to manage planning, orchestration, and adaptation. The architectural distinction from Level 2 is straightforward: where Level 2 systems execute explicitly programmed decision logic, Level 3 systems employ neural networks (especially LLMs) or other learned models for these functions. In practice, modern Level 3 systems are typically hybrid: LLMs handle high-level planning and natural language interaction, while traditional code manages structured operations like database queries or API calls. This is what most practitioners mean by ``AI agents.''

\begin{definitionbox}[unbreakable,title={\textbf{Level 3: AI-Powered Agentic Systems}}]
\glsadd{ai-agent}
	AI-powered agentic systems implement the six operational properties through strategic integration of artificial intelligence, typically neural network models such as large language models (LLMs) and vision-language models (VLMs), with traditional computational components.

	\textcolor{border-neutral}{\rule{\linewidth}{0.4pt}}

	\textbf{Additional properties beyond agentic systems:}
	\begin{itemize}[nosep,leftmargin=1.5em]
		\item \textit{No new properties (same 6 as agentic systems)}
	\end{itemize}
\end{definitionbox}

The architectural choice to use AI/ML for planning and orchestration has practical implications. LLMs enable natural language interfaces so users can specify goals conversationally rather than through structured formats. These neural network models handle pattern recognition tasks that would require extensive rule engineering. Yet this approach trades some predictability for flexibility: LLM outputs can vary across runs, and behavior may be harder to audit than explicit rule chains.\glsadd{llm-as-agent} The boundary between Level 2 and Level 3 can blur: is a system using gradient boosting for fraud detection Level 2 or Level 3? Today's practical dividing line focuses on whether LLMs manage high-level planning and orchestration.

Take an AI contract risk analyzer as a Level 3 exemplar. It possesses all six operational properties: goals (identifies and assesses contract risks), perception (reads contract text and clause context), action (flags problematic provisions, generates risk assessments), iteration (reviews document section by section), adaptation (adjusts risk scoring based on clause combinations and jurisdiction), and termination (stops when complete review is done or high-severity risk triggers immediate escalation). The LLM manages high-level planning such as deciding which clauses merit detailed analysis, how to interpret ambiguous language, and when to flag issues versus when to request human review, while traditional code handles document parsing, clause extraction, jurisdiction lookup, and risk score calculation. This hybrid architecture is typical of Level 3 systems: AI handles interpretation and strategic decisions, traditional programming handles structured data operations. Beyond contract analysis, AI-powered agentic systems are emerging across professional domains:

\begin{examplebox}[title={AI-Powered Agentic System Examples}]
\begin{itemize}[nosep,leftmargin=1.5em]
\item \textbf{Legal}: AI research assistants (iterative case law search); contract risk analyzers (clause interpretation); document review (adaptive classification)
\item \textbf{Financial}: AI trading assistants (market analysis); portfolio advisors (natural language recommendations); risk assessment (pattern recognition)
\end{itemize}
\end{examplebox}

Table~\ref{tab:property-levels} maps the progression from minimal agency (three properties) through agentic systems (six properties) to implementation paradigms (traditional vs. AI).

\begin{table}[htbp]
	\centering
	\footnotesize
	\begin{tabular}{@{}lccc@{}}
		\toprule
		\textbf{Property} & \textbf{Level 1} & \textbf{Level 2} & \textbf{Level 3} \\
		                  & \textbf{Agent}   & \textbf{Traditional} & \textbf{AI} \\
		\midrule
		Goal              & \tikz\fill[primary] (0,0) circle (0.08cm); & \tikz\fill[primary] (0,0) circle (0.08cm); & \tikz\fill[primary] (0,0) circle (0.08cm); \\
		Perception        & \tikz\fill[primary] (0,0) circle (0.08cm); & \tikz\fill[primary] (0,0) circle (0.08cm); & \tikz\fill[primary] (0,0) circle (0.08cm); \\
		Action            & \tikz\fill[primary] (0,0) circle (0.08cm); & \tikz\fill[primary] (0,0) circle (0.08cm); & \tikz\fill[primary] (0,0) circle (0.08cm); \\
		\cmidrule{1-4}
		Iteration         & \tikz\draw[primary] (0,0) circle (0.08cm); & \tikz\fill[key-base] (0,0) circle (0.08cm); & \tikz\fill[key-base] (0,0) circle (0.08cm); \\
		Adaptation        & \tikz\draw[primary] (0,0) circle (0.08cm); & \tikz\fill[key-base] (0,0) circle (0.08cm); & \tikz\fill[key-base] (0,0) circle (0.08cm); \\
		Termination       & \tikz\draw[primary] (0,0) circle (0.08cm); & \tikz\fill[key-base] (0,0) circle (0.08cm); & \tikz\fill[key-base] (0,0) circle (0.08cm); \\
		\midrule
		AI-Powered        & \tikz\draw[primary] (0,0) circle (0.08cm); & \tikz\draw[primary] (0,0) circle (0.08cm); & \tikz\fill[example-base] (0,0) circle (0.08cm); \\
		\bottomrule
	\end{tabular}
	\caption{Property requirements by level. Filled = required; empty = optional.}
	\label{tab:property-levels}
\end{table}

While Levels 2 and 3 share identical property requirements, they differ fundamentally in \textit{how} each property is implemented. Table~\ref{tab:implementation-contrast} contrasts implementation approaches.

\begin{table}[htbp]
	\centering
	\footnotesize
	\rowcolors{2}{white}{bg-note}
	\begin{tabular}{@{}l>{\raggedright\arraybackslash}p{3.8cm}>{\raggedright\arraybackslash}p{3.8cm}@{}}
		\toprule
		\textbf{Property} & \textbf{Level 2 (Traditional)} & \textbf{Level 3 (AI)} \\
		\midrule
		Goal & Config files, explicit targets & Natural language instructions \\
		Perception & APIs, SQL, regex & LLM understanding, semantic search \\
		Action & Function calls, API invocations & LLM tool orchestration \\
		Iteration & Control loops, state machines & LLM reasoning loop \\
		Adaptation & Rule tuning, A/B tests & Chain-of-thought,\glsadd{chain-of-thought} in-context learning \\
		Termination & Max iterations, timeouts & LLM goal satisfaction check \\
		\midrule
		Planning & Decision trees, rule engines & LLM-generated plans \\
		Logs & Structured audit trails & Reasoning traces \\
		\bottomrule
	\end{tabular}
	\caption{Implementation contrast: Level 2 vs Level 3.}
	\label{tab:implementation-contrast}
\end{table}

\subsection{Key Distinctions}

Having traced the progression from minimal agency (three properties) through agentic systems (six properties) to implementation paradigms (traditional vs. AI), we can now synthesize what this hierarchy reveals. Level 1 establishes conceptual qualification, the operational definition adds production-readiness requirements, and Levels 2 and 3 distinguish implementation paradigms. This structure clarifies three critical distinctions that cut through definitional confusion:

The critical distinction lies between \textit{properties} and \textit{implementation}. The major jump in capability occurs between Level 1 (three properties) and agentic systems (six properties). Levels 2 and 3, by contrast, have \textit{identical property requirements}, differing only in how those properties are implemented: through explicit rules and logic (Level 2) or through AI/ML models (Level 3).

This hierarchy permits precise terminology. An \textbf{``agent''} (noun) refers to anything that exhibits Level 1's three minimal properties. The adjective \textbf{``agentic''} describes systems meeting all six operational properties at the system level; we may also use it descriptively at the feature or behavior level (e.g., ``agentic behavior'' or ``agentic properties''), but reserve ``agentic system'' for six-of-six conformance. Finally, an \textbf{``AI agent''} is an agentic system specifically powered by AI/ML capabilities (Level 3).

These definitions allow clear exclusions. Compilers and databases lack goals entirely, failing even Level 1's minimal requirements. A single chatbot response, however sophisticated, lacks iteration and therefore does not qualify as an operational agentic system. Traditional ML classifiers (image recognizers, spam filters, sentiment analyzers) lack both iteration and autonomous goals, processing inputs without pursuing objectives across perceive-act cycles. Rule-based expert systems present a more nuanced case: they can be fully agentic (meeting all six operational properties) but are not AI-powered, placing them at Level 2 rather than Level 3.

This framework provides scaffolding for the historical and theoretical analysis that follows. Section~\ref{sec:practical} provides a practical decision rubric for immediate application. The remaining sections trace where these definitions came from and why they take this particular form, building toward the professional implications explored in Section~\ref{sec:furtherlearning}.

% ============================================================================
% 02-triggers.tex
% Q1: How Does an Agent Know When It Has Work to Do?
% Part of: Chapter 07 - Agents Part II: How to Build an Agent
% ============================================================================

\section{How Does an Agent Know When It Has Work to Do?}
\label{sec:agents2-triggers}

% ----------------------------------------------------------------------------
% Opening: Q1 Framing and Organizational Analogy
% ----------------------------------------------------------------------------

Consider how work reaches a professional. A client calls with an urgent question, the court docket updates with a new filing, the calendar reminds you that a motion is due tomorrow, and a junior associate realizes an issue exceeds their expertise and brings it to your office. These four channels define how work enters your day: the phone, the inbox, the calendar, and escalation from colleagues.

Agentic systems operate in the same way. A system with tools, memory, and planning capabilities remains idle until work arrives; the architectural question is how tasks enter the system and what events trigger execution.

\begin{definitionbox}[title={Triggers}]
\keyterm{Triggers} are the events that start agent execution. In practice, a trigger might be a docket alert, a price crossing a threshold, a calendar deadline coming due, or an internal ``I can't proceed safely'' signal from the agent itself. Without a trigger, even a highly capable system sits idle.
\end{definitionbox}

\begin{definitionbox}[title={Channels}]
\keyterm{Channels} are how triggers reach the agent. In professional practice, four channels cover almost all work intake:

\textbf{External feeds}: The world pushes work to you (court filings, market data, regulatory updates).

\textbf{Human prompts}: People request work directly (chat, email, collaboration platforms).

\textbf{Scheduled jobs}: Time itself triggers execution (deadlines, periodic checks, end-of-day).

\textbf{Escalation events}: Internal signals that ask for human help (budget exhaustion, low confidence).
\end{definitionbox}

Before an agent can reason or act, it must first notice that work exists. Channels are the sensory apparatus of the system: the ways it becomes aware of its environment and the tasks it must accomplish—just as a lawyer cannot respond to a motion they never received.

% Four Channel Types for Agent Triggers - 2x2 Grid Visualization
% Uses semantic color palette with neutral backgrounds and colored accents

\begin{figure}[htbp]
\centering
\begin{tikzpicture}[scale=0.85, every node/.style={scale=0.85},
    % Top row card style
    channel top/.style={
        rectangle,
        minimum width=5.4cm,
        minimum height=4.8cm,
        fill=white,
        draw=gray-300,
        line width=0.5pt,
        rounded corners=3pt,
        inner sep=0pt
    },
    % Bottom row card style
    channel bottom/.style={
        rectangle,
        minimum width=5.4cm,
        minimum height=4.8cm,
        fill=white,
        draw=gray-300,
        line width=0.5pt,
        rounded corners=3pt,
        inner sep=0pt
    },
    % Header text style - larger and bold
    header/.style={
        font=\sffamily\bfseries,
        anchor=north west
    },
    % Tagline style
    tagline/.style={
        font=\sffamily\scriptsize,
        text=gray-700,
        anchor=north west
    },
    % Example text style - smaller
    examples/.style={
        font=\sffamily,
        text=gray-600,
        anchor=north west,
        text width=4.4cm,
        scale=0.85,
        transform shape
    }
]

% Define positions - center cards symmetrically around origin
\def\hgap{2.9}   % Half the horizontal gap between card centers
\def\vgap{5.3}   % Vertical gap between cards

% ============================================================
% TOP LEFT: External Feeds (Green accent)
% ============================================================
\node[channel top] (feeds) at (-\hgap, 0) {};
\fill[example-base, rounded corners=2pt]
    ([xshift=0.15cm, yshift=-0.15cm]feeds.north west)
    rectangle ([xshift=0.45cm, yshift=0.15cm]feeds.south west);
\node[header, text=example-dark] at ([xshift=0.7cm, yshift=-0.4cm]feeds.north west)
    {\textbf{External Feeds}};
\node[tagline] at ([xshift=0.7cm, yshift=-0.9cm]feeds.north west)
    {The world pushes work to you};
\node[examples] at ([xshift=0.7cm, yshift=-1.4cm]feeds.north west) {
    \textbullet~Court filings\\
    \quad\textcolor{gray-400}{CM/ECF, PACER}\\[0.5ex]
    \textbullet~Market data\\
    \quad\textcolor{gray-400}{Bloomberg, Reuters}\\[0.5ex]
    \textbullet~Regulatory updates\\
    \quad\textcolor{gray-400}{EDGAR, Federal Register}\\[0.5ex]
    \textbullet~Research alerts\\
    \quad\textcolor{gray-400}{Westlaw, Lexis}
};

% ============================================================
% TOP RIGHT: Human Prompts (Blue/Slate accent)
% ============================================================
\node[channel top] (human) at (\hgap, 0) {};  % Right card at +4.4
\fill[definition-base, rounded corners=2pt]
    ([xshift=0.15cm, yshift=-0.15cm]human.north west)
    rectangle ([xshift=0.45cm, yshift=0.15cm]human.south west);
\node[header, text=definition-dark] at ([xshift=0.7cm, yshift=-0.4cm]human.north west)
    {\textbf{Human Prompts}};
\node[tagline] at ([xshift=0.7cm, yshift=-0.9cm]human.north west)
    {People request work directly};
\node[examples] at ([xshift=0.7cm, yshift=-1.4cm]human.north west) {
    \textbullet~Chat interfaces\\
    \quad\textcolor{gray-400}{Direct queries}\\[0.5ex]
    \textbullet~Email routing\\
    \quad\textcolor{gray-400}{Forwarded questions}\\[0.5ex]
    \textbullet~Collaboration tools\\
    \quad\textcolor{gray-400}{Slack, Teams}\\[0.5ex]
    \textbullet~Voice interfaces\\
    \quad\textcolor{gray-400}{Transcribed speech}
};

% ============================================================
% BOTTOM LEFT: Scheduled Jobs (Amber accent)
% ============================================================
\node[channel bottom] (sched) at (-\hgap, -\vgap) {};
\fill[key-base, rounded corners=2pt]
    ([xshift=0.15cm, yshift=-0.15cm]sched.north west)
    rectangle ([xshift=0.45cm, yshift=0.15cm]sched.south west);
\node[header, text=key-dark] at ([xshift=0.7cm, yshift=-0.4cm]sched.north west)
    {\textbf{Scheduled Jobs}};
\node[tagline] at ([xshift=0.7cm, yshift=-0.9cm]sched.north west)
    {Time itself triggers execution};
\node[examples] at ([xshift=0.7cm, yshift=-1.4cm]sched.north west) {
    \textbullet~Calendar deadlines\\
    \quad\textcolor{gray-400}{Motion due dates}\\[0.5ex]
    \textbullet~Compliance checks\\
    \quad\textcolor{gray-400}{Nightly monitoring}\\[0.5ex]
    \textbullet~End-of-day workflows\\
    \quad\textcolor{gray-400}{P\&L, reconciliation}\\[0.5ex]
    \textbullet~Recurring reports\\
    \quad\textcolor{gray-400}{Monthly, quarterly}
};

% ============================================================
% BOTTOM RIGHT: Escalation Events (Red accent)
% ============================================================
\node[channel bottom] (esc) at (\hgap, -\vgap) {};
\fill[caution-base, rounded corners=2pt]
    ([xshift=0.15cm, yshift=-0.15cm]esc.north west)
    rectangle ([xshift=0.45cm, yshift=0.15cm]esc.south west);
\node[header, text=caution-dark] at ([xshift=0.7cm, yshift=-0.4cm]esc.north west)
    {\textbf{Escalation Events}};
\node[tagline] at ([xshift=0.7cm, yshift=-0.9cm]esc.north west)
    {Internal signals requiring intervention};
\node[examples] at ([xshift=0.7cm, yshift=-1.4cm]esc.north west) {
    \textbullet~Budget exhaustion\\
    \quad\textcolor{gray-400}{Token or cost limits}\\[0.5ex]
    \textbullet~Low confidence\\
    \quad\textcolor{gray-400}{Uncertain results}\\[0.5ex]
    \textbullet~Approval gates\\
    \quad\textcolor{gray-400}{Filings, trades}\\[0.5ex]
    \textbullet~Errors and anomalies\\
    \quad\textcolor{gray-400}{Tool failures}
};

\end{tikzpicture}
\caption{Four channel types through which work reaches agentic systems. External feeds push events from outside systems; human prompts arrive through interactive interfaces; scheduled jobs trigger on time-based conditions; and escalation events signal internal limits requiring human intervention. All channels converge on the agentic system's event router.}
\label{fig:agents2-trigger-channels}
\end{figure}


% ----------------------------------------------------------------------------
% External Feeds
% ----------------------------------------------------------------------------

\subsection{External Feeds: The World Pushes Work to You}
\label{sec:agents2-external-feeds}

External feeds deliver events from systems outside the agent's direct control. The external system pushes notifications when events occur, much like receiving service of process rather than checking the courthouse daily to see if you have been sued.

\subsubsection{Legal and Regulatory Feeds}

Court docket systems like CM/ECF send email notifications when documents are filed. An agent monitoring litigation can receive these alerts, retrieve filed documents via PACER, analyze contents, and trigger appropriate responses. When opposing counsel files a motion, the agent alerts the litigation team, extracts key arguments, searches for responsive authority, and calculates response deadlines.

The SEC's EDGAR system publishes corporate filings with API access for programmatic retrieval. For corporate counsel, EDGAR feeds trigger review workflows: when a competitor files a 10-K, an agent retrieves the filing, extracts financial data and risk factors, compares them to your company's disclosures, and flags material differences. Regulatory agencies publish rules and guidance through the Federal Register and agency websites, enabling agents to monitor for changes that affect compliance obligations.

Legal research platforms like Westlaw and Lexis offer citator alerts when monitored cases are cited, distinguished, or overruled. An agent can subscribe to these alerts, retrieve and analyze citing opinions, and notify attorneys of developments affecting their arguments.

\subsubsection{Financial Market Feeds}

Financial institutions receive real-time market data through providers like Bloomberg and Reuters. These feeds push price updates and market events continuously. A portfolio management agent can subscribe to price alerts, receive notifications when thresholds are crossed, evaluate rebalancing rules, and either execute trades within risk limits or escalate to a portfolio manager.

Position and P\&L updates cascade through financial systems: when trades execute, position systems update holdings, risk agents recalculate exposure, compliance agents check concentration limits, and reporting agents update dashboards. This architecture treats agents as event processors that consume upstream events, reason about implications, and produce downstream events.

News feeds deliver breaking headlines, earnings announcements, and sentiment analytics in machine-readable format. When material news hits, agents can retrieve the content, assess sentiment, compare to historical patterns, and alert portfolio managers if the news appears significant.

\begin{highlightbox}[title={Speed vs. Reasoning: A Critical Distinction}]
Market data arrives at millisecond granularity. LLM-based reasoning operates at second-to-minute timescales. This fundamental mismatch determines where agents add value in financial workflows.

\textbf{Agents are not suited for:} High-frequency trading, market-making, latency-sensitive execution. These domains require deterministic algorithms operating at microsecond latencies. An LLM reasoning loop, even a fast one, cannot compete.

\textbf{Agents are suited for:} Strategic portfolio decisions, investment thesis development, rebalancing analysis, compliance monitoring, research synthesis. These tasks operate on timescales of minutes to hours, where reasoning quality matters more than latency.

\textbf{The architecture pattern:} Fast deterministic systems handle real-time data capture and threshold detection. When thresholds trigger (position approaching limit, price target hit, anomaly detected), they generate events that LLM agents process. The agent's role is strategic reasoning and recommendation, not execution speed. This complements latency-sensitive pipelines: keep the microsecond path deterministic, hand off to the agent only once an alert is raised.

Match agent capabilities to task requirements. Speed-critical tasks need traditional algorithms; reasoning-critical tasks need agents.
\end{highlightbox}

\subsubsection{Integration Patterns}

External feeds reach agents through \keyterm{webhooks} (HTTP callbacks for immediate notification) or \keyterm{message queues} (durable event streams with delivery guarantees). Webhooks work well for low-volume, time-sensitive events where immediate delivery matters and occasional missed events are acceptable. Message queues provide ordering, durability, and replay capabilities essential for regulated applications requiring audit trails. In practice, many systems use both: a portfolio management system might use webhooks to receive immediate notification when a stock price crosses a stop-loss threshold, while using message queues to process daily trade confirmations that require guaranteed delivery and audit logging.

% ----------------------------------------------------------------------------
% Human Prompts as Events
% ----------------------------------------------------------------------------

\subsection{Human Prompts as Events}
\label{sec:agents2-human-events}

Human prompts feel different from external feeds because they are interactive and synchronous. However, at the architectural level a human prompt is still just another event type: the user generates an event, the agent receives it through a channel, processes it, and responds. Treating prompts this way simplifies design, because all events can flow through common routing and prioritization logic rather than requiring separate code paths for ``chat'' versus ``background'' work.

\textbf{Chat interfaces} are the most direct channel. The associate types ``Find Fifth Circuit authority on personal jurisdiction for e-commerce defendants,'' the agent searches and presents summaries, and the associate follows up with refinements. The analyst asks for revenue growth comparisons across portfolio companies, receives a table, and requests additional filtering. Chat enables iterative clarification while maintaining architectural consistency: each message is simply an event processed through the standard agent loop, with tighter latency expectations than background tasks.

\textbf{Email routing} enables agents to process work arriving through existing communication channels. A general counsel forwards a business unit's compliance question to an agent mailbox; the agent extracts the question, searches relevant guidance, and emails back an assessment. The challenge is intent classification: email bodies are unstructured and may include forwarded threads with multiple topics.

\textbf{Collaboration platforms} like Slack and Teams allow agents to appear as team members. Users @mention the agent in channels, send direct messages, or use slash commands. The litigation team discussing strategy can invoke research directly in their coordination channel. Security requires authorization checks at the agent layer, since collaboration platforms may log responses and channels may include unauthorized viewers.

\textbf{Voice interfaces} work best for short, urgent requests where typing is impractical. They introduce transcription errors (legal jargon like ``Chevron deference'' may transcribe incorrectly) and authentication challenges. High-stakes voice requests should require explicit confirmation before execution.

% ----------------------------------------------------------------------------
% Scheduled Jobs
% ----------------------------------------------------------------------------

\subsection{Scheduled Jobs: Time as Trigger}
\label{sec:agents2-scheduled}

Some work follows predictable schedules rather than arriving from external events or human prompts: end-of-day reconciliation, monthly compliance reporting, quarterly reviews, annual filings. For these recurring tasks, time itself triggers execution.

\textbf{Calendar-driven deadlines} govern legal practice. Answer the complaint within 21 days. File motions 30 days before hearings. Respond to discovery within 30 days. Agents can monitor litigation calendars, calculate deadlines accounting for court holidays, schedule reminders as deadlines approach, and escalate if work remains incomplete. Sophisticated deadline agents go further, retrieving the complaint, extracting claims, generating draft answers with standard defenses, and presenting drafts for attorney review before filing. Financial institutions face similar deadline-driven work, from SEC reporting deadlines to tax filings to contractual obligations to lenders.

\textbf{Periodic compliance checks} run even when no external event triggers review. An investment compliance agent runs nightly to check portfolios against client guidelines and flag violations. A law firm conflicts agent retrieves new docket entries, extracts party names, and checks them against the conflicts database. These scheduled checks enable continuous monitoring that would be impractical manually across thousands of matters or client accounts.

\textbf{End-of-day workflows} in financial institutions reconcile trades, calculate valuations at market close, generate P\&L reports, and prepare risk reports for the next morning. At market close, an EOD agent retrieves final prices, marks positions to market, calculates P\&L, and identifies unexplained variances. The agent then distributes reports to stakeholders. If any step fails, the agent escalates rather than proceeding with incomplete data. Law firms run similar periodic workflows, reminding attorneys to enter time, generating draft invoices at month-end, and flagging anomalies for partner review.

% ----------------------------------------------------------------------------
% Escalation Events (Brief - Full Treatment in Section 09)
% ----------------------------------------------------------------------------

\subsection{Escalation Events: When Agents Reach Their Limits}
\label{sec:agents2-escalation-brief}

The previous three channel types bring work into the agent system from outside. Escalation events operate internally: the agent generates an event signaling it has reached a limit and requires human intervention, transferring control to human decision-makers when the agent cannot proceed autonomously.

Four escalation triggers appear most frequently:

\textbf{Budget exhaustion}: The agent approaches resource limits (token consumption, iteration counts, time limits, or cost caps) and must decide whether to stop or request additional budget.

\textbf{Low confidence}: Uncertainty is too high for autonomous action. Conflicting authority, novel situations, or results that seem implausible warrant human review.

\textbf{Approval requirements}: Certain actions require explicit human authorization regardless of the agent's confidence: filing court documents, sending client communications, executing large trades.

\textbf{Errors and anomalies}: Tools fail repeatedly, data is inconsistent, or the agent detects red flags that require human investigation.

\Cref{sec:agents2-escalation} provides comprehensive treatment of when and how agents should escalate to humans.

% ----------------------------------------------------------------------------
% Event Routing and Prioritization
% ----------------------------------------------------------------------------

\subsection{Event Routing and Prioritization}
\label{sec:agents2-routing}

With events arriving from multiple channels, agents need routing and prioritization logic. A law firm routes work similarly. Client calls go to appropriate attorneys, court filings route to the litigation coordinator, research requests go to assigned associates. Agent systems implement the same pattern through a central router that receives events, examines metadata, applies routing rules, and dispatches to appropriate handlers.

\textbf{Routing rules} map event attributes to handlers. Court filing notifications for Matter 12345 route to that matter's litigation agent. SEC filings by portfolio companies route to the monitoring agent. Routing can be static (predefined rules) or dynamic (classifiers that analyze content and identify topics). For multi-agent architectures, routing determines delegation: an orchestrator receives high-level tasks, classifies them, and routes to specialist agents.

\textbf{Priority queues} implement tiered processing. Urgent events (emergency motions, margin calls) enter the high-priority queue and are processed immediately, potentially interrupting lower-priority work. Routine tasks enter standard queues. Background work (database updates, model retraining) runs when resources are idle. Priority can be rule-based (certain event types always urgent) or adaptive (priority escalates as deadlines approach).

\textbf{Temporal constraints} require processing within specific windows. Court filings have deadlines, trading must occur during market hours, EOD reports must complete before the next morning. Agents track these constraints, calculate time remaining, and escalate priority as deadlines approach.

\textbf{Overload management} prevents cascading failures when events arrive faster than processing capacity. Rate limiting caps how many events agents accept per minute, protecting downstream APIs. Backpressure signals upstream systems to slow down. Load shedding drops low-priority work to preserve capacity for critical tasks during peak demand. During a market crash, trade execution and risk calculations take precedence; routine reporting can wait.

% ----------------------------------------------------------------------------
% Diagram: Event Routing Architecture
% ----------------------------------------------------------------------------

\begin{figure}[htbp]
\centering
\begin{tikzpicture}[
  node distance=1.5cm and 2.5cm,
  box/.style={rectangle, draw=border-definition, fill=bg-definition, thick, minimum width=2.8cm, minimum height=1cm, align=center, rounded corners=2pt, font=\small},
  source/.style={rectangle, draw=border-example, fill=bg-example, thick, minimum width=2.5cm, minimum height=0.9cm, align=center, rounded corners=2pt, font=\small},
  handler/.style={rectangle, draw=border-key, fill=bg-key, thick, minimum width=2.5cm, minimum height=0.9cm, align=center, rounded corners=2pt, font=\small},
  arrow/.style={-Stealth, thick, draw=primary},
  label/.style={font=\scriptsize\itshape, text=text-secondary}
]

% Event Sources (Left Column)
\node[source] (feed) {External Feeds\\{\scriptsize Court, SEC, Markets}};
\node[source, below=0.8cm of feed] (human) {Human Prompts\\{\scriptsize Chat, Email, Slack}};
\node[source, below=0.8cm of human] (sched) {Scheduled Jobs\\{\scriptsize EOD, Deadlines}};
\node[source, below=0.8cm of sched] (esc) {Escalations\\{\scriptsize Budget, Confidence}};

% Router/Dispatcher (Center)
\node[box, right=of human, yshift=-0.9cm, minimum height=3.5cm] (router) {
  \textbf{Event Router}\\[0.3em]
  {\scriptsize Classify}\\
  {\scriptsize Route}\\
  {\scriptsize Prioritize}\\
  {\scriptsize Queue}
};

% Agent Handlers (Right Column)
\node[handler, right=of router, yshift=1.5cm] (research) {Research Agent\\{\scriptsize Legal/Financial}};
\node[handler, below=0.8cm of research] (compliance) {Compliance Agent\\{\scriptsize Checks, Alerts}};
\node[handler, below=0.8cm of compliance] (report) {Reporting Agent\\{\scriptsize EOD, Dashboards}};
\node[handler, below=0.8cm of report] (human-handler) {Human Review\\{\scriptsize High-Stakes}};

% Arrows from sources to router
\draw[arrow] (feed) -- (router) node[midway, above, label] {events};
\draw[arrow] (human) -- (router);
\draw[arrow] (sched) -- (router);
\draw[arrow] (esc) -- (router);

% Arrows from router to handlers
\draw[arrow] (router.east) -- ++(0.5,0) |- (research.west) node[near start, above, label] {high priority};
\draw[arrow] (router.east) -- ++(0.5,0) |- (compliance.west) node[near start, above, label] {normal};
\draw[arrow] (router.east) -- ++(0.5,0) |- (report.west) node[near start, above, label] {low priority};
\draw[arrow] (router.east) -- ++(0.5,0) |- (human-handler.west) node[near start, above, label] {escalation};

\end{tikzpicture}
\caption{Event routing architecture showing how events from multiple channels flow through a central router that classifies, prioritizes, and dispatches to appropriate handlers.}
\label{fig:agents2-event-routing}
\end{figure}

% ----------------------------------------------------------------------------
% Surfaces (Brief)
% ----------------------------------------------------------------------------

\subsection{Surfaces: How Users Experience Agent Systems}
\label{sec:agents2-surfaces}

The same underlying architecture can manifest through different user interfaces, or \textit{surfaces}. Understanding surfaces matters because the appropriate surface depends on the task, the user's expertise, and how the output will be used. Three primary surfaces serve different purposes. Chat surfaces suit interactive exploration, where the partner thinking through case strategy or the analyst exploring market conditions remains actively engaged, refining direction through dialogue. Automation surfaces suit continuous monitoring such as portfolio surveillance, docket tracking, and compliance alerts, where the agent works in the background and users receive outputs only when relevant. Document surfaces suit defined deliverables like research memos, due diligence reports, and client presentations, where the agent produces work products for human review and editing before distribution.

Most deployments combine these surfaces in practice: chat for ad hoc queries and exploratory thinking, automation for continuous monitoring and alerting, and document generation for formal deliverables that must be filed, sent to clients, or presented to committees. The underlying agent architecture supports all three; the surface simply determines how users encounter the system in their day-to-day work.

% ----------------------------------------------------------------------------
% Evaluating Trigger Systems
% ----------------------------------------------------------------------------

\subsection{Evaluating Trigger Systems}
\label{sec:agents2-evaluating-triggers}

When evaluating agent systems, whether building or buying, assess trigger capabilities against five criteria:

\textbf{Coverage}: Does the system receive events from all relevant sources? A litigation agent that monitors CM/ECF but not state court dockets has incomplete coverage.

\textbf{Latency}: How quickly do events reach the agent? Real-time market data requires sub-second delivery; docket alerts can tolerate minutes.

\textbf{Reliability}: What happens when feeds fail? Systems need retry logic, fallback sources, and alerting when data goes stale.

\textbf{Priority mechanisms}: Can the system distinguish urgent from routine? During a market crash or litigation crisis, the right events must reach the right handlers immediately.

\textbf{Auditability}: Is every trigger logged? When a regulator asks why the agent took action, you need a complete record of the triggering event.

% ----------------------------------------------------------------------------
% Connection to Other Questions
% ----------------------------------------------------------------------------

\subsection{From Triggers to Action}
\label{sec:agents2-triggers-to-action}

Triggers answer how work reaches the agent, but triggering is only the beginning. Once an event arrives, the agent must:

\begin{itemize}[nosep]
\item \textbf{Understand intent} (Q2, \Cref{sec:agents2-intent}): What is being asked?
\item \textbf{Perceive information} (Q3, \Cref{sec:agents2-perception}): What does the agent need to know?
\item \textbf{Take action} (Q4, \Cref{sec:agents2-action}): What should the agent do?
\item \textbf{Remember context} (Q5, \Cref{sec:agents2-memory}): What should persist across sessions?
\item \textbf{Plan execution} (Q6, \Cref{sec:agents2-planning}): How should work be decomposed?
\item \textbf{Recognize completion} (Q7, \Cref{sec:agents2-termination}): When is the task done?
\item \textbf{Escalate when needed} (Q8, \Cref{sec:agents2-escalation}): When should humans intervene?
\end{itemize}

The connection between questions is direct. An external feed delivers a court filing notification. The router classifies it as urgent litigation work and dispatches to the litigation agent. The agent retrieves case context from memory, downloads the filed document through PACER, analyzes content, searches for responsive authority, generates deadline calculations, and drafts a response strategy. At each step, the agent might escalate: low confidence in legal analysis triggers escalation to a senior litigator; filing a responsive document requires approval; approaching budget limits prompts a status update.

\Cref{sec:agents2-intent} examines the next question: once work arrives, how does the agent understand what's being asked?

% ============================================================================
% 03-intent.tex
% Q2: How Does an Agent Understand What's Being Asked?
% Part of: Chapter 07 - Agents Part II: How to Build an Agent
% ============================================================================

\section{How Does an Agent Understand What's Being Asked?}
\label{sec:agents2-intent}

% ----------------------------------------------------------------------------
% Opening: Q2 Framing and Organizational Analogy
% ----------------------------------------------------------------------------

When a partner walks into your office and says ``look into the Johnson matter,'' your first job is understanding what that actually means. You must determine whether this is a quick status check or a request for deep analysis, whether you should answer a specific question or identify all issues, and whether this is urgent work for today's call or background work for next week's meeting. The words you hear are the \textbf{instruction}; the underlying purpose that those words point toward is the \textbf{intent}.

Every professional develops this skill over time: reading the assignment memo, clarifying ambiguous instructions, and understanding not just what was said but what was meant. Junior associates tend to over-clarify; senior associates internalize firm norms and client expectations and infer appropriately. The best professionals know when to ask and when to proceed.

Agent systems face the same challenge. The user provides an instruction: natural language, often ambiguous, sometimes contradictory. The agent must extract intent: what goal is being pursued, what constraints apply, what success looks like. This is the second fundamental question: \textit{How does an agent understand what's being asked?}

\begin{definitionbox}[title={Instruction, Intent, Goal, and Task}]
\keyterm{Instruction}: The words the user provides (e.g., ``Review this credit agreement'').

\keyterm{Intent}: The underlying purpose, constraints, and success criteria behind the instruction (e.g., ``identify material risks to the lender by tomorrow'').

\keyterm{Goal}: The desired end state the intent points to (e.g., ``produce a lender-risk memo that meets policy''), as defined in the GPA framework.

\keyterm{Task}: The concrete unit of work the agent will execute to advance the goal (e.g., ``extract covenants and compare to template'').

\textbf{Intent bridges instruction to goal and shapes the tasks the agent will plan.}
\end{definitionbox}

% ----------------------------------------------------------------------------
% From Instruction to Intent
% ----------------------------------------------------------------------------

\subsection{From Instruction to Intent}
\label{sec:agents2-from-instruction-to-intent}

Consider three real instructions a legal or financial professional might give:

\begin{itemize}[nosep]
\item ``Review this credit agreement for risks''
\item ``Rebalance to reduce tech exposure''
\item ``Look into the Johnson matter''
\end{itemize}

Each is ambiguous. The word ``risks'' raises immediate questions about perspective (lender or borrower), scope (material risks or all risks), and domain (legal, financial, or both). ``Reduce tech exposure'' leaves open the target level, the mechanism (sales, hedges, or both), and the tax and timing constraints. ``Look into'' specifies neither depth, urgency, nor deliverable format. The instruction is clear enough to start; the intent is not.

Pre-LLM systems handled intent through rigid parsing: keyword matching, slot filling, decision trees. These systems worked for narrow domains with controlled vocabularies but broke on natural language variation. ``Find cases on personal jurisdiction'' and ``What's the law on where you can sue someone?'' express similar intents but look nothing alike to a keyword matcher. The gap between instruction and intent has always existed; what has changed is our ability to bridge it.

Large language models dramatically improved the ability to infer intent from natural language. Where rule-based systems required exact matches, LLMs handle variation, implicit context, and domain-specific jargon. Modern LLMs excel at handling noisy input (misspellings, shorthand, tangential information), resolving references using conversational context, inferring domain-specific intent, and detecting implicit constraints that professionals take for granted.

Despite these capabilities, intent understanding remains imperfect. As conversations extend, LLMs may lose track of earlier context or constraints. When clarification is needed, LLMs sometimes proceed with a default interpretation rather than asking, resulting in a ``helpful but wrong'' failure: the agent does \textit{something} reasonable rather than confirming it understood correctly. Professionals communicate through implication---``This needs to be right'' signals high stakes; ``When you get a chance'' signals low urgency---and these signals may not be explicitly parsed. Research has shown that LLMs can ``exploit loopholes'' by selectively misunderstanding ambiguous requests in ways that appear helpful but avoid difficult work. Governance must monitor for this failure mode.

\begin{keybox}[title={Intent Inference Is Not Mind Reading}]
LLMs infer \textit{probable} intent from language patterns; they do not read minds. Inference fails when:

\begin{itemize}[nosep]
\item The instruction is genuinely ambiguous (multiple reasonable interpretations)
\item The user's intent differs from typical patterns for similar language
\item Critical context exists outside the conversation (prior meetings, firm norms)
\item The user themselves is unclear about what they want
\end{itemize}

\textbf{Design for clarification, not guessing.} The aim is an agent that surfaces uncertainty and asks, not one that pretends certainty.
\end{keybox}

% ----------------------------------------------------------------------------
% Goal Extraction from Natural Language
% ----------------------------------------------------------------------------

\subsection{Goal Extraction from Natural Language}
\label{sec:agents2-goal-extraction}

Once the agent receives an instruction, it must extract structured goals that can guide execution. This extraction transforms natural language into actionable specifications.

\textbf{Intent classification}: The first step classifies the instruction into task types that determine workflow. This translates raw words into candidate \textbf{tasks} that can advance the underlying \textbf{goal}.

\begin{itemize}[nosep]
\item \textbf{Information retrieval}: ``What's the current NAV?'' ``Find the latest 10-K''
\item \textbf{Research and analysis}: ``Research whether we can pierce the corporate veil''
\item \textbf{Document review}: ``Review the acquisition agreement for change-of-control provisions''
\item \textbf{Document generation}: ``Draft an engagement letter for the Smith matter''
\item \textbf{Calculation}: ``Calculate the IRR assuming a 5-year hold''
\item \textbf{Monitoring}: ``Alert me if tech exposure exceeds 30\%''
\end{itemize}

Different task types invoke different tools, planning patterns, and success criteria. A research task requires search and synthesis; a calculation task requires structured computation; a monitoring task requires continuous observation.

\textbf{Entity and constraint recognition}: Beyond classification, the agent must extract entities (what the task concerns) and constraints (what bounds apply):

\textbf{Entities}: Matters, clients, securities, parties, documents, jurisdictions. ``Review the Smith acquisition agreement'' references a specific document; ``Research Delaware fiduciary duties'' references a jurisdiction.

\textbf{Temporal constraints}: Deadlines, as-of dates, time windows. ``By Friday'' sets a deadline; ``as of year-end 2024'' sets a reference date; ``over the past quarter'' defines a window.

\textbf{Resource constraints}: Budget limits, scope bounds. ``Spend no more than 2 hours'' limits effort; ``focus on Articles 3 and 4'' limits scope.

\textbf{Format constraints}: Deliverable specifications. ``Summarize in one page'' constrains length; ``prepare a memo for the file'' specifies format; ``I need something to show the client'' signals external audience.

\textbf{Audience and privilege constraints}: Who will see the output (internal team, client, regulator) and what privilege or confidentiality must be preserved.

\textbf{Risk and compliance constraints}: Limits that may not be stated but must be inferred. A compliance review implicitly requires flagging violations; a client communication implicitly requires privilege protection.

\textbf{Structured goal representation}: Extracted components can be organized into structured representations that guide execution. This structured representation resembles a short assignment memo that the planning system can act on (\Cref{sec:agents2-planning}) and the termination system can measure against (\Cref{sec:agents2-termination}):

\begin{verbatim}
task_type: document_review
document: Smith Acquisition Agreement
objective: identify change-of-control provisions
constraints:
  deadline: 2025-01-15
  scope: sections 5-8
  deliverable: summary memo
success_criteria:
  - all CoC provisions identified
  - triggering events listed
  - consent requirements noted
\end{verbatim}

% ----------------------------------------------------------------------------
% Ambiguity Detection and Clarification
% ----------------------------------------------------------------------------

\subsection{Ambiguity Detection and Clarification}
\label{sec:agents2-clarification}

Not all instructions can be unambiguously interpreted. The agent must detect ambiguity and decide whether to clarify or proceed.

\textbf{When to clarify}: The decision to clarify depends on ambiguity severity and action stakes:

\textbf{Low stakes, low ambiguity}: Proceed with best interpretation. If the user asks ``What's Apple's market cap?'' and you're unsure whether they mean Apple Inc. or Apple Hospitality REIT, the dominant interpretation is obvious and the cost of being wrong is low (easy to correct).

\textbf{Low stakes, high ambiguity}: Clarify briefly. If the user asks ``Research the statute of limitations'' without specifying the claim type, a quick clarification prevents wasted effort.

\textbf{High stakes, low ambiguity}: Confirm before acting. If the instruction is clear but consequential (``File this motion''), confirmation prevents irreversible errors.

\textbf{High stakes, high ambiguity}: Clarify thoroughly. If the user says ``Handle the regulatory response'' for a complex matter, extended clarification is appropriate before taking any action.

% When to Clarify: Compact Stakes × Ambiguity Decision Matrix

\begin{figure}[!htb]
\centering
\begin{tikzpicture}[
    cell/.style={
        rounded corners=4pt,
        line width=1.2pt,
        minimum width=5.8cm,
        minimum height=1.8cm,
        align=center,
        inner sep=10pt
    },
    action/.style={
        font=\small\bfseries\sffamily
    },
    example/.style={
        font=\scriptsize\itshape,
        text=gray-700
    },
    axis label/.style={
        font=\scriptsize\bfseries\sffamily,
        text=gray-700
    }
]

% Column headers
\node[axis label] at (-3.1, 2.4) {Low Ambiguity};
\node[axis label] at (3.1, 2.4) {High Ambiguity};

% Row headers
\node[axis label, rotate=90, anchor=south, align=center] at (-6.5, 0.95) {Low\\Stakes};
\node[axis label, rotate=90, anchor=south, align=center] at (-6.5, -0.95) {High\\Stakes};

% Top-left: Low Stakes, Low Ambiguity
\node[cell, draw=green-600, fill=green-100] (tl) at (-3.1, 0.95) {
    \textcolor{green-900}{\textsf{\textbf{PROCEED}}}\\[2pt]
    \textcolor{gray-600}{\scriptsize\itshape ``Apple market cap'' → Apple Inc.}
};

% Top-right: Low Stakes, High Ambiguity
\node[cell, draw=amber-600, fill=amber-100] (tr) at (3.1, 0.95) {
    \textcolor{amber-900}{\textsf{\textbf{CLARIFY BRIEFLY}}}\\[2pt]
    \textcolor{gray-600}{\scriptsize\itshape ``Research SOL'' → which claim type?}
};

% Bottom-left: High Stakes, Low Ambiguity
\node[cell, draw=amber-600, fill=amber-100] (bl) at (-3.1, -0.95) {
    \textcolor{amber-900}{\textsf{\textbf{CONFIRM}}}\\[2pt]
    \textcolor{gray-600}{\scriptsize\itshape ``File this motion'' → verify before acting}
};

% Bottom-right: High Stakes, High Ambiguity
\node[cell, draw=red-600, fill=red-100] (br) at (3.1, -0.95) {
    \textcolor{red-900}{\textsf{\textbf{CLARIFY THOROUGHLY}}}\\[2pt]
    \textcolor{gray-600}{\scriptsize\itshape ``Handle the regulatory response''}
};

% Subtle grid lines
\draw[gray-300, line width=0.5pt] (0, 2.0) -- (0, -1.9);
\draw[gray-300, line width=0.5pt] (-6.2, 0) -- (6.2, 0);

\end{tikzpicture}
\caption{Decision matrix for when agents should clarify user intent. Stakes measure consequence of error; ambiguity measures interpretation confidence.}
\label{fig:agents2-when-to-clarify}
\end{figure}


\textbf{How to clarify}: Effective clarification is specific, contextual, and actionable:

\textbf{Specific}: ``Which jurisdiction's statute of limitations---Delaware or New York?'' not ``Can you clarify?''

\textbf{Contextual}: Reference what the agent already understands. ``I understand you want me to review the credit agreement. Should I focus on lender protections, borrower obligations, or both?''

\textbf{Actionable}: Offer options rather than open-ended questions. ``Should I (a) provide a comprehensive review of all provisions, (b) focus on the financial covenants, or (c) flag only provisions that differ from our standard template?''

\textbf{Bounded}: Limit clarification rounds. If the agent needs extensive clarification, it may be the wrong tool for the task, or the user may need to think through requirements before delegating.

\begin{quote}
\textit{Poor clarification}: ``Can you clarify?'' \\
\textit{Better}: ``Should I assess lender risks, borrower risks, or both?'' \\
\textit{Best}: ``You asked to reduce tech exposure. Should I (a) sell tech to 25\% target, (b) hedge with options, or (c) add non-tech positions? Which deadline matters—this week or month-end reporting?''
\end{quote}

\begin{highlightbox}[title={The Default Interpretation Risk}]
Research has documented that LLMs sometimes select a default interpretation rather than asking for clarification, even when ambiguity is significant. This ``proceed without asking'' behavior can be problematic:

\textbf{The risk}: The agent interprets ``review the contract'' as a surface-level summary when the user expected deep issue-spotting. Work product is delivered, but it is wrong.

\textbf{Mitigation strategies}:
\begin{itemize}[nosep]
\item Prompt engineering that emphasizes clarification for ambiguous requests
\item Confidence thresholds that trigger clarification below a certainty level
\item User training to provide detailed initial instructions
\item Checkpoint reviews before significant work begins
\end{itemize}

\textbf{Governance implication}: Monitor for cases where the agent proceeded confidently but delivered unexpected results. These may indicate calibration problems in ambiguity detection.
\end{highlightbox}

% ----------------------------------------------------------------------------
% Constraint Identification
% ----------------------------------------------------------------------------

\subsection{Constraint Identification}
\label{sec:agents2-constraints}

Beyond explicit instructions, agents must identify constraints that bound acceptable execution. \textbf{Temporal constraints}: Deadlines and time windows. Some are explicit (``by Friday''); others are implicit (court filing deadlines calculated from rules); still others are contextual (``before the board meeting'' requires knowing when the meeting is).

\textbf{Resource constraints}: Budget and effort limits. Token budgets limit API costs; time budgets limit calendar impact; scope constraints focus effort on high-value areas.

\textbf{Scope constraints}: What is in and out of bounds. ``Focus on Articles 3 and 4'' excludes other articles; ``just the Delaware analysis'' excludes other jurisdictions.

\textbf{Format and style constraints}: How deliverables should appear. Memo versus email versus presentation; formal versus casual tone; internal versus client-facing.

\textbf{Risk and compliance constraints}: What must be avoided. Privilege protection, conflicts of interest, regulatory restrictions, confidentiality obligations. These constraints often apply implicitly based on context.

\textbf{Inferring implicit constraints}: Professionals operate under constraints they rarely state explicitly. When a partner says ``research Section 10(b) liability,'' implicit constraints include:

\begin{itemize}[nosep]
\item Use authoritative sources (binding precedent, not blog posts)
\item Focus on the relevant jurisdiction (probably the circuit where the case is filed)
\item Assume current law (not historical analysis unless specified)
\item Protect privilege (don't disclose strategy in external searches)
\item Operate within budget norms (don't spend 40 hours on a 2-hour task)
\end{itemize}

Agents must infer these constraints from context, domain knowledge, and organizational norms. Memory systems (\Cref{sec:agents2-memory}) help by preserving firm-specific expectations; user profiles track individual preferences; matter context provides case-specific constraints.

% ----------------------------------------------------------------------------
% Validation and Domain Examples
% ----------------------------------------------------------------------------

\subsection{Validation and Domain Examples}
\label{sec:agents2-validation-examples}

Before executing, agents should validate their understanding of intent. Several patterns support validation:

\textbf{Reflection and summarization}: The agent restates its understanding before proceeding, giving the user an opportunity to correct misunderstandings before work begins.

\textbf{Chunked validation}: For complex tasks, validate in phases rather than all at once. After completing research, summarize findings and confirm direction before drafting. After drafting, confirm the approach before finalizing. Each checkpoint prevents error propagation.

\textbf{Confidence signaling}: The agent should indicate how confident it is in its own understanding. When confidence is high, the agent can proceed with light oversight; when confidence is low, the right move is to pause and ask for clarification rather than press ahead. Clear confidence signaling helps users decide how much review is needed and whether to treat the output as a draft, a starting point, or a near-final product.

\textbf{Legal example---credit agreement review}: Consider how intent extraction and validation work together for a legal task. Given the instruction ``Review this credit agreement for risks,'' the agent classifies this as a document review task and detects that ``risks'' is ambiguous (risks to whom? what types?). Context gathering reveals this is a lender-side engagement for a senior secured facility. The agent infers implicit constraints (focus on lender risks, prioritize material issues, assume current market terms as baseline) and clarifies: ``I'll review from the lender's perspective, focusing on credit risk, collateral coverage, and covenant adequacy. Should I also flag documentation risks (drafting issues, missing provisions) or focus only on substantive credit terms?'' The extracted goal:

\begin{verbatim}
{
  "task_type": "document_review",
  "perspective": "lender",
  "document": "[attached credit agreement]",
  "focus_areas": ["credit_risk", "collateral", "covenants"],
  "deliverable": "risk_summary_memo",
  "success_criteria": [
    "material_risks_identified",
    "comparison_to_market_terms",
    "recommendations_provided"
  ]
}
\end{verbatim}

The agent validates: ``I'll prepare a memo identifying material risks to the lender, comparing key terms to market standards, and recommending negotiation points. I'll have a draft for your review by tomorrow afternoon.''

\textbf{Financial example---portfolio rebalancing}: The same pattern applies to financial tasks. Given the instruction ``Rebalance to reduce tech exposure,'' the agent classifies this as a portfolio action task and immediately detects multiple ambiguities: how much reduction? through what mechanism? with what constraints? Context gathering reveals current tech exposure at 35\% against a 25\% target. Through clarification dialogue, the agent confirms the user wants to reach target through sales while minimizing tax impact (preferring loss harvesting and long-term gains over short-term). The extracted goal:

\begin{verbatim}
{
  "task_type": "portfolio_rebalancing",
  "objective": "reduce_tech_exposure",
  "target": "25%",
  "mechanism": "sales",
  "constraints": {
    "tax_optimization": true,
    "prefer_loss_harvesting": true
  },
  "approval_required": true
}
\end{verbatim}

The agent validates: ``I'll prepare a rebalancing plan to reduce tech from 35\% to 25\% through sales, optimized for tax efficiency. I'll present the plan for your approval before executing any trades.''

\begin{keybox}[title={Intent Understanding Is Continuous}]
Intent extraction is not a one-time step at task initiation. As the agent works, it may discover:

\begin{itemize}[nosep]
\item The original understanding was incomplete (new constraints emerge)
\item The user's intent has evolved (priorities shift mid-task)
\item Implicit constraints conflict (cannot optimize for both)
\item The task is impossible as specified (constraints are mutually exclusive)
\end{itemize}

Effective agents surface these discoveries through clarification rather than proceeding with outdated or impossible goals. Intent understanding is iterative, not instantaneous.
\end{keybox}

% ----------------------------------------------------------------------------
% Transition to Q3
% ----------------------------------------------------------------------------

Intent understanding connects to other framework questions. Memory (\Cref{sec:agents2-memory}) improves intent extraction over time by preserving user preferences, matter history, and firm norms. Planning (\Cref{sec:agents2-planning}) depends on clear intent; extracted goals feed the planning system, while ambiguous intent propagates through the plan as uncertainty. Governance (\Cref{sec:agents2-governance}) must address intent misalignment as a core risk, verifying goal alignment before deployment and monitoring for drift during operation.

Understanding intent bridges the gap between what users say (instruction) and what they mean (intent), shaping the goals and tasks the agent will plan. Clarification beats guessing when ambiguity is significant and stakes are high. Constraints---time, scope, audience, compliance, and budget---matter as much as goals. Validation prevents wasted effort by confirming understanding before significant work begins.

With triggers delivering work (Q1) and intent extraction revealing what's being asked (Q2), the agent needs capabilities to gather information and effect change. \Cref{sec:agents2-perception} examines the next question: how does an agent find things out?

% ============================================================================
% 04-perception.tex
% Q3: How Does an Agent Find Things Out?
% Part of: Chapter 2 - How to Design an Agent
% ============================================================================

\section{How Does an Agent Find Things Out?}
\label{sec:agents2-perception}

% ----------------------------------------------------------------------------
% Opening: Q3 Framing and Organizational Analogy
% ----------------------------------------------------------------------------

Understanding what someone wants is not the same as being able to deliver it. The previous section examined how agents interpret instructions, extracting goals, detecting ambiguity, and gathering the context needed to proceed. But even an agent that perfectly understands ``Research Ninth Circuit authority on personal jurisdiction for foreign corporations'' cannot help if it lacks access to case law databases. Where Chapter~1 introduced perception as the ability to observe the world, we now examine the concrete mechanisms---tools and integrations---that enable agents to gather the information they need. An agent that correctly interprets ``Analyze this credit agreement from the lender's perspective'' is useless without the credit agreement itself.

This constraint will feel familiar to anyone who has onboarded a new professional. A junior associate's effectiveness depends as much on access as on reasoning ability. Can they query Westlaw? Do they have credentials for the Bloomberg terminal? Can they search the firm's precedent database and document management system? The answers to these questions determine which problems they can solve. A brilliant analyst without access to portfolio data reasons in a vacuum; a talented associate without access to the case file works blind.

\glsadd{llm}Agentic systems face exactly the same constraint. A large language model can reason impressively about legal doctrines and financial concepts, but without integrations into external or internal sources, it cannot access current case law, live market prices, client documents, or regulatory filings. These connections are what we call \textit{perception tools}: the interfaces through which an agent observes the world beyond its training data.

\begin{definitionbox}[title={Tools and Perception}, breakable=false]
\glsadd{tools}
\glsadd{perception}
	A \keyterm{tool} is a function that allows an agent to interact with external systems. \keyterm{Perception tools} are the subset that are read-only: they let the agent observe without changing anything. When an agent queries a database, retrieves a document, or fetches market data, the external system's state remains unchanged.

	\vspace{0.5em}
	Perception defines the boundary of what information the agent can access, and that boundary shapes every downstream decision. An agent with access to public filings reasons differently than one with access to internal deal documents. An agent that can query real-time market data operates differently than one limited to end-of-day prices.
\end{definitionbox}

Here we examine perception: the read-only tools that let agents gather the information they need. The next section, on action, examines write tools that let agents change things in the world. The distinction matters for governance: reading a document and sending an email carry fundamentally different risks, and your controls should reflect that difference. \Cref{sec:agents2-action} develops these distinctions and the different oversight mechanisms each requires.

% ----------------------------------------------------------------------------
% Perception Tool Categories
% ----------------------------------------------------------------------------

\subsection{Perception Tool Categories}
\label{sec:agents2-perception-categories}

Not all perception tools work the same way, and choosing the right tool for the task matters. Perception tools generally fall into three categories: information retrieval, document processing, and computation.

\textbf{Information retrieval tools} query external platforms and databases to bring back answers. On the legal side, these tools connect to research services like Westlaw and Lexis in the United States, Beck-Online in Germany, LawNet in Singapore, and similar platforms in other jurisdictions, allowing agents to search case law, retrieve full opinions, check citing references, and download court filings. On the finance side, similar tools connect to platforms like Bloomberg, FactSet, or Refinitiv, enabling agents to pull real-time prices, retrieve company fundamentals, and access analyst research. Many organizations also have internal knowledge bases, including document management systems, deal archives, and precedent databases, that agents need to search to find prior work product relevant to a current matter.

\textbf{Document processing tools} transform raw files into data an agent can actually work with. A credit agreement arrives as a PDF; a financial statement comes as a scanned image; a data room contains thousands of files in mixed formats. Before an agent can reason about this content, it needs to extract the text (using OCR for scanned documents), identify what type of document each file is, and pull out structured information like party names, dates, and dollar amounts. During due diligence, for example, an agent reviewing a data room must distinguish contracts from correspondence, extract key terms from each agreement, and organize findings in a way that supports analysis. Without document processing tools, the agent sees only filenames and metadata.

\textbf{Computation tools} generate new information through calculation rather than lookup. Take deadline calculation: the Federal Rules of Civil Procedure require an answer within 21 days of service \parencite{frcp-rule-12}, but determining the actual due date requires accounting for weekends, court holidays, and local rules. That is a computational task, not a database query. Citation formatters perform a similar function, converting case information into proper Bluebook or other citation styles. In finance, computation tools normalize security identifiers (mapping between tickers, CUSIPs, and ISINs), calculate risk metrics like Value at Risk from position data, or derive analytics that inform investment decisions. These tools produce new information for the agent to use without changing anything in the external world.

% ----------------------------------------------------------------------------
% In-Context Learning and Retrieval Fundamentals
% ----------------------------------------------------------------------------

\subsection{In-Context Learning and Retrieval}
\label{sec:agents2-icl-retrieval}

Before examining specific retrieval mechanisms, we must understand the fundamental capability that makes retrieval useful: \keyterm{in-context learning}. This concept is central to how modern language models work and why retrieval matters.

\begin{definitionbox}[title={In-Context Learning}, breakable=false]
\glsadd{in-context-learning}
	\keyterm{In-context learning (ICL)} is the ability of large language models to learn from information provided in their input---without any update to the model's underlying parameters \parencite{brown2020gpt3}. When you provide examples, documents, or instructions in a prompt, the model adapts its responses based on that context. This is not ``learning'' in the traditional machine learning sense of updating weights; it is learning \textit{within the conversation}.

	\vspace{0.5em}
	In-context learning is what makes retrieval valuable. When an agent retrieves a relevant statute or prior memo and includes it in its prompt, the model can reason about that specific content even though it never saw it during training. The model's behavior changes based on what appears in its context window.
\end{definitionbox}

In-context learning explains why agents can work with information far newer than their training data. A model trained in 2024 can analyze a regulation enacted in 2025---provided that regulation appears in its context. The limitation is the \keyterm{context window}: the maximum amount of text the model can process at once. Current models handle roughly 200,000 to 1,000,000 tokens (where a token is roughly three-quarters of a word),\footnote{Context window sizes as of December 2025 (vendor-reported): Claude Opus 4.5 (200k tokens), Gemini 3 (up to 1M tokens), GPT-5.2 (256k tokens). These specifications change frequently; consult current model documentation.} but the cost, speed, and quality of results vary widely---and dangerously---when operating on large amounts of input. A model may technically accept a million tokens while producing degraded results on tasks requiring attention to details scattered throughout. Professional knowledge bases contain millions of documents regardless; the gap between what fits in context and what exists in the world creates the need for retrieval.

\begin{keybox}[title={Why Agentic Architecture Works}]
	The practical limits of context windows explain much of the value of agentic systems. Rather than attempting to process an entire knowledge base in a single prompt---which would exceed limits and degrade quality regardless---an agent decomposes work into focused steps, retrieves only what is relevant to each step, and reasons over smaller, manageable contexts.

	\vspace{0.5em}
	This is not merely automation; it is an architectural solution to a fundamental constraint. Planning (\Cref{sec:agents2-planning}) breaks complex tasks into subtasks, ensuring each retrieval step is targeted rather than overwhelming. Retrieval fetches specific information for each subtask. The LLM operates within its effective range on each step, even when the overall task would be impossible to handle monolithically.

	\vspace{0.5em}
	This decomposition explains both why agentic systems are powerful and why they fail in predictable ways. The reliability cliff (\Cref{sec:agents2-termination}) shows that short, focused tasks succeed while long tasks fail---precisely because decomposition keeps individual steps within the model's effective operating range, but errors compound across many steps.
\end{keybox}

\paragraph{Retrieval Methods}

To retrieve relevant information from large collections, we need a way to find what matters. Several approaches exist, each with distinct strengths:

\textbf{Keyword and boolean search} is the oldest and most familiar approach. Platforms like Westlaw, Lexis, and Bloomberg have offered sophisticated keyword search for decades. Boolean operators (AND, OR, NOT), proximity searches, and field-specific queries give users precise control. When you need documents containing exact terms---a specific case citation, a statutory section, a company name---keyword search excels. Its limitation is vocabulary mismatch: a search for ``breach of fiduciary duty'' will not find documents discussing ``violation of trust obligations'' unless both phrases happen to appear.

\textbf{BM25 and probabilistic search} extends keyword matching with statistical weighting \parencite{robertson2009bm25}. Terms that appear rarely in the corpus but frequently in a document signal relevance more strongly than common terms. BM25 powers many production search systems, including Elasticsearch and the baseline retrievers in academic benchmarks. It requires no machine learning infrastructure---just an inverted index---making it fast, interpretable, and cheap to operate.

\textbf{Embedding-based semantic search} uses machine learning to encode meaning numerically. An \keyterm{embedding} is a vector---a list of numbers---that represents semantic content. Similar concepts produce similar vectors; different concepts produce distant vectors. The phrases ``breach of fiduciary duty'' and ``violation of trust obligations'' have embeddings close together in vector space, even though they share no words. This enables \textit{meaning matching} rather than vocabulary matching. Embedding models vary in quality and domain fit; legal-specific models may outperform general-purpose ones for professional content. The infrastructure cost is higher: you need embedding models, vector storage, and similarity search capabilities.

\textbf{Structured queries} retrieve from databases and APIs. An agent checking a company's SEC filings queries EDGAR; an agent researching a case retrieves from a court's electronic filing system. These are not ``search'' in the traditional sense---they are direct data access---but they serve the same function in RAG: finding relevant information to inject into context.

\textbf{Hybrid approaches} combine methods. A common pattern uses BM25 for initial retrieval (fast, cheap, catches exact matches) followed by semantic reranking (slower, more expensive, captures meaning). Many production systems blend keyword and semantic scores, getting the best of both approaches.

\begin{keybox}[title={Choosing Retrieval Methods}, breakable=false]
	No single retrieval method dominates. The right choice depends on your content, queries, and constraints:
	\begin{itemize}[nosep]
		\item \textbf{Exact terms matter}: Keyword or BM25. Case citations, statutory references, and party names must match precisely.
		\item \textbf{Conceptual similarity matters}: Embeddings. Finding documents about ``materiality'' when the query says ``significance'' requires semantic matching.
		\item \textbf{Structured data}: Database queries or APIs. Retrieving a specific filing from EDGAR or a price from Bloomberg.
		\item \textbf{Production constraints}: BM25 is cheaper and faster; embeddings require more infrastructure but capture meaning better.
	\end{itemize}
	Most professional systems use hybrid approaches, combining methods to balance precision, recall, and cost.
\end{keybox}

\paragraph{Vector Stores for Embedding-Based Search}

When using embeddings, storing and searching millions of vectors efficiently requires specialized infrastructure.

\begin{definitionbox}[title={Vector Stores}, breakable=false]
\glsadd{vector-store}
	A \keyterm{vector store} (or vector database) is a system optimized for storing embeddings and performing similarity search. Given a query embedding, a vector store returns the most similar document embeddings from its collection. Vector stores use specialized indexing structures (like HNSW or IVF) that make approximate nearest-neighbor search practical at scale.

	\vspace{0.5em}
	Vector stores are \textit{infrastructure}---the retrieval mechanism that powers semantic search. They are not the same as the retrieval pattern itself.
\end{definitionbox}

\paragraph{Retrieval-Augmented Generation (RAG)}

With these retrieval methods in hand, we can now define RAG precisely.

\begin{definitionbox}[title={Retrieval-Augmented Generation (RAG)}, breakable=false]
\glsadd{rag}
	\keyterm{Retrieval-Augmented Generation (RAG)} is a \textit{prompt pattern}---not a software system or mathematical technique \parencite{lewis2020rag}. RAG augments a model's context with retrieved information before generation. The pattern has three steps:
	\begin{enumerate}[nosep]
		\item \textbf{Retrieve}: Given a query, find relevant content using \textit{any} retrieval method---keyword search, BM25, embeddings, database queries, API calls, or combinations thereof.
		\item \textbf{Augment}: Insert the retrieved content into the model's prompt as additional context.
		\item \textbf{Generate}: The model produces a response informed by both its training and the retrieved content.
	\end{enumerate}

	\vspace{0.5em}
	RAG works because of in-context learning. The retrieved content appears in the prompt, and the model adapts its response to incorporate that specific information. The retrieval step is completely method-agnostic: what matters is getting relevant content into context, not how you find it.
\end{definitionbox}

The distinction matters. When evaluating a system, you should ask: What retrieval infrastructure does it use? (Keyword index, vector store, database, external API, hybrid?) What RAG pattern does it implement? (Single-stage, multi-stage, with reranking?) These are separate architectural decisions with different tradeoffs.

\begin{keybox}[title={RAG Is Retrieval-Agnostic}, breakable=false]
	A common misconception equates RAG with embeddings and vector stores. In practice, RAG implementations vary widely:
	\begin{itemize}[nosep]
		\item \textbf{Keyword search}: Westlaw, Lexis, or internal full-text search---no embeddings required.
		\item \textbf{BM25}: Fast probabilistic retrieval powering many production systems.
		\item \textbf{Semantic search}: Embeddings and vector stores for meaning-based matching.
		\item \textbf{Structured queries}: SQL databases, EDGAR filings, court record APIs.
		\item \textbf{Hybrid}: Combining methods to balance precision, recall, and cost.
	\end{itemize}
	The ``embedding plus vector store'' pattern dominates tutorials because it is general-purpose and handles semantic similarity well. But many production systems---especially those leveraging existing search infrastructure---use simpler approaches that work well for their specific needs.
\end{keybox}

% ----------------------------------------------------------------------------
% Model Context Protocol (MCP) for Perception
% ----------------------------------------------------------------------------

\subsection{Model Context Protocol (MCP)}
\label{sec:agents2-mcp-perception}

\glsadd{mcp}
One of the persistent challenges in building agentic systems is integration. Every database, every document management system, every market data feed has its own way of accepting queries and returning results. Historically, connecting an agent to a new information source meant writing custom code for that specific system. Multiply that by the number of agents and the number of tools an organization uses, and the integration burden grows quickly.

The Model Context Protocol (MCP) addresses this problem by standardizing how agents access tools \parencite{anthropic-mcp,mcp-spec}. Instead of learning a different integration pattern for each system, agents learn the protocol once and can then access any compatible tool. For organizations, this means a single point of audit and control rather than dozens of bespoke connections to monitor.

The architecture has three components. The \keyterm{MCP Host} manages the agent and controls what it can access, functioning like a firm's IT department deciding which systems a new employee can use. The \keyterm{MCP Client} is the agent-side component that discovers available tools and makes requests. The \keyterm{MCP Server} is what each information source runs to expose its capabilities through the standard interface. A document management system, an internal knowledge base, or a proprietary analytics platform can each run as an MCP server, and any MCP-compatible agent can then access them without custom integration work.

For perception specifically, MCP defines \keyterm{Resources} as read-only data access endpoints. These might include document repositories, market data feeds, regulatory filing databases, or internal knowledge bases. The read-only designation matters: an agent with resource access can retrieve documents but cannot modify or file them. This separation enables fine-grained access control that mirrors how organizations already think about permissions.

\begin{keybox}[title={The Integration Landscape Is Changing}]
	Historically, many information providers in law and finance offered no programmatic access at all. Legal research platforms required human users working through web interfaces; market data terminals assumed a person at the keyboard. This created a barrier for agentic systems, which need machine-readable interfaces to function.

	\vspace{0.5em}
	That landscape is shifting. Competitive pressure and customer demand are pushing providers to open up. Document management systems, e-discovery platforms, and financial data providers increasingly offer APIs that agents can use. Some legal research providers are beginning to follow. As this trend continues, the range of information sources available to agentic systems will expand significantly.

	\vspace{0.5em}
	Standards like MCP accelerate this shift by reducing the cost of integration. Without MCP, connecting 10 agents to 10 tools requires 100 custom integrations. With MCP, the same setup requires only 20 implementations, since each agent and each tool learns the protocol once. Recent benchmarks found over ten thousand MCP servers already in the ecosystem \parencite{livemcpbench-2025}, and that number will likely grow substantially in the years ahead.
\end{keybox}

% ----------------------------------------------------------------------------
% Memory as Perception
% ----------------------------------------------------------------------------

\subsection{Memory as Perception into Institutional Knowledge}
\label{sec:agents2-memory-perception}

Memory systems (\Cref{sec:agents2-memory}) serve as perception tools for institutional knowledge. The retrieval concepts introduced above---embeddings, vector stores, and RAG---enable agents to perceive accumulated expertise that would otherwise be inaccessible.

When an agent queries a precedent database using the RAG pattern (\Cref{sec:agents2-icl-retrieval}), it perceives institutional knowledge through in-context learning. The retrieved content appears in the agent's prompt, allowing it to reason about specific precedents, prior analyses, and established approaches. A search for ``breach of fiduciary duty'' retrieves documents about ``violation of trust obligations'' because their embeddings are similar---not because they share keywords.

This mechanism enables perception into knowledge bases far too large to fit in any model's context window. A law firm's precedent database might contain decades of work product; a financial institution's research archive might span thousands of analyst reports. No agent can hold all of this in active context. RAG allows selective retrieval: the agent perceives only the most relevant fragments, guided by semantic similarity to the current task.

Institutional memory provides access to prior work product. When drafting a new registration statement, an agent can perceive prior S-1 filings (IPO registrations), SEC comment histories, and successful disclosure language. This access allows current work to build on verified precedents rather than starting from first principles.

Memory-as-perception distinguishes experienced agents from novices. A junior associate reasons from what they learned in law school; a senior associate draws on pattern recognition from hundreds of matters. Memory provides agents with this accumulated experience---but only if the retrieval infrastructure connects them to the right knowledge at the right time. \Cref{sec:agents2-memory} develops these requirements in detail, including how memory systems must enforce the authority, temporal validity, and isolation constraints introduced above.

% ----------------------------------------------------------------------------
% Domain-Specific Perception Requirements
% ----------------------------------------------------------------------------

\subsection{Domain-Specific Perception Requirements}
\label{sec:agents2-perception-domain}

Perception for regulated professional services requires specialized enhancements. General-purpose search is insufficient; professional agents require authority tracking, jurisdictional awareness, and confidentiality boundaries.

\paragraph{Authority and Verification}

Information varies in authority. Perception systems must track provenance to ensure reliability. Authority weighting ranks primary sources (statutes, binding precedent) above secondary sources (law reviews, news). When searching for ``insider trading liability,'' a Supreme Court opinion outranks a commentary article. Source verification confirms that retrieved information originates from the claimed source. Perception tools must return verifiable citations, not just text. Currency validation ensures the authority remains valid. Integrated citators (like Shepard's or KeyCite) verify that retrieved cases have not been overruled.

\paragraph{Jurisdiction and Temporal Scope}

Legal and regulatory information is bounded by jurisdiction. California precedent does not bind Texas courts; SEC rules differ from CFTC rules. Perception tools must filter results by relevant jurisdiction. Temporal validity is equally critical. Laws change, and financial data expires. Perception systems must track effective dates. In finance, validity varies by context: milliseconds for trading prices, quarters for compliance reporting. Identifier resolution manages the proliferation of formats. ``123 F.3d 456'' and ``123 F3d 456'' refer to the same case. Financial identifiers include tickers, CUSIPs, and LEIs (Legal Entity Identifiers). Perception must normalize these to ensure consistent retrieval.

\paragraph{Matter and Client Isolation}

Critically, perception must respect confidentiality boundaries. Whether a human or AI, an agent working on Matter A cannot perceive documents from adverse Matter B. This enforcement of \keyterm{ethical walls} arguably must occur at the perception layer. In financial contexts, an agent advising Client X cannot perceive material non-public information (MNPI) from Client Y's engagement. Every perception event must be logged, capturing the agent, the query, and the matter context. This audit trail enables compliance review and breach detection. See \Cref{sec:agents2-memory} for detailed treatment of isolation requirements; Chapter~3 (\textit{How to Govern an Agent}) addresses the professional responsibility obligations---including attorney confidentiality duties and fiduciary obligations---that mandate these controls.

% ----------------------------------------------------------------------------
% Tool Design Principles for Perception
% ----------------------------------------------------------------------------

\subsection{Tool Design Principles}
\label{sec:agents2-perception-design}

Robust perception tools follow design principles that enable reliable operation in professional environments.

\paragraph{Single Responsibility}

Each tool should perform one function well. Poorly designed tools bundle multiple functions---searching, formatting, and validation---into opaque interfaces. Untyped return values obscure what callers can expect.

\begin{listingbox}[title={Poor Design: Bundled Functions, Untyped Returns}, listing options={language=Python}]
def legal_research(query: str) -> dict:
  """Returns... something. Good luck."""
  ...
\end{listingbox}

When such a tool fails, diagnosing the error is difficult. A better approach separates tools by function with typed returns. This allows the agent to compose them and isolates failures.

\begin{listingbox}[title={Better Design: Single Responsibility, Typed Returns}, listing options={language=Python}, breakable=false]
def search_cases(query: str, jurisdiction: str) -> list[Citation]:
  """Returns matching citations from case law database."""

def retrieve_case(citation: Citation) -> CaseText:
  """Fetches full text for a specific citation."""

def shepardize(citation: Citation) -> CitatorResult:
  """Checks validity: good law, distinguished, overruled."""

def format_citation(case: CaseText, style: str) -> str:
  """Converts to Bluebook, ALWD, or other format."""
\end{listingbox}

\paragraph{Graceful Failure}

Production systems inevitably fail. Tools should return informative errors rather than generic exceptions. A poor approach raises exceptions that provide no context.

\begin{listingbox}[title={Poor: Opaque Exception}, listing options={language=Python}]
def retrieve_case(citation: str) -> dict:
  result = db.query(citation)
  return result["text"]  # raises KeyError if not found
\end{listingbox}

A better approach uses typed result objects that make success and failure explicit.

\begin{listingbox}[title={Better: Typed Result with Structured Errors}, listing options={language=Python}]
class CaseNotFoundError(BaseModel):
  citation: str
  reason: str
  suggestions: list[str]

def retrieve_case(citation: Citation) -> CaseText | CaseNotFoundError:
  """Returns case text or structured error with recovery options."""
  if not (result := db.query(citation)):
    return CaseNotFoundError(
      citation=str(citation),
      reason="Case may not be in database",
      suggestions=["Check citation format", "Try alternative reporter"]
    )
  return CaseText(...)
\end{listingbox}

In professional practice, graceful failure prevents malpractice. When an agent cannot find authority, it must report that explicitly rather than proceeding silently.

\paragraph{Least Privilege and Rate Limiting}

Perception tools should request minimum necessary permissions. A legal research tool requires read access to case databases, not write access to the document management system. If a compromised agent gains perception credentials, damage is limited to information disclosure rather than destruction. Rate limiting addresses a common failure mode: infinite search loops. Tools should track invocation frequency and refuse requests beyond reasonable thresholds. If an agent searches five times without results, the tool should force a stop and escalation (\Cref{sec:agents2-escalation}).

% ----------------------------------------------------------------------------
% Evaluating Perception Capabilities
% ----------------------------------------------------------------------------

\subsection{Evaluating Perception Capabilities}
\label{sec:agents2-perception-eval}

When evaluating agentic systems, you should assess perception against criteria that matter for professional practice.

\textbf{Coverage} determines which sources the agent can access. A litigation agent that queries Westlaw but not state-specific databases has incomplete coverage. You must map available perception tools against information needs to identify gaps.

\textbf{Retrieval quality} measures whether the agent finds relevant information. Test with known-good queries where the correct result is established. Measure both \keyterm{precision} (the fraction of retrieved documents that are actually relevant) and \keyterm{recall} (the fraction of all relevant documents that the system successfully retrieves). These metrics will be familiar to legal professionals from technology-assisted review (TAR) in e-discovery, where courts have recognized precision and recall as the standard measures of retrieval effectiveness \parencite{grossman2011tar}. The same framework applies to agent perception: high precision means the agent does not waste time on irrelevant results; high recall means the agent does not miss important authorities. The tradeoff between them---casting a wider net improves recall but may reduce precision---is a design decision that should be calibrated to the task's risk profile.

\textbf{Verification} confirms that the system distinguishes authoritative from secondary sources. You must ensure that retrieved information is traceable to its source and that citations are independently verifiable.

\textbf{Access controls} ensure that permissions are appropriate. The agent must access only what it should, and confidentiality boundaries must hold across matter and client lines.

\textbf{Failure handling} reveals system behavior when perception fails. Does it retry, try alternatives, or escalate? It must not crash or proceed with incomplete information.

\textbf{Audit capability} confirms that every perception event is logged. You must be able to reconstruct what information the agent accessed during a task for compliance review.

% ----------------------------------------------------------------------------
% Connection to Other Questions
% ----------------------------------------------------------------------------

\subsection{From Perception to Action}
\label{sec:agents2-perception-action}

Perception enables agents to gather information, but professional value ultimately requires effecting change: filing documents, sending communications, executing trades. This distinction between observing the world and changing it is fundamental to agent architecture, and the protocols we use reflect it. MCP, introduced earlier in this section, explicitly separates \textbf{Resources} (read-only data access) from \textbf{Tools} (operations that modify state). A single MCP server might expose both capabilities---a document management system could offer read access to files alongside the ability to create, modify, or delete them---but the protocol distinguishes these so that access control and governance can treat them differently.

The distinction matters because the consequences of failure differ fundamentally. When perception tools fail---a search returns wrong results, a document fails to load---the external world remains unchanged. The agent can retry, try alternatives, or escalate to human review without having caused any harm beyond wasted time. Action tools carry different stakes entirely. They file documents that become part of court records, send emails that reach recipient inboxes, execute trades that transfer ownership at market prices. Once executed, many actions cannot be undone, or can only be undone at significant cost. The agent's mistakes become facts in the world.

\Cref{sec:agents2-action} examines action capabilities in detail, beginning with the conceptual foundations that distinguish actions from observations and developing the governance frameworks that these differences require.

% ============================================================================
% 05-action.tex
% Q4: How Does an Agent Make Things Happen?
% Part of: Chapter 07 - Agents Part II: How to Build an Agent
% ============================================================================

\section{How Does an Agent Make Things Happen?}
\label{sec:agents2-action}

% ----------------------------------------------------------------------------
% Opening: Q4 Framing and Organizational Analogy
% ----------------------------------------------------------------------------

A junior associate's job extends beyond research to producing work product. They draft memos, send emails, file documents, schedule meetings. A trader's job extends beyond analysis to execution. They enter orders, route trades, confirm allocations. The value comes from action, not just observation.

Agentic systems face the same imperative. An agent that only reads---searching databases, retrieving documents, analyzing information---produces no deliverable. To complete tasks, agents must \textit{act}: generate documents, send communications, update systems, execute transactions. Action implements the ``A'' in the GPA framework.

\begin{definitionbox}[title={Action Tools}]
\keyterm{Action tools} enable agents to change the state of external systems. Unlike perception tools (read-only), action tools \textit{write}: they file documents, send messages, execute trades, update databases. Once executed, some actions cannot be undone.

The distinction between perception and action is fundamental to governance. Perception risks include accessing wrong information or missing relevant data. Action risks include taking wrong actions that harm clients, violate regulations, or create liability.
\end{definitionbox}

% ----------------------------------------------------------------------------
% Action Tool Categories
% ----------------------------------------------------------------------------

\subsection{Action Tool Categories}
\label{sec:agents2-action-categories}

Action tools vary in consequence. The key dimension is \keyterm{reversibility}: can the action be undone if something goes wrong?

\textbf{Communication Tools (Partially Reversible)}: Communication tools send information to others:

\textbf{Internal communications}: Emails to colleagues, messages in collaboration platforms, updates to internal systems. These are partially reversible: you can follow up with corrections, but you cannot unsend.

\textbf{External communications}: Emails to clients, letters to opposing counsel, regulatory notifications. Higher stakes than internal; recipients outside your control. Retractions are possible but create their own problems.

\textbf{Automated alerts}: System-generated notifications, compliance alerts, deadline reminders. Often templated with limited customization.

Governance implication: Internal communications may proceed with post-hoc review; external communications typically require pre-approval.

\textbf{Document Management Tools (Largely Reversible)}: Document management tools create and organize work product:

\textbf{Document creation}: Drafting memos, generating reports, producing analysis. The documents exist internally and can be revised before distribution.

\textbf{Document organization}: Filing documents in management systems, tagging and categorizing, maintaining matter files. Generally reversible through re-organization.

\textbf{Template application}: Generating documents from templates, populating forms, producing standard documents. Low-risk if templates are validated.

Governance implication: Document creation is relatively low-risk; documents can be revised before external sharing. Validation before distribution is the key control.

\textbf{Filing and Submission Tools (Largely Irreversible)}: Filing tools submit documents to external authorities:

\textbf{Court filings}: E-filing through CM/ECF or state systems. Once filed, documents become part of the public record. Amendments are possible but do not erase the original.

\textbf{Regulatory submissions}: SEC filings through EDGAR, FINRA submissions, state regulatory filings. Subject to regulatory requirements; errors can trigger enforcement.

\textbf{Contract execution}: Signature, execution, delivery of binding agreements. Creates legal obligations that may be difficult or impossible to unwind.

Governance implication: Filing and submission require mandatory pre-approval. The irreversibility demands human verification before execution.

\textbf{Transaction Execution Tools (Irreversible or Costly)}: Transaction tools execute binding transactions:

\textbf{Trade execution}: Entering orders, executing trades, confirming allocations. Once executed, trades settle and create positions. Reversal requires offsetting trades at market prices.

\textbf{Payment processing}: Wire transfers, payment initiation, fund disbursements. Once sent, funds are gone. Recovery requires recipient cooperation.

\textbf{System updates}: Modifying production databases, updating client records, changing configurations. May affect downstream systems; reversal may be complex.

Governance implication: Transaction execution requires the strictest controls---multi-factor approval, segregation of duties, real-time monitoring.

% ----------------------------------------------------------------------------
% The Reversibility Framework
% ----------------------------------------------------------------------------

\subsection{The Reversibility Framework}
\label{sec:agents2-reversibility}

% Reversibility Spectrum - Horizontal Layout with Governance Requirements

\begin{figure}[!htb]
\centering
\resizebox{\textwidth}{!}{%
\begin{tikzpicture}[
    remember picture,
    level circle/.style={circle, minimum size=2.0cm, inner sep=0pt, draw=#1, line width=1.5pt, font=\bfseries\small, align=center},
    level box/.style={rounded corners=4pt, draw=#1, line width=1pt, inner sep=8pt, text width=3.2cm, font=\small\raggedright, align=left}
]

% Horizontal connecting line at top
\draw[gray-600, line width=1.5pt] (0,0) -- (14,0);

% Arrow showing increasing governance requirements
\draw[-{Stealth[length=3mm,width=2mm]}, gray-800, line width=1.2pt]
    (0.5,1.0) -- (13.5,1.0) node[midway, above, font=\small\bfseries] {Increasing Governance Requirements};

% Level 1: Fully Reversible (leftmost)
\node[level circle={green-900}, fill=green-100, text=green-900] (level1) at (0,0) {Fully\\Reversible};

\node[level box={green-900}, fill=green-100, below=0.8cm of level1] (box1) {%
\textbf{\textcolor{green-900}{Oversight}}\par
Post-hoc review\par\vspace{4pt}
\textbf{\textcolor{green-900}{Examples}}\par
Research; internal drafts\par\vspace{4pt}
\textbf{\textcolor{green-900}{Recovery}}\par
Delete/revise
};

% Level 2: Partially Reversible
\node[level circle={amber-900}, fill=amber-100, text=amber-900] (level2) at (4.67,0) {Partially\\Reversible};

\node[level box={amber-900}, fill=amber-100, below=0.8cm of level2] (box2) {%
\textbf{\textcolor{amber-900}{Oversight}}\par
Checkpoint review\par\vspace{4pt}
\textbf{\textcolor{amber-900}{Examples}}\par
Internal emails\par\vspace{4pt}
\textbf{\textcolor{amber-900}{Recovery}}\par
Correction message
};

% Level 3: Largely Irreversible (uses key-dark which is amber/orange-toned)
\node[level circle={key-dark}, fill=key-light, text=key-dark] (level3) at (9.33,0) {Largely\\Irreversible};

\node[level box={key-dark}, fill=key-light, below=0.8cm of level3] (box3) {%
\textbf{\textcolor{key-dark}{Oversight}}\par
Pre-approval required\par\vspace{4pt}
\textbf{\textcolor{key-dark}{Examples}}\par
Client communications\par\vspace{4pt}
\textbf{\textcolor{key-dark}{Recovery}}\par
Amendment/\\retraction
};

% Level 4: Irreversible (rightmost)
\node[level circle={red-900}, fill=red-100, text=red-900] (level4) at (14,0) {Irreversible};

\node[level box={red-900}, fill=red-100, below=0.8cm of level4] (box4) {%
\textbf{\textcolor{red-900}{Oversight}}\par
Multi-party approval\par\vspace{4pt}
\textbf{\textcolor{red-900}{Examples}}\par
Court filings; trades; wires\par\vspace{4pt}
\textbf{\textcolor{red-900}{Recovery}}\par
Offsetting transaction
};

\end{tikzpicture}
}%
\caption{Action reversibility spectrum and corresponding governance requirements. As actions become less reversible, oversight shifts from post-hoc review to pre-approval and multi-party authorization. Recovery mechanisms range from simple deletion for fully reversible actions to complex offsetting transactions for irreversible ones.}
\label{fig:agents2-reversibility-spectrum}
\end{figure}


Reversibility determines required oversight. Consider how you delegate to a junior associate:

\textbf{Fully reversible actions} (research, internal drafts): The associate works independently. If they make mistakes, you catch them in review. No external harm occurs.

\textbf{Partially reversible actions} (internal communications, document organization): Checkpoint review. The associate completes work; you review before it affects others significantly.

\textbf{Largely irreversible actions} (client communications, filings): Pre-approval required. The associate prepares; you approve before execution.

\textbf{Irreversible actions} (trade execution, fund transfers): Multi-party approval with controls. Multiple people must verify before execution.

\begin{table}[htbp]
\centering
\caption{Action reversibility and required oversight}
\label{tab:agents2-reversibility}
\small
\begin{tabular}{p{0.20\textwidth}p{0.25\textwidth}p{0.22\textwidth}p{0.20\textwidth}}
\toprule
\textbf{Reversibility} & \textbf{Examples} & \textbf{Oversight} & \textbf{Recovery} \\
\midrule
Fully reversible & Research, internal drafts, calculations & Post-hoc review & Delete/revise \\
\midrule
Partially reversible & Internal emails, document filing, alerts & Checkpoint review & Correction/follow-up \\
\midrule
Largely irreversible & Client communications, court filings, regulatory submissions & Pre-approval required & Amendment/retraction (visible) \\
\midrule
Irreversible & Trade execution, wire transfers, contract execution & Multi-party approval & Offsetting transaction (costly) \\
\bottomrule
\end{tabular}
\end{table}

Agent governance should track reversibility. The architecture should enforce appropriate controls based on action classification.

% ----------------------------------------------------------------------------
% MCP Tools for Action
% ----------------------------------------------------------------------------

\subsection{MCP Tools and Prompts for Action}
\label{sec:agents2-mcp-action}

The Model Context Protocol defines two capability types relevant to action:

\textbf{MCP Tools}: \keyterm{MCP Tools} are executable functions that may change state. Unlike Resources (read-only), Tools can:

\begin{itemize}[nosep]
\item Create and modify documents
\item Send communications
\item Submit filings
\item Execute transactions
\item Update external systems
\end{itemize}

Tool manifests should include risk metadata: reversibility classification, approval requirements, audit requirements. This enables hosts to enforce appropriate controls.

\textbf{MCP Prompts}: \keyterm{MCP Prompts} are reusable templates for common tasks. For action tools, prompts encode standard operating procedures:

\textbf{Legal examples}: Contract review checklist prompts, due diligence workflow prompts, filing preparation prompts.

\textbf{Financial examples}: Trade compliance check prompts, client onboarding prompts, regulatory submission prompts.

Prompts standardize action sequences, reducing variation and error. They are particularly valuable for high-stakes actions where consistency matters.

% ----------------------------------------------------------------------------
% Action Security
% ----------------------------------------------------------------------------

\subsection{Action Security}
\label{sec:agents2-action-security}

Every action interface is a potential security boundary. Actions access external systems, process inputs, and create real-world consequences.

\textbf{Core Controls}: All action tools must implement:

\textbf{Authentication}: Verify the agent is who it claims to be. Service accounts with strong credentials; no shared or default passwords.

\textbf{Authorization}: Verify the agent has permission for this specific action. Role-based access control; principle of least privilege.

\textbf{Input validation}: Reject malformed or suspicious requests. Validate all parameters before execution.

\textbf{Output confirmation}: For high-stakes actions, require confirmation before execution completes.

\textbf{Rate limiting}: Cap action frequency to prevent runaway execution. Escalate after repeated actions.

\textbf{Audit logging}: Record every action with full context: agent identifier, action name, parameters, timestamp, result, matter/client context.

\textbf{Threat-Specific Mitigations}: \textbf{Prompt injection through action parameters}: Adversaries embed instructions in parameters the agent passes to action tools. \textit{Mitigation}: Sanitize all parameters; never pass raw user input directly to action interfaces; validate parameter formats against strict schemas.

\textbf{Privilege escalation through tool chaining}: An agent chains multiple tools to achieve capabilities no single tool grants. \textit{Mitigation}: Analyze tool combinations for escalation paths; require human approval for tool sequences that span security boundaries.

\textbf{Action replay}: A captured action request is replayed to re-execute the action. \textit{Mitigation}: Implement nonces or timestamps; reject duplicate requests; maintain action logs for detection.

% ----------------------------------------------------------------------------
% Approval Workflows
% ----------------------------------------------------------------------------

\subsection{Approval Workflows}
\label{sec:agents2-approval}

For non-reversible actions, the agent prepares; humans approve. Several patterns implement this:

\textbf{Single Approver}: The agent completes preparation and presents to one designated approver. Appropriate for routine actions with clear approval authority.

\textbf{Pattern}: Agent prepares court filing $\rightarrow$ presents to supervising attorney $\rightarrow$ attorney reviews and approves $\rightarrow$ agent submits.

\textbf{Multi-Party Approval}: High-stakes actions require multiple independent approvers. Appropriate for actions with significant financial or legal exposure.

\textbf{Pattern}: Agent prepares wire transfer $\rightarrow$ operations reviews amounts and accounts $\rightarrow$ compliance reviews purpose and restrictions $\rightarrow$ manager provides final approval $\rightarrow$ agent executes.

\textbf{Escalating Approval}: Approval authority escalates with transaction size or risk. Routine actions have lower approval thresholds; exceptional actions escalate to senior personnel.

\textbf{Pattern}: Trades under \$100K $\rightarrow$ desk manager approval. Trades \$100K-\$1M $\rightarrow$ senior trader approval. Trades over \$1M $\rightarrow$ CIO approval.

\textbf{Approval Request Design}: Effective approval requests include:

\begin{itemize}[nosep]
\item Clear description of the proposed action
\item Context: why is this action needed?
\item Risk assessment: what could go wrong?
\item Reversibility: can this be undone?
\item Supporting evidence: what analysis supports this action?
\item Deadline: when is approval needed?
\end{itemize}

The approver should be able to make an informed decision from the request alone, without needing to investigate further.

% ----------------------------------------------------------------------------
% Rate Limiting and Circuit Breakers
% ----------------------------------------------------------------------------

\subsection{Rate Limiting and Circuit Breakers}
\label{sec:agents2-circuit-breakers}

Agents can get stuck in action loops: submitting the same request repeatedly, sending multiple messages, attempting failed transactions again and again. Controls prevent runaway execution:

\textbf{Rate Limiting}: Cap how many actions the agent can take per time period:

\textbf{Per-action limits}: No more than 5 emails per minute; no more than 10 trades per hour.

\textbf{Per-matter limits}: No more than 20 actions on any single matter per day without human review.

\textbf{Cost limits}: No more than \$1,000 in transaction costs per session.

When limits are reached, the agent pauses and escalates rather than continuing.

\textbf{Circuit Breakers}: Automatic stops when anomalies are detected:

\textbf{Repeated failures}: If the same action fails 3 times, stop and escalate. Do not retry indefinitely.

\textbf{Unusual patterns}: If action rate suddenly spikes, pause for review. May indicate agent malfunction or compromise.

\textbf{Threshold breaches}: If cumulative actions exceed daily limits, stop automatically. Resume requires human authorization.

Circuit breakers transform potential runaway failures into controlled pauses that allow human intervention.

% ----------------------------------------------------------------------------
% Evaluating Action Capabilities
% ----------------------------------------------------------------------------

\subsection{Evaluating Action Capabilities}
\label{sec:agents2-action-eval}

When evaluating agentic systems, assess action capabilities against these criteria:

\textbf{Action inventory}: What actions can the agent take? Map available action tools against workflow requirements.

\textbf{Reversibility classification}: Is each action properly classified? Are controls appropriate to reversibility level?

\textbf{Approval workflows}: Are approval gates implemented for non-reversible actions? Do approvers receive sufficient information?

\textbf{Security controls}: Are authentication, authorization, and audit logging implemented? Have penetration tests been conducted?

\textbf{Rate limiting}: Are limits in place? Do they match acceptable risk tolerances?

\textbf{Rollback capability}: What happens when actions fail? Are recovery procedures documented and tested?

% ----------------------------------------------------------------------------
% Connection to Other Questions
% ----------------------------------------------------------------------------

\subsection{From Action to Governance}
\label{sec:agents2-action-governance}

Action tools are where agentic systems create real-world consequences. The governance implications are significant:

\textbf{Perception risks} (Q3) include accessing wrong or incomplete information. The agent reasons from bad data, but no external harm has occurred yet.

\textbf{Action risks} (Q4) include taking wrong actions that harm clients, violate regulations, or create liability. The consequences are external and may be irreversible.

This section introduced action control concepts at the architectural level: the reversibility framework, approval workflow patterns, and rate limiting mechanisms. The following chapter---\textit{How to Govern an Agent}---provides comprehensive governance treatment, examining the five-layer governance stack, dimensional controls that calibrate oversight to risk, regulatory compliance frameworks specific to legal and financial services, and accountability structures that assign responsibility when things go wrong.

Within this chapter, \Cref{sec:agents2-escalation} examines when agents should \textit{not} act---recognizing situations that require human decision-making rather than autonomous execution. The interplay between action capability and escalation judgment is central to safe agent deployment.

\Cref{sec:agents2-memory} examines the next question: how does an agent remember things? Memory enables agents to maintain context across sessions, learn from experience, and access institutional knowledge.

% ============================================================================
% 06-memory.tex
% Q5: How Does an Agent Remember Things?
% Part of: Chapter 07 - Agents Part II: How to Design an Agent
% ============================================================================

\section{How Does an Agent Remember Things?}
\label{sec:agents2-memory}

% ----------------------------------------------------------------------------
% Opening: Q5 Framing and Organizational Analogy
% ----------------------------------------------------------------------------

The previous two sections examined how agents gather information (perception) and effect change (action). However, these capabilities operate in the moment: the agent perceives the current state, reasons about it, and acts. Without memory, every interaction resets. The agent cannot recall prior research, successful strategies, or case history. Memory transforms an agent from a stateless tool into a system that learns and adapts.

Every experienced professional knows that institutional memory distinguishes efficient work from reinventing the wheel. When starting a new securities registration or revisiting an investment thesis, you do not begin from scratch. You pull prior filings, review comment history, check precedent databases, and update existing models with new data. The firm maintains templates and research files that incorporate years of accumulated knowledge. Memory in agentic systems serves this same function: context retention across sessions and learning from experience, building on prior work rather than starting over.

\begin{definitionbox}[title={Agent Memory}]
	\keyterm{Agent memory} stores and retrieves information across timescales \parencite{park2023generative,wang2024llmagents}.
	\begin{itemize}[nosep]
		\item \textbf{Working Memory:} The documents actively loaded in the agent's context (like papers on a desk).
		\item \textbf{Episodic Memory:} The history of actions and outcomes for a specific matter (like a case file).
		\item \textbf{Semantic Memory:} The general principles and institutional knowledge available for retrieval (like the firm's precedent archive).
	\end{itemize}
\end{definitionbox}

% ----------------------------------------------------------------------------
% Memory Types
% ----------------------------------------------------------------------------

\subsection{Memory Types: From Desk to Archive}
\label{sec:agents2-memory-types}

Law firms use layered filing systems, each suited to different timescales. The associate's desk holds active work; the matter file contains engagement history; the precedent database archives institutional knowledge. Each layer trades immediacy for capacity. Agent memory systems follow the same pattern.

\textbf{Working memory} utilizes the \keyterm{context window}---the text currently loaded in the LLM's active attention. Just like desk space, context windows have strict limits. An associate can only have so many documents open at once; an agent can only hold so many \keyterm{tokens} (units of text, roughly 0.75 words) in active context. As of late 2025, leading models handle roughly 200,000 tokens. When the task exceeds this limit, you need other storage systems. In finance, this parallels the active trading screen: live prices and positions are immediate but transient.

\textbf{Episodic memory} corresponds to the matter file. Every memo, correspondence, and research result related to an engagement goes here. The associate does not re-research answered questions; the file provides the history. In agentic systems, episodic memory captures the log of actions and outcomes \parencite{park2023generative}. The agent records: ``I searched for Ninth Circuit venue cases, found three opinions, and drafted the analysis.'' When asked a follow-up, the agent retrieves that prior state. This mirrors the financial research file: you pull prior analysis and update it, rather than starting fresh.

\textbf{Semantic memory} is the firm's precedent database. Institutional knowledge accumulates over decades. When you need a force majeure clause, the database offers fifty examples. Agentic systems access semantic memory through the RAG pattern introduced in \Cref{sec:agents2-icl-retrieval}: retrieve relevant content from a large corpus, inject it into the prompt, and generate a response informed by that specific knowledge \parencite{lewis2020rag}.

The retrieval infrastructure powering semantic memory---embeddings and vector stores---was defined in \Cref{sec:agents2-icl-retrieval}. Here, the key insight is that semantic memory works because of \textit{in-context learning}: the model adapts its behavior based on what appears in its prompt. Retrieved precedents become part of the agent's working context, allowing it to reason about specific examples even when those examples were created after the model's training.

% ----------------------------------------------------------------------------
% RAG Implementation
% ----------------------------------------------------------------------------

\subsection{Implementing the RAG Pattern}
\label{sec:agents2-rag}

The RAG pattern (\Cref{sec:agents2-icl-retrieval}) requires careful implementation to work reliably in professional contexts. The basic pattern---retrieve, augment, generate---becomes a pipeline with several stages, each introducing design decisions that affect quality, latency, and cost.

\paragraph{The RAG Pipeline}

A typical implementation has four stages:
\begin{enumerate}
	\item \textbf{Chunking:} Breaks documents into semantic units (sentences, paragraphs, sections, or sliding windows) while preserving metadata (source, date, jurisdiction). Chunk size affects retrieval: too small loses context; too large dilutes relevance.
	\item \textbf{Embedding:} Converts each chunk into a vector using an embedding model. Different models have different strengths; legal-specific embeddings may outperform general-purpose ones for professional content.
	\item \textbf{Retrieval:} Finds chunks similar to the query by comparing vectors (using cosine similarity or similar metrics). Returns the top-$k$ most similar chunks.
	\item \textbf{Augmentation and Generation:} Injects retrieved chunks into the prompt and generates a response. The model's in-context learning capability allows it to reason about the retrieved content.
\end{enumerate}

% fig-rag-pipeline.tex
% RAG Pipeline: Embedding-Based Implementation
% Part of: Chapter 2 - How to Design an Agent
% Section: 06-memory (RAG)

\begin{figure}[htbp]
\centering
\begin{tikzpicture}[
    % Step box style
    step box/.style={
        rectangle,
        rounded corners=6pt,
        minimum width=2.8cm,
        minimum height=1.5cm,
        align=center,
        line width=1.5pt,
        font=\small\bfseries
    },
    % Input/Output box style
    io box/.style={
        rectangle,
        rounded corners=6pt,
        minimum width=2.4cm,
        minimum height=1.5cm,
        align=center,
        line width=1.5pt,
        font=\small\bfseries,
        fill=bg-key,
        draw=key-dark,
        text=key-dark
    },
    % Description style
    desc/.style={
        font=\scriptsize,
        text width=2.6cm,
        align=center,
        anchor=north
    },
    % Arrow style
    arrow/.style={
        -stealth,
        line width=1.5pt,
        color=gray-500
    }
]

% Row 1 (top): Documents → Chunking → Embedding
\node[io box] (docs) at (0, 0) {Documents};
\node[step box, fill=bg-definition, draw=definition-dark, text=definition-dark] (chunk) at (3.8, 0) {1. Chunking};
\node[step box, fill=bg-definition, draw=definition-dark, text=definition-dark] (embed) at (7.6, 0) {2. Embedding};

% Descriptions for top row (above boxes)
\node[desc, text=gray-600, anchor=south] at (chunk.north) [yshift=0.15cm] {Break documents\\into semantic units};
\node[desc, text=gray-600, anchor=south] at (embed.north) [yshift=0.15cm] {Convert chunks\\into vectors};

% Row 2 (bottom): Retrieval → Generation → Answer
\node[step box, fill=bg-definition, draw=definition-dark, text=definition-dark] (retrieve) at (0, -2.6) {3. Retrieval};
\node[step box, fill=bg-definition, draw=definition-dark, text=definition-dark] (generate) at (3.8, -2.6) {4. Generation};
\node[io box] (answer) at (7.6, -2.6) {Answer};

% Descriptions for bottom row (below boxes)
\node[desc, text=gray-600, anchor=north] at (retrieve.south) [yshift=-0.15cm] {Find similar chunks\\by vector comparison};
\node[desc, text=gray-600, anchor=north] at (generate.south) [yshift=-0.15cm] {Inject context\\into LLM prompt};

% Arrows - top row
\draw[arrow] (docs.east) -- (chunk.west);
\draw[arrow] (chunk.east) -- (embed.west);

% Arrow connecting rows (curved down)
\draw[arrow] (embed.south) -- ++(0, -0.4) -| (retrieve.north);

% Arrows - bottom row
\draw[arrow] (retrieve.east) -- (generate.west);
\draw[arrow] (generate.east) -- (answer.west);

\end{tikzpicture}
\caption{An embedding-based RAG pipeline---one common implementation pattern. Documents are chunked into semantic units, embedded as vectors, and retrieved by semantic similarity. Alternative implementations may skip chunking and embedding entirely, using keyword search, BM25, database queries, or hybrid approaches to retrieve relevant content.}
\label{fig:agents2-rag-pipeline}
\end{figure}


\paragraph{Advanced RAG Patterns}

Robust implementations enhance basic RAG with patterns that improve precision and authority:
\begin{itemize}[nosep]
	\item \textbf{Hybrid retrieval} combines semantic search (embeddings) with keyword search (BM25) \parencite{robertson2009bm25}, catching both conceptual matches and exact terms (like specific case citations that must match precisely).
	\item \textbf{Query rewriting} transforms vague questions (``What's the rule?'') into specific search queries based on conversation history and domain knowledge \parencite{ma2023queryrewriting}.
	\item \textbf{Reranking} scores initial results by authority or relevance, ensuring binding precedent ranks above secondary sources \parencite{yu2024rankrag}. A Supreme Court opinion should outrank a blog post, even if both are semantically similar to the query.
	\item \textbf{Filtered retrieval} constrains results by metadata---jurisdiction, date, document type---preventing agents from citing inapplicable authority.
\end{itemize}

\begin{keybox}[title={Warning: The Hallucination Risk}]
	Fabricated citations remain a critical failure mode even with RAG. Studies show that RAG-enabled tools can hallucinate 17--33\% of the time \parencite{dahl2024hallucinations}. The model may generate plausible-sounding citations that do not exist in the retrieved context---or anywhere.

	\vspace{0.5em}
	You must implement verification: before any citation reaches the user, confirm that the source exists in the retrieved context. This is not optional for professional applications.
\end{keybox}

% ----------------------------------------------------------------------------
% Domain-Specific Memory Considerations
% ----------------------------------------------------------------------------

\subsection{Domain-Specific Memory Considerations}
\label{sec:agents2-memory-domain}

Memory systems for regulated industries require enhancements that go well beyond generic implementations. Three dimensions matter most: authority, jurisdiction, and time.

Consider how a legal researcher thinks about sources. When investigating insider trading liability, a Supreme Court opinion carries far more weight than a law review article discussing the same topic. The human researcher instinctively applies this hierarchy; an agent's memory system must do the same. This means tagging documents with their position in the authority hierarchy---primary sources like statutes and binding precedent at the top, secondary sources like treatises and articles below---and ensuring that retrieval algorithms surface higher-authority documents more prominently. Financial systems face analogous challenges: an SEC no-action letter provides more reliable guidance than a client alert summarizing the same issue, and memory systems should reflect that difference.

Jurisdiction adds another layer of complexity. Legal information does not exist in a vacuum; it operates within territorial boundaries. California precedent does not bind Texas courts, and New York banking regulations have no force in London. A memory system that fails to account for these boundaries will produce results that are not merely unhelpful but actively misleading. When an agent researches Delaware corporate law, it must filter results to surface Delaware authorities as controlling while clearly distinguishing persuasive authority from other jurisdictions. This requires rich metadata tagging and retrieval logic that can enforce strict jurisdictional filtering when the task demands it.

Time presents perhaps the most challenging dimension. Law evolves through legislative amendments, regulatory updates, and judicial decisions that overturn or limit prior holdings. A case from 1985 may have been explicitly overruled, limited to its facts, or superseded by statute. Memory systems must integrate with citator services like Shepard's or KeyCite to validate that retrieved precedents remain good law. Financial data introduces even more varied temporal requirements: market prices become stale in milliseconds, earnings reports remain relevant for quarters, and industry analyses may hold value for years. Effective memory systems tag all data with effective dates and expiration windows, triggering refresh processes when content ages beyond its useful life.

Finally, professional domains create identifier resolution challenges that generic systems rarely encounter. Legal citations appear in multiple formats---``123 F.3d 456'' and ``123 F3d 456'' refer to the same case, but a naive system might treat them as distinct. Companies accumulate multiple identifiers across different contexts: stock tickers, CUSIPs, ISINs, and Legal Entity Identifiers (LEIs) all point to the same entity but appear in different documents and databases. Without careful normalization, retrieval systems fail to connect related records, fragmenting information that belongs together.

% ----------------------------------------------------------------------------
% Matter and Client Isolation (CRITICAL)
% ----------------------------------------------------------------------------

\subsection{Matter and Client Isolation}
\label{sec:agents2-memory-isolation}

Perhaps no aspect of memory architecture matters more for professional services than enforcing strict \keyterm{ethical walls} between matters and clients. The consequences of failure are severe: if an agent working on Matter A inadvertently accesses privileged information from Matter B---particularly when the matters involve adverse parties---the result may be privilege waiver, disqualification, and malpractice liability. Financial services face parallel risks when Material Non-Public Information (MNPI) leaks across the walls that separate investment banking from trading operations.

Implementing effective separation requires a layered approach that begins with architectural isolation. Each matter should occupy its own namespace within the memory system, creating a logical partition that separates its documents, notes, and work product from all other matters. Retrieval operations must be scoped to respect these boundaries: when an agent queries memory in the context of a particular matter, the system should only search within that matter's namespace, never reaching across into other partitions regardless of how relevant the results might appear.

Access controls provide the next layer of protection. Role-based permissions determine which agents and human users can access each namespace, mirroring the ethical wall policies that law firms and financial institutions already maintain for their human professionals. An associate staffed on a merger transaction should have access to that deal's namespace but not to the namespace for litigation against the same company being handled by a different team. When delegation introduces multiple agents accessing the same matter namespace (\Cref{sec:agents2-delegation}), isolation becomes even more critical: coordination among agents must not inadvertently leak information across matter boundaries.

Audit trails complete the picture by creating a verifiable record of every interaction with the memory system. Each read and write operation should be logged with a timestamp, the identity of the requesting agent or user, and the matter identifier. These logs serve multiple purposes: they enable compliance review, support investigations if a breach is suspected, and provide documentation that the organization maintained appropriate controls.

Retention and deletion policies add a temporal dimension to isolation. Organizational policy often requires that matter files be retained for specified periods and then destroyed; legal and regulatory requirements may impose additional constraints depending on the engagement type, jurisdiction, and applicable rules. When a matter closes or a client relationship ends, the associated memory namespace should be archived or deleted according to a documented retention schedule. Critically, deletion must be verifiable---the organization needs confidence that purged data is truly gone, not merely hidden from normal retrieval. These isolation and audit requirements exemplify the architectural patterns for privilege enforcement and logging detailed in \Cref{sec:agents2-logging-arch}.

% ----------------------------------------------------------------------------
% Evaluating Memory Systems
% ----------------------------------------------------------------------------

\subsection{Evaluating Memory Systems}
\label{sec:agents2-memory-eval}

Memory evaluation requires four assessments: \textit{retrieval quality} (precision and recall against expert work product---does it find the authorities a skilled attorney would cite?), \textit{isolation integrity} (adversarial testing to verify matter/client boundaries hold), \textit{temporal validity} (tracking whether retrieved precedent remains good law and how quickly updates propagate), and \textit{scale performance} (latency as corpus grows to tens of thousands of documents). See the consolidated Evaluation Checklist in \Cref{sec:agents2-evaluation-checklist} for memory-specific assessment criteria including isolation integrity testing, retention policy verification, and temporal validity checks.

% ----------------------------------------------------------------------------
% Adaptation: In-Context Learning as a System Design Element
% ----------------------------------------------------------------------------

\subsection{Adaptation: Learning and Its Consequences}
\label{sec:agents2-adaptation}

Memory enables \keyterm{adaptation}---the ``A'' in the GPA+IAT framework from \href{https://papers.ssrn.com/sol3/papers.cfm?abstract_id=5806982}{\textit{What is an Agent?}}. An agent that adapts changes its behavior based on experience, ideally improving over time. This capability transforms agents from static tools into systems that learn. But adaptation also introduces risks that static systems avoid.

\begin{definitionbox}[title={Adaptation}, breakable=false]
	\keyterm{Adaptation} is behavioral change based on experience. In agentic systems, adaptation occurs through three mechanisms:
	\begin{itemize}[nosep]
		\item \textbf{In-context adaptation:} The agent changes behavior within a session based on instructions, examples, or retrieved content. This is in-context learning in action.
		\item \textbf{Cross-session adaptation:} The agent changes behavior across sessions by persisting information in memory and retrieving it later.
		\item \textbf{Model-level adaptation:} The model's weights are updated through fine-tuning or reinforcement learning. This is less common in deployed systems due to cost and complexity.
	\end{itemize}
\end{definitionbox}

In-context adaptation is immediate and powerful. You provide the agent with examples of preferred output format, and it adapts its responses accordingly. You correct an error, and subsequent responses reflect that correction. This is why retrieval matters: by controlling what appears in the agent's context, you control how it adapts.

Cross-session adaptation requires persistent memory. The agent stores information---user preferences, successful strategies, matter-specific context---and retrieves it in future sessions. This creates continuity: the agent ``remembers'' that this client prefers concise memos, that this portfolio manager wants risk metrics in basis points, that this matter involves a specific contractual provision.

\paragraph{Opportunities from Adaptation}

Adaptation enables capabilities that static systems cannot match:

\textbf{Personalization.} The agent learns user preferences---communication style, level of detail, preferred formats---and tailors responses accordingly. A partner who always asks follow-up questions about procedural history gets proactive procedural context. An analyst who focuses on downside scenarios gets risk-weighted analysis by default.

\textbf{Performance improvement.} The agent learns from feedback. Corrections persist: ``Actually, this jurisdiction uses different venue rules'' becomes part of the agent's knowledge for future queries. Successful strategies are reinforced: the research approach that found relevant authority is remembered and reapplied.

\textbf{Domain specialization.} Through accumulated episodic and semantic memory, agents develop expertise in specific practice areas or market segments. An agent that has worked on dozens of M\&A transactions ``knows'' the typical deal structure, common negotiation points, and relevant precedents---not through training, but through memory.

\textbf{Continuity across handoffs.} When a professional leaves or a matter transfers, institutional knowledge often walks out the door. Memory-enabled agents maintain continuity. The research history, strategic decisions, and accumulated context persist in retrievable form.

\paragraph{Risks from Adaptation}

The same mechanisms that enable adaptation create risks that require governance:

\textbf{Security: Memory poisoning.} If an adversary can write to an agent's memory, they can influence future behavior \parencite{clop2024ragpoisoning}. A document designed to be retrieved---containing misleading instructions or false information---can corrupt the agent's responses. This is prompt injection via the memory layer. The attack surface expands beyond the current conversation to include everything the agent might retrieve.

\textbf{Security: Data leakage through memory.} Memory that persists across sessions can leak information across contexts. If the agent stores sensitive information from Matter A and retrieves it during Matter B, confidentiality is breached. This risk intensifies with shared memory systems that serve multiple users or matters.

\textbf{Governance: Behavioral drift.} As agents adapt, their behavior changes in ways that may not be predictable or desirable. An agent that learns from user feedback might learn bad habits---shortcuts that work in routine cases but fail in edge cases. Accumulated memory can shift the agent's responses away from its original design.

\textbf{Governance: Accountability gaps.} When behavior depends on memory state, accountability becomes complex. Which version of the agent produced this output? What was in its memory at the time? If the agent's behavior changes based on accumulated experience, audit trails must capture not just inputs and outputs but memory state.

Detecting and responding to these risks---particularly drift and leakage---requires escalation triggers and monitoring mechanisms explored in \Cref{sec:agents2-escalation}, which addresses when anomalies warrant human review.

\textbf{Cost: Storage and retrieval overhead.} Memory consumes resources. Episodic memory for a complex litigation matter can grow to millions of tokens. Semantic memory for a precedent database can require terabytes of embeddings. Every retrieval operation has latency and cost. These costs compound as memory grows.

\textbf{Latency: Retrieval time.} RAG adds latency to every query. The agent must embed the query, search the vector store, retrieve results, and inject them into the prompt before generation begins. For real-time applications---trading, live client calls---this latency may be unacceptable.

\begin{keybox}[title={The Adaptation Tradeoff}]
	Adaptation is not optional. Any system that uses RAG, maintains conversation history, or persists user preferences is adapting. The question is not whether to allow adaptation, but how to govern it.

	\vspace{0.5em}
	Key governance questions:
	\begin{itemize}[nosep]
		\item \textbf{What can be written to memory?} Control the write path to prevent poisoning.
		\item \textbf{What can be retrieved?} Scope retrieval to prevent leakage.
		\item \textbf{How is memory validated?} Review accumulated memory for accuracy and appropriateness.
		\item \textbf{When is memory cleared?} Define retention policies and implement secure deletion.
		\item \textbf{How is drift detected?} Monitor behavior changes over time.
	\end{itemize}
\end{keybox}

These governance questions are not abstract---they map directly to architectural controls developed in \Cref{sec:agents2-governance}, which synthesizes logging, override mechanisms, and isolation patterns specifically designed to manage adaptation risks.

% ----------------------------------------------------------------------------
% Connection to Other Questions
% ----------------------------------------------------------------------------

\subsection{From Memory to Planning}
\label{sec:agents2-memory-planning}

Memory provides the context agents need to plan effectively. Without memory, agents cannot learn from failed strategies, cannot build on prior work, and cannot maintain the continuity that complex tasks require. The adaptation mechanisms discussed above (\Cref{sec:agents2-adaptation}) transform memory from passive storage into active learning---but that learning must be channeled into productive planning.

Consider how an experienced associate approaches a new research task. They do not start from scratch; they recall similar matters, retrieve relevant precedents, and apply strategies that worked before. Memory-enabled agents can do the same: retrieve prior research, recall successful approaches, and avoid repeating failures.

\Cref{sec:agents2-planning} examines the next question: how does an agent break a big job into steps? Just as the case file enables strategic litigation planning, agent memory enables systematic task decomposition. The adaptation capabilities discussed here---personalization, domain specialization, performance improvement---all feed into more effective planning.

% ============================================================================
% 07-planning.tex
% Q6: How Does an Agent Break a Big Job into Steps?
% Part of: Chapter 07 - Agents Part II: How to Build an Agent
% ============================================================================

\section{How Does an Agent Break a Big Job into Steps?}
\label{sec:agents2-planning}

% ----------------------------------------------------------------------------
% Opening: Q6 Framing and Organizational Analogy
% ----------------------------------------------------------------------------

A litigation partner approaching a new matter does not start by drafting motions. The partner develops a strategy: discovery first (what facts do we need?), then dispositive motions if the law clearly favors us, settlement discussions in parallel, trial prep as a backstop. Discovery breaks into phases: initial disclosures, document requests, interrogatories, depositions. Tasks distribute across the team: senior associate handles briefing, junior associate does document review, paralegal manages scheduling and filings. Throughout, the partner monitors progress: are we on track for deadlines? Are discovery responses revealing helpful facts or should we adjust our theory?

This is \keyterm{planning}: decomposing complex goals into action sequences, much like the litigation roadmap or deal timeline that guides execution. Without planning, agents react to immediate observations without strategy. With planning, they work systematically toward objectives, adapt when circumstances change, and know when they're done.

\begin{definitionbox}[title={Planning}]
\keyterm{Planning} decomposes complex goals into sequences of actions. It encompasses:

\begin{itemize}[nosep]
\item \textbf{Decomposition}: Breaking large tasks into manageable steps
\item \textbf{Sequencing}: Ordering steps logically (what depends on what?)
\item \textbf{Allocation}: Assigning steps to tools or agents
\item \textbf{Monitoring}: Tracking progress toward the goal
\item \textbf{Adaptation}: Adjusting the plan when circumstances change
\end{itemize}

Without planning, an agent is like an associate who keeps running searches without a research strategy, busy but not progressing toward a deliverable.
\end{definitionbox}

% ----------------------------------------------------------------------------
% Planning Patterns
% ----------------------------------------------------------------------------

\subsection{Planning Patterns}
\label{sec:agents2-planning-patterns}

Three patterns dominate agent planning, each suited to different task types:

\textbf{ReAct: Reasoning + Acting.} The most fundamental pattern interleaves reasoning with action \parencite{yao2022react}. The partner asks for authority that a forum selection clause is unenforceable. The associate reasons: ``Key grounds are unconscionability and public policy. Start with \textit{Atlantic Marine}.'' They search, observe results, reason again: ``The unconscionability cases involve consumer adhesion contracts---not our commercial situation. The public policy line is closer.'' They search again, refine based on results.

Each cycle has three components:

\begin{itemize}[nosep]
\item \textbf{Thought}: Explicit reasoning about what to do next
\item \textbf{Action}: Tool call to gather information or effect change
\item \textbf{Observation}: Tool output that informs the next thought
\end{itemize}

Reasoning traces make decisions transparent and auditable. ReAct works well for exploratory tasks where you learn as you go---research questions, fact investigation, market analysis.

\textbf{Plan-Execute.} This pattern separates planning from execution. For document review (``Review 50 contracts for choice-of-law, forum selection, arbitration, and liquidated damages provisions''), the associate makes a plan: checklist of provisions, open each contract, record findings. Then they execute systematically. The plan does not change because the task is well-defined.

Plan-Execute fits workflows with established procedures: due diligence checklists, compliance reviews, document assembly. You create the plan upfront and execute methodically. Research variants like ReWOO (which separates reasoning from observation to reduce token usage) and LLMCompiler (which optimizes execution graphs for parallelism) enable parallel tool calling when steps are independent, though the basic pattern remains: plan first, then execute.

\textbf{Hierarchical Planning.} Law firms decompose matters into workstreams delegated through layers. A parent agent receives a high-level goal, breaks it into sub-goals, and delegates to specialists.

``Prepare for trial'' becomes:
\begin{itemize}[nosep]
\item Finalize witness list (delegated to one agent)
\item Prepare exhibits (another agent)
\item Draft jury instructions (another agent)
\end{itemize}

Each specialist may decompose further. This enables parallelization and specialization, mirroring how litigation teams work with multiple associates and paralegals handling different workstreams simultaneously.

See \Cref{sec:agents2-delegation} for detailed treatment of multi-agent coordination patterns.

% ----------------------------------------------------------------------------
% Choosing the Right Pattern
% ----------------------------------------------------------------------------

\subsection{Choosing the Right Planning Pattern}
\label{sec:agents2-planner-selection}

Selecting the right pattern depends on task structure and required autonomy level:

\begin{table}[htbp]
\centering
\caption{Planning pattern selection guide}
\label{tab:agents2-planner-selection}
\small
\begin{tabular}{p{0.22\textwidth}p{0.14\textwidth}p{0.14\textwidth}p{0.32\textwidth}}
\toprule
\textbf{Task Type} & \textbf{Pattern} & \textbf{Autonomy} & \textbf{Example} \\
\midrule
Well-defined steps, known scope & Plan-Execute & Moderate & Credit review, compliance audit, due diligence checklist \\
\midrule
Exploratory, learns as it goes & ReAct & Higher & Legal research, fact investigation, market analysis \\
\midrule
Complex, parallel workstreams & Hierarchical & Distributed & M\&A transaction, portfolio construction, multi-jurisdiction filing \\
\bottomrule
\end{tabular}
\end{table}

The autonomy column matters for governance. Higher-autonomy patterns require more sophisticated oversight:

\textbf{Plan-Execute (Moderate autonomy)}: The agent operates within tight bounds defined by the plan. Oversight focuses on plan validation and output review.

\textbf{ReAct (Higher autonomy)}: The agent makes decisions about what to search, what to pursue, when to stop. Oversight requires explicit termination mechanisms, confidence thresholds, and reasoning trace review.

\textbf{Hierarchical (Distributed autonomy)}: Multiple agents make decisions. Oversight requires clear delegation contracts, escalation paths between agents, and coordination monitoring.

Match oversight rigor to autonomy level.

% ----------------------------------------------------------------------------
% Understanding the Task
% ----------------------------------------------------------------------------

\subsection{Understanding the Task Before Planning}
\label{sec:agents2-pre-planning}

Before planning, agents must understand what they're being asked to do. \Cref{sec:agents2-intent} covers intent extraction in detail. For planning purposes, the key outputs are:

\textbf{Task classification}: Is this exploratory (ReAct), structured (Plan-Execute), or complex (Hierarchical)?

\textbf{Constraints}: What bounds the work? Deadlines, budgets, scope limitations.

\textbf{Success criteria}: How will we know when we're done? What deliverable is expected?

Effective planning requires clear inputs. Ambiguous goals produce unfocused plans; unclear success criteria make termination difficult.

% ----------------------------------------------------------------------------
% Budget Architecture
% ----------------------------------------------------------------------------

\subsection{Budget Architecture}
\label{sec:agents2-budgets}

Without explicit resource budgets, agents can run indefinitely. This is the ``runaway associate'' problem: you asked for two cases, the associate gives you fifty because they didn't know when the answer was sufficient.

\textbf{Budget Types.} Four budget types provide control over agent execution, each addressing a different dimension of resource consumption. Token budgets limit LLM API consumption, preventing expensive runaway reasoning loops where the agent keeps elaborating without making progress. Time budgets enforce deadlines by stopping execution after a fixed duration---perhaps 10 minutes---if no meaningful progress has occurred. Tool call budgets prevent runaway tool loops by capping the number of external calls; after 20 searches without progress, the agent should escalate rather than continuing to search. Cost budgets cap total spending in dollars, particularly important when using expensive models or external APIs where unconstrained execution could generate substantial charges.

These budgets cascade through levels: session budgets constrain entire engagements, task budgets allocate resources to specific work items, and subtask budgets subdivide further. A legal research task might receive a 30-minute time budget and 50,000-token limit; if it spawns subtasks, those subtasks share the parent budget rather than each receiving unlimited resources.

\textbf{Cost at Scale.} Token costs compound across agentic workflows. Consider a credit facility review: a 200-page document requires roughly 80,000 tokens to ingest. Each section analysis might consume 10,000--20,000 tokens across reasoning and tool calls. Retrieval from precedent databases adds tokens. Multi-iteration refinement multiplies costs.

A comprehensive review might consume 500,000--1,000,000 tokens. At illustrative pricing (late 2025: roughly \$3--15 per million input tokens for leading models; verify current rates), that's \$2--15 per review in API costs alone---before infrastructure, storage, or human review time.

For portfolio management running continuously, costs accumulate differently: thousands of small queries per day rather than occasional large tasks. Monitor aggregate daily/weekly costs, not just per-task.

\textbf{Economic Considerations.} When does agent assistance cost less than human work? Retrieval-heavy tasks (research, document review) show the clearest ROI when agents reduce hours substantially. Judgment-intensive tasks show less clear ROI when extensive human revision is required. The critical variable is human review time: agent output requiring extensive correction may cost more than human-only work.

Billing norms are evolving: some firms pass efficiency gains through as reduced hours, others add technology fees, others use fixed-fee arrangements. ABA Formal Opinion 512 requires competence regardless of tools and reasonable billing \parencite{aba-formal-opinion-512}. Transparency about AI assistance enables clients to evaluate the value proposition.

\textbf{Graceful Degradation.} When budgets tighten, agents should degrade gracefully rather than failing completely. Tiered outputs provide value at every budget level: minimal budget delivers the controlling statute with citation; moderate budget adds key holdings; full budget delivers comprehensive analysis. The user receives something useful regardless of where termination occurs.

Soft limits at 75--80\% of budget warn the agent that resources are running low, prompting it to prioritize completion over exploration. If the agent has found adequate authority, it should synthesize rather than searching for more. Hard limits at 100\% terminate execution and return whatever partial results exist. A budget-aware agent that delivers partial results is more useful than one that fails completely, and partial results often suffice for the user's immediate needs.

% ----------------------------------------------------------------------------
% Knowing When to Stop
% ----------------------------------------------------------------------------

\subsection{Knowing When to Stop}
\label{sec:agents2-termination-preview}

Perhaps the most critical planning capability is knowing when to stop. \Cref{sec:agents2-termination} provides comprehensive treatment; for planning purposes, four categories of stopping conditions guide agent behavior.

Success conditions terminate execution when the goal is achieved: the research question is answered, the document is reviewed, the analysis is complete. The agent returns its result and stops. Resource exhaustion terminates execution when budget limits are reached, returning partial results or escalating for additional allocation. Confidence thresholds terminate execution when uncertainty is too high for autonomous action, escalating for human review rather than proceeding with unreliable conclusions. Error conditions terminate execution when repeated failures indicate a problem that retrying will not solve.

Define explicit stopping rules, just as you would instruct an associate: ``If you find three on-point circuit opinions that all agree, you're done. If you've searched for two hours and found nothing, come talk to me.'' Agents need the same clarity about when their work is complete.

% ----------------------------------------------------------------------------
% Guardrails and Loop Detection
% ----------------------------------------------------------------------------

\subsection{Guardrails and Loop Detection}
\label{sec:agents2-guardrails}

Even with budgets and termination conditions, agents can get stuck in unproductive loops. Multiple mechanisms detect and prevent these patterns: step limits, reflection checkpoints, external watchdogs, and meta-policies. \Cref{sec:agents2-loop-detection} provides comprehensive treatment of loop detection and guardrail mechanisms in the context of termination.

% ----------------------------------------------------------------------------
% Connection to Other Questions
% ----------------------------------------------------------------------------

\subsection{From Planning to Termination}
\label{sec:agents2-planning-termination}

Planning answers how agents decompose work, but every plan must end. The next two questions address the boundaries that contain autonomous execution.

Termination (Q7, \Cref{sec:agents2-termination}) answers: how does an agent know when it is done? This involves defining success criteria so the agent can recognize completion, budget limits so the agent cannot run indefinitely, and completion recognition so the agent delivers results rather than continuing to refine. Escalation (Q8, \Cref{sec:agents2-escalation}) answers: how does an agent know when to ask for help? This involves confidence thresholds that trigger human review when uncertainty is high, authority boundaries that prevent agents from exceeding their mandate, and human-in-the-loop integration that makes escalation smooth rather than disruptive.

Without clear termination, agents run forever. Without escalation, agents exceed authority. These boundaries define the safe operating envelope for autonomous execution, ensuring that agents remain useful tools rather than becoming uncontrolled processes.

% ============================================================================
% 08-termination.tex
% Q7: How Does an Agent Know When It's Done?
% Part of: Chapter 07 - Agents Part II: How to Build an Agent
% ============================================================================

\section{How Does an Agent Know When It's Done?}
\label{sec:agents2-termination}

% ----------------------------------------------------------------------------
% Opening: Q7 Framing and Organizational Analogy
% ----------------------------------------------------------------------------

Every professional learns to recognize completion. The research memo is done when you've found sufficient authority and synthesized it coherently. The due diligence is done when you've reviewed all material documents and reported findings. The trade is done when the order executes and settles. Knowing when work is complete---and when it isn't---distinguishes effective professionals from those who over-research or under-deliver.

Agents face the same challenge. Without explicit termination conditions, agents can run indefinitely: searching one more database, trying one more approach, refining one more time. We call this the ``runaway associate'' problem: you asked for two relevant cases, the associate gives you fifty because they did not know when enough was enough.

\begin{definitionbox}[title={Termination}]
\keyterm{Termination} conditions define when an agent should stop executing. Three outcomes are possible:

\textbf{Success}: The goal is achieved. Deliver the result.

\textbf{Failure}: The goal cannot be achieved. Report why and stop.

\textbf{Escalation}: The agent cannot determine success or failure. Transfer to human judgment.

Termination implements the ``T'' in the GPA+IAT framework from Part I. Without termination, agents lack the sixth property that distinguishes agentic systems from runaway processes.
\end{definitionbox}

% ----------------------------------------------------------------------------
% Termination Condition Categories
% ----------------------------------------------------------------------------

\subsection{Termination Condition Categories}
\label{sec:agents2-termination-categories}

Five categories of termination conditions bound agent execution:

\subsubsection{Success Conditions}

The most obvious termination: the goal is achieved, return the result.

\textbf{Completeness criteria}: Have all required elements been produced? For document review: all provisions on the checklist have been analyzed. For research: the legal question has been answered with supporting authority. For portfolio rebalancing: allocations match targets within tolerance.

\textbf{Quality thresholds}: Is the output good enough? For a research memo: are conclusions supported by binding authority? For a risk assessment: have material risks been identified and analyzed? Quality thresholds often require human judgment---the agent can check completeness but may not assess quality reliably.

\textbf{Convergence criteria}: Has the agent stopped learning new information? If the last three searches returned no new relevant authority, the research may be saturated. If the last five portfolio adjustments produced diminishing improvement, optimization may have converged.

\subsubsection{Resource Budgets}

Hard limits prevent runaway execution by capping consumption across multiple dimensions. Token budgets stop execution after a specified threshold---say, 50,000 tokens---preventing expensive reasoning loops that would otherwise continue indefinitely. Time budgets enforce deadlines by terminating after a fixed duration, ensuring that research tasks do not consume an entire day when an hour was expected. Iteration budgets cap tool calls, stopping after twenty searches to prevent infinite loops where the agent keeps trying slightly different queries without progress. Cost budgets provide the most direct control, halting execution after spending a dollar amount (perhaps \$5 in API calls) to limit financial exposure.

These budgets cascade and interact: a task might hit its time limit before exhausting its token budget, or vice versa. Budget exhaustion does not mean failure; partial results may still be valuable. But the agent must stop and report rather than continuing indefinitely.

\subsubsection{Confidence Thresholds}

Confidence thresholds gate actions on certainty, creating a decision boundary between autonomous execution and human review. When confidence is high, the agent delivers its answer. When confidence drops below a calibrated threshold---perhaps 80\%---the agent stops and escalates rather than proceeding with uncertain information. This mirrors how associates should work: ``I'm not confident this is right. Let me ask the partner before proceeding.''

Calibrating these thresholds is challenging. Agents may be overconfident, proceeding when they should escalate, or underconfident, escalating unnecessarily and providing no value. Effective calibration requires testing against known outcomes, comparing agent confidence to actual accuracy, and adjusting thresholds until the agent escalates at appropriate uncertainty levels.

\subsubsection{Error Conditions}

Agents must recognize when things are going wrong and terminate rather than compounding errors. Repeated tool failures signal infrastructure problems: if Westlaw times out three times consecutively, the agent should stop rather than retrying indefinitely while consuming budget. Inconsistent data---a revenue figure in the 10-K that does not match the earnings release---requires human investigation, not agent guesswork about which source is correct. Constraint violations demand immediate termination: if the planned action would exceed position limits or breach confidentiality, the agent must stop before acting, not after.

Some tasks prove impossible as specified. Analysis may reveal conflicting requirements, missing prerequisites, or logical impossibilities. In these cases, the agent should report the impossibility honestly rather than proceeding with a compromised approach that satisfies the letter of the instruction while violating its spirit.

\subsubsection{Escalation Triggers}

Some situations require human judgment regardless of whether the task has succeeded, failed, or consumed its budget. Novel situations that do not match training patterns need human expertise to navigate. High-stakes decisions warrant human approval even when the agent is confident, because the consequences of error justify the overhead of review. Authority boundaries define what the agent can do autonomously; actions beyond those boundaries require escalation by design, not because something went wrong.

\Cref{sec:agents2-escalation} provides comprehensive treatment of when and how to escalate.

% ----------------------------------------------------------------------------
% Explicit Success Criteria
% ----------------------------------------------------------------------------

\subsection{Defining Success Criteria}
\label{sec:agents2-success-criteria}

Vague goals produce unclear termination. ``Research the statute of limitations'' could mean finding one relevant case or exhaustively surveying all circuits. Effective success criteria take several forms, each providing a different signal that work is complete.

Completeness checklists enumerate what must be delivered. For credit agreement review, the checklist might require identifying all financial covenants, comparing them to market terms, flagging provisions that differ from the firm's template, and summarizing material risks. The agent terminates when all checklist items are complete. Sufficiency thresholds define ``enough'' without requiring exhaustive coverage. For case research, sufficiency might mean finding at least three on-point circuit opinions, or if circuits conflict, identifying the leading case from each side. The agent knows when sufficiency is reached without searching every database.

Convergence criteria recognize diminishing returns. If three consecutive searches return no new relevant authority, the research may be saturated; further searching is unlikely to yield value. Deliverable specifications define the output format---a two-page memo with executive summary, analysis, and recommendation---so the agent knows what success looks like, not just what success requires.

Consider how experienced attorneys instruct associates: ``If you find clear Ninth Circuit authority, you're done. If the circuits are split, map the split and recommend which approach applies to our facts. If you can't find binding authority after two hours, come talk to me.'' Agents need the same clarity.

% ----------------------------------------------------------------------------
% Failure Recognition
% ----------------------------------------------------------------------------

\subsection{Recognizing Failure}
\label{sec:agents2-failure-recognition}

Not every task succeeds, and agents must recognize failure and report it honestly. Negative results are still results: ``I searched all major databases and found no authority on point'' is a valid finding that the attorney needs. The absence of authority is information. Agents should report negative results explicitly rather than continuing to search indefinitely in hope of finding something.

When failure occurs, diagnostic reporting explains what was attempted and why it failed. A useful failure report might state: ``I searched Westlaw, Lexis, and Bloomberg Law using [specific queries]. Zero results suggest either the issue is novel or the search terms are wrong. Recommend manual review of secondary sources or consultation with practice group expert.'' This gives the human actionable information rather than a bare ``task failed'' message.

Partial completion should be acknowledged honestly. If the agent completed analysis of Articles 1-4 before a tool failure, it should report what was accomplished and what remains: ``Articles 5-8 remain unanalyzed.'' Partial results may still be valuable, and preserving them prevents wasted effort. Where possible, root cause identification helps humans decide next steps: Was this a tool failure that will resolve itself? Impossible requirements that need rethinking? Insufficient information that requires additional input? The agent's diagnosis informs the human's response.

% ----------------------------------------------------------------------------
% Guardrails and Loop Detection
% ----------------------------------------------------------------------------

\subsection{Guardrails and Loop Detection}
\label{sec:agents2-loop-detection}

Even with well-defined termination conditions, agents can get stuck in unproductive loops---searching repeatedly without progress, rephrasing queries slightly, finding nothing, rephrasing again. Multiple mechanisms detect and prevent these loops.

Step limits provide the simplest guardrail: after N steps, stop and require human approval to continue. This prevents unbounded execution regardless of what the agent thinks it is accomplishing. Progress detection monitors whether recent actions produced value: if the last five actions yielded no new information, the agent may be stuck and should trigger reflection or escalation. Reflection steps build self-assessment into the workflow, periodically asking meta-questions: ``Am I making progress toward the goal? Have my recent actions been productive? Should I try a different approach or escalate?''

External watchdogs monitor agent behavior from outside the agent's own reasoning. If the same tool is called repeatedly with nearly identical parameters, an external system can recognize the loop pattern and intervene. Meta-policies encode loop detection rules directly: calling the same tool with the same parameters more than three times is probably a loop, so stop and escalate.

Without loop detection, agents will eventually get stuck in production. The question is not whether it will happen, but whether you will detect it when it does.

% ----------------------------------------------------------------------------
% The Reliability Cliff
% ----------------------------------------------------------------------------

\subsection{The Reliability Cliff}
\label{sec:agents2-reliability}

Independent benchmarking reveals a sharp reliability boundary. METR (Model Evaluation and Threat Research) tested agents across standardized task suites varying in duration and complexity. The results:

\begin{keybox}[title={The Four-Minute Cliff}]
METR's 2025 study found that agents achieve \textbf{near-perfect success on tasks under 4 minutes}, but \textbf{under 10\% success on tasks over 4 hours} \parencite{metr2024autonomy}.

This gap---from near-100\% to under-10\%---defines the current boundary between reliable and unreliable agent deployment.

\textbf{Implication}: Decompose tasks aggressively. Keep individual agent tasks short. Insert human checkpoints between phases. Don't expect autonomous completion of multi-hour workflows.
\end{keybox}

% TikZ figure: The Reliability Cliff
% Shows dramatic drop in agent success rates as task duration increases
% Based on METR 2024 findings
% Uses semantic color palette (example-light, key-light, caution-light)

\begin{figure}[!htb]
  \centering
  \begin{tikzpicture}[
    scale=1.0,
    font=\footnotesize,
    >=stealth
  ]

    % Define key x-coordinates for readability
    \def\xmin{0}
    \def\xfourmin{2}      % 4 minutes mark
    \def\xfourhours{8}    % 4 hours mark
    \def\xmax{10}

    % Define y-coordinates
    \def\ymin{0}
    \def\ymax{5}
    \def\yhigh{4.75}      % ~95% success
    \def\ylow{0.5}        % ~10% success

    % ZONE 1: Safe Zone (under 4 min) - example-light (green)
    \fill[example-light, opacity=0.3]
      (\xmin,\ymin) rectangle (\xfourmin,\ymax);

    % ZONE 2: Transition Zone (4 min to 4 hours) - key-light (amber)
    \fill[key-light, opacity=0.3]
      (\xfourmin,\ymin) rectangle (\xfourhours,\ymax);

    % ZONE 3: Unreliable Zone (over 4 hours) - caution-light (red)
    \fill[caution-light, opacity=0.3]
      (\xfourhours,\ymin) rectangle (\xmax,\ymax);

    % Draw axes
    \draw[->, line width=1pt, gray-800]
      (\xmin,\ymin) -- (\xmax,\ymin) node[right, black] {Task Duration};
    \draw[->, line width=1pt, gray-800]
      (\xmin,\ymin) -- (\xmin,\ymax) node[above, black] {Success Rate};

    % Y-axis labels
    \foreach \y/\label in {0/0\%, 1.25/25\%, 2.5/50\%, 3.75/75\%, 5/100\%} {
      \draw[gray-800] (\xmin,\y) -- ++(-0.1,0);
      \node[left, black] at (\xmin,\y) {\label};
    }

    % X-axis labels
    \draw[gray-800] (\xfourmin,\ymin) -- ++(0,-0.1);
    \node[below, black] at (\xfourmin,\ymin) {4 min};

    \draw[gray-800] (\xfourhours,\ymin) -- ++(0,-0.1);
    \node[below, black] at (\xfourhours,\ymin) {4 hours};

    % Draw the reliability cliff line
    \draw[line width=2pt, draw=gray-800]
      (\xmin,\yhigh) -- (\xfourmin,\yhigh)           % Flat high success
      -- (\xfourhours,\ylow)                          % Sharp drop (the cliff)
      -- (\xmax,\ylow);                               % Flat low success

    % Add circle markers at key points
    \fill[gray-800] (\xfourmin,\yhigh) circle (2pt);
    \fill[gray-800] (\xfourhours,\ylow) circle (2pt);

    % Zone annotations
    \node[example-dark, font=\small\bfseries] at (1, 4.25) {Safe Zone};
    \node[example-dark, font=\footnotesize] at (1, 3.75) {$\sim$95\% success};

    \node[caution-dark, font=\small\bfseries] at (9, 4.25) {Unreliable};
    \node[caution-dark, font=\footnotesize] at (9, 3.75) {$\sim$10\% success};

    % Callout annotation pointing to the cliff
    \node[
      draw=gray-800,
      fill=white,
      rounded corners=2pt,
      font=\small\bfseries,
      align=center,
      inner sep=4pt
    ] (callout) at (5.5, 3.5) {
      The Reliability Cliff\\
      \footnotesize\normalfont(METR 2024)
    };

    % Arrow from callout to cliff
    \draw[->, line width=1pt, gray-600]
      (callout.south) -- (\xfourhours-0.5,\ylow+1.5);

    % Key insight box at bottom
    \node[
      draw=key-dark,
      fill=key-light,
      rounded corners=3pt,
      text width=8cm,
      align=center,
      font=\small,
      inner sep=6pt,
      opacity=0.9
    ] at (5, -1.2) {
      \textbf{Implication:} Decompose tasks to stay under 4 minutes for reliable execution
    };

  \end{tikzpicture}

  \caption{
    The Reliability Cliff: Agent success rates drop dramatically as task duration increases.
    METR (2024) found that agents maintain approximately 95\% success on tasks under 4 minutes,
    but success rates fall to roughly 10\% for tasks exceeding 4 hours.
    This sharp transition motivates the decomposition of complex tasks into shorter subtasks
    and the use of aggressive timeout policies in agent architectures.
  }
  \label{fig:agents2-reliability-cliff}
\end{figure}


The reliability cliff has several causes, each contributing to the dramatic drop in success rates as task duration increases. Compounding errors are perhaps the most fundamental: each step introduces error probability, and these probabilities multiply. A 95\%-accurate retrieval step followed by 90\%-accurate reasoning followed by 85\%-accurate action yields roughly 73\% end-to-end accuracy, before accounting for sequencing decisions. Over a four-hour task with dozens of steps, these compounding errors accumulate into near-certain failure.

Planning fragility compounds the problem. Agents frequently select suboptimal tool sequences, get stuck in loops, or fail to recognize when their approach is not working. A human would step back and reconsider; agents often persist with failing strategies. Integration brittleness adds another failure mode: tool APIs return unexpected formats, authentication tokens expire, rate limits trigger. Each integration point is a potential failure mode, and complex tasks touch many integration points.

Design for this reality. Decompose aggressively. Validate at checkpoints. Assume agents will fail and design for graceful degradation.

% ----------------------------------------------------------------------------
% Graceful Degradation
% ----------------------------------------------------------------------------

\subsection{Graceful Degradation}
\label{sec:agents2-graceful-degradation}

When termination occurs before full completion, agents should degrade gracefully rather than failing completely. Tiered outputs provide value at every budget level: at 20\% budget, deliver the controlling statute with citation; at 60\% budget, add key holdings; at 100\% budget, deliver comprehensive analysis. Partial results are better than nothing, and users can decide whether partial results suffice or warrant additional investment.

Progress preservation saves intermediate state so humans can resume where the agent stopped. If the agent analyzed 30 of 50 contracts before budget exhaustion, that work should not be lost when execution terminates. Clear status reporting communicates exactly where things stand: ``Completed 60\% of task. Remaining: Articles 5-8 unreviewed due to budget exhaustion. Findings so far: [summary].'' The human knows what was accomplished and what remains.

Beyond reporting status, agents should recommend next steps when possible. ``Recommend allocating additional 30 minutes to complete review'' gives the human a concrete decision to make. ``Remaining work is routine; recommend proceeding with partial findings'' helps humans assess whether additional effort is worthwhile. The goal is a handoff that enables informed human decision-making, not a handoff that forces the human to start over.

% ----------------------------------------------------------------------------
% Evaluating Termination
% ----------------------------------------------------------------------------

\subsection{Evaluating Termination Capabilities}
\label{sec:agents2-termination-eval}

When evaluating agent systems, assess termination capabilities against six criteria that distinguish robust systems from fragile ones.

Success criteria clarity determines whether termination is predictable. Are termination conditions explicit? Can you predict when the agent will stop? Systems with vague or implicit termination conditions produce unpredictable behavior. Budget enforcement determines whether limits actually constrain execution. Test by setting tight budgets and verifying the agent actually stops; some systems log budget exhaustion but continue anyway. Loop detection determines whether the agent recognizes when it is stuck. Test with impossible tasks or unavailable tools; a system without loop detection will spin indefinitely.

Failure reporting determines whether failures are actionable. When tasks fail, does the agent explain why? A bare ``task failed'' message forces the human to investigate; a detailed explanation enables informed response. Graceful degradation determines whether early termination preserves value. When stopped early, does the agent preserve partial results? Is status clearly reported? Escalation integration determines whether handoffs work smoothly. When termination requires human judgment, does the human receive sufficient context to decide? A handoff that requires the human to start from scratch is a handoff that failed.

% ----------------------------------------------------------------------------
% Connection to Other Questions
% ----------------------------------------------------------------------------

\subsection{From Termination to Escalation}
\label{sec:agents2-termination-escalation}

Termination defines when agents stop, but not all stopping is the same. Success termination means the task is complete; deliver the results. Failure termination means the task is impossible; report why. Escalation termination means the agent cannot determine success or failure on its own; human judgment is required.

The third category is critical. An agent might complete its search and find conflicting authority, leaving it unable to determine whether the research question has been answered. It might approach a decision that exceeds its authorization. It might recognize that the situation is novel in ways that make its confidence unreliable. In each case, the right response is not to terminate with a result or a failure, but to terminate with a request for human input.

\Cref{sec:agents2-escalation} examines this closely related question: when should an agent stop autonomous operation and ask for human help? Termination and escalation together define the boundaries of autonomous execution. Without termination, agents run forever. Without escalation, agents exceed authority. These boundaries make agent deployment safe---or at least safer.

% ============================================================================
% 09-escalation.tex
% Q8: How Does an Agent Know When to Ask for Help?
% Part of: Chapter 07 - Agents Part II: How to Build an Agent
% ============================================================================

\section{How Does an Agent Know When to Ask for Help?}
\label{sec:agents2-escalation}

% ----------------------------------------------------------------------------
% Opening: Q8 Framing and Organizational Analogy
% ----------------------------------------------------------------------------

The best junior associates know when to go to the supervisor. They don't interrupt the partner with every question, but they also don't proceed confidently into territory beyond their expertise. They recognize authority boundaries: ``I can draft this motion, but I need partner review before filing.'' They recognize competence limits: ``I've researched for two hours and can't find clear authority---I should ask someone with more experience.'' They recognize high-stakes situations: ``The client is asking about strategy, not just research---this needs partner involvement.''

Agentic systems need the same judgment. An agent that never escalates will eventually exceed its competence, authority, or the bounds of safe autonomous operation. An agent that escalates everything provides no value: it becomes a complicated way to route work to humans. The challenge is knowing where to draw the line.

\begin{definitionbox}[title={Escalation}]
\keyterm{Escalation} transfers control from the agent to a human when autonomous execution should stop. Unlike termination, which ends the task (success or failure), escalation pauses the task and requests human input before continuing.

This reflects professionalism, not failure. Recognizing when you need help and asking for it is exactly what we want from junior professionals. Agents should do the same.
\end{definitionbox}

% ----------------------------------------------------------------------------
% When to Escalate: Decision Framework
% ----------------------------------------------------------------------------

\subsection{When to Escalate}
\label{sec:agents2-when-escalate}

Three categories of triggers warrant escalation: mandatory triggers, confidence-based triggers, and error detection \parencite{hitl-survey-2022,human-ai-taxonomy-2024}.

Some situations require human involvement regardless of the agent's confidence. Budget exhaustion is a clear mandatory trigger. When the agent approaches resource limits (tokens, time, iterations, cost), it should escalate with a progress summary rather than stopping silently: ``I've used 80\% of the research budget. Here's what I found. Options: (a) grant additional budget, (b) conclude with current findings, (c) provide strategic guidance on where to focus remaining effort.'' High-stakes actions require human approval regardless of agent confidence: filing court documents, sending client communications, executing large trades, making regulatory submissions. These are approval gates where the agent prepares and the human authorizes. Authority boundaries trigger escalation when the action exceeds what the agent is authorized to do autonomously; even if the agent is confident in its recommendation, organizational policy may require human sign-off above certain thresholds. Irreversible actions warrant escalation because once you file with the court or execute the trade, you cannot take it back.

Confidence-based escalation occurs when uncertainty about the right answer or approach exceeds acceptable thresholds \parencite{ai-uncertainty-2023}. Low confidence on output might manifest as: ``I found conflicting circuit authority. I'm not confident which rule applies in our jurisdiction'' for legal research, or ``Correlations have spiked beyond historical norms and model assumptions may be violated'' for portfolio analysis. Conflicting information triggers escalation when data sources disagree: the 10-K revenue does not match the earnings release, or two authoritative sources give different answers. Novel situations that do not match patterns the agent has seen warrant human expertise: novel legal questions, unusual market conditions, unprecedented fact patterns. Ambiguous instructions should trigger escalation when, despite clarification attempts (\Cref{sec:agents2-intent}), the agent remains uncertain about what is being asked.

Error and anomaly detection triggers escalation when something has gone wrong and the agent cannot fix it. Repeated tool failures warrant escalation: if Westlaw times out three times consecutively, the agent should report ``Research tool unavailable'' with options to wait, use alternatives, or proceed manually. Data anomalies, such as revenue figures that do not reconcile or filing dates that seem wrong or parties that appear on multiple sides of a transaction, warrant human investigation. Constraint violations occur when the task as specified would violate a policy, and the agent should report: ``Executing this trade would exceed the position limit. Please confirm override or adjust the order.'' Impossible requirements should be reported when analysis reveals the task cannot be completed as specified due to conflicting requirements, missing prerequisites, or logical impossibilities.

% ----------------------------------------------------------------------------
% How to Escalate: Information Handoff
% ----------------------------------------------------------------------------

\subsection{How to Escalate}
\label{sec:agents2-how-escalate}

Effective escalation provides the human with everything needed to make a decision. A five-part structure ensures completeness.

The escalation should begin with a situation summary that provides brief context for a human who may not have been following closely. Progress to date explains what has been accomplished and what remains, ensuring the human does not have to start from scratch. The escalation trigger explains why the agent is escalating now and what specific condition triggered the handoff. Information gathered presents what relevant information the agent has found; even if incomplete, partial findings are valuable. Options or recommendations describe the possible paths forward; if the agent has a recommendation, it should state it with supporting reasoning.

\paragraph{Example: Legal Research Escalation}

\begin{quote}
\textbf{Situation}: Researching statute of limitations for Section 10(b) securities fraud claim.

\textbf{Progress}: Searched Westlaw and Lexis. Found clear authority on the 2-year discovery period. Found conflicting circuit authority on when discovery is triggered.

\textbf{Trigger}: Low confidence. The Ninth Circuit and Second Circuit apply different tests for inquiry notice. I cannot determine which applies to our facts.

\textbf{Findings}: [Summary of key cases with citations]

\textbf{Options}: (a) Apply the more conservative test and note the split; (b) research district court authority in our jurisdiction; (c) seek partner guidance on which test likely applies.

\textbf{Recommendation}: Option (c): this appears to be a fact-intensive question where partner judgment on the strength of our facts would be valuable.
\end{quote}

\paragraph{Example: Financial Escalation}

\begin{quote}
\textbf{Situation}: Executing rebalancing trades to reduce tech exposure from 35\% to 25\%.

\textbf{Progress}: Generated trade list. Compliance check passed. Ready to execute.

\textbf{Trigger}: Trade size exceeds single-approver threshold (\$500K total).

\textbf{Findings}: Recommended trades would realize \$45K in short-term gains and \$12K in losses. Net tax impact: approximately \$8K additional liability.

\textbf{Options}: (a) Approve full trade list; (b) modify to prioritize tax-loss positions; (c) execute in tranches over multiple days.

\textbf{Recommendation}: Option (a) preferred if reducing exposure is urgent; Option (b) if tax optimization is priority.
\end{quote}

% ----------------------------------------------------------------------------
% Human-in-the-Loop Patterns
% ----------------------------------------------------------------------------

\subsection{Human-in-the-Loop Patterns}
\label{sec:agents2-hitl}

Five patterns integrate human oversight into agent workflows, each suited to different risk profiles and organizational needs.

Approval gates separate preparation from authorization. The agent prepares work product; the human authorizes execution. This pattern is essential for irreversible or high-stakes actions. A litigation agent drafts the court filing, presents it for review, receives approval, then submits. The agent handles preparation; the human controls execution.

Checkpoint reviews provide human verification at milestones to prevent error propagation. A research agent completes legal research, presents authorities, receives confirmation of direction, then proceeds to drafting. Each checkpoint catches misalignment before significant effort is wasted.

Confidence-based escalation ties autonomy to certainty. High confidence triggers autonomous execution; low confidence triggers escalation. A compliance agent processes clear-pass cases automatically while escalating ambiguous situations, balancing efficiency with safety.

The human-as-tool pattern treats human expertise like any other tool. When encountering questions beyond its capabilities, the agent queries the relevant expert, incorporates the response, and proceeds. Human expertise becomes a resource invoked when needed, not a bottleneck for all decisions.

Reversibility classification matches oversight level to action consequences. Fully reversible actions like drafts and research proceed autonomously. Partially reversible actions like communications receive checkpoint review. Irreversible actions like filings and trades require pre-approval. Oversight is proportional to risk.

% Oversight Spectrum: How Reversibility Determines Human Control

\begin{figure}[htbp]
\centering
\begin{tikzpicture}[
    every node/.style={inner sep=0pt}
]

% Arrow labels above cards
\node[font=\scriptsize\itshape, text=text-secondary, anchor=south] (rev-arrow) at (0, 3.8) {Decreasing Reversibility};
\draw[->, line width=0.8pt, text-muted] (-4.8, 3.6) -- (4.8, 3.6);

\node[font=\scriptsize\itshape, text=text-secondary, anchor=north] (control-arrow) at (0, 3.4) {Increasing Human Control};
\draw[<-, line width=0.8pt, text-muted] (-4.8, 3.2) -- (4.8, 3.2);

% Card 1: Fully Reversible (Left)
\node[
    draw=example-base,
    line width=1.5pt,
    fill=white,
    minimum width=4.5cm,
    minimum height=3.5cm,
    rounded corners=4pt
] (card1) at (-5.2, 0) {};

% Card 1 Header bar
\fill[example-base, rounded corners=4pt]
    ([yshift=-0.1cm]card1.north west) -- ([yshift=-0.1cm]card1.north east)
    [rounded corners=0pt] -- ([yshift=-0.7cm]card1.north east) -- ([yshift=-0.7cm]card1.north west)
    [rounded corners=4pt] -- cycle;

% Card 1 Header text
\node[font=\scriptsize\bfseries\sffamily, text=white, anchor=center]
    at ([yshift=-0.4cm]card1.north) {FULLY REVERSIBLE};

% Card 1 Body
\node[
    text width=4.0cm,
    anchor=north,
    font=\scriptsize,
    align=left
] at ([yshift=-0.9cm]card1.north) {
\textbf{\textsf{Control:}}\\[2pt]
\textcolor{example-dark}{AUTONOMOUS}\\[2pt]
\textit{(no pre-check required)}\\[6pt]
\textbf{\textsf{Example:}}\\[2pt]
Internal drafts, research notes, preliminary analysis
};

% Card 2: Partially Reversible (Middle)
\node[
    draw=key-base,
    line width=1.5pt,
    fill=white,
    minimum width=4.5cm,
    minimum height=3.5cm,
    rounded corners=4pt
] (card2) at (0, 0) {};

% Card 2 Header bar
\fill[key-base, rounded corners=4pt]
    ([yshift=-0.1cm]card2.north west) -- ([yshift=-0.1cm]card2.north east)
    [rounded corners=0pt] -- ([yshift=-0.7cm]card2.north east) -- ([yshift=-0.7cm]card2.north west)
    [rounded corners=4pt] -- cycle;

% Card 2 Header text
\node[font=\scriptsize\bfseries\sffamily, text=white, anchor=center]
    at ([yshift=-0.4cm]card2.north) {PARTIALLY REVERSIBLE};

% Card 2 Body
\node[
    text width=4.0cm,
    anchor=north,
    font=\scriptsize,
    align=left
] at ([yshift=-0.9cm]card2.north) {
\textbf{\textsf{Control:}}\\[2pt]
\textcolor{key-dark}{CHECKPOINT REVIEW}\\[2pt]
\textit{(review before send)}\\[6pt]
\textbf{\textsf{Example:}}\\[2pt]
Client communications, internal emails, routine responses
};

% Card 3: Irreversible (Right)
\node[
    draw=caution-dark,
    line width=1.5pt,
    fill=white,
    minimum width=4.5cm,
    minimum height=3.5cm,
    rounded corners=4pt
] (card3) at (5.2, 0) {};

% Card 3 Header bar
\fill[caution-dark, rounded corners=4pt]
    ([yshift=-0.1cm]card3.north west) -- ([yshift=-0.1cm]card3.north east)
    [rounded corners=0pt] -- ([yshift=-0.7cm]card3.north east) -- ([yshift=-0.7cm]card3.north west)
    [rounded corners=4pt] -- cycle;

% Card 3 Header text
\node[font=\scriptsize\bfseries\sffamily, text=white, anchor=center]
    at ([yshift=-0.4cm]card3.north) {IRREVERSIBLE};

% Card 3 Body
\node[
    text width=4.0cm,
    anchor=north,
    font=\scriptsize,
    align=left
] at ([yshift=-0.9cm]card3.north) {
\textbf{\textsf{Control:}}\\[2pt]
\textcolor{caution-dark}{PRE-APPROVAL}\\[2pt]
\textit{(required before action)}\\[6pt]
\textbf{\textsf{Example:}}\\[2pt]
Court filings, trade execution, contract signatures
};

\end{tikzpicture}
\caption{Oversight spectrum showing how the reversibility of agent actions determines the level of human control required. Fully reversible actions may proceed autonomously, partially reversible actions require checkpoint review, and irreversible actions demand pre-approval before execution.}
\label{fig:agents2-oversight-spectrum}
\end{figure}


% ----------------------------------------------------------------------------
% Domain-Specific Escalation
% ----------------------------------------------------------------------------

\subsection{Domain-Specific Escalation Requirements}
\label{sec:agents2-escalation-domain}

Legal and financial services impose domain-specific escalation requirements rooted in professional responsibility rules and regulatory obligations.

\paragraph{Legal Practice}

Legal practice escalation requirements arise from professional responsibility rules \parencite{aba-model-rule-1-1,aba-formal-opinion-512}:

\begin{itemize}[nosep]
\item \textbf{Competence limits}: Escalate when matters exceed agent training or supervising attorney capacity (ABA Model Rule 1.1)
\item \textbf{Privilege protection}: Escalate any action that might expose privileged information to third parties
\item \textbf{Conflicts of interest}: Escalate potential conflict situations to conflicts counsel
\item \textbf{Candor to tribunal}: Escalate immediately if adverse authority is discovered that may require disclosure
\end{itemize}

\paragraph{Financial Services}

Financial services escalation requirements arise from regulatory obligations and fiduciary duties \parencite{finra-notice-24-09,fed-sr11-7}:

\begin{itemize}[nosep]
\item \textbf{Suitability and fiduciary duty}: Escalate investment recommendations for adviser review before client delivery
\item \textbf{Regulatory thresholds}: Escalate when trades approach reporting thresholds or disclosure requirements
\item \textbf{Material non-public information}: Escalate immediately if potential MNPI is encountered
\item \textbf{Risk limits}: Escalate when proposed actions would breach position limits or risk thresholds
\end{itemize}

% ----------------------------------------------------------------------------
% Evaluating Escalation
% ----------------------------------------------------------------------------

\subsection{Evaluating Escalation Mechanisms}
\label{sec:agents2-escalation-eval}

When evaluating agentic systems, assess escalation mechanisms against six criteria that distinguish effective systems from those that either escalate too much or too little.

Coverage determines whether all appropriate situations trigger escalation. Test with edge cases: novel situations that should clearly require human judgment, conflicting data that the agent cannot resolve, near-threshold conditions that might slip through. A system with coverage gaps will occasionally proceed autonomously when it should not. Calibration determines whether thresholds are set appropriately. Too sensitive and the agent escalates everything, providing no value; too loose and it proceeds when it should not. Calibrate thresholds against real scenarios, adjusting until the agent escalates when practitioners agree it should.

Latency determines how quickly escalation reaches the right human. For urgent matters---a margin call, a filing deadline, a client emergency---escalation must be immediate. For routine matters, queued escalation may suffice. Routing determines whether escalation reaches the right person. Complex legal questions should reach senior attorneys, not paralegals; risk limit breaches should reach risk managers, not operations staff. Misrouted escalation wastes time and may produce inadequate responses.

Context quality determines whether the human can actually decide. Test by reviewing escalation messages and asking: could you make a decision from this information alone? If the human must investigate further before responding, the escalation is incomplete. Response handling determines whether the agent correctly incorporates human guidance. Test the full cycle, not just escalation initiation; an agent that escalates well but ignores responses provides only the illusion of human oversight.

% ----------------------------------------------------------------------------
% Connection to Other Questions
% ----------------------------------------------------------------------------

\subsection{From Escalation to Delegation}
\label{sec:agents2-escalation-delegation}

Escalation moves control \textit{up}: from agent to human supervisor. But agents can also move control \textit{sideways} by delegating subtasks to other agents. Where escalation says ``I need human help,'' delegation says ``I need specialist help.''

\Cref{sec:agents2-delegation} examines the next question: how does an agent work with other agents? Delegation patterns enable complex workflows where multiple specialized agents collaborate, each with its own escalation paths back to human oversight. A coordinating agent might delegate research to a legal research specialist, analysis to a financial modeling specialist, and drafting to a document generation specialist---while each specialist retains the ability to escalate to humans when it reaches its own limits.

The combination of escalation (vertical) and delegation (horizontal) defines the full topology of human-agent collaboration. Escalation ensures human oversight. Delegation enables specialization and scale. Together, they make complex agentic workflows possible while maintaining the human control that regulated professions require.

% ============================================================================
% 10-delegation.tex
% Q9: How Does an Agent Work with Other Agents?
% Part of: Chapter 07 - Agents Part II: How to Build an Agent
% ============================================================================

\section{How Does an Agent Work with Other Agents?}
\label{sec:agents2-delegation}

% ----------------------------------------------------------------------------
% Opening: Q9 Framing and Organizational Analogy
% ----------------------------------------------------------------------------

Complex matters require coordination. An M\&A partner does not execute everything personally---they coordinate specialists, with corporate counsel reviewing governance, tax specialists analyzing structure, and antitrust counsel assessing regulatory risk. Each specialist contributes deep domain expertise while the partner orchestrates: defining deliverables, integrating work products, and synthesizing conclusions for the client.

A portfolio manager coordinates similarly, with analysts providing company analysis, traders handling execution, risk managers monitoring exposure, and compliance officers verifying adherence. Complex trades require all these perspectives because no single person possesses all necessary expertise.

Agentic systems face the same coordination challenge. A single agent trying to do everything quickly exceeds its competence, permission boundaries, or context limits. Multi-agent architectures mirror professional teams: specialized agents with deep expertise, orchestrators that coordinate them, and structured protocols that enable seamless collaboration.

\begin{definitionbox}[title={Delegation}]
\keyterm{Delegation} assigns subtasks from one agent (the coordinator) to another (the specialist). Unlike escalation (agent to human), delegation is agent to agent. The coordinator defines \textit{what} needs to be done; the specialist determines \textit{how}.

Delegation enables parallelization (multiple specialists work simultaneously), specialization (each agent is optimized for its domain), and security isolation (each agent has only the permissions it needs).
\end{definitionbox}

% ----------------------------------------------------------------------------
% Why Multi-Agent?
% ----------------------------------------------------------------------------

\subsection{Why Multi-Agent Architectures?}
\label{sec:agents2-why-multi-agent}

Several factors drive multi-agent designs \parencite{wu2023autogen,guo2024multiagent}, each reflecting challenges familiar to professional practice.

\textbf{Specialization} allows agents to excel in narrow domains, just as a securities law agent can be optimized for SEC regulations and equipped with EDGAR tools while a tax agent handles tax implications---neither needs expertise in the other's domain. \textbf{Security isolation} enforces least privilege: a research agent can read legal databases but cannot file documents, while a filing agent can submit to CM/ECF but cannot access client financial data; if one agent is compromised, damage is contained. \textbf{Parallel execution} lets independent workstreams proceed simultaneously, so a document review agent can analyze contracts while a research agent investigates legal issues without either waiting for the other.

Two additional factors favor multi-agent designs in production settings. \textbf{Vendor diversity} enables best-of-breed selection---a specialized legal model handles research, a general model handles drafting, and a fast model handles classification---matching capability to task. \textbf{Scale management} addresses context window limits by decomposing tasks across agents, each with focused context, rather than cramming everything into one overwhelmed process.

These benefits come with tradeoffs: coordination overhead increases communication costs, debugging complexity grows when failures span agents, and the attack surface expands with each additional component. The protocols discussed in \Cref{sec:agents2-mcp-perception} and the security controls in \Cref{sec:agents2-action-security} help manage these risks.

% ----------------------------------------------------------------------------
% Agent-to-Agent Protocol (A2A)
% ----------------------------------------------------------------------------

\subsection{Agent-to-Agent Protocol (A2A)}
\label{sec:agents2-a2a}

Just as MCP standardizes how agents access tools and data (see \Cref{sec:agents2-mcp-perception}), the Agent-to-Agent Protocol (A2A) standardizes how agents delegate work to each other \parencite{google-a2a,anthropic-mcp}. The distinction is straightforward: MCP handles agent-to-tool communication, A2A handles agent-to-agent coordination.

A2A uses familiar professional concepts. \textbf{Agent Cards} function as capability statements---digital résumés listing expertise, accepted inputs, and expected outputs. \textbf{Tasks} are units of delegated work, analogous to engagement letters, that specify scope, constraints, and deadlines. \textbf{Artifacts} are deliverables returned upon completion: memos, analyses, or structured data. And \textbf{Channels} provide communication streams for status updates and clarification during execution.

\begin{keybox}[title={Delegation Lifecycle}]
The A2A task lifecycle mirrors professional delegation: (1) \textit{discovery}---finding the right specialist via Agent Card, like identifying co-counsel with relevant expertise; (2) \textit{delegation}---creating a Task with goals and constraints, like drafting an engagement letter; (3) \textit{execution}---the specialist works independently, requesting clarification if needed; (4) \textit{delivery}---the specialist returns Artifacts; and (5) \textit{completion}---the coordinator reviews and approves the work. This structure ensures that delegated work is scoped, tracked, and auditable.
\end{keybox}

% ----------------------------------------------------------------------------
% Multi-Agent Patterns
% ----------------------------------------------------------------------------

\subsection{Multi-Agent Patterns}
\label{sec:agents2-orchestration}

Three patterns organize multi-agent collaboration, each with distinct tradeoffs \parencite{wang2024tdag}.

\textbf{Sequential Delegation} processes work in series: the Coordinator delegates to a Research Agent, whose output flows to an Analysis Agent, then to a Drafting Agent. This pattern is simple to implement and debug---each handoff is clear---but slow, as each stage must wait for the previous one.

\textbf{Parallel Delegation} runs independent workstreams simultaneously. Securities, Tax, and Employment Agents can analyze an acquisition concurrently while the Coordinator integrates findings afterward. This pattern trades coordination complexity for speed, but only works when tasks are truly independent; dependencies between specialists require careful orchestration.

\textbf{Hierarchical Delegation} creates nested authority structures: a Lead Due Diligence Agent delegates to Document Review and Legal Research sub-agents, who may further delegate to specialized tools. This pattern enables deep task decomposition for complex matters but introduces orchestration overhead and debugging challenges when failures occur deep in the hierarchy.

Figure~\ref{fig:agents2-orchestration-patterns} illustrates these three approaches. Most production systems blend multiple patterns: parallel agents handle independent analyses while hierarchical structures decompose complex, tightly-coupled tasks. The choice depends on task structure---independent subtasks favor parallelism, dependent workflows favor sequencing, and complex matters with natural subdivisions favor hierarchy.

% fig-orchestration-patterns.tex
% Three Multi-Agent Orchestration Patterns
% Part of: Chapter 07 - Agents Part II: How to Build an Agent
% Section: 10-Delegation

\begin{figure}[!htb]
\centering
\resizebox{\textwidth}{!}{%
\begin{tikzpicture}[
    % Card styles
    card/.style={
        rounded corners=6pt,
        line width=1.5pt,
        minimum width=5.2cm,
        minimum height=8.0cm,
        align=center,
        inner sep=0pt
    },
    card header/.style={
        font=\small\bfseries,
        text=white,
        minimum height=0.8cm,
        text width=4.8cm,
        align=center,
        rounded corners=4pt
    },
    section text/.style={
        font=\scriptsize,
        text=gray-800,
        text width=4.4cm,
        align=left,
        anchor=north west
    },
    % Mini-flowchart styles (from agent-loop.tex)
    flow box/.style={
        font=\tiny,
        align=center,
        text width=1.6cm,
        rounded corners=2pt,
        inner sep=3pt,
        minimum height=1.2em,
        line width=0.8pt
    },
    flow arrow/.style={
        -stealth,
        line width=1pt,
        color=border-neutral
    }
]

% ========== LEFT CARD: SEQUENTIAL ==========
\node[card, draw=definition-dark, fill=white] (seq-card) at (-5.5, 0) {};

% Header
\node[card header, fill=definition-dark] at (seq-card.north) [yshift=0.13cm] {\textbf{Sequential}};

% Mini-flowchart (vertical chain)
\node[flow box, fill=bg-definition, draw=border-definition, text width=2.0cm] (seq-coord) at (-5.5, 3.1) {Coordinator};
\node[flow box, fill=bg-definition, draw=border-definition] (seq-a) at (-5.5, 2.2) {Agent A};
\node[flow box, fill=bg-definition, draw=border-definition] (seq-b) at (-5.5, 1.3) {Agent B};
\node[flow box, fill=bg-definition, draw=border-definition] (seq-c) at (-5.5, 0.4) {Agent C};
\node[flow box, fill=bg-definition, draw=border-definition, text width=2.0cm] (seq-synth) at (-5.5, -0.5) {Synthesize};

\draw[flow arrow] (seq-coord) -- (seq-a);
\draw[flow arrow] (seq-a) -- (seq-b);
\draw[flow arrow] (seq-b) -- (seq-c);
\draw[flow arrow] (seq-c) -- (seq-synth);

% Best for and Trade-off
\node[section text] at (-7.7, -1.6) {\textbf{Best for:} Tasks with dependencies between steps\\[0.3em]\textbf{Trade-off:} Slower; blocked by bottlenecks};

% ========== MIDDLE CARD: PARALLEL ==========
\node[card, draw=key-dark, fill=white] (par-card) at (0, 0) {};

% Header
\node[card header, fill=key-dark] at (par-card.north) [yshift=0.13cm] {\textbf{Parallel}};

% Mini-flowchart (fan-out, fan-in)
\node[flow box, fill=bg-key, draw=border-key, text width=2.0cm] (par-coord) at (0, 3.1) {Coordinator};
\node[flow box, fill=bg-key, draw=border-key, text width=1.0cm] (par-a) at (-1.3, 1.8) {Agent\\A};
\node[flow box, fill=bg-key, draw=border-key, text width=1.0cm] (par-b) at (0, 1.8) {Agent\\B};
\node[flow box, fill=bg-key, draw=border-key, text width=1.0cm] (par-c) at (1.3, 1.8) {Agent\\C};
\node[flow box, fill=bg-key, draw=border-key, text width=2.0cm] (par-synth) at (0, 0.5) {Synthesize};

\draw[flow arrow] (par-coord) -- (par-a);
\draw[flow arrow] (par-coord) -- (par-b);
\draw[flow arrow] (par-coord) -- (par-c);
\draw[flow arrow] (par-a) -- (par-synth);
\draw[flow arrow] (par-b) -- (par-synth);
\draw[flow arrow] (par-c) -- (par-synth);

% Best for and Trade-off
\node[section text] at (-2.2, -1.6) {\textbf{Best for:} Independent parallel work\\[0.3em]\textbf{Trade-off:} Coordination overhead};

% ========== RIGHT CARD: HIERARCHICAL ==========
\node[card, draw=example-dark, fill=white] (hier-card) at (5.5, 0) {};

% Header
\node[card header, fill=example-dark] at (hier-card.north) [yshift=0.13cm] {\textbf{Hierarchical}};

% Mini-flowchart (tree with sub-agents)
\node[flow box, fill=bg-example, draw=border-example, text width=2.0cm] (hier-coord) at (5.5, 3.1) {Coordinator};
\node[flow box, fill=bg-example, draw=border-example] (hier-a) at (4.4, 1.9) {Agent A};
\node[flow box, fill=bg-example, draw=border-example, font=\tiny, text width=0.8cm] (hier-a1) at (6.6, 2.2) {A1};
\node[flow box, fill=bg-example, draw=border-example, font=\tiny, text width=0.8cm] (hier-a2) at (6.6, 1.6) {A2};
\node[flow box, fill=bg-example, draw=border-example] (hier-b) at (4.4, 0.5) {Agent B};
\node[flow box, fill=bg-example, draw=border-example, font=\tiny, text width=0.8cm] (hier-b1) at (6.6, 0.9) {B1};
\node[flow box, fill=bg-example, draw=border-example, font=\tiny, text width=0.8cm] (hier-b2) at (6.6, 0.35) {B2};
\node[flow box, fill=bg-example, draw=border-example, font=\tiny, text width=0.8cm] (hier-b3) at (6.6, -0.2) {B3};
\node[flow box, fill=bg-example, draw=border-example, text width=2.0cm] (hier-synth) at (5.5, -0.9) {Synthesize};

\draw[flow arrow] (hier-coord.south west) -- (hier-a.north);
\draw[flow arrow] (hier-a.east) -- (hier-a1.west);
\draw[flow arrow] (hier-a.east) -- (hier-a2.west);
\draw[flow arrow] (hier-coord.south) -- (hier-b.north);
\draw[flow arrow] (hier-b.east) -- (hier-b1.west);
\draw[flow arrow] (hier-b.east) -- (hier-b2.west);
\draw[flow arrow] (hier-b.east) -- (hier-b3.west);
\draw[flow arrow] (hier-a.south) -- (hier-synth.north west);
\draw[flow arrow] (hier-b.south) -- (hier-synth.north);

% Best for and Trade-off
\node[section text] at (3.3, -1.6) {\textbf{Best for:} Complex tasks with sub-delegation\\[0.3em]\textbf{Trade-off:} Complexity; harder to debug};

\end{tikzpicture}
}%
\caption{Three multi-agent orchestration patterns. Sequential delegation chains agents in order, ideal for dependent tasks but vulnerable to bottlenecks. Parallel delegation runs agents concurrently, maximizing throughput for independent work but requiring coordination. Hierarchical delegation enables sub-agents to handle specialized sub-tasks, providing flexibility for complex workflows at the cost of debugging complexity.}
\label{fig:agents2-orchestration-patterns}
\end{figure}


Consider M\&A due diligence as an example. The Orchestrator delegates to specialists in parallel: a Document Processing Agent accesses the data room while a Financial Analysis Agent queries financial databases. Each specialist may use hierarchical delegation internally, with sub-agents handling specific document types or analysis categories. The specialists return structured Artifacts, which the Orchestrator synthesizes into a unified assessment.

% ----------------------------------------------------------------------------
% Multi-Agent Workflows
% ----------------------------------------------------------------------------

\subsection{Multi-Agent Workflow Examples}
\label{sec:agents2-workflows}

\paragraph{Legal: Regulatory Assessment}
A fintech company asks about regulatory approvals for a new product. The Orchestrator decomposes the request across four specialists working in parallel. A \textbf{Securities Agent} analyzes SEC guidance and identifies potential registration requirements. A \textbf{Banking Agent} checks OCC and FDIC rules, flagging money transmitter licensing needs. A \textbf{Consumer Agent} reviews CFPB regulations and highlights disclosure requirements. An \textbf{AML Agent} analyzes Bank Secrecy Act obligations and determines KYC needs. Each specialist accesses relevant regulatory databases through the tool integrations discussed in \Cref{sec:agents2-mcp-perception}. The Orchestrator synthesizes these parallel findings into a prioritized regulatory roadmap, identifying which approvals are prerequisites for others.

\paragraph{Finance: Block Trade Execution}
A portfolio manager requests a \$50M block trade. The Orchestrator coordinates four specialists with different responsibilities. A \textbf{Market Agent} analyzes liquidity conditions and recommends VWAP execution over two days to minimize market impact. A \textbf{Compliance Agent} verifies position limits and notes 13F reporting requirements. A \textbf{Risk Agent} assesses portfolio impact, confirming that the trade keeps exposure within policy bounds. An \textbf{Execution Agent} interfaces with the order management system to place and monitor orders. The Orchestrator tracks progress across all specialists, escalating if any specialist encounters problems that require human judgment (see \Cref{sec:agents2-escalation}).

% ----------------------------------------------------------------------------
% Multi-Agent Risks
% ----------------------------------------------------------------------------

\subsection{Multi-Agent Risks}
\label{sec:agents2-coordination-failures}

Multi-agent systems introduce failure modes beyond those of single-agent systems \parencite{cemri2025multiagentfail}.

\textbf{Coordination failures} manifest in three primary ways. \textit{Deadlock} occurs when agents wait cyclically for each other---Agent A waits for B, B waits for C, and C waits for A---requiring timeouts and dependency analysis to prevent. \textit{Divergence} results when specialists reach incompatible conclusions, such as conflicting legal interpretations or inconsistent financial projections; the orchestrator must reconcile these differences or escalate to human judgment. \textit{Cascading errors} propagate when incorrect output from one specialist becomes flawed input for another, compounding the original mistake; prevention requires validation at each handoff, applying the input validation principles discussed in \Cref{sec:agents2-action-security}.

\textbf{Security risks} in multi-agent systems require the same rigor applied to human teams \parencite{openid-ai-identity-2024}. Agents must have verifiable, auditable identities to prevent impersonation and enable compliance tracking. Access policies must restrict which agents can delegate to whom, preventing unauthorized capability escalation. Ethical walls must enforce conflict rules across agent boundaries just as they do across human teams---an agent working on one side of a transaction cannot delegate to specialists with access to the other side's confidential information. And every delegation must be logged (who delegated, what was delegated, when, and with what result) to support the audit requirements discussed in \Cref{sec:agents2-memory-isolation}.

% ----------------------------------------------------------------------------
% Protocol Selection
% ----------------------------------------------------------------------------

\subsection{Protocol Selection Guidance}
\label{sec:agents2-protocol-selection}

Table~\ref{tab:agents2-protocol-selection} summarizes how to choose between MCP and A2A based on task characteristics.

\begin{table}[htbp]
\centering
\caption{Selecting between MCP and A2A based on task characteristics}
\label{tab:agents2-protocol-selection}
\small
\begin{tabular}{
  >{\raggedright\arraybackslash}p{0.28\textwidth}
  >{\raggedright\arraybackslash}p{0.12\textwidth}
  >{\raggedright\arraybackslash}p{0.16\textwidth}
  >{\raggedright\arraybackslash}p{0.30\textwidth}
}
\toprule
\textbf{Task Characteristic} & \textbf{Protocol} & \textbf{Typical Duration} & \textbf{Examples} \\
\midrule
Immediate, well-defined operation & MCP & Seconds & Query database; retrieve document; run calculation \\
\midrule
Delegated work requiring judgment & A2A & Minutes to hours & Assign research; request analysis; coordinate specialists \\
\midrule
End-to-end workflow combining both & MCP + A2A & Varies & Due diligence; portfolio rebalancing; regulatory assessment \\
\bottomrule
\end{tabular}
\end{table}

The decision rule is straightforward: use MCP for well-defined, auditable operations (``fetch this filing,'' ``calculate this metric''), A2A for delegated work requiring specialist judgment (``research and synthesize,'' ``draft and revise,'' ``coordinate across constraints''), and both together for complex workflows. As of late 2025, MCP is production-ready \parencite{mcp-spec}, while A2A is maturing with variable cross-vendor reliability \parencite{a2a-spec}. Until A2A standardization solidifies, design systems with fallbacks to human coordination for critical decisions.

% ----------------------------------------------------------------------------
% Connection to Other Questions
% ----------------------------------------------------------------------------

\subsection{From Delegation to Governance}
\label{sec:agents2-delegation-governance}

Delegation distributes work, creating governance challenges. Accountability becomes complex: does responsibility lie with the coordinator, the specialist, or the approving human? Information barriers must apply to agents just as they do to humans. Audit trails must span the entire delegation tree.

\Cref{sec:agents2-governance} previews these requirements. Chapter~8 (Agents Part III: Governing Agents) develops them in detail, translating multi-agent delegation into concrete controls: delegation policy, identity and authorization, logging and retention, and organizational accountability.

% ============================================================================
% 11-governance.tex
% Q10: How Do We Keep the Agent Safe?
% Part of: Chapter 07 - Agents Part II: How to Build an Agent
% ============================================================================

\section{How Do We Keep the Agent Safe?}
\label{sec:agents2-governance}

% ----------------------------------------------------------------------------
% Opening: Q10 Framing and Organizational Analogy
% ----------------------------------------------------------------------------

Every professional organization has compliance programs, audit functions, and oversight structures. The law firm has conflicts committees, billing review, and quality control. The financial institution has risk management, compliance monitoring, and internal audit. These functions do not do the work themselves; they ensure the work is done safely, ethically, and in compliance with applicable rules.

Agentic systems require the same infrastructure. Governance is not a single question but a lens through which all other questions must be viewed. Every capability creates governance requirements. Every architectural choice enables or constrains oversight. This section previews governance across all ten questions; the following chapter explores these requirements in depth.

\begin{definitionbox}[title={Agentic System Governance}]
\keyterm{Agentic system governance} encompasses the policies, controls, and oversight mechanisms that ensure agentic systems operate safely, ethically, and in compliance with applicable requirements. Governance spans the agentic system lifecycle: design, deployment, operation, and retirement.

\vspace{0.5em}
Governance is not optional for regulated professional services \parencite{aba-formal-opinion-512,finra-notice-24-09}. Professional duties are non-delegable: attorneys remain liable for AI-assisted work product, and fiduciaries remain accountable for AI-informed recommendations.
\end{definitionbox}

% ----------------------------------------------------------------------------
% Architecture Enables Governance
% ----------------------------------------------------------------------------

\subsection{Architecture Enables Governance}
\label{sec:agents2-arch-enables-gov}

The architectural choices throughout this chapter are not merely technical decisions. They are the \textit{infrastructure} that makes governance possible.

You cannot audit what you did not log. You cannot enforce privilege boundaries that were never implemented. You cannot demonstrate bounded operation without termination mechanisms. When a regulator asks how the compliance agent detected a breach, when opposing counsel demands production of the agent's reasoning, when a client questions why the agent recommended a particular strategy---architecture determines whether you can answer.

Professional duties are non-delegable: attorneys remain liable for AI-assisted work product, and fiduciaries remain accountable for AI-informed recommendations. The following chapter details those obligations. This chapter provides the architecture to meet them. \Cref{sec:agents2-conclusion-enables} provides a comprehensive mapping of architectural choices to governance implications.

% ----------------------------------------------------------------------------
% Ten-Question Governance Mapping
% ----------------------------------------------------------------------------

\subsection{Governance Requirements by Question}
\label{sec:agents2-governance-mapping}

Each of the ten questions creates specific governance requirements:

\begin{table}[htbp]
\centering
\caption{Ten-question governance mapping}
\label{tab:agents2-governance-mapping}
\small
\begin{tabular}{clp{6cm}}
\toprule
\textbf{Q} & \textbf{Question} & \textbf{Governance Requirement} \\
\midrule
1 & Triggers & Event authorization, audit logging of all triggers \\
2 & Intent & Purpose limitation, goal alignment verification \\
3 & Perception & Data governance, access controls, provenance tracking \\
4 & Action & Actuation controls, approval gates, rollback capability \\
5 & Memory & State integrity, retention policies, isolation enforcement \\
6 & Planning & Bounded operation, resource budgets, plan validation \\
7 & Termination & Exit protocols, success criteria, graceful degradation \\
8 & Escalation & Human oversight, escalation triggers, response tracking \\
9 & Delegation & Agent identity, delegation contracts, barrier enforcement \\
10 & Governance & Meta-governance, audit architecture, compliance monitoring \\
\bottomrule
\end{tabular}
\end{table}

% ----------------------------------------------------------------------------
% Security Essentials
% ----------------------------------------------------------------------------

\subsection{Security Essentials}
\label{sec:agents2-security-essentials}

Five security controls are essential for any agentic system deployment in regulated contexts \parencite{owasp-llm-top10,iso-iec-42001}:

\begin{keybox}[title={Security Controls for Regulated Practice}]
\begin{enumerate}[nosep]
\item \textbf{Input separation}: Isolate user inputs from system prompts to prevent prompt injection attacks

\item \textbf{Output validation}: Verify agent outputs before execution to detect hallucinations and constraint violations

\item \textbf{Least privilege}: Grant minimum necessary tool access to limit the scope and impact of failures

\item \textbf{Audit logging}: Maintain comprehensive action logs for accountability and investigation

\item \textbf{Matter/client isolation}: Enforce confidentiality boundaries to protect privileged and confidential information
\end{enumerate}
\end{keybox}

These controls map to the ten-question framework:

\begin{itemize}[nosep]
\item Input separation protects Q2 (Intent) from manipulation
\item Output validation governs Q4 (Action)
\item Least privilege limits Q3 (Perception) and Q4 (Action)
\item Audit logging enables Q7 (Termination) review and Q8 (Escalation) tracking
\item Matter/client isolation enforces Q5 (Memory) boundaries
\end{itemize}

% ----------------------------------------------------------------------------
% Transparency and Explainability
% ----------------------------------------------------------------------------

\subsection{Transparency and Explainability}
\label{sec:agents2-transparency-preview}

Regulators and clients increasingly require explanations for agentic system decisions \parencite{zhong2024xai-auditing,ai-auditing-systematic-review-2024}. Four levels of transparency serve different audiences, illustrated here with a breach detection agentic system:

Level 0 provides output only: just the answer, such as ``Suspicious transaction flagged.'' This level suffices for routine, low-stakes queries where the consumer trusts the system.

Level 1 adds a summary with sources: the conclusion plus citations, such as ``Transaction \#45921 flagged; exceeds threshold in Rule 203(b)(1).'' This enables verification without requiring full reasoning.

Level 2 provides a reasoning outline: key analytical steps plus sources, such as ``Flagged because: (1) \$150K exceeds \$100K threshold, (2) counterparty on watchlist, (3) timing matches known pattern.'' This level is appropriate for substantive work product requiring review.

Level 3 provides a full execution trace: a structured record of tool calls, retrieved documents, and decision points, including database queries, rule evaluation steps, and confidence scores. This level enables audit and debugging.

The architecture should capture Level 3 traces for all operations, then generate audience-appropriate summaries (Levels 0--2) on demand.

% ----------------------------------------------------------------------------
% Auditability vs. Retention
% ----------------------------------------------------------------------------

\subsection{Auditability vs. Retention}
\label{sec:agents2-retention-preview}

A tension exists between comprehensive logging (for audit) and data minimization (for privacy and compliance). The resolution is \textit{not} ``log everything forever.'' Instead, four practices balance these competing demands.

Structured logging captures structured decisions rather than raw chain-of-thought. Structure enables selective retention because decision points can be preserved while ephemeral reasoning is discarded. Tiered retention implements different periods for different purposes: short-term operational logs with full detail retained for days to weeks, medium-term audit logs with structured decisions retained for months to years, and long-term compliance archives containing minimal but sufficient information as required by regulation.

Redaction at capture applies privacy and confidentiality filters before logging, not after; once sensitive information enters logs, it becomes difficult to remove systematically. Finally, legal hold integration ensures that retention schedules yield to preservation obligations when litigation is anticipated.

The following chapter provides detailed retention frameworks for legal and financial contexts.

% ----------------------------------------------------------------------------
% Forward to Chapter 8
% ----------------------------------------------------------------------------

\subsection{Forward to Chapter 8}
\label{sec:agents2-forward-ch8}

This chapter answered \textit{how to build an agentic system}. The ten questions (Table~\ref{tab:agents2-governance-mapping}) provide architectural foundations; each creates governance requirements that the architecture must support.

The following chapter answers: \textit{how do we govern these systems?} It examines the five-layer governance stack (legal, model, system, process, culture), dimensional controls (autonomy, persistence, goal dynamics), accountability structures, and regulatory compliance frameworks---including Federal Reserve model risk management guidance \parencite{fed-sr11-7}, the NIST AI Risk Management Framework \parencite{nist-ai-rmf}, and GAO findings on AI oversight in financial services \parencite{gao-ai-financial-services-2025}---with worked examples in legal, financial, and audit contexts.

Architecture provides the foundation; governance provides the controls. Together, they enable responsible deployment of agentic systems in regulated professional services.

% ============================================================================
% 12-conclusion.tex
% Conclusion: From Architecture to Governance
% Part of: Chapter 07 - Agents Part II: How to Build an Agent
% ============================================================================

\section{Conclusion: From Architecture to Governance}
\label{sec:agents2-conclusion}

% ----------------------------------------------------------------------------
% Opening
% ----------------------------------------------------------------------------

This chapter opened with a claim: \textbf{agents are not magic; they are architecture}. Ten sections later, that claim should feel concrete. The ten-question framework from \Cref{sec:agents2-ten-questions} has provided the spine for understanding how agents receive work, understand intent, gather information, take action, maintain context, plan, terminate, escalate, coordinate, and operate under governance.

You now know how work reaches an agent (triggers), how instructions become goals (intent), how agents gather information (perception) and take action (action), how context persists (memory), how complex work decomposes (planning), how agents recognize completion (termination), when they hand off to humans (escalation), how they coordinate (delegation), and what controls keep them safe (governance).

Each capability involves tradeoffs. Richer memory improves context but increases latency. Aggressive escalation improves safety but reduces autonomy. Tighter approval gates reduce risk but slow execution. There are no free lunches, only choices that must be calibrated to your context, risk tolerance, and professional obligations. Additional resources, including protocol specifications, regulatory guidance links, and research references, are available on the book's companion website. Chapter~8 (Agents Part III: Governing Agents) provides a risk-based calibration approach rather than a one-size-fits-all checklist.

The organizational analogy is not merely pedagogical. Law firms and investment teams are cognitive work systems that have evolved infrastructure for exactly these challenges: distributing work, maintaining context, and ensuring quality. When you evaluate an agentic system, ask the same questions you would ask about a professional team.

% ----------------------------------------------------------------------------
% What You Can Now Do
% ----------------------------------------------------------------------------

\subsection{What This Understanding Enables}
\label{sec:agents2-conclusion-enables}

With architectural literacy, you can evaluate vendor claims. When a vendor says their agent ``handles legal research,'' ask: What triggers it (\Cref{sec:agents2-triggers})? How does it understand the question (\Cref{sec:agents2-intent})? What databases does it access (\Cref{sec:agents2-perception})? How does it know when it is done (\Cref{sec:agents2-termination})? What happens when confidence is low (\Cref{sec:agents2-escalation})? The ten questions from \Cref{sec:agents2-ten-questions} provide your evaluation framework.

You can participate meaningfully in procurement. Assess whether a system meets requirements: Does it enforce matter isolation? Maintain audit trails? Integrate with approval workflows? Escalate appropriately? You have the vocabulary to specify requirements.

You can demand governance artifacts, not promises. Ask vendors to demonstrate action gating, escalation behavior under low confidence, and reconstruction via logs. If a system cannot show you what it accessed, what it did, and why it stopped, it is not deployable in regulated practice.

You can design governance that maps to architecture. Governance is enabled by architecture. If you want audit trails, the system must log reasoning. If you want approval gates, the system must pause before action. If you want confidentiality, the system must isolate context. You can design systems where governance is built in, not bolted on.

Finally, you can communicate with technical teams. Describe requirements precisely: ``I need perception tools for these databases, action tools behind approval gates, escalation triggers for low confidence, and memory isolation between matters.'' Shared vocabulary enables collaboration.

% ----------------------------------------------------------------------------
% Evaluation Checklist
% ----------------------------------------------------------------------------

\subsection{Evaluation Checklist}
\label{sec:agents2-evaluation-checklist}

Table~\ref{tab:agents2-eval-checklist} consolidates evaluation criteria across all ten architectural questions. Use this checklist when assessing vendor systems, designing your own, or conducting due diligence on agentic deployments.

\begin{table}[htbp]
\centering
\caption{Evaluation checklist by architectural component}
\label{tab:agents2-eval-checklist}
\small
\begin{tabular}{ll}
\toprule
\textbf{Component} & \textbf{Key Topics} \\
\midrule
Triggers & Source coverage; latency; failure detection; audit logging \\
Intent & Ambiguity detection; constraint validation; clarification requests \\
Perception & Data source access; access controls; provenance; jurisdictional filtering \\
Action & Approval gates; reversibility classification; circuit breakers \\
Memory & Matter/client isolation; retention policies; state reconstruction; ethical walls \\
Planning & Budget enforcement; pattern selection; plan validation \\
Termination & Success criteria; failure recognition; graceful degradation \\
Escalation & Confidence thresholds; handoff context; routing accuracy \\
Delegation & Agent identity; delegation contracts; cross-agent ethical walls \\
Governance & Logging infrastructure; override mechanisms; state snapshots; least privilege \\
\bottomrule
\end{tabular}
\end{table}

% ----------------------------------------------------------------------------
% Current Limitations
% ----------------------------------------------------------------------------

\subsection{Honest Assessment of Current Capabilities}
\label{sec:agents2-conclusion-limitations}

Architectural understanding requires honest acknowledgment of limitations.

\textbf{The reliability cliff} (\Cref{sec:agents2-reliability}) is the most significant constraint. Agents exhibit near-perfect success on short tasks but fail frequently on multi-hour workflows. Design systems that decompose work aggressively and checkpoint progress frequently.

\textbf{Judgment limitations} constrain value. Agents excel at retrieval and pattern matching but struggle with nuance and novelty. The effective deployments pair agent capabilities with human judgment.

\textbf{Brittleness} causes failures due to API changes or edge cases. Build monitoring and graceful degradation into every deployment.

\textbf{Compounding errors} affect multi-step workflows. Error probabilities multiply, so long autonomous chains fail. Shorter workflows with human checkpoints perform better.

% ----------------------------------------------------------------------------
% Bridge to Next Chapter
% ----------------------------------------------------------------------------

\subsection{From Architecture to Governance}
\label{sec:agents2-conclusion-bridge}

This chapter answered: \textit{How do you build an agent?}
The next chapter answers: \textit{How do you govern one?}

As \Cref{sec:agents2-governance} demonstrated, five architectural patterns enable governance: logging across all components, override and circuit breaker mechanisms, state snapshots and checkpoints, least privilege enforcement, and escalation hooks. These patterns form the \textit{governance surface}---the interface through which governance systems interact with agents. If you did not architect for logging, you cannot audit; if you did not architect for isolation, you cannot enforce boundaries; if you did not architect for override, you cannot intervene.

The governance imperative is that agents are not just tools. They interpret goals, select what to perceive, and take actions that can create liability. That shift introduces predictable accountability problems: purpose drift (misaligned goals), perceptual opacity (bad or manipulated inputs), and actuation risk (high-consequence actions). \href{https://papers.ssrn.com/abstract=5911464}{\textit{Governing Agents}} provides dimensional calibration for matching controls to risk and presents a five-layer framework that translates these challenges into concrete controls: calibrated autonomy, input and action constraints, auditability, retention, escalation and override mechanisms, and accountability structures.

With the architectural literacy developed in this chapter, you can engage meaningfully with the governance frameworks that follow---specifying requirements precisely, evaluating vendor claims critically, and ensuring that controls are built into systems from the start rather than retrofitted after deployment. Agents are not magic; they are architecture. And architecture, once understood, can be governed.


\end{document}
