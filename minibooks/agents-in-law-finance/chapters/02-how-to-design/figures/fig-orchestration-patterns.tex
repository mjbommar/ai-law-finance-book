% fig-orchestration-patterns.tex
% Three Multi-Agent Orchestration Patterns
% Part of: Chapter 2 - How to Design an Agent
% Section: 10-Delegation

\begin{figure}[htbp]
\centering
\resizebox{\textwidth}{!}{%
\begin{tikzpicture}[
    % Card styles - reduced height since text moved to table
    card/.style={
        rounded corners=6pt,
        line width=1.5pt,
        minimum width=5.5cm,
        minimum height=6.5cm,
        align=center,
        inner sep=0pt
    },
    card header/.style={
        font=\small\bfseries\sffamily,
        text=white,
        minimum height=0.9cm,
        text width=5.0cm,
        align=center,
        rounded corners=4pt
    },
    % Mini-flowchart styles - increased size for legibility
    flow box/.style={
        font=\scriptsize\sffamily,
        align=center,
        text width=2.0cm,
        rounded corners=3pt,
        inner sep=4pt,
        minimum height=1.4em,
        line width=1pt
    },
    flow box wide/.style={
        flow box,
        text width=2.5cm
    },
    flow box narrow/.style={
        flow box,
        text width=1.25cm
    },
    flow box small/.style={
        flow box,
        text width=1.0cm,
        font=\scriptsize\sffamily
    },
    flow arrow/.style={
        -stealth,
        line width=1.2pt,
        color=border-neutral
    }
]

% ========== LEFT CARD: SEQUENTIAL ==========
\node[card, draw=definition-dark, fill=white] (seq-card) at (-5.5, 0) {};

% Header
\node[card header, fill=definition-dark] at (seq-card.north) [yshift=0.15cm] {\textbf{Sequential}};

% Mini-flowchart (vertical chain) - scaled positions, shifted down 0.5cm
\node[flow box wide, fill=bg-definition, draw=border-definition] (seq-coord) at (-5.5, 2.0) {Coordinator};
\node[flow box, fill=bg-definition, draw=border-definition] (seq-a) at (-5.5, 0.9) {Agent A};
\node[flow box, fill=bg-definition, draw=border-definition] (seq-b) at (-5.5, -0.2) {Agent B};
\node[flow box, fill=bg-definition, draw=border-definition] (seq-c) at (-5.5, -1.3) {Agent C};
\node[flow box wide, fill=bg-definition, draw=border-definition] (seq-synth) at (-5.5, -2.4) {Synthesize};

\draw[flow arrow] (seq-coord) -- (seq-a);
\draw[flow arrow] (seq-a) -- (seq-b);
\draw[flow arrow] (seq-b) -- (seq-c);
\draw[flow arrow] (seq-c) -- (seq-synth);

% ========== MIDDLE CARD: PARALLEL ==========
\node[card, draw=key-dark, fill=white] (par-card) at (0, 0) {};

% Header
\node[card header, fill=key-dark] at (par-card.north) [yshift=0.15cm] {\textbf{Parallel}};

% Mini-flowchart (fan-out, fan-in) - scaled positions, shifted down 0.5cm
\node[flow box wide, fill=bg-key, draw=border-key] (par-coord) at (0, 1.4) {Coordinator};
\node[flow box narrow, fill=bg-key, draw=border-key] (par-a) at (-1.6, -0.2) {Agent A};
\node[flow box narrow, fill=bg-key, draw=border-key] (par-b) at (0, -0.2) {Agent B};
\node[flow box narrow, fill=bg-key, draw=border-key] (par-c) at (1.6, -0.2) {Agent C};
\node[flow box wide, fill=bg-key, draw=border-key] (par-synth) at (0, -1.8) {Synthesize};

\draw[flow arrow] (par-coord) -- (par-a);
\draw[flow arrow] (par-coord) -- (par-b);
\draw[flow arrow] (par-coord) -- (par-c);
\draw[flow arrow] (par-a) -- (par-synth);
\draw[flow arrow] (par-b) -- (par-synth);
\draw[flow arrow] (par-c) -- (par-synth);

% ========== RIGHT CARD: HIERARCHICAL ==========
\node[card, draw=example-dark, fill=white] (hier-card) at (5.5, 0) {};

% Header
\node[card header, fill=example-dark] at (hier-card.north) [yshift=0.15cm] {\textbf{Hierarchical}};

% Mini-flowchart (tree with sub-agents) - scaled positions, shifted down 0.5cm
\node[flow box wide, fill=bg-example, draw=border-example] (hier-coord) at (5.5, 2.3) {Coordinator};
\node[flow box, fill=bg-example, draw=border-example] (hier-a) at (4.1, 0.8) {Agent A};
\node[flow box small, fill=bg-example, draw=border-example] (hier-a1) at (6.9, 1.2) {A1};
\node[flow box small, fill=bg-example, draw=border-example] (hier-a2) at (6.9, 0.4) {A2};
\node[flow box, fill=bg-example, draw=border-example] (hier-b) at (4.1, -1.0) {Agent B};
\node[flow box small, fill=bg-example, draw=border-example] (hier-b1) at (6.9, -0.5) {B1};
\node[flow box small, fill=bg-example, draw=border-example] (hier-b2) at (6.9, -1.1) {B2};
\node[flow box small, fill=bg-example, draw=border-example] (hier-b3) at (6.9, -1.8) {B3};
\node[flow box wide, fill=bg-example, draw=border-example] (hier-synth) at (5.5, -2.7) {Synthesize};

\draw[flow arrow] (hier-coord.south west) -- (hier-a.north);
\draw[flow arrow] (hier-a.east) -- (hier-a1.west);
\draw[flow arrow] (hier-a.east) -- (hier-a2.west);
\draw[flow arrow] (hier-coord.south) -- (hier-b.north);
\draw[flow arrow] (hier-b.east) -- (hier-b1.west);
\draw[flow arrow] (hier-b.east) -- (hier-b2.west);
\draw[flow arrow] (hier-b.east) -- (hier-b3.west);
\draw[flow arrow] (hier-a.south) -- (hier-synth.north west);
\draw[flow arrow] (hier-b.south) -- (hier-synth.north);

\end{tikzpicture}
}%

\small
\begin{flushleft}
\textbf{Sequential}: Best for tasks with dependencies between steps. Trade-off: slower; blocked by bottlenecks.
\textbf{Parallel}: Best for independent work that can run concurrently. Trade-off: coordination overhead.
\textbf{Hierarchical}: Best for complex tasks requiring sub-delegation. Trade-off: harder to debug.
\end{flushleft}

\caption{Three multi-agent orchestration patterns. Sequential delegation chains agents in order, ideal for dependent tasks but vulnerable to bottlenecks. Parallel delegation runs agents concurrently, maximizing throughput for independent work but requiring coordination. Hierarchical delegation enables sub-agents to handle specialized sub-tasks, providing flexibility for complex workflows at the cost of debugging complexity.}
\label{fig:agents2-orchestration-patterns}
\end{figure}
