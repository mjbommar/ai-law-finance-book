% fig-react-cycle.tex
% ReAct Cycle: Thought, Action, Observation
% Part of: Chapter 07 - Agents Part II: How to Design an Agent
% Section: 07-planning

\begin{figure}[!htb]
\centering
\begin{tikzpicture}[
    % Description style
    desc/.style={
        font=\small,
        text width=4cm,
        align=center
    }
]

% Circular nodes at 120-degree intervals
\foreach \i/\label/\angle in {1/Thought/90, 2/Action/-30, 3/Observation/210} {
    \node[
        circle,
        fill=bg-definition,
        draw=definition-dark,
        line width=1.5pt,
        text=definition-dark,
        minimum size=2.4cm,
        font=\normalsize\bfseries,
        align=center,
    ] (n\i) at (\angle:2.8cm) {\label};
}

% Arrows between nodes - bend left for clockwise = follows the circle path
\foreach \from/\to in {1/2, 2/3, 3/1} {
    \draw[-{Stealth[length=3mm]}, gray-500, line width=2pt]
        (n\from) to[bend left=40] (n\to);
}

% Descriptions positioned radially outside
\node[desc, anchor=south] at (90:5cm) {Explicit reasoning about\\what to do next};
\node[desc, anchor=west] at (-30:5cm) {A tool call to gather\\information or effect change};
\node[desc, anchor=east] at (210:5cm) {The tool output that\\informs the next thought};

\end{tikzpicture}
\caption{The ReAct cycle interleaves reasoning with action. The agent reasons about what to do (Thought), executes a tool call (Action), observes the result (Observation), and uses that observation to inform the next round of reasoning.}
\label{fig:agents2-react-cycle}
\end{figure}
