% ============================================================================
% pattern-selection.tex
% TikZ flowchart for pattern selection decision tree
% Part of: Chapter 2 - How to Design an Agent
% Section: Synthesis and Applications
% ============================================================================

\begin{figure}[htbp]
\centering
\begin{tikzpicture}[
  % Node styles
  start/.style={
    ellipse,
    draw=definition-dark,
    fill=bg-definition,
    thick,
    minimum width=2.5cm,
    minimum height=1cm,
    align=center,
    font=\small\bfseries\sffamily
  },
  decision/.style={
    diamond,
    draw=definition-dark,
    fill=bg-definition,
    thick,
    minimum width=2.8cm,
    minimum height=2.8cm,
    align=center,
    font=\small\sffamily,
    inner sep=2pt,
    aspect=2
  },
  outcome-low/.style={
    rectangle,
    draw=border-example,
    fill=bg-example,
    thick,
    rounded corners=3pt,
    minimum width=3.5cm,
    minimum height=1.2cm,
    align=left,
    font=\small\sffamily,
    inner sep=6pt
  },
  outcome-med/.style={
    rectangle,
    draw=border-definition,
    fill=bg-definition,
    thick,
    rounded corners=3pt,
    minimum width=3.5cm,
    minimum height=1.2cm,
    align=left,
    font=\small\sffamily,
    inner sep=6pt
  },
  outcome-high/.style={
    rectangle,
    draw=border-key,
    fill=bg-key,
    thick,
    rounded corners=3pt,
    minimum width=3.5cm,
    minimum height=1.2cm,
    align=left,
    font=\small\sffamily,
    inner sep=6pt
  },
  % Arrow styles
  arrow/.style={
    -stealth,
    thick,
    draw=definition-dark
  },
  yes/.style={
    arrow,
    draw=text-primary
  },
  no/.style={
    arrow,
    draw=text-muted
  },
  % Label styles
  yeslabel/.style={
    font=\scriptsize\bfseries\sffamily,
    text=text-primary,
    pos=0.3
  },
  nolabel/.style={
    font=\scriptsize\sffamily,
    text=text-muted,
    pos=0.3
  }
]

% Start node
\node[start] (start) at (0,0) {START};

% Decision 1: Irreversible actions?
\node[decision] (d1) at (0,-2.5) {Irreversible\\external\\actions?};

% Pattern 4 (High risk - amber/orange)
\node[outcome-high] (p4) at (5,-2.5) {
  \textbf{Pattern 4}\\
  Autonomous Filing\\
  \textit{High Risk}
};

% Decision 2: Multiple subtasks?
\node[decision] (d2) at (0,-5.5) {Multiple\\specialized\\subtasks?};

% Pattern 3 (High risk - amber/orange)
\node[outcome-high] (p3) at (5,-5.5) {
  \textbf{Pattern 3}\\
  Multi-Agent Workflow\\
  \textit{High Risk}
};

% Decision 3: High-volume documents?
\node[decision] (d3) at (0,-8.5) {High-volume\\document\\processing?};

% Pattern 2 (Medium risk - blue)
\node[outcome-med] (p2) at (5,-8.5) {
  \textbf{Pattern 2}\\
  Document Review\\
  \textit{Medium Risk}
};

% Pattern 1 (Low risk - green)
\node[outcome-low] (p1) at (0,-11.5) {
  \textbf{Pattern 1}\\
  Assisted Research\\
  \textit{Low Risk}\\
  (Default starting point)
};

% Arrows and labels
\draw[arrow] (start) -- (d1);

\draw[yes] (d1) -- node[yeslabel, above] {YES} (p4);
\draw[no] (d1) -- node[nolabel, right] {NO} (d2);

\draw[yes] (d2) -- node[yeslabel, above] {YES} (p3);
\draw[no] (d2) -- node[nolabel, right] {NO} (d3);

\draw[yes] (d3) -- node[yeslabel, above] {YES} (p2);
\draw[no] (d3) -- node[nolabel, right] {NO} (p1);

\end{tikzpicture}

\caption{Pattern selection decision tree for legal AI agent deployment. The flowchart guides practitioners from task characteristics to appropriate deployment pattern, with risk levels indicated by color: green (low), blue (medium), amber (high). Start with Pattern 1 unless task requirements justify higher-risk patterns.}
\label{fig:agents2-pattern-selection}
\end{figure}
