% fig-rag-pipeline.tex
% RAG Pipeline: Embedding-Based Implementation
% Part of: Chapter 2 - How to Design an Agent
% Section: 06-memory (RAG)

\begin{figure}[htbp]
\centering
\begin{tikzpicture}[
    % Step box style
    step box/.style={
        rectangle,
        rounded corners=6pt,
        minimum width=2.8cm,
        minimum height=1.5cm,
        align=center,
        line width=1.5pt,
        font=\small\bfseries
    },
    % Input/Output box style
    io box/.style={
        rectangle,
        rounded corners=6pt,
        minimum width=2.4cm,
        minimum height=1.5cm,
        align=center,
        line width=1.5pt,
        font=\small\bfseries,
        fill=bg-key,
        draw=key-dark,
        text=key-dark
    },
    % Description style
    desc/.style={
        font=\scriptsize,
        text width=2.6cm,
        align=center,
        anchor=north
    },
    % Arrow style
    arrow/.style={
        -stealth,
        line width=1.5pt,
        color=gray-500
    }
]

% Row 1 (top): Documents → Chunking → Embedding
\node[io box] (docs) at (0, 0) {Documents};
\node[step box, fill=bg-definition, draw=definition-dark, text=definition-dark] (chunk) at (3.8, 0) {1. Chunking};
\node[step box, fill=bg-definition, draw=definition-dark, text=definition-dark] (embed) at (7.6, 0) {2. Embedding};

% Descriptions for top row (above boxes)
\node[desc, text=gray-600, anchor=south] at (chunk.north) [yshift=0.15cm] {Break documents\\into semantic units};
\node[desc, text=gray-600, anchor=south] at (embed.north) [yshift=0.15cm] {Convert chunks\\into vectors};

% Row 2 (bottom): Retrieval → Generation → Answer
\node[step box, fill=bg-definition, draw=definition-dark, text=definition-dark] (retrieve) at (0, -2.6) {3. Retrieval};
\node[step box, fill=bg-definition, draw=definition-dark, text=definition-dark] (generate) at (3.8, -2.6) {4. Generation};
\node[io box] (answer) at (7.6, -2.6) {Answer};

% Descriptions for bottom row (below boxes)
\node[desc, text=gray-600, anchor=north] at (retrieve.south) [yshift=-0.15cm] {Find similar chunks\\by vector comparison};
\node[desc, text=gray-600, anchor=north] at (generate.south) [yshift=-0.15cm] {Inject context\\into LLM prompt};

% Arrows - top row
\draw[arrow] (docs.east) -- (chunk.west);
\draw[arrow] (chunk.east) -- (embed.west);

% Arrow connecting rows (curved down)
\draw[arrow] (embed.south) -- ++(0, -0.4) -| (retrieve.north);

% Arrows - bottom row
\draw[arrow] (retrieve.east) -- (generate.west);
\draw[arrow] (generate.east) -- (answer.west);

\end{tikzpicture}
\caption{An embedding-based RAG pipeline---one common implementation pattern. Documents are chunked into semantic units, embedded as vectors, and retrieved by semantic similarity. Alternative implementations may skip chunking and embedding entirely, using keyword search, BM25, database queries, or hybrid approaches to retrieve relevant content.}
\label{fig:agents2-rag-pipeline}
\end{figure}
