% When to Clarify: Compact Stakes × Ambiguity Decision Matrix

\begin{figure}[htbp]
\centering
\begin{tikzpicture}[
    cell/.style={
        rounded corners=4pt,
        line width=1.2pt,
        minimum width=4.2cm,
        minimum height=1.6cm,
        align=center,
        inner sep=8pt
    },
    action/.style={
        font=\small\bfseries\sffamily
    },
    example/.style={
        font=\scriptsize\itshape,
        text=text-secondary
    },
    axis label/.style={
        font=\scriptsize\bfseries\sffamily,
        text=text-secondary
    }
]

% Column headers
\node[axis label] at (-2.3, 2.1) {Low Ambiguity};
\node[axis label] at (2.3, 2.1) {High Ambiguity};

% Row headers
\node[axis label, rotate=90, anchor=south, align=center] at (-4.8, 0.9) {Low\\Stakes};
\node[axis label, rotate=90, anchor=south, align=center] at (-4.8, -0.9) {High\\Stakes};

% Top-left: Low Stakes, Low Ambiguity
\node[cell, draw=example-base, fill=bg-example] (tl) at (-2.3, 0.9) {
    \textcolor{example-dark}{\textsf{\textbf{PROCEED}}}\\[2pt]
    \textcolor{text-muted}{\scriptsize\itshape ``Apple market cap''}
};

% Top-right: Low Stakes, High Ambiguity
\node[cell, draw=key-base, fill=bg-key] (tr) at (2.3, 0.9) {
    \textcolor{key-dark}{\textsf{\textbf{CLARIFY BRIEFLY}}}\\[2pt]
    \textcolor{text-muted}{\scriptsize\itshape ``Research SOL''}
};

% Bottom-left: High Stakes, Low Ambiguity
\node[cell, draw=key-base, fill=bg-key] (bl) at (-2.3, -0.9) {
    \textcolor{key-dark}{\textsf{\textbf{CONFIRM}}}\\[2pt]
    \textcolor{text-muted}{\scriptsize\itshape ``File this motion''}
};

% Bottom-right: High Stakes, High Ambiguity
\node[cell, draw=caution-base, fill=bg-caution] (br) at (2.3, -0.9) {
    \textcolor{caution-dark}{\textsf{\textbf{CLARIFY THOROUGHLY}}}\\[2pt]
    \textcolor{text-muted}{\scriptsize\itshape ``Handle regulatory response''}
};

\end{tikzpicture}
\caption{Decision matrix for when agents should clarify user intent. Stakes measure consequence of error; ambiguity measures interpretation confidence.}
\label{fig:agents2-when-to-clarify}
\end{figure}
