% fig-mcp-architecture.tex
% Model Context Protocol (MCP) Architecture
% Part of: Chapter 2 - How to Design an Agent
% Section: 04-perception (MCP subsection)

\begin{figure}[htbp]
\centering
\begin{tikzpicture}[
    % Host box style
    hostbox/.style={
        rectangle,
        rounded corners=6pt,
        minimum width=6.3cm,
        minimum height=1.1cm,
        fill=bg-key,
        draw=key-dark,
        line width=1.2pt,
        text=key-dark,
        font=\small\bfseries,
        align=center
    },
    % Client box style
    clientbox/.style={
        rectangle,
        rounded corners=5pt,
        minimum width=1.85cm,
        minimum height=0.9cm,
        fill=bg-definition,
        draw=definition-dark,
        line width=1pt,
        text=definition-dark,
        font=\scriptsize\bfseries,
        align=center
    },
    % Server box style
    serverbox/.style={
        rectangle,
        rounded corners=5pt,
        minimum width=1.85cm,
        minimum height=1.1cm,
        fill=bg-example,
        draw=example-dark,
        line width=1pt,
        text=example-dark,
        font=\scriptsize\bfseries,
        align=center
    },
    % Primitive label style
    primitive/.style={
        rectangle,
        rounded corners=2pt,
        minimum width=1.6cm,
        minimum height=0.5cm,
        fill=gray-100,
        draw=gray-300,
        line width=0.6pt,
        text=gray-600,
        font=\tiny,
        align=center
    },
    % Arrow style
    arrow/.style={
        -stealth,
        line width=1pt,
        gray-500
    },
    % Bidirectional arrow
    biarrow/.style={
        stealth-stealth,
        line width=1pt,
        gray-500
    },
    % Label style
    lbl/.style={
        font=\tiny,
        text=gray-500
    }
]

% Host (top)
\node[hostbox] (host) at (0, 0) {MCP Host {\scriptsize\normalfont (Agent Application)}};

% Clients (middle row)
\node[clientbox] (client1) at (-2.55, -1.6) {Client};
\node[clientbox] (client2) at (0, -1.6) {Client};
\node[clientbox] (client3) at (2.55, -1.6) {Client};

% Servers (bottom row)
\node[serverbox] (server1) at (-2.55, -3.4) {Server\\[-1pt]{\tiny\normalfont Docs}};
\node[serverbox] (server2) at (0, -3.4) {Server\\[-1pt]{\tiny\normalfont Data}};
\node[serverbox] (server3) at (2.55, -3.4) {Server\\[-1pt]{\tiny\normalfont Research}};

% Primitives (below servers, compact)
\node[primitive] (prim1) at (-2.55, -4.7) {Resources};
\node[primitive] (prim2) at (0, -4.7) {Tools};
\node[primitive] (prim3) at (2.55, -4.7) {Prompts};

% Host to Clients
\draw[arrow] (host.south) ++(-2.55, 0) -- (client1.north);
\draw[arrow] (host.south) -- (client2.north);
\draw[arrow] (host.south) ++(2.55, 0) -- (client3.north);

% Clients to Servers
\draw[biarrow] (client1.south) -- (server1.north);
\draw[biarrow] (client2.south) -- (server2.north);
\draw[biarrow] (client3.south) -- (server3.north);

% Servers to Primitives
\draw[arrow, gray-400] (server1.south) -- (prim1.north);
\draw[arrow, gray-400] (server2.south) -- (prim2.north);
\draw[arrow, gray-400] (server3.south) -- (prim3.north);

% Labels
\node[lbl] at (-0.9, -0.8) {manages};
\node[lbl] at (1.6, -2.5) {JSON-RPC};

\end{tikzpicture}
\caption{MCP architecture. The Host manages access control. Each Client connects to one Server. Servers expose primitives: Resources (read-only data), Tools (executable functions), and Prompts (reusable templates).}
\label{fig:agents2-mcp-architecture}
\end{figure}
