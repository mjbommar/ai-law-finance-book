% How to Read This Chapter - Reading paths for different audiences
% This section appears before the main content to guide readers

\section*{How to Read This Chapter}
\addcontentsline{toc}{section}{How to Read This Chapter}

This chapter answers a deceptively simple question: \textit{What is an agent?}

The term ``agent'' appears everywhere in industry and academia today---from ``AI agents'' to ``agentic workflows''---yet rarely with clear definition. This creates confusion, miscommunication, and inevitably, disappointment.

We hope to help you avoid this disappointment, but \textbf{this chapter is not short and your time is valuable}. You do not need to read everything, especially on first reading. Three reading paths correspond to different goals:

\paragraph{Path 1: Get the working definition} If you need practical clarity fast, read Sections~\ref{sec:intro}--\ref{sec:practical} and stop. You will get the three-level hierarchy, the six operational properties, and a practical evaluation rubric. This is enough to use the term correctly, evaluate vendor claims, and participate in informed discussions. You can always return for deeper context later.

\paragraph{Path 2: Understand where this came from} If you want to understand why we define agents this way and how the concept evolved, continue through Section~\ref{sec:disciplines}. You will learn how agency emerged from 1950s philosophy through 1990s distributed systems to today's LLM-powered tools (Section~\ref{sec:history}), and why eight different disciplines---from law to cognitive science---emphasize different aspects of agency (Section~\ref{sec:disciplines}). This historical and disciplinary context explains current debates and grounds our synthesis in established scholarship.

\paragraph{Path 3: Master the analytical framework} If you are writing research, crafting policy, or need comprehensive understanding, read everything. Sections~\ref{sec:dimensions}--\ref{sec:synthesis} provide analytical dimensions (autonomy spectrum, entity frames, goal dynamics), formal specifications, boundary cases, and professional deployment requirements. Section~\ref{sec:furtherlearning} synthesizes key takeaways and connects to subsequent chapters. This path gives you the theoretical foundations needed for rigorous analysis.

\vspace{1em}

This chapter provides conceptual foundations and analytical frameworks. Part II (\textit{How to Build an Agent}) covers architectures, protocols, and technical evaluation. Part III (\textit{How to Govern an Agent}) addresses regulation, controls, and deployment.

\vspace{0.5em}
\noindent\textcolor{border-neutral}{\rule{\textwidth}{1.5pt}}

\pagebreak

\section*{At a Glance: The Framework in One Page}
\addcontentsline{toc}{section}{At a Glance: The Framework in One Page}

\begin{definitionbox}[title={\textbf{Six Properties (Mnemonic: GPA + IAT)}}]
\small
\textbf{G (Goal):} Objective directing behavior toward desired outcomes \\
\textbf{P (Perception):} Observing environment to inform action selection \\
\textbf{A (Action):} Executing decisions that affect environment or state \\
\textbf{I (Iteration):} Multiple perception-action cycles, not single-shot \\
\textbf{A (Adaptation):} Modifying policy based on observations within session \\
\textbf{T (Termination):} Explicit or implicit stopping conditions
\medskip

\textit{Note: the mnemonic reuses ``A'' twice---first for Action, then for Adaptation. When we need extra precision, we distinguish $A_{\text{act}}$ (action) from $A_{\text{adapt}}$ (adaptation).}
\end{definitionbox}

\begin{keybox}[title={\textbf{Three-Level Hierarchy}}]
\textbf{Level 1 (Agent):} G + P + A — 3 properties minimum

\textbf{Level 2 (Agentic System):} G + P + A + I + A + T — All 6 via rules/algorithms

\textbf{Level 3 (Agentic AI):} G + P + A + I + A + T — All 6 via LLMs/neural networks

\vspace{0.3em}
\small\textit{Quick check: Q1--Q3 yes = Agent | Q1--Q6 yes = Agentic System | All 6 + AI/ML = Agentic AI}
\end{keybox}

\begin{table}[h!]
\centering
\small
\begin{tabular}{@{}>{\raggedright\arraybackslash}p{7.2cm}>{\raggedright\arraybackslash}p{7.2cm}@{}}
\toprule
\textbf{Example: Portfolio Rebalancer (L2)} & \textbf{Example: Legal Research (L3)} \\
\midrule
Monitors positions, executes trades, adapts to volatility, stops when balanced. \textit{(6 properties via rules)} & Searches case law, refines queries based on results, escalates when stuck. \textit{(6 properties via LLM)} \\
\bottomrule
\end{tabular}
\end{table}

\vspace{-0.5em}

\begin{highlightbox}[title={\textbf{Professional Deployment Adds Four Safeguards}}]
\small
\textbf{Attribution}: Every claim traceable to authoritative sources \\
\textbf{Provenance}: Auditable logs showing reasoning steps and tool invocations \\
\textbf{Escalation}: Clear triggers for human review of high-stakes decisions \\
\textbf{Confidentiality}: Ethical walls and privilege protection mechanisms
\end{highlightbox}

\pagebreak
