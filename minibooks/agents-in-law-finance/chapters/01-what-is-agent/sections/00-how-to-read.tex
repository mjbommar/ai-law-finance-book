% How to Read This Chapter - Reading paths for different audiences
% This section appears before the main content to guide readers

\section*{How to Read This Chapter}
\addcontentsline{toc}{section}{How to Read This Chapter}

This chapter answers a deceptively simple question: \textit{What is an agent?}

The term ``agent'' appears everywhere in industry and academia today, from ``AI agents'' to ``agentic workflows,'' yet rarely with clear definition. This creates confusion, miscommunication, and inevitably, disappointment.

We hope to help you avoid this disappointment, but \textbf{this chapter is not short and your time is valuable}. You do not need to read everything, especially on first reading. Three reading paths correspond to different goals:

\paragraph{Path 1: Get the working definition} If you need practical clarity fast, read Sections~\ref{sec:intro}--\ref{sec:practical} and stop. You will get the three-level hierarchy, the six operational properties, and a practical evaluation rubric. This is enough to use the term correctly, evaluate vendor claims, and participate in informed discussions. You can always return for deeper context later.

\paragraph{Path 2: Understand where this came from} If you want to understand why we define agents this way and how the concept evolved, continue through Section~\ref{sec:disciplines}. You will learn how agency emerged from 1950s philosophy through 1990s distributed systems to today's LLM-powered tools (Section~\ref{sec:history}), and why eight different disciplines---from law to cognitive science---emphasize different aspects of agency (Section~\ref{sec:disciplines}). This historical and disciplinary context explains current debates and grounds our synthesis in established scholarship.

\paragraph{Path 3: Master the analytical framework} If you are writing research, crafting policy, or need comprehensive understanding, read everything. Sections~\ref{sec:dimensions}--\ref{sec:synthesis} provide analytical dimensions (autonomy spectrum, entity frames, goal dynamics), formal specifications, boundary cases, and professional deployment requirements. Section~\ref{sec:furtherlearning} synthesizes key takeaways and connects to subsequent chapters. This path gives you the theoretical foundations needed for rigorous analysis.

\bigskip

This chapter provides conceptual foundations and analytical frameworks. Chapter~2 (\textit{How to Design an Agent}) covers architectures, protocols, and technical evaluation. Chapter~3 (\textit{How to Govern an Agent}) addresses regulation, controls, and deployment.

