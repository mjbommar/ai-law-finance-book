% ============================================================================
% HOW TO READ THIS MINI-BOOK
% ============================================================================

\chapter*{How to Read This Book}
\addcontentsline{toc}{chapter}{How to Read This Book}
\markboth{How to Read This Book}{}

\section*{Chapter Outline}

The three chapters follow a logical sequence: \textit{definition} \textrightarrow{} \textit{design} \textrightarrow{} \textit{governance}. Read straight through for cumulative understanding, or jump to whichever chapter addresses your immediate need.

\textbf{Chapter~1: What is an Agent?} provides definitional clarity. If you need to cut through marketing hype, evaluate vendor claims, or simply use the term ``agent'' with precision, start here. The chapter synthesizes perspectives from philosophy, computer science, law, and economics into a unified framework with practical evaluation tools. For a shorter read, Sections~1.1--1.2 deliver the core framework; the remaining sections add historical context and disciplinary depth.

\textbf{Chapter~2: How to Design an Agent} addresses architectural decisions. If you are building, evaluating, or procuring an agentic system, this chapter provides a structured way to think through key design choices---from how work enters the system to how it coordinates with humans and other agents. Each section stands alone, so you can read selectively based on which decisions you face.

\glsadd{autonomy-spectrum}\textbf{Chapter~3: How to Govern an Agent} covers risk, compliance, and accountability. If you are responsible for approving agent deployments, this chapter provides frameworks for calibrating oversight to your specific context---what controls to implement, how to structure accountability, and when to escalate. Note that this chapter provides conceptual tools, not legal advice; consult qualified experts for your jurisdiction and sector.

\needspace{8\baselineskip}
\section*{Visual Elements}

Colored boxes signal different types of content. Color is functional, not decorative---it indicates how to read what follows.

\begin{keybox}[title={Key Takeaways}]
Orange boxes highlight essential points to remember or operationalize.
\end{keybox}

\begin{definitionbox}[title={Definitions}]
Blue boxes introduce formal definitions and terms of art used throughout.
\end{definitionbox}

\begin{examplebox}[title={Concrete Examples}]
Green boxes provide scenarios and case studies from law and finance.
\end{examplebox}

\begin{practicebox}[title={Practice Checklists}]
Teal boxes offer checklists and workflows for direct application.
\end{practicebox}

\begin{cautionbox}[title={Warnings and Risks}]
Red boxes flag error modes, compliance risks, and failure cases.
\end{cautionbox}

\begin{technicalbox}[title={Technical Details}]
Indigo boxes contain protocol details and implementation mechanics---optional on first reading.
\end{technicalbox}

\section*{Reference Materials}

\paragraph{Glossary} Technical terms with specific meanings in the agent context appear in the glossary at the end. First uses are often marked with \keyterm{key term formatting}.

\paragraph{References} The bibliography consolidates citations from all three chapters. Parenthetical citations \parencite{russellnorvig2020aima} indicate background sources; narrative citations like \textcite{russellnorvig2020aima} appear when the author is part of the sentence.

