% ============================================================================
% SHARED PREAMBLE - Agents in Law and Finance (Mini-Book)
% ============================================================================
% This preamble synthesizes configurations from three source chapters:
%   - Chapter 06: agents-part-1 (What is an Agent?)
%   - Chapter 07: agents-part-2 (How to Design an Agent)
%   - Chapter 08: agents-part-3 (How to Govern an Agent)
%
% Key synthesis decisions:
%   - TikZ: Superset of all chapter libraries (9 total)
%   - TikZ: NO global font override (preserves explicit \tiny, \scriptsize in figures)
%   - tcolorbox: Core semantic boxes (definitionbox, examplebox, highlightbox,
%                keybox, cautionbox, questionbox, theorembox, technicalbox,
%                practicebox, listingbox)
%   - Listings: Enhanced syntax highlighting from chapter 07
%   - Ligatures: Enabled (for proper dash rendering -- and ---)
%   - Colors: Full 4-layer system with legacy compatibility
%   - ToC: Styled for book class with chapter/section/subsection hierarchy
%
% Used by:
%   - Complete mini-book compilation (via subfiles)
%
% Do NOT include \documentclass or \begin{document} here
% ============================================================================

% ============================================================================
% MODERN TYPOGRAPHY AND LAYOUT
% ============================================================================
% US Trade size: 6in × 9in (standard for professional/technical books)
% Margins optimized for readability and POD printing
\usepackage[
  papersize={6in,9in},
  inner=0.75in,       % Gutter margin (for binding)
  outer=0.5in,        % Outside margin
  top=0.7in,
  bottom=0.8in,
  footskip=0.35in
]{geometry}
% Font encoding - conditional for XeLaTeX/LuaLaTeX vs pdfLaTeX compatibility
\usepackage{iftex}
\ifxetex
  % XeLaTeX: use fontspec (loaded automatically by libertinus)
\else\ifluatex
  % LuaLaTeX: use fontspec (loaded automatically by libertinus)
\else
  % pdfLaTeX: use traditional encoding
  \usepackage[utf8]{inputenc}
  \usepackage[T1]{fontenc}
\fi\fi

% ============================================================================
% MATHEMATICS AND SYMBOLS (load before libertinus to avoid conflicts)
% ============================================================================
\usepackage{amsmath}
\usepackage{amssymb}

% Modern font: Libertinus (elegant, readable serif)
% Auto-detects engine and uses fontspec with LuaLaTeX
\usepackage{libertinus}
\usepackage{libertinust1math} % Math support for Libertinus

% Match monospaced font to Libertinus for consistent kerning/sizing
% Fall back safely if the mono font is not installed.
\ifxetex
  \IfFontExistsTF{Libertinus Mono}
    {\setmonofont{Libertinus Mono}[Scale=MatchLowercase]}
    {\IfFontExistsTF{LibertinusMono}
      {\setmonofont{LibertinusMono}[Scale=MatchLowercase]}
      {\IfFileExists{LibertinusMono-Regular.otf}
        {\setmonofont{LibertinusMono-Regular.otf}[Scale=MatchLowercase]}
        {\setmonofont{Latin Modern Mono}[Scale=MatchLowercase]}}}
\else\ifluatex
  \IfFontExistsTF{Libertinus Mono}
    {\setmonofont{Libertinus Mono}[Scale=MatchLowercase]}
    {\IfFontExistsTF{LibertinusMono}
      {\setmonofont{LibertinusMono}[Scale=MatchLowercase]}
      {\IfFileExists{LibertinusMono-Regular.otf}
        {\setmonofont{LibertinusMono-Regular.otf}[Scale=MatchLowercase]}
        {\setmonofont{Latin Modern Mono}[Scale=MatchLowercase]}}}
\fi\fi

% Typography enhancements
\usepackage{microtype}  % Improved typography and spacing
% Note: Ligatures are enabled to preserve proper dash rendering (-- and ---)
% If f-ligatures cause print issues, can selectively disable with microtype settings
\usepackage{setspace}   % Line spacing control
\setstretch{1.1}        % Line spacing optimized for US Trade size

% Paragraph formatting
\usepackage{parskip}    % Space between paragraphs instead of indentation
\setlength{\parskip}{0.5em}
\setlength{\parindent}{0pt}

% Language and quotation support (must come before biblatex)
\usepackage[english]{babel}
\usepackage{csquotes}

% ============================================================================
% GRAPHICS AND FIGURES
% ============================================================================
\usepackage{graphicx}
\usepackage{pdfpages}   % Include external PDF pages (for cover)
\usepackage{float}      % Better float control

% ============================================================================
% COLOR SYSTEM - Educational Semantic Palette
% ============================================================================
%
% ARCHITECTURE:
%   This color system uses a four-layer architecture for modularity and clarity:
%
%   Layer 1: PRIMITIVES - Raw RGB values (the actual color palette)
%            Named by visual appearance (slate-900, green-600, etc.)
%            Change these to rebrand the entire document
%
%   Layer 2: SEMANTICS - Educational content types
%            Named by pedagogical purpose (definition, example, key, caution)
%            Maps content types to primitive colors
%
%   Layer 3: COMPONENTS - Specific usage contexts
%            Named by component role (bg-*, border-*, text-*)
%            Makes code self-documenting and explicit
%
%   Layer 4: LEGACY - Backward compatibility aliases
%            Preserves old naming for gradual migration
%
% EDUCATIONAL CONTENT TYPES:
%   - definition: Formal definitions, theoretical concepts (Blue)
%   - example:    Concrete examples, demonstrations, applications (Green)
%   - key:        Important takeaways, essential concepts (Orange/Amber)
%   - caution:    Warnings, pitfalls, common mistakes (Red)
%   - note:       Supplementary info, asides, optional reading (Neutral)
%   - theorem:    Formal statements, proofs, logical arguments (Indigo)
%   - practice:   Exercises, problems, hands-on activities (Teal)
%
% VARIANT PATTERN:
%   Each semantic type has three consistent variants:
%   - [type]-dark:  For text and strong borders (high contrast, readable)
%   - [type]-base:  For medium emphasis (icons, accents, highlights)
%   - [type]-light: For backgrounds (subtle, doesn't overwhelm content)
%
% ============================================================================

\usepackage{xcolor}

% ============================================================================
% LAYER 1: PRIMITIVE COLOR PALETTE
% ============================================================================
% Raw color values named by visual appearance, not purpose.
% These form the foundation - modify these to rebrand the entire document.
% Organized by color family with numeric scale: 900 (darkest) → 100 (lightest)
% ============================================================================

% --- SLATE/BLUE FAMILY: Professional, Formal, Structural ---
\definecolor{slate-900}{RGB}{30,58,95}          % Deep slate blue - maximum depth/authority
\definecolor{slate-700}{RGB}{71,101,135}        % Medium slate - readable structure
\definecolor{slate-100}{RGB}{240,246,252}       % Ice blue - subtle professional backgrounds

% --- GREEN FAMILY: Natural, Practical, Concrete ---
\definecolor{green-900}{RGB}{46,125,50}         % Forest green - maximum depth/seriousness
\definecolor{green-600}{RGB}{67,160,71}         % Vibrant green - clear emphasis
\definecolor{green-100}{RGB}{232,245,233}       % Light green - fresh, approachable backgrounds

% --- AMBER/ORANGE FAMILY: Warm, Important, Attention ---
\definecolor{amber-900}{RGB}{230,124,0}         % Deep amber - strong attention signal
\definecolor{amber-600}{RGB}{251,140,0}         % Medium amber - warm emphasis
\definecolor{amber-100}{RGB}{255,243,224}       % Light amber - gentle highlight backgrounds

% --- RED FAMILY: Alert, Warning, Critical ---
\definecolor{red-900}{RGB}{198,40,40}           % Deep red - serious warnings/errors
\definecolor{red-600}{RGB}{229,57,53}           % Medium red - clear alert signal
\definecolor{red-100}{RGB}{255,235,238}         % Light red - soft warning backgrounds

% --- TEAL FAMILY: Modern, Technical, Alternative ---
\definecolor{teal-700}{RGB}{0,128,128}          % Deep teal - strong technical tone
\definecolor{teal-600}{RGB}{0,137,123}          % Professional teal - modern accent
\definecolor{teal-100}{RGB}{224,242,241}        % Light teal - fresh technical backgrounds

% --- INDIGO/PURPLE FAMILY: Abstract, Theoretical, Formal ---
\definecolor{indigo-700}{RGB}{63,81,181}        % Medium indigo - thoughtful, abstract tone
\definecolor{indigo-600}{RGB}{92,107,192}       % Lighter indigo - approachable formal tone
\definecolor{indigo-100}{RGB}{232,234,246}      % Light indigo - subtle theoretical backgrounds

% --- GRAY FAMILY: Neutral, Text, Structure ---
\definecolor{gray-900}{RGB}{30,32,34}           % Almost black - primary body text
\definecolor{gray-800}{RGB}{52,58,64}           % Very dark gray - strong headings
\definecolor{gray-700}{RGB}{73,80,87}           % Dark gray - secondary text
\definecolor{gray-600}{RGB}{95,100,105}         % Medium-dark gray - muted/de-emphasized text
\definecolor{gray-500}{RGB}{134,142,150}        % Medium gray - borders, dividers
\definecolor{gray-400}{RGB}{173,181,189}        % Medium-light gray - subtle dividers
\definecolor{gray-300}{RGB}{210,215,220}        % Light gray - soft borders
\definecolor{gray-200}{RGB}{233,236,239}        % Very light gray - hover states
\definecolor{gray-50}{RGB}{252,253,254}         % Near white - ultra subtle backgrounds
\definecolor{gray-100}{RGB}{247,248,250}        % Near white - subtle backgrounds

% --- WARM NEUTRALS ---
\definecolor{cream-100}{RGB}{252,250,246}       % Warm cream - inviting, friendly backgrounds


% ============================================================================
% LAYER 2: SEMANTIC COLOR MAPPINGS (Educational Content Types)
% ============================================================================
% Maps educational content types to primitive colors.
% USE THESE when creating educational content boxes and callouts.
% Each type follows the pattern: [type]-dark, [type]-base, [type]-light
% ============================================================================

% --- DEFINITION: Formal definitions, theoretical concepts, foundational knowledge ---
% Color: Blue (professional, formal, authoritative, structural)
% Use for: Formal term definitions, conceptual frameworks, theoretical foundations
\definecolor{definition-dark}{RGB}{30,58,95}    % slate-900 → text, strong borders
\definecolor{definition-base}{RGB}{71,101,135}  % slate-700 → medium borders, accents
\definecolor{definition-light}{RGB}{240,246,252}% slate-100 → box backgrounds

% --- EXAMPLE: Concrete examples, demonstrations, applications, case studies ---
% Color: Green (practical, real-world, natural, accessible)
% Use for: Code examples, real-world applications, demonstrations, case studies
\definecolor{example-dark}{RGB}{46,125,50}      % green-900 → text, strong borders
\definecolor{example-base}{RGB}{67,160,71}      % green-600 → medium borders, accents
\definecolor{example-light}{RGB}{232,245,233}   % green-100 → box backgrounds

% --- KEY: Important takeaways, essential concepts, memorable points ---
% Color: Orange/Amber (warm, attention-getting, important, energetic)
% Use for: Key takeaways, must-remember concepts, essential points, summaries
\definecolor{key-dark}{RGB}{230,124,0}          % amber-900 → text, strong borders
\definecolor{key-base}{RGB}{251,140,0}          % amber-600 → medium borders, accents
\definecolor{key-light}{RGB}{255,243,224}       % amber-100 → box backgrounds

% --- CAUTION: Warnings, pitfalls, common mistakes, things to avoid ---
% Color: Red (alert, warning, danger, critical attention)
% Use for: Common mistakes, pitfalls to avoid, deprecated patterns, errors
\definecolor{caution-dark}{RGB}{198,40,40}      % red-900 → text, strong borders
\definecolor{caution-base}{RGB}{229,57,53}      % red-600 → medium borders, accents
\definecolor{caution-light}{RGB}{255,235,238}   % red-100 → box backgrounds

% --- NOTE: General notes, asides, supplementary information, optional reading ---
% Color: Neutral/Gray (calm, supportive, non-intrusive, optional)
% Use for: Sidebar notes, historical context, tangential info, further reading
\definecolor{note-dark}{RGB}{73,80,87}          % gray-700 → text, strong borders
\definecolor{note-base}{RGB}{134,142,150}       % gray-500 → medium borders, accents
\definecolor{note-light}{RGB}{252,250,246}      % cream-100 → box backgrounds

% --- THEOREM: Mathematical statements, formal proofs, logical arguments ---
% Color: Indigo/Purple (abstract, theoretical, formal, rigorous)
% Use for: Theorems, lemmas, proofs, formal logical statements
\definecolor{theorem-dark}{RGB}{63,81,181}      % indigo-700 → text, strong borders
\definecolor{theorem-base}{RGB}{92,107,192}     % indigo-600 → medium borders, accents
\definecolor{theorem-light}{RGB}{232,234,246}   % indigo-100 → box backgrounds

% --- PRACTICE: Exercises, problems, hands-on activities, try-it-yourself ---
% Color: Teal (modern, technical, interactive, hands-on)
% Use for: Practice problems, exercises, coding challenges, interactive tasks
\definecolor{practice-dark}{RGB}{0,128,128}     % teal-700 → text, strong borders
\definecolor{practice-base}{RGB}{0,137,123}     % teal-600 → medium borders, accents
\definecolor{practice-light}{RGB}{224,242,241}  % teal-100 → box backgrounds


% ============================================================================
% LAYER 3: COMPONENT-SPECIFIC ALIASES
% ============================================================================
% Explicit naming for specific component usage (backgrounds, borders, text).
% These make code self-documenting and usage intentions clear.
% Use these when building custom components or tcolorbox definitions.
% ============================================================================

% --- General Text Colors (Body Text, Headings, Emphasis) ---
\definecolor{text-primary}{RGB}{30,32,34}       % gray-900 → main body text
\definecolor{text-secondary}{RGB}{73,80,87}     % gray-700 → secondary text, captions
\definecolor{text-muted}{RGB}{95,100,105}       % gray-600 → de-emphasized, metadata

% --- Background Colors for Content Boxes ---
\definecolor{bg-definition}{RGB}{240,246,252}   % definition-light → definition boxes
\definecolor{bg-example}{RGB}{232,245,233}      % example-light → example boxes
\definecolor{bg-key}{RGB}{255,243,224}          % key-light → key takeaway boxes
\definecolor{bg-caution}{RGB}{255,235,238}      % caution-light → warning boxes
\definecolor{bg-note}{RGB}{252,250,246}         % note-light → note/aside boxes
\definecolor{bg-theorem}{RGB}{232,234,246}      % theorem-light → theorem boxes
\definecolor{bg-practice}{RGB}{224,242,241}     % practice-light → practice boxes
\definecolor{bg-neutral}{RGB}{247,248,250}      % gray-100 → neutral backgrounds

% --- Border Colors for Content Boxes ---
\definecolor{border-definition}{RGB}{71,101,135}% definition-base → definition frames
\definecolor{border-example}{RGB}{67,160,71}    % example-base → example frames
\definecolor{border-key}{RGB}{251,140,0}        % key-base → key takeaway frames
\definecolor{border-caution}{RGB}{229,57,53}    % caution-base → warning frames
\definecolor{border-note}{RGB}{210,215,220}     % gray-300 → note frames
\definecolor{border-theorem}{RGB}{92,107,192}   % theorem-base → theorem frames
\definecolor{border-practice}{RGB}{0,137,123}   % practice-base → practice frames
\definecolor{border-neutral}{RGB}{210,215,220}  % gray-300 → subtle neutral borders

% --- Text Colors for Content Boxes ---
\definecolor{text-definition}{RGB}{30,58,95}    % definition-dark → definition text/titles
\definecolor{text-example}{RGB}{46,125,50}      % example-dark → example text/titles
\definecolor{text-key}{RGB}{230,124,0}          % key-dark → key takeaway text/titles
\definecolor{text-caution}{RGB}{198,40,40}      % caution-dark → warning text/titles
\definecolor{text-note}{RGB}{73,80,87}          % note-dark → note text/titles
\definecolor{text-theorem}{RGB}{63,81,181}      % theorem-dark → theorem text/titles
\definecolor{text-practice}{RGB}{0,128,128}     % practice-dark → practice text/titles

% --- Primary Structural Colors (Headings, Links, Main Theme) ---
\definecolor{primary}{RGB}{30,58,95}            % slate-900 → main brand/theme color
\definecolor{primary-light}{RGB}{71,101,135}    % slate-700 → lighter brand variant
\definecolor{accent}{RGB}{251,140,0}            % amber-600 → attention/emphasis accent


% ============================================================================
% LAYER 4: LEGACY COMPATIBILITY ALIASES
% ============================================================================
% Backward-compatible aliases preserving old color names.
% Allows gradual migration to new semantic system without breaking existing code.
% TODO: Gradually replace these throughout the document with semantic names.
% ============================================================================

\definecolor{agentblue}{RGB}{30,58,95}          % → slate-900 / primary
\definecolor{agentlightblue}{RGB}{240,246,252}  % → slate-100 / bg-definition
\definecolor{accentorange}{RGB}{191,97,35}      % Legacy (slightly warmer than amber-900)
\definecolor{highlightgray}{RGB}{252,250,246}   % → cream-100 / bg-note
\definecolor{bordergray}{RGB}{210,215,220}      % → gray-300 / border-neutral
\definecolor{darkgray}{RGB}{95,100,105}         % → gray-600 / text-muted

% Old semantic names from previous color system
\definecolor{primary-slate}{RGB}{30,58,95}      % → slate-900
\definecolor{accent-amber}{RGB}{191,97,35}      % Legacy warm amber
\definecolor{secondary-sage}{RGB}{82,121,111}   % Legacy sage green (no longer in palette)
\definecolor{bg-ice}{RGB}{240,246,252}          % → slate-100
\definecolor{bg-cream}{RGB}{252,250,246}        % → cream-100
\definecolor{bg-amber-light}{RGB}{254,245,237}  % Legacy (slightly different than amber-100)
\definecolor{bg-gray-cool}{RGB}{247,248,250}    % → gray-100
\definecolor{border-slate}{RGB}{71,101,135}     % → slate-700
\definecolor{border-sage}{RGB}{118,145,137}     % Legacy sage border (no longer in palette)

% ============================================================================
% TABLES - Modern styling
% ============================================================================
\usepackage{booktabs}    % Professional quality tables
\usepackage{longtable}   % Tables spanning multiple pages
\usepackage{array}
\usepackage{multirow}
\usepackage{tabularx}    % Better column sizing

% Table row coloring
\usepackage{colortbl}

% Landscape pages and rotated figures
\usepackage{pdflscape}
\usepackage{rotating}  % For sidewaysfigure
\renewcommand{\arraystretch}{1.3}  % Better row spacing

% ============================================================================
% BOXES AND VISUAL ELEMENTS
% ============================================================================
\usepackage{tcolorbox}
\tcbuselibrary{breakable,skins,theorems,listings}

% Listings package for code formatting
\usepackage{listings}
\lstset{
  basicstyle=\small\ttfamily\color{gray-900},
  breaklines=true,
  breakatwhitespace=false,
  columns=flexible,
  keepspaces=true,
  showstringspaces=false,
  tabsize=2,
  xleftmargin=0pt,
  framexleftmargin=0pt,
  % Syntax highlighting colors (from chapter 07)
  keywordstyle=\color{slate-700}\bfseries,
  stringstyle=\color{green-900},
  commentstyle=\color{gray-600}\itshape,
  emphstyle=\color{amber-900},
}

% Importance scaling for tcolorbox environments (book-wide)
% Usage in any box: [importance=low|medium|high]
\tcbset{
  importance/.is choice,
  importance/high/.style={
    boxrule=2pt,
    top=10pt,bottom=10pt,left=12pt,right=12pt
  },
  importance/medium/.style={},
  importance/low/.style={
    boxrule=0.8pt,
    opacityback=0.92,
    opacityframe=0.6,
    boxed title style={opacityback=0.85},
    top=6pt,bottom=6pt,left=8pt,right=8pt
  },
  importance/.default=medium
}

% Definition box style - Most prominent, formal definitions
\newtcolorbox{definitionbox}[1][]{
  enhanced,
  colback=bg-definition,
  colframe=border-definition,
  fonttitle=\bfseries\large,
  coltitle=white,
  boxrule=1.5pt,
  arc=3pt,
  left=8pt,
  right=8pt,
  top=8pt,
  bottom=8pt,
  breakable,
  attach boxed title to top left={yshift=-2mm, xshift=4mm},
  boxed title style={
    colback=definition-dark,
    colframe=definition-dark,
    arc=2pt,
    boxrule=0pt
  },
  #1
}

% Example box style - Concrete examples and demonstrations
\newtcolorbox{examplebox}[1][]{
  enhanced,
  colback=bg-example,
  colframe=border-example,
  fonttitle=\bfseries\large,
  coltitle=white,
  boxrule=1.5pt,
  arc=3pt,
  left=8pt,
  right=8pt,
  top=8pt,
  bottom=8pt,
  breakable,
  borderline west={3pt}{0pt}{example-base},
  attach boxed title to top left={yshift=-2mm, xshift=4mm},
  boxed title style={
    colback=example-dark,
    colframe=example-dark,
    arc=2pt,
    boxrule=0pt
  },
  #1
}

% Highlight box style - Warm, inviting context and notes
\newtcolorbox{highlightbox}[1][]{
  enhanced,
  colback=bg-note,
  colframe=border-note,
  fonttitle=\bfseries,
  coltitle=note-dark,
  boxrule=1pt,
  arc=2.5pt,
  left=10pt,
  right=10pt,
  top=8pt,
  bottom=8pt,
  breakable,
  #1
}

% Key takeaway box - Stands out for important points
\newtcolorbox{keybox}[1][]{
  enhanced,
  colback=bg-key,
  colframe=key-base,
  fonttitle=\bfseries\large,
  coltitle=white,
  boxrule=2pt,
  arc=3pt,
  left=8pt,
  right=8pt,
  top=8pt,
  bottom=8pt,
  breakable,
  borderline west={3pt}{0pt}{key-base},
  attach boxed title to top left={yshift=-2mm, xshift=4mm},
  boxed title style={
    colback=key-base,
    colframe=key-base,
    arc=2pt,
    boxrule=0pt
  },
  #1
}

% Question box style - Distinct professional style for evaluation questions
\newtcolorbox{questionbox}[1][]{
  enhanced,
  colback=gray-100,
  colframe=gray-600,
  fonttitle=\bfseries,
  coltitle=white,
  boxrule=1.5pt,
  arc=2pt,
  left=10pt,
  right=10pt,
  top=8pt,
  bottom=8pt,
  breakable,
  borderline west={4pt}{0pt}{slate-700},
  attach boxed title to top left={yshift=-2mm, xshift=4mm},
  boxed title style={
    colback=slate-700,
    colframe=slate-700,
    arc=2pt,
    boxrule=0pt
  },
  #1
}

% Theorem box style - For formal logical statements and proofs
\newtcolorbox{theorembox}[1][]{
  enhanced,
  colback=bg-theorem,
  colframe=border-theorem,
  fonttitle=\bfseries\large,
  coltitle=white,
  boxrule=1.5pt,
  arc=3pt,
  left=8pt,
  right=8pt,
  top=8pt,
  bottom=8pt,
  breakable,
  attach boxed title to top left={yshift=-2mm, xshift=4mm},
  boxed title style={
    colback=theorem-dark,
    colframe=theorem-dark,
    arc=2pt,
    boxrule=0pt
  },
  #1
}

% Caution box style - For warnings and critical alerts
\newtcolorbox{cautionbox}[1][]{
  enhanced,
  colback=bg-caution,
  colframe=border-caution,
  fonttitle=\bfseries\large,
  coltitle=white,
  boxrule=1.5pt,
  arc=3pt,
  left=8pt,
  right=8pt,
  top=8pt,
  bottom=8pt,
  breakable,
  borderline west={4pt}{0pt}{caution-dark},
  attach boxed title to top left={yshift=-2mm, xshift=4mm},
  boxed title style={
    colback=caution-dark,
    colframe=caution-dark,
    arc=2pt,
    boxrule=0pt
  },
  #1
}

% Technical detail box - For optional deeper mechanics (algorithms, protocols, internals)
% Uses indigo/theorem colors to distinguish from practice boxes
\newtcolorbox{technicalbox}[1][]{
  enhanced,
  colback=bg-theorem,
  colframe=border-theorem,
  fonttitle=\bfseries\large,
  coltitle=white,
  boxrule=1.5pt,
  arc=3pt,
  left=8pt,
  right=8pt,
  top=8pt,
  bottom=8pt,
  breakable,
  borderline west={4pt}{0pt}{theorem-dark},
  attach boxed title to top left={yshift=-2mm, xshift=4mm},
  boxed title style={
    colback=theorem-dark,
    colframe=theorem-dark,
    arc=2pt,
    boxrule=0pt
  },
  #1
}

% Practice box style - Exercises and hands-on activities
\newtcolorbox{practicebox}[1][]{
  enhanced,
  colback=bg-practice,
  colframe=border-practice,
  fonttitle=\bfseries\large,
  coltitle=white,
  boxrule=1.5pt,
  arc=3pt,
  left=8pt,
  right=8pt,
  top=8pt,
  bottom=8pt,
  breakable,
  borderline west={4pt}{0pt}{practice-dark},
  attach boxed title to top left={yshift=-2mm, xshift=4mm},
  boxed title style={
    colback=practice-dark,
    colframe=practice-dark,
    arc=2pt,
    boxrule=0pt
  },
  #1
}

% Listing box style - For code examples, API signatures, and technical content
\newtcblisting{listingbox}[1][]{
  enhanced,
  colback=gray-100,
  colframe=gray-400,
  fonttitle=\bfseries\sffamily,
  coltitle=gray-800,
  boxrule=1pt,
  arc=2pt,
  left=8pt,
  right=8pt,
  top=6pt,
  bottom=6pt,
  breakable,
  listing only,
  listing options={
    basicstyle=\small\ttfamily\color{gray-900},
    breaklines=true,
    columns=flexible,
    keepspaces=true,
  },
  attach boxed title to top left={yshift=-2mm, xshift=4mm},
  boxed title style={
    colback=gray-500,
    colframe=gray-500,
    arc=2pt,
    boxrule=0pt
  },
  #1
}

% ============================================================================
% LISTS - Better spacing and appearance
% ============================================================================
\usepackage{enumitem}
\setlist{
  itemsep=0.4em,
  parsep=0.2em,
  topsep=0.5em,
  leftmargin=1.5em
}

% Custom list for evaluation questions
\newlist{evallist}{enumerate}{1}
\setlist[evallist]{
  label=\textbf{Q\arabic*.},
  leftmargin=2em,
  itemsep=0.5em,
  labelsep=0.5em
}

% ============================================================================
% TIKZ for diagrams
% ============================================================================
\usepackage{tikz}
\usetikzlibrary{
  positioning,
  arrows.meta,
  shapes.geometric,
  shapes.misc,
  calc,
  decorations.pathreplacing,
  backgrounds,
  fit,
  shadows.blur
}

% Note: Do NOT override TikZ node fonts globally here.
% The figures use explicit font sizes (\tiny, \scriptsize, etc.) that must be preserved.
% Global font overrides break figures like fig-incident-report-legal.tex.

% ============================================================================
% SECTION STYLING
% ============================================================================
\usepackage{titlesec}
\usepackage{titletoc}

% Section formatting
\titleformat{\section}
  {\Large\bfseries\color{primary}}
  {\thesection}{1em}{}
  [\vspace{0.3em}{\color{primary}\titlerule[1.5pt]}\vspace{0.5em}]

% Subsection formatting
\titleformat{\subsection}
  {\large\bfseries\color{primary}}
  {\thesubsection}{1em}{}
  [\vspace{-0.3em}]

% Subsubsection formatting
\titleformat{\subsubsection}
  {\normalsize\bfseries\color{text-secondary}}
  {\thesubsubsection}{1em}{}

% Paragraph formatting (for \paragraph)
\titleformat{\paragraph}[runin]
  {\normalsize\bfseries\color{primary}}
  {}{0em}{}[.]
\titlespacing*{\paragraph}{0pt}{1em}{0.5em}

% ============================================================================
% TABLE OF CONTENTS STYLING
% ============================================================================
% Customize ToC title
\renewcommand{\contentsname}{%
  {\color{primary}\Large\bfseries Contents}%
}

% Chapter entries in ToC (for book class)
\titlecontents{chapter}
  [0em]                            % left margin
  {\vspace{1em}\large\bfseries}    % above code
  {\color{primary}\contentslabel{2em}}  % numbered entry format
  {\color{primary}}                % numberless entry format (Preface, etc.)
  {\titlerule*[0.75pc]{.}\contentspage}  % filler and page

% Section entries in ToC (with chapter.section numbering like 2.12)
\titlecontents{section}
  [2em]                            % left margin (indented from chapter)
  {\vspace{0.3em}}                 % above code
  {\color{primary}\contentslabel{2.5em}}  % numbered entry format (wider for X.XX)
  {}                               % numberless entry format
  {\titlerule*[0.75pc]{.}\contentspage}  % filler and page

% Subsection entries in ToC (with chapter.section.subsection numbering like 2.12.3)
\titlecontents{subsection}
  [4.5em]                          % left margin (indented from section)
  {\vspace{0.1em}}                 % above code
  {\contentslabel{3.2em}}          % numbered entry format (wider for X.XX.X)
  {}                               % numberless entry format
  {\titlerule*[0.75pc]{.}\contentspage}  % filler and page

% ============================================================================
% CAPTIONS - Cleaner appearance
% ============================================================================
\usepackage[
  font={small},
  labelfont={bf,color=primary},
  format=plain,
  justification=justified,
  skip=8pt
]{caption}

% ============================================================================
% HYPERLINKS AND PDF PROPERTIES
% ============================================================================
\usepackage{hyperref}
\hypersetup{
  colorlinks=true,
  linkcolor=primary,
  citecolor=accent,
  urlcolor=primary,
  bookmarksnumbered=true,
  pdfborder={0 0 0}
}

% ============================================================================
% GLOSSARY SUPPORT
% ============================================================================
% The glossaries package:
%   - Links terms in-text to their glossary entry (via hyperref)
%   - Shows hyperlinked page numbers in the glossary ("where this term occurs")
%
% Note: glossaries recommends loading hyperref before glossaries.
\usepackage[
  toc,                    % Add glossary to table of contents
  section=chapter,        % Format glossary as chapter-level
  nopostdot,              % No period after descriptions
  automake=immediate      % Auto-run makeglossaries via shell escape
]{glossaries}
\usepackage{glossary-longbooktabs} % Beautiful longtable + booktabs glossary styles

% Create the glossary
\makeglossaries

% --- Glossary formatting (readable + navigable) ---
% Wrap long term names, give the description room, and make page links obvious.
\renewcommand*{\glsnamefont}[1]{\textbf{\color{primary}#1}}
\renewcommand*{\glsnumberformat}[1]{\textcolor{text-secondary}{\glshypernumber{#1}}}

\newglossarystyle{lawfin-long3col}{%
  \setglossarystyle{longragged3col-booktabs}%
  \renewenvironment{theglossary}%
    {%
      \begingroup
      \setlength{\LTleft}{0pt}%
      \setlength{\LTright}{0pt}%
      \setlength{\LTpre}{-0.5em}%  % Reduce top margin
      \setlength{\LTpost}{-0.5em}% % Reduce bottom margin
      \setlength{\tabcolsep}{5pt}% % Tighter column spacing
      \renewcommand{\arraystretch}{1.15}% % Slightly tighter row spacing
      \small% Reduce glossary font size
      \begin{longtable}{%
        @{}%
        >{\raggedright\arraybackslash}p{0.22\textwidth}%
        >{\arraybackslash}p{0.65\textwidth}%  % Justified text for descriptions
        >{\raggedright\arraybackslash}p{0.11\textwidth}%
        @{}%
      }%
    }%
    {%
      \end{longtable}%
      \endgroup
    }%
  \renewcommand{\glossentry}[2]{%
    \glsentryitem{##1}\glstarget{##1}{\glossentryname{##1}} &
    \glossentrydesc{##1} &
    ##2\tabularnewline[0.35em]%
  }%
  \renewcommand{\subglossentry}[3]{%
    &%
    \glssubentryitem{##2}\glstarget{##2}{\strut}\glossentrydesc{##2} &
    ##3\tabularnewline[0.35em]%
  }%
  \renewcommand*{\glossaryheader}{%
    \toprule
    \bfseries\color{primary}\entryname &
    \bfseries\color{primary}\descriptionname &
    \bfseries\color{primary}\pagelistname\tabularnewline
    \midrule\endhead
    \bottomrule\endfoot
  }%
}
\setglossarystyle{lawfin-long3col}

% Better cross-referencing
\usepackage{cleveref}
\crefname{section}{Section}{Sections}
\crefname{figure}{Figure}{Figures}
\crefname{table}{Table}{Tables}
\crefname{chapter}{Chapter}{Chapters}

% ============================================================================
% BIBLIOGRAPHY
% ============================================================================
\usepackage[
  backend=biber,
  style=authoryear,
  maxcitenames=2,
  maxbibnames=99,
  uniquename=false,
  uniquelist=false,
  sorting=nyt,
  dashed=false
]{biblatex}

% Citation formatting
\DeclareFieldFormat{citetitle}{\mkbibquote{#1}}
\DeclareFieldFormat[article]{citetitle}{#1}
\DeclareFieldFormat[inproceedings]{citetitle}{#1}
\DeclareFieldFormat[book]{citetitle}{\mkbibemph{#1}}
\DeclareFieldFormat[techreport]{citetitle}{#1}
\DeclareFieldFormat[misc]{citetitle}{#1}

% Bibliography formatting - smaller text and tighter spacing
\renewcommand*{\bibfont}{\small}
\setlength{\bibitemsep}{0.5em}
\setlength{\bibhang}{1.5em}

% ============================================================================
% CUSTOM COMMANDS
% ============================================================================

% Highlighted text for key terms
\newcommand{\keyterm}[1]{\textbf{\color{primary}#1}}

% Three-level hierarchy terms
\newcommand{\levelone}[1]{\textbf{#1}}
\newcommand{\leveltwo}[1]{\textbf{#1}}
\newcommand{\levelthree}[1]{\textbf{#1}}
