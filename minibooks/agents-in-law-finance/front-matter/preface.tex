% ============================================================================
% PREFACE
% ============================================================================

\chapter*{Preface}
\addcontentsline{toc}{chapter}{Preface}

\section*{Why This Book}

Everyone is talking about agents. Vendors announce ``agentic AI'' capabilities, researchers develop ``agent benchmarks,'' and practitioners deploy ``AI agents'' in production---yet when you ask what makes something an agent, the answers vary wildly. This ambiguity is not just an academic problem. It creates confusion in vendor evaluations, architectural decisions, and regulatory discussions. In law and finance, where precision matters and mistakes carry consequences, we need clearer thinking.

This book offers that clarity. We synthesize seven decades of scholarship---from philosophy and psychology through economics and computer science---into a unified framework for understanding, designing, and governing AI agents. The framework is practical enough for immediate application yet grounded in foundations that will remain relevant as the technology evolves.

\section*{A Book Written with Agents}

We should acknowledge from the outset: this book about AI agents was written with substantial assistance from AI agents.

Our previous textbook took three years to complete, even with professional support from Cambridge University Press. This book took under three months from conception to completion. The difference is not that we worked harder or faster. The difference is that we worked \textit{with} agents.

Throughout the drafting, editing, and production process, we used Claude Code, Codex CLI, Google Gemini CLI, and OpenCode to assist with research synthesis, bibliographic work, figure generation, LaTeX formatting, and iterative revision. These tools helped us locate and integrate sources, maintain consistent terminology and cross-references, and identify gaps in our arguments.

We do not claim that agents wrote this book. The intellectual framework, the selection and interpretation of sources, the judgments about what matters and why---these remain human contributions. But the production process has been fundamentally transformed. We see this acceleration as direct evidence of the capabilities we analyze in the chapters that follow.

If you find value in this book, you are experiencing what well-governed agentic collaboration can produce. If you find errors, you are experiencing why the governance frameworks in Chapter~3 matter.

\section*{Who This Book Is For}

We wrote this book for professionals who need to work with AI agents thoughtfully and responsibly:

\begin{itemize}
  \item \textbf{Legal practitioners} evaluating AI tools for research, document review, and client service;
  \item \textbf{Financial professionals} deploying automated systems for analysis, trading, and compliance;
  \item \textbf{Technology leaders} making architectural and vendor selection decisions;
  \item \textbf{Risk and compliance officers} developing governance frameworks for AI adoption;
  \item \textbf{Regulators and policymakers} seeking to understand the systems they oversee; and
  \item \textbf{Researchers} building the next generation of agentic systems.
\end{itemize}

We assume technical literacy but not specialized expertise. Legal professionals need not be engineers; technologists need not be lawyers. The goal is mutual understanding across disciplines.

\section*{A Note on Scope}

This book is extracted from a forthcoming textbook, \textit{Artificial Intelligence for Law and Finance}, and revised to stand alone. The full work covers foundational topics---machine learning, natural language processing, knowledge representation---alongside domain applications in legal research, contract analysis, regulatory compliance, and financial modeling. We selected these chapters on agents because they form a self-contained arc: what agents are, how to build them, and how to govern them responsibly.

You will notice that we focus on concepts rather than code. This is intentional. Today's hot framework is tomorrow's legacy system; the model everyone uses this year may be obsolete by the next. We want this material to remain useful whether you are working with LangChain or AutoGen, Python or TypeScript, OpenAI or Anthropic or whatever comes next. The architectural questions---how agents perceive, decide, act, and terminate---do not change when you swap out the underlying technology.

For readers who want hands-on implementation, the forthcoming textbook includes a companion website with working source code, interactive exercises, and practical tutorials. That material will satisfy the appetite for technical depth while this book provides the conceptual foundation that makes implementation choices intelligible.

The field evolves rapidly. We have included citations through December 2025 and will update this material as developments warrant.

\section*{Acknowledgments}

This work synthesizes insights from decades of scholarship across multiple disciplines. We are grateful to the researchers whose foundational work made this synthesis possible, from Anscombe and Bratman in philosophy, through Bandura in psychology, to Russell, Norvig, and the contemporary LLM agent community in computer science.

\vspace{1em}

\noindent\textit{Michael J Bommarito II, Daniel Martin Katz, and Jillian Bommarito}\\
\noindent\textit{December 2025}

