% ============================================================================
% PREFACE
% ============================================================================

\chapter*{Preface}
\addcontentsline{toc}{chapter}{Preface}

\section*{Why This Book}

The term ``agent'' has become ubiquitous in discussions of artificial intelligence, yet its meaning remains remarkably fluid. Vendors announce ``agentic AI'' capabilities, researchers develop ``agent benchmarks,'' and practitioners deploy ``AI agents'' in production---often with fundamentally different conceptions of what these terms entail.

This ambiguity creates real problems. How do you evaluate vendor claims about agentic systems? What architectural decisions matter when building agents for professional contexts? How do existing regulatory frameworks apply to systems that act with increasing autonomy? These questions demand rigorous answers, particularly in law and finance where mistakes carry significant consequences.

This mini-book provides those answers. Drawing on seven decades of scholarship across philosophy, psychology, law, economics, and computer science, we develop a unified framework for understanding, designing, and governing AI agents. The framework is practical enough for immediate application yet grounded in theoretical foundations that will remain relevant as the technology evolves.

\section*{A Book Written with Agents}

We should acknowledge from the outset: this book about AI agents was written with substantial assistance from AI agents.

Our previous textbook took three years to complete---from initial drafts through copy editing and typesetting---even with the professional support of Cambridge University Press. This book, by contrast, took under three months from conception to completion. The difference is not that we worked harder or faster. The difference is that we worked \textit{with} agents.

Throughout the drafting, editing, and production process, we used Claude Code, Codex CLI, Google Gemini CLI, and OpenCode to assist with research synthesis, bibliographic work, figure generation, LaTeX formatting, and iterative revision. These tools helped us locate and integrate sources across seven decades of scholarship, maintain consistent terminology and cross-references, and identify gaps in our arguments. They suggested alternative phrasings, caught inconsistencies, and accelerated the mechanical aspects of academic writing.

We do not claim that agents wrote this book. The intellectual framework, the selection and interpretation of sources, the judgments about what matters and why---these remain human contributions. But the production process that transforms ideas into a finished manuscript has been fundamentally transformed. We see this acceleration not as a curiosity to mention in passing, but as direct evidence of the capabilities we analyze in the chapters that follow.

If you find value in this book, you are experiencing what well-governed agentic collaboration can produce. If you find errors, you are experiencing why the governance frameworks in Part III matter.

\section*{What You Will Learn}

This book is organized into three parts, each addressing a fundamental question:

\textbf{Part I: What is an Agent?} We establish a rigorous conceptual foundation by synthesizing definitions from multiple disciplines into a coherent three-level hierarchy. You will learn to distinguish genuine agents from sophisticated tools, evaluate systems using a six-property framework (Goals, Perception, Action, Iteration, Adaptation, Termination), and understand the historical evolution from philosophical accounts of human agency to modern LLM-powered systems.

\textbf{Part II: How to Design an Agent?} We translate conceptual understanding into architectural decisions through ten fundamental questions that every agent designer must answer. From trigger mechanisms and intent understanding through memory systems and planning patterns to escalation and multi-agent coordination, this part provides actionable guidance grounded in distributed cognition theory and contemporary agent research.

\textbf{Part III: How to Govern an Agent?} We address the governance imperative for agents operating in regulated domains. You will learn to calibrate oversight to agent autonomy, navigate the five-layer regulatory stack (from foundational law through voluntary frameworks), implement practical controls (audit logging, human oversight, incident response), and establish organizational accountability structures.

\section*{Who This Book Is For}

We wrote this book for professionals who need to work with AI agents thoughtfully and responsibly:

\begin{itemize}
  \item \textbf{Legal practitioners} evaluating AI tools for research, document review, and client service
  \item \textbf{Financial professionals} deploying automated systems for analysis, trading, and compliance
  \item \textbf{Technology leaders} making architectural and vendor selection decisions
  \item \textbf{Risk and compliance officers} developing governance frameworks for AI adoption
  \item \textbf{Regulators and policymakers} seeking to understand the systems they oversee
  \item \textbf{Researchers} building the next generation of agentic systems
\end{itemize}

We assume technical literacy but not specialized expertise. Legal professionals need not be engineers; technologists need not be lawyers. The goal is mutual understanding across disciplines.

\section*{How to Read This Book}

Each part can be read independently, though they build upon each other:

\textbf{For conceptual clarity}, start with Part I. The six-property framework and three-level hierarchy provide vocabulary and evaluation tools useful throughout.

\textbf{For architectural guidance}, focus on Part II. The ten questions structure provides a systematic approach to agent design regardless of implementation technology.

\textbf{For governance frameworks}, emphasize Part III. The dimensional calibration logic and five-layer stack apply to any agent deployment in regulated contexts.

\textbf{For comprehensive understanding}, read all three parts in sequence. The conceptual foundations of Part I inform the design decisions of Part II, which in turn shape the governance requirements of Part III.

\section*{A Note on Scope}

This mini-book is extracted from a larger work, \textit{Artificial Intelligence for Law and Finance}, which covers additional topics including knowledge graphs, machine learning foundations, and domain-specific applications. We have selected and refined these three chapters because they form a self-contained treatment of the agent problem---from definition through design to governance.

The field continues to evolve rapidly. We have included citations through December 2025 and will update this material as significant developments warrant. The current version reflects the state of agent architectures, regulatory frameworks, and best practices as of the publication date.

\section*{Acknowledgments}

This work synthesizes insights from decades of scholarship across multiple disciplines. We are grateful to the researchers whose foundational work---from Anscombe and Bratman in philosophy, through Bandura in psychology, to Russell, Norvig, and the contemporary LLM agent community in computer science---made this synthesis possible.

\vspace{1em}

\noindent\textit{Michael J Bommarito II, Daniel Martin Katz, and Jillian Bommarito}\\
\noindent\textit{December 2025}
