% =============================================================================
% Further Learning — Tool Use
% Purpose: Annotated bibliography for tool use and function calling
% Label: sec:llmC3-further
% =============================================================================

\section{Further Learning}
\label{sec:llmC3-further}

This section provides an annotated guide to resources for readers who wish to deepen their understanding of tool use, function calling, and AI system integration.

\subsection{Tool Use Foundations}
\label{sec:llmC3-further-foundations}

\paragraph{Toolformer.}
\fullcite{schick2023toolformer}

The foundational paper demonstrating that language models can learn to use tools through self-supervised training. Establishes the conceptual framework for tool-augmented language models.

\paragraph{ReAct Pattern.}
\fullcite{yao2022react}

Introduces the Reasoning + Acting pattern where models interleave reasoning traces with tool calls. Essential for understanding how tool use integrates with structured reasoning.

\paragraph{Function Calling Documentation.}
OpenAI and Anthropic both provide comprehensive documentation on function calling APIs. These are essential references for implementation:
\begin{itemize}
  \item OpenAI: \url{https://platform.openai.com/docs/guides/function-calling}
  \item Anthropic: \url{https://docs.anthropic.com/claude/docs/tool-use}
\end{itemize}

\subsection{Security and Governance}
\label{sec:llmC3-further-security}

\paragraph{OWASP Top 10 for LLMs.}
\fullcite{owasp2025llm}

The OWASP security risk taxonomy for LLM applications, including tool-related vulnerabilities like insecure plugin design and excessive agency. Essential reference for security assessments.

\paragraph{Prompt Injection Attacks.}
\fullcite{liu2023prompt}

Analyzes prompt injection vulnerabilities including attacks that manipulate tool calls. Understanding these attacks is essential for designing secure tool-using systems.

\paragraph{Indirect Prompt Injection.}
\fullcite{greshake2023youve}

Demonstrates how malicious content in external data sources can compromise AI applications. Particularly relevant for tools that retrieve and process external content.

\subsection{Implementation Frameworks}
\label{sec:llmC3-further-frameworks}

\begin{itemize}
  \item \textbf{LangChain Tools:} Comprehensive framework for building tool-using agents with pre-built integrations. \url{https://python.langchain.com/docs/modules/agents/tools}

  \item \textbf{Semantic Kernel:} Microsoft's framework for AI orchestration with function calling support. \url{https://github.com/microsoft/semantic-kernel}

  \item \textbf{OpenAI Assistants API:} Production API for building tool-using assistants with built-in code interpreter and file handling. \url{https://platform.openai.com/docs/assistants}
\end{itemize}

\subsection{Enterprise Integration}
\label{sec:llmC3-further-enterprise}

\paragraph{API Design.}
The OpenAPI specification is the standard for describing tool interfaces. Understanding OpenAPI helps in designing tools that integrate cleanly with LLM function calling. \url{https://www.openapis.org}

\paragraph{Authentication and Authorization.}
OAuth 2.0 and related standards govern secure API access. Tools that call external services must implement proper authentication flows. IETF OAuth specifications provide the foundation.

\paragraph{Rate Limiting and Circuit Breakers.}
Production systems require rate limiting (to protect external services) and circuit breakers (to handle service failures gracefully). Patterns from resilience engineering apply directly to tool-using AI.

\subsection{Legal and Financial Domain Tools}
\label{sec:llmC3-further-domain}

\paragraph{Legal Research APIs.}
Services like Westlaw Edge, LexisNexis, and CourtListener provide APIs for legal research. Understanding their capabilities and limitations helps in designing legal AI tools.

\paragraph{Financial Data APIs.}
Bloomberg, Refinitiv, and SEC EDGAR provide financial data APIs. Regulatory data from FINRA and CFTC is available through official APIs.

\paragraph{Regulatory Filing.}
EDGAR, SEDAR+, and other regulatory portals provide submission APIs. Integrating with these requires understanding both technical specifications and regulatory requirements.

\subsection{Staying Current}
\label{sec:llmC3-further-current}

Tool use capabilities evolve rapidly. Resources for staying current:

\begin{itemize}
  \item \textbf{Vendor documentation:} OpenAI, Anthropic, and Google regularly update their function calling capabilities
  \item \textbf{arXiv cs.CL and cs.AI:} New research on tool-augmented language models appears as preprints
  \item \textbf{Security advisories:} Follow OWASP and vendor security bulletins for emerging vulnerabilities
\end{itemize}

