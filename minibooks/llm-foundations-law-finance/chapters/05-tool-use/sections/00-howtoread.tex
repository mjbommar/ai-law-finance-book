% =============================================================================
% How to Read — Tool Use
% Purpose: Audience paths; scope
% Label: sec:llmC3-howtoread
% =============================================================================

\section*{How to Read This Chapter}
\addcontentsline{toc}{section}{How to Read This Chapter}

This chapter extends LLM capabilities from passive analysis to active interaction with external systems. Building on the structured output foundations from Chapter~4, we examine how LLMs can invoke tools---calculators, databases, APIs, and enterprise systems---while maintaining the governance and auditability that professional practice demands.

\subsection*{Prerequisites}

This chapter assumes familiarity with:
\begin{itemize}
  \item \textbf{Structured outputs} (\Cref{sec:llmC-structured}): Schema design and constrained generation
  \item \textbf{Evidence records} (\Cref{sec:llmC-evidence}): Provenance tracking and audit trails
  \item \textbf{Reasoning patterns} (\Cref{sec:llmB-reason}): ReAct and tool-augmented reasoning
\end{itemize}

If these concepts are unfamiliar, review Chapters~3--4 before proceeding.

\subsection*{Reading Paths}

\begin{highlightbox}[colback=bg-definition, colframe=definition-base, title={Path 1: Quick Implementation (25--35 minutes)}]
If you need to add tool calling to an existing system:
\begin{itemize}
  \item \textbf{\Cref{sec:llmC-tools}}: Function calling mechanics, tool design, and parameter handling
  \item \textbf{\Cref{sec:llmC-pitfalls}}: Common failure modes and how to avoid them
\end{itemize}
\end{highlightbox}

\begin{highlightbox}[colback=bg-example, colframe=example-base, title={Path 2: Full Technical Understanding (45--60 minutes)}]
If you need to understand governance, security, and enterprise integration:
\begin{itemize}
  \item Read all sections in order, starting with the introduction
  \item Pay special attention to governance metadata and audit logging
  \item Review security considerations including OWASP Top 10 for LLMs
\end{itemize}
\end{highlightbox}

\begin{highlightbox}[colback=bg-note, colframe=border-note, title={Path 3: Security and Compliance Focus (30--40 minutes)}]
If your primary concern is secure, compliant tool integration:
\begin{itemize}
  \item \textbf{\Cref{sec:llmC-tools}}: Focus on permissions, sandboxing, and governance metadata sections
  \item \textbf{\Cref{sec:llmC-pitfalls}}: Security vulnerabilities and mitigation strategies
  \item \textbf{\Cref{sec:llmC3-synthesis}}: Integration with audit and compliance frameworks
\end{itemize}
\end{highlightbox}

\begin{keybox}[title={Key Objectives}]
By the end of this chapter, you will be able to:
\begin{itemize}
  \item Design \textbf{tool interfaces} with clear contracts, pre/postconditions, and error handling
  \item Implement \textbf{function calling} with proper parameter validation and response handling
  \item Apply \textbf{governance metadata} to log who, what, why, and under what context
  \item Avoid \textbf{common pitfalls} including injection attacks, permission escalation, and error cascades
\end{itemize}
\end{keybox}

\subsection*{What This Chapter Does Not Cover}

\begin{itemize}
  \item \textbf{Multimodal inputs}: PDFs, tables, and audio processing are addressed in Chapter~6
  \item \textbf{Prompt optimization}: Systematic improvement of tool-calling prompts appears in Chapter~7
  \item \textbf{Agent architectures}: Autonomous planning with tool chains is addressed in \textit{Agentic AI in Law and Finance}
\end{itemize}

