% =============================================================================
% Further Learning — Structured Outputs
% Purpose: Annotated bibliography for structured generation and retrieval
% Label: sec:llmC2-further
% =============================================================================

\section{Further Learning}
\label{sec:llmC2-further}

This section provides an annotated guide to resources for readers who wish to deepen their understanding of structured generation, production retrieval, and evidence-based systems.

\subsection{Structured Output Generation}
\label{sec:llmC2-further-structured}

\paragraph{OpenAI Structured Outputs.}
\fullcite{openai2024structured}

OpenAI's documentation on structured outputs, including JSON mode and schema-constrained generation. Essential for understanding production implementations of constrained decoding.

\paragraph{Outlines Library.}
The Outlines library provides constrained generation for open-source models, implementing JSON schema enforcement at the logit level. Demonstrates how constrained decoding works at a technical level. \url{https://github.com/outlines-dev/outlines}

\paragraph{Instructor Library.}
A Python library that simplifies structured output extraction using Pydantic models with LLM APIs. Provides practical patterns for validation and retry logic. \url{https://github.com/jxnl/instructor}

\subsection{Validation and Type Safety}
\label{sec:llmC2-further-validation}

\paragraph{Pydantic Documentation.}
Pydantic provides data validation using Python type annotations. Essential for implementing validation loops in LLM pipelines. \url{https://docs.pydantic.dev}

\paragraph{Zod Documentation.}
Zod provides TypeScript-first schema validation. Useful for JavaScript/TypeScript implementations of structured output validation. \url{https://zod.dev}

\paragraph{JSON Schema Specification.}
The JSON Schema specification defines the vocabulary for schema-constrained generation. Understanding JSON Schema is essential for designing effective output schemas. \url{https://json-schema.org}

\subsection{Production Retrieval}
\label{sec:llmC2-further-retrieval}

\paragraph{ColBERT and Late Interaction.}
\fullcite{khattab2020colbert}

Introduces late interaction for efficient cross-encoder-quality retrieval. Important for understanding modern re-ranking approaches.

\paragraph{Query Rewriting.}
\fullcite{gao2022hyde}

HyDE (Hypothetical Document Embeddings) demonstrates query rewriting by generating a hypothetical answer and using it for retrieval. One of several query expansion techniques for improving retrieval.

\paragraph{Self-RAG.}
\fullcite{asai2023selfrag}

Self-RAG teaches models to retrieve, generate, and critique adaptively. Demonstrates how retrieval can be integrated into the generation loop rather than as a preprocessing step.

\subsection{Evidence and Provenance}
\label{sec:llmC2-further-evidence}

\paragraph{Attributed QA.}
\fullcite{bohnet2022attributed}

Examines how to train and evaluate systems that attribute claims to sources. Important for understanding the research foundations of evidence-based generation.

\paragraph{Citation Generation.}
Research on generating accurate citations is rapidly evolving. Key concerns include citation accuracy (does the citation exist?), faithfulness (does the citation support the claim?), and coverage (are all claims cited?).

\subsection{Security and Governance}
\label{sec:llmC2-further-security}

\paragraph{OWASP Top 10 for LLMs.}
\fullcite{owasp2025llm}

The OWASP security risk taxonomy for LLM applications. Essential reference for security assessments and compliance frameworks.

\paragraph{Prompt Injection.}
\fullcite{liu2023prompt}

Analyzes prompt injection vulnerabilities. Understanding these attacks is essential for designing secure structured output pipelines.

\subsection{Implementation Frameworks}
\label{sec:llmC2-further-frameworks}

\begin{itemize}
  \item \textbf{LangChain:} Comprehensive framework for building LLM applications with structured outputs and retrieval. \url{https://python.langchain.com/docs}

  \item \textbf{LlamaIndex:} Framework focused on data indexing and structured extraction. \url{https://docs.llamaindex.ai}

  \item \textbf{Haystack:} Open-source framework for building production-ready LLM applications. \url{https://haystack.deepset.ai}
\end{itemize}

\subsection{Staying Current}
\label{sec:llmC2-further-current}

Structured output techniques evolve rapidly. Resources for staying current:

\begin{itemize}
  \item \textbf{Vendor documentation:} OpenAI, Anthropic, Google, and other providers regularly update their structured output capabilities
  \item \textbf{arXiv cs.CL and cs.IR:} New research on constrained generation and retrieval appears as preprints
  \item \textbf{Community resources:} The applied AI community shares practical techniques through blog posts and open-source implementations
\end{itemize}

The field moves quickly; techniques from 2023 may already be superseded by more capable approaches. Check publication dates and look for subsequent citations when evaluating sources.

