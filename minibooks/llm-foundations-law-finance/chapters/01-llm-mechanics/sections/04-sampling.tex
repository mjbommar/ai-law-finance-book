% =============================================================================
% Sampling and Decoding — LLM Primer & Mechanics
% Purpose: Temperature, top-p, determinism, constrained decoding
% Label: sec:llm1-sampling
% =============================================================================

\section{Controlling Randomness: Temperature, Sampling, and Consistency}
\label{sec:llm1-sampling}

Once the model processes input tokens through its attention mechanisms, it does not simply ``pick'' the next word. Instead, it computes a probability distribution---called \keyterm{logits}---over its entire vocabulary (often 50,000--100,000 possible tokens) for what should come next. How the system selects the actual next token from this distribution is determined by \keyterm{sampling parameters}. Understanding these parameters is essential for controlling the hallucination rate, reproducibility, and appropriateness of outputs for regulated applications.

\subsection{The Generation Process}
\label{sec:llm1-sampling-process}

At each generation step, the model:

\begin{enumerate}
  \item Processes all preceding tokens (prompt + already-generated output) through its Transformer layers
  \item Produces a probability distribution over the vocabulary for the next token
  \item Selects one token according to the sampling strategy
  \item Appends that token to the sequence
  \item Repeats until a stopping condition is met
\end{enumerate}

This process is \keyterm{autoregressive}: each token depends on all previous tokens, and the model's own outputs become inputs for subsequent steps. A small error early in generation can cascade through all subsequent tokens.

\begin{highlightbox}[title={Example: Token Probabilities}]
For the prompt ``The verdict was...'', the model might compute:
\begin{itemize}
  \item ``guilty'' --- 45\% probability
  \item ``not'' --- 30\% probability (likely followed by ``guilty'')
  \item ``delivered'' --- 10\% probability
  \item ``unexpected'' --- 5\% probability
  \item ``rendered'' --- 3\% probability
  \item ... thousands of other tokens with tiny probabilities ...
  \item ``banana'' --- 0.0001\% probability
\end{itemize}
The sampling strategy determines how likely we are to select ``guilty'' versus ``unexpected'' versus extremely rare options.
\end{highlightbox}

\subsection{Temperature: The Primary Control}
\label{sec:llm1-sampling-temperature}

\keyterm{Temperature} is the most important sampling parameter. It scales the logits before they are converted to probabilities through the softmax function. Think of it as controlling how ``peaked'' versus ``flat'' the probability distribution is.

\begin{definitionbox}[title={Temperature Effects}]
\textbf{Temperature = 0 (or very low):}
\begin{itemize}
  \item The highest-probability token is selected with near-certainty
  \item Output is highly deterministic and consistent
  \item Generates ``safe'' but potentially repetitive text
\end{itemize}

\textbf{Temperature = 1.0 (default):}
\begin{itemize}
  \item Probabilities are used as-is
  \item Moderate variety in outputs
  \item Balance between creativity and coherence
\end{itemize}

\textbf{Temperature > 1.0 (high):}
\begin{itemize}
  \item Probabilities are ``flattened''---less likely tokens get higher chances
  \item Outputs become more creative, diverse, and surprising
  \item Significantly higher risk of incoherence and hallucination
\end{itemize}
\end{definitionbox}

\paragraph{Mathematical Intuition.} Temperature $T$ modifies the softmax function:
\[
P(x_i) = \frac{e^{z_i / T}}{\sum_j e^{z_j / T}}
\]
where $z_i$ are the raw logits. As $T \to 0$, the distribution approaches a one-hot vector at the maximum logit. As $T \to \infty$, the distribution approaches uniform over all tokens.

\paragraph{Practical Recommendations.}

\begin{table}[htbp]
\centering
\caption{Temperature Settings by Use Case}
\label{tab:llm1-temperature}
\small
\begin{tabular}{@{}llp{3.5cm}@{}}
\toprule
\textbf{Temp.} & \textbf{Use Case} & \textbf{Rationale} \\
\midrule
0.0--0.2 & Extraction, JSON, legal & Max consistency; minimal hallucination \\
0.2--0.5 & Summarization, drafting & Variety while maintaining accuracy \\
0.5--0.8 & Creative, brainstorming & Explores alternatives and ideas \\
0.8--1.2 & Fiction, marketing & Max creativity; needs human review \\
\bottomrule
\end{tabular}
\end{table}

For regulated outputs in legal and financial contexts, we generally recommend \textbf{temperature 0.0--0.3}. The small loss in linguistic variety is far outweighed by improved reliability and reproducibility.

\subsection{Top-p (Nucleus Sampling)}
\label{sec:llm1-sampling-topp}

\keyterm{Top-p} sampling, also called \keyterm{nucleus sampling} \parencite{holtzman2020curious}, is often preferred over simple temperature scaling because it adapts dynamically to the model's confidence.

Instead of considering all possible tokens, top-p considers only the smallest set of tokens whose cumulative probability exceeds a threshold $p$:

\begin{enumerate}
  \item Sort tokens by probability (highest first)
  \item Include tokens until their cumulative probability exceeds $p$ (e.g., 0.9 or 90\%)
  \item Sample only from this ``nucleus'' of tokens
\end{enumerate}

\paragraph{Dynamic Adaptation.} The power of top-p is its context-sensitivity:

\begin{itemize}
  \item \textbf{High-confidence predictions:} If the model is confident (e.g., ``The United States of...''), the nucleus might contain only ``America'' (99\% probability). Sampling from this tiny nucleus is effectively deterministic.

  \item \textbf{Low-confidence predictions:} If the model is uncertain (e.g., ``The best approach is...''), the nucleus might contain dozens of viable completions, allowing variety.
\end{itemize}

This adaptivity makes top-p generally preferable to fixed-$k$ approaches (top-$k$ sampling), which always consider exactly $k$ tokens regardless of the distribution shape.

\begin{keybox}[title={Recommended Defaults}]
\begin{itemize}
  \item \textbf{Temperature:} 0.0--0.2 for extraction and structured output; 0.2--0.4 for analysis
  \item \textbf{Top-p:} 0.9--0.95 (truncates only the extreme long tail)
  \item \textbf{Combined:} Low temperature with moderate top-p provides consistent outputs while avoiding rare degenerate tokens
\end{itemize}
This combination is often called ``focused sampling''---deterministic when the model is confident, slightly variable when genuinely uncertain.
\end{keybox}

\subsection{Top-k Sampling}
\label{sec:llm1-sampling-topk}

\keyterm{Top-k} sampling restricts consideration to exactly the $k$ most likely tokens, regardless of their probability mass. While simpler than top-p, it has a significant drawback: it does not adapt to the distribution shape.

\begin{itemize}
  \item If $k = 40$ and only one token is plausible, the model still considers 40 options (potentially including nonsense)
  \item If $k = 40$ and 100 tokens are equally plausible, the model arbitrarily excludes 60 of them
\end{itemize}

Top-k is rarely used alone in modern systems but sometimes combined with temperature or top-p as an additional constraint.

\subsection{Greedy Decoding and Beam Search}
\label{sec:llm1-sampling-greedy}

\keyterm{Greedy decoding} is the simplest strategy: always select the highest-probability token. This is equivalent to temperature = 0. It produces deterministic outputs but can lead to:

\begin{itemize}
  \item \textbf{Repetition loops:} The model gets stuck repeating phrases
  \item \textbf{Suboptimal global choices:} The locally highest-probability token at each step may not lead to the globally best sequence
\end{itemize}

\keyterm{Beam search} partially addresses the global optimization problem by maintaining multiple candidate sequences (``beams'') and selecting the overall highest-probability complete sequence. However, beam search often produces outputs that are technically higher-probability but less natural-sounding than sampled outputs. It is more commonly used in translation than in general text generation.

For most practical applications, we recommend nucleus sampling (top-p) with low temperature rather than greedy or beam search. The slight stochasticity improves output quality while low temperature keeps outputs reliable.

\subsection{Determinism, Seeds, and Reproducibility}
\label{sec:llm1-sampling-determinism}

A common misconception is that setting temperature = 0 guarantees perfectly reproducible outputs. In theory, it should. In practice, it often does not.

\subsubsection{Sources of Non-Determinism}

Even with identical prompts and temperature = 0:

\paragraph{GPU Non-Determinism.} Modern GPUs perform parallel floating-point operations that are not always associative at high precision. The computation $A + B + C$ may yield a slightly different result than $A + C + B$ at the least significant bits. These microscopic differences can cascade: a tiny change in logits might flip which token has the maximum probability, especially when two tokens are nearly tied.

\paragraph{Model Updates.} Cloud-hosted models are updated periodically. What you call a model today may be a subtly different version than what you called it last month. Behavior can change even with identical prompts.

\paragraph{Infrastructure Variation.} Different server hardware, different load balancing, different batching decisions---all can introduce variation in how prompts are processed.

\subsubsection{Strategies for Reproducibility}

For contexts requiring audit trails (regulatory filings, legal analysis, financial reporting):

\begin{enumerate}
  \item \textbf{Set a random seed} if the API supports it. This controls the pseudo-random number generator used in sampling.

  \item \textbf{Log the system fingerprint.} Advanced APIs (e.g., OpenAI) return a system fingerprint that identifies the exact backend configuration. Store this with your outputs.

  \item \textbf{Store the full prompt.} Complete reproducibility requires the exact input, not a summary.

  \item \textbf{Store the output.} Don't assume you can regenerate it---keep the actual output.

  \item \textbf{Document model version.} Note the specific model identifier, not just ``GPT-5'' but ``gpt-5-2025-06'' or equivalent.

  \item \textbf{Accept near-determinism.} For many applications, outputs that are 99\% identical across runs are sufficient; plan for the 1\% variation in your validation workflows.
\end{enumerate}

\begin{highlightbox}[title={Audit Trail Example}]
For regulatory purposes, log:
\begin{verbatim}
{
  "timestamp": "2024-12-20T14:30:00Z",
  "model": "claude-sonnet-4.5-20251015",
  "temperature": 0.1,
  "top_p": 0.95,
  "seed": 42,
  "system_fingerprint": "fp_abc123...",
  "input_tokens": 1547,
  "output_tokens": 423,
  "prompt_hash": "sha256:8f3a...",
  "output_hash": "sha256:2d7b..."
}
\end{verbatim}
This enables demonstrating to auditors how outputs were generated.
\end{highlightbox}

\subsection{Stop Sequences and Maximum Length}
\label{sec:llm1-sampling-stops}

LLMs will continue generating tokens until a stopping condition is met. Without explicit controls, models can produce run-on responses, continue past the intended answer, or generate unwanted content.

\subsubsection{Stop Sequences}

A \keyterm{stop sequence} is a string that, when generated, immediately halts further output. Common uses:

\begin{itemize}
  \item \textbf{Natural terminators:} Stop on double newlines (``\textbackslash n\textbackslash n'') to end at paragraph boundaries
  \item \textbf{Format delimiters:} Stop on ``\}'' when generating JSON, to halt after closing the object
  \item \textbf{Custom markers:} Stop on ``END\_RESPONSE'' or similar sentinel tokens you define in your prompt
\end{itemize}

\paragraph{Practical Tip.} Define stop sequences that would not naturally appear in valid output. If analyzing legal text, ``\#\#\# END ANALYSIS \#\#\#'' is safer than common punctuation.

\subsubsection{Maximum Token Limits}

\keyterm{Max tokens} caps the number of output tokens the model can generate. This serves multiple purposes:

\begin{itemize}
  \item \textbf{Cost control:} Bounds the maximum expense per request
  \item \textbf{Latency control:} Long outputs take time; caps keep response times predictable
  \item \textbf{Quality control:} Prevents rambling; forces conciseness
\end{itemize}

\paragraph{Warning: Truncation.} If the model hits the max token limit before finishing its thought, output is simply cut off mid-sentence. This is different from a natural completion. Check for truncation by:
\begin{itemize}
  \item Examining the finish reason in API responses (``length'' vs. ``stop'')
  \item Looking for incomplete sentences or missing closing brackets in structured output
\end{itemize}

\begin{cautionbox}[title={Setting Appropriate Limits}]
\begin{itemize}
  \item \textbf{Too low:} Outputs truncated, answers incomplete
  \item \textbf{Too high:} Unnecessary cost; risk of verbose, off-topic content
  \item \textbf{Right-sized:} Match expected output length plus margin. For a one-paragraph summary, 200--300 tokens; for detailed analysis, 1,000--2,000 tokens.
\end{itemize}
\end{cautionbox}

\subsection{Constrained Decoding: Enforcing Structure}
\label{sec:llm1-sampling-constrained}

In agentic workflows, LLMs often need to output structured data (JSON, XML, function calls) to integrate with downstream systems. A major failure mode is \keyterm{formatting drift}: the model produces valid JSON 99\% of the time but occasionally:

\begin{itemize}
  \item Adds conversational preamble: ``Here is the JSON you requested: \{...\}''
  \item Misses a comma or quote
  \item Uses integers where strings are expected (or vice versa)
  \item Outputs markdown code blocks around JSON
\end{itemize}

Any of these breaks the parser and crashes the integration.

\subsubsection{How Constrained Decoding Works}

\keyterm{Constrained decoding} (or ``structured output'') solves this by manipulating the sampling process itself. At each generation step:

\begin{enumerate}
  \item The model produces its probability distribution over all tokens
  \item A grammar-aware layer \emph{masks out} tokens that would be syntactically invalid given the current partial output
  \item Sampling proceeds only over valid continuations
\end{enumerate}

\paragraph{Example.} If the model has generated \verb|{"price":|, the next token must be a valid JSON value (number, string quote, bracket, etc.). The constraint layer prevents selecting a closing brace \verb|}| or plain text. Syntactic correctness becomes a \emph{guarantee} rather than a probabilistic hope.

\subsubsection{Implementation Approaches}

Modern systems offer several approaches to structured output:

\begin{itemize}
  \item \textbf{Native structured output modes:} OpenAI's ``JSON mode'' or ``function calling,'' Anthropic's tool use, etc.
  \item \textbf{Grammar specifications:} Provide a JSON Schema or grammar; the API enforces it during generation
  \item \textbf{Post-hoc validation with retry:} Generate output, validate against schema, retry if invalid (less efficient but more flexible)
\end{itemize}

\begin{keybox}[title={Structured Output Best Practices}]
For reliable integration:
\begin{enumerate}
  \item Use native structured output modes when available
  \item Provide explicit JSON schemas defining expected fields and types
  \item Implement validation even with constrained decoding (defense in depth)
  \item Design schemas with the LLM in mind: use descriptive field names, provide examples
  \item Log validation failures for debugging and model improvement
\end{enumerate}
\end{keybox}

\subsection{Self-Consistency and Multiple Samples}
\label{sec:llm1-sampling-selfconsistency}

For complex reasoning tasks, \textcite{wang2022selfconsistency} demonstrated that generating multiple independent responses and aggregating them often outperforms a single response. This is a sampling-time technique: it uses the model's stochasticity to probe whether the answer is stable across plausible continuations.

Operationally, it can be viewed as ``sample multiple reasoning paths and vote.'' If the model is confident and the task is well-specified, answers tend to converge; if answers diverge, the disagreement itself is a useful signal to escalate to retrieval, tools, or human review.

\paragraph{Where to go deeper.} Chapter~3 (Reasoning and Conversations) provides the primary treatment of self-consistency as a reasoning reliability strategy, including when to use it (and when not to) in professional workflows.

\subsection{Putting It Together: Sampling Configuration}
\label{sec:llm1-sampling-configuration}

We summarize recommended configurations for common regulated-domain use cases:

\begin{table}[htbp]
\centering
\caption{Sampling Configurations by Application}
\label{tab:llm1-sampling-configs}
\begin{tabular}{@{}p{2.8cm}ccp{1.8cm}p{3.2cm}@{}}
\toprule
\textbf{Application} & \textbf{Temp} & \textbf{Top-p} & \textbf{Max Tok.} & \textbf{Notes} \\
\midrule
JSON extraction & 0.0 & 1.0 & Schema & Structured output mode \\
Legal summarization & 0.2 & 0.9 & 500--1000 & Clear length instruction \\
Document classif. & 0.0 & 1.0 & 50 & Single word/phrase \\
Contract drafting & 0.3 & 0.95 & 2000+ & Review all output \\
Risk analysis & 0.2 & 0.9 & 1000 & Consider self-consistency \\
Brainstorming & 0.7 & 0.95 & 1000 & Human curation needed \\
\bottomrule
\end{tabular}
\end{table}

These are starting points. Optimal settings depend on your specific model, task, and tolerance for variation. We recommend:

\begin{enumerate}
  \item Start conservative (low temperature)
  \item Evaluate outputs on representative examples
  \item Increase temperature only if outputs are too rigid or repetitive
  \item Always implement validation for structured outputs
  \item Log sampling parameters for reproducibility
\end{enumerate}

With tokenization and sampling mechanics established, we now turn to how models represent meaning through embeddings---the foundation for semantic search and retrieval-augmented generation.
