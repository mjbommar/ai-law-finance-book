% =============================================================================
% Tokens, Tokenizers, and Context Windows — LLM Primer & Mechanics
% Purpose: Explain tokenization, BPE, context limits, Lost in Middle
% Label: sec:llm1-tokens
% =============================================================================

\section{Tokens, Tokenizers, and Context Windows}
\label{sec:llm1-tokens}

When you type a prompt into an LLM, the model does not ``see'' words, sentences, or paragraphs. It sees a sequence of integers. This conversion process---\keyterm{tokenization}---is the source of many peculiar LLM behaviors, including their struggles with arithmetic, their difficulty with specific string manipulation tasks (like reversing words), and their occasional incoherence with non-English scripts or specialized notation.

Understanding tokenization is essential for legal and financial practitioners because it directly affects:
\begin{itemize}
  \item \textbf{Cost:} LLM pricing is typically per-token, so verbose formats cost more
  \item \textbf{Context limits:} Fixed token budgets constrain how much text you can process
  \item \textbf{Reliability:} Certain tokenization artifacts cause systematic errors (especially with numbers)
  \item \textbf{Cross-language performance:} Non-English text often tokenizes less efficiently
\end{itemize}

\subsection{Tokenization: From Text to Token IDs}
\label{sec:llm1-tokens-bpe}

In practice, a \keyterm{tokenizer} is a deterministic mapping between (1) text and (2) a sequence of token IDs (integers). This matters because \emph{the tokenizer is part of the model interface}: token counts determine cost, context window fit, and many ``mysterious'' error patterns.

\paragraph{A practitioner-first mental model.} A tokenizer is a domain-specific ``compression'' scheme for text. Common strings become short representations (few tokens); rare strings become longer representations (more tokens). This is why legal citations, identifiers, and numbers often take more tokens than you expect.

\subsubsection{Tokenization Strategies (High Level)}

Tokenization is not standardized across providers or models. The most common strategy families are:
\begin{itemize}
  \item \textbf{Word-level tokenization:} A fixed dictionary of whole words. Simple to understand, but brittle: out-of-vocabulary terms (new company names, ticker symbols, citations) break easily and vocabularies become huge.

  \item \textbf{Character-level tokenization:} Every character is a token. Extremely robust (it can represent any string), but sequences become very long, increasing cost and slowing attention.

  \item \textbf{Byte-level tokenization:} A close cousin of character-level tokenization that operates on raw bytes (e.g., UTF-8 bytes). Also robust, and often used as a base layer so \emph{any} string can be represented.

  \item \textbf{Subword tokenization (the modern default):} Tokens correspond to frequent \emph{pieces} of words (prefixes, suffixes, common character sequences). This balances robustness and efficiency and is the dominant approach for modern LLMs.
\end{itemize}

\subsubsection{Subword Tokenizers in Practice (BPE, WordPiece, Unigram)}

Most modern LLMs use subword tokenization based on algorithms such as \keyterm{Byte Pair Encoding} (BPE) \parencite{sennrich2016bpe} or SentencePiece-style tokenizers \parencite{kudo2018sentencepiece}. The key practical implication is the same across variants: common strings become single tokens (``contract'', ``liability'', ``defendant''), while rare words, proper nouns, and technical strings are broken into multiple subword pieces.

\begin{technicalbox}[title={Engineering Note (Optional): How Subword Tokenizers Are Built}]
Subword tokenizers learn a vocabulary from training text so that frequent character sequences become single tokens.

In BPE-style tokenizers, the training process repeatedly merges frequent adjacent units (characters or bytes) into larger units until a target vocabulary size is reached (often tens of thousands of tokens).

You do \emph{not} need to know the algorithm to use LLMs safely. You \emph{do} need to understand the consequences: token counts vary by content type; identifiers and numbers fragment; and tokenization is model-specific.
\end{technicalbox}

\begin{highlightbox}[title={Tokenization Examples}]
\textbf{Common words} (likely single tokens):
\begin{itemize}
  \item ``the'', ``contract'', ``agreement'', ``court''
\end{itemize}

\textbf{Rare words} (likely split):
\begin{itemize}
  \item ``Anthropocene'' $\rightarrow$ ``An'' + ``thro'' + ``po'' + ``cene''
  \item ``indemnification'' $\rightarrow$ ``ind'' + ``em'' + ``nification'' (or similar)
  \item ``counterparty'' $\rightarrow$ ``counter'' + ``party''
\end{itemize}

\textbf{Numbers and codes} (often fragmented):
\begin{itemize}
  \item ``\$1,234,567.89'' $\rightarrow$ ``\$'' + ``1'' + ``,'' + ``234'' + ``,'' + ``567'' + ``.'' + ``89''
  \item ``Section 409A'' $\rightarrow$ ``Section'' + `` 4'' + ``09'' + ``A''
  \item ``Rule 10b-5'' $\rightarrow$ tokens for ``Rule'' + `` 10'' + ``b'' + ``-'' + ``5'' (often split)
  \item ``CUSIP 037833100'' $\rightarrow$ multiple tokens (letters, digits, and spacing often fragment)
\end{itemize}
\end{highlightbox}

\paragraph{Reproducible token traces (forthcoming).} In later revisions, we will include concrete tokenization traces generated with open-source libraries (e.g., Hugging Face \texttt{tokenizers}/\texttt{transformers}) and model-specific tokenizers (e.g., \texttt{tiktoken}) using a reproducible command such as \texttt{uv run --with tokenizers,transformers,tiktoken python -c "..."}.

\subsubsection{The Token Counting Heuristic}

A widely cited approximation: \textbf{1,000 tokens $\approx$ 750 words} in standard English prose. However, this ratio varies significantly:

\begin{itemize}
  \item \textbf{English prose:} Close to the 750-word heuristic
  \item \textbf{Legal documents:} Often more tokens per word due to technical terms, citations, and formatting
  \item \textbf{Code and JSON:} Significantly more tokens per character (punctuation, brackets, quotes all consume tokens)
  \item \textbf{Non-Latin scripts:} Chinese, Japanese, Arabic, and other scripts often require many more tokens per word equivalent
  \item \textbf{Tables and structured data:} High token overhead for formatting characters
\end{itemize}

\paragraph{Cost Implications.} Since LLM pricing is typically per-million tokens, these differences matter economically:

\begin{itemize}
  \item Processing XML versus JSON can differ by 2--3$\times$ in token count for equivalent data
  \item Verbose natural-language prompts cost more than concise structured prompts
  \item Non-English documents may cost significantly more to process than English equivalents
\end{itemize}

\begin{definitionbox}[title={Token Economics}]
\textbf{Input tokens:} Tokens in the prompt (less expensive, processed in parallel)

\textbf{Output tokens:} Tokens generated by the model (more expensive, generated sequentially)

\textbf{Context tokens:} Total tokens (input + output) that fit within the context window

Typical pricing (as of 2024--2025 for frontier models):
\begin{itemize}
  \item Input: \$2--15 per million tokens
  \item Output: \$8--60 per million tokens
\end{itemize}
A 50-page legal document might contain 25,000--35,000 tokens, costing \$0.10--0.50 in input tokens alone for a frontier model.
\end{definitionbox}

\subsection{The ``Numbers'' Problem: Tokenization Artifacts}
\label{sec:llm1-tokens-numbers}

One of the most persistent failure modes in LLMs---particularly problematic for financial applications---is unreliable arithmetic. This is largely a tokenization artifact.

\subsubsection{Why LLMs Struggle with Math}

Consider how numbers are tokenized:

\begin{itemize}
  \item The number ``345'' might be a single token in the vocabulary
  \item The number ``346'' might be a \emph{different} single token
  \item The number ``3455'' might be tokenized as ``34'' + ``55'' (two tokens)
  \item The number ``34567'' might be tokenized as ``345'' + ``67'' or ``34'' + ``567''
\end{itemize}

The model sees these as distinct integer IDs---similar to how it distinguishes ``cat'' from ``dog.'' It has no inherent understanding of the base-10 place value system that would connect these representations mathematically.

\paragraph{The Decimal Comparison Problem.} A classic failure case involves comparing decimals:

\begin{itemize}
  \item Question: ``Which is larger, 9.11 or 9.9?''
  \item Tokenization: ``9.11'' $\rightarrow$ ``9'' + ``.'' + ``11''; ``9.9'' $\rightarrow$ ``9'' + ``.'' + ``9''
  \item Failure mode: The model's attention might compare the final tokens (``11'' vs ``9'') and conclude incorrectly that 9.11 > 9.9 because 11 > 9.
\end{itemize}

This is not a failure of ``reasoning''---it is a failure of \emph{representation}. The tokenizer has destroyed the mathematical structure of the numbers.

\begin{cautionbox}[title={Never Trust LLM Arithmetic}]
For financial applications requiring reliable calculation:
\begin{enumerate}
  \item \textbf{Use tool-calling:} Have the LLM generate code (Python, Excel formulas) that performs calculations deterministically
  \item \textbf{Validate externally:} Cross-check any numerical outputs against known sources
  \item \textbf{Be especially cautious} with percentages, ratios, comparisons, and multi-step calculations
\end{enumerate}
The model can \emph{understand} that a calculation is needed and \emph{describe} the calculation steps, but actually performing the calculation should be delegated to deterministic tools.
\end{cautionbox}

\subsubsection{Other Tokenization Artifacts}

Beyond arithmetic, tokenization causes other systematic issues:

\paragraph{String Manipulation.} Tasks like reversing a word, counting letters, or finding specific character positions often fail because the model operates on tokens, not characters. ``Reversal'' of ``contract'' should yield ``tcartnoc'', but the model has never seen ``contract'' as individual letters---it sees it as a single token or small set of subword tokens.

\paragraph{Rare Words and Names.} Unusual proper nouns (company names, jurisdiction names, technical terms) may be split into fragments that individually carry misleading associations. ``Anthropic'' might be tokenized into ``An'' + ``throp'' + ``ic'', where ``throp'' activates associations with ``philanthropy'' that are irrelevant to the technology company.

\paragraph{Code and Syntax.} Programming languages, JSON, and other structured formats are tokenized character-by-character for punctuation, making outputs fragile. A missing comma or bracket is just as ``plausible'' to the model as the correct syntax---it does not have a syntactic parser built in.

\subsection{The Context Window: Capacity and Constraints}
\label{sec:llm1-tokens-context}

The \keyterm{context window} is the maximum sequence length---input tokens plus output tokens---that a model can process in a single pass. Everything the model ``knows'' about your specific query must fit within this window; it has no persistent memory between calls (unless explicitly designed into the system).

\subsubsection{Context Window Sizes}

Context windows have expanded dramatically over the LLM era:

\begin{table}[htbp]
\centering
\caption{Evolution of Context Window Sizes}
\label{tab:llm1-context-sizes}
\begin{tabular}{@{}lrl@{}}
\toprule
\textbf{Model/Era} & \textbf{Context Window} & \textbf{Approximate Equivalent} \\
\midrule
GPT-2 (2019) & 1,024 tokens & $\sim$2 pages \\
GPT-3 (2020) & 2,048--4,096 tokens & $\sim$4--8 pages \\
GPT-4 (2023) & 8,192--32,768 tokens & $\sim$15--60 pages \\
Claude 2.1 (2023) & 100,000 tokens & $\sim$200 pages \\
GPT-4 Turbo (2024) & 128,000 tokens & $\sim$250 pages \\
Gemini 1.5 Pro (2024) & 1,000,000 tokens & $\sim$2,000 pages \\
\bottomrule
\end{tabular}
\end{table}

\paragraph{What Fits in Context?} To give practical intuition:

\begin{itemize}
  \item \textbf{8K tokens:} A few pages of legal text plus detailed instructions
  \item \textbf{32K tokens:} A typical contract or short legal brief
  \item \textbf{128K tokens:} A substantial merger agreement or regulatory filing
  \item \textbf{1M tokens:} Multiple lengthy documents, discovery sets, or a small codebase
\end{itemize}

\subsubsection{Cost and Latency Dynamics}

Larger context windows are not free:

\paragraph{Input Processing.} Reading the prompt (input tokens) can be parallelized, but longer prompts still take more time. \keyterm{Time-to-First-Token} (TTFT)---the latency before the model begins generating---increases with input length.

\paragraph{Output Generation.} Each output token is generated sequentially (autoregressively), and each generation step must attend to all preceding tokens. Longer contexts slow generation and increase memory requirements.

\paragraph{Pricing Tiers.} Some providers charge differently for long-context usage, or offer optimized pricing for cached/repeated contexts.

\begin{highlightbox}[title={Practical Context Management}]
Best practices for efficient context usage:
\begin{enumerate}
  \item \textbf{Retrieve, don't dump:} Use retrieval (RAG) to select relevant excerpts rather than pasting entire documents
  \item \textbf{Summarize when possible:} For background context, summaries may suffice
  \item \textbf{Structure your prompt:} Keep critical context near the query; long contexts do not guarantee uniform retrieval (see Chapter~2)
  \item \textbf{Monitor token counts:} Log input and output tokens for cost attribution
\end{enumerate}
\end{highlightbox}

\subsection{The ``Lost in the Middle'' Phenomenon}
\label{sec:llm1-tokens-lost}

Long context windows do not imply uniform retrieval. Empirical work shows a \textbf{U-shaped retrieval curve}: models more reliably use information placed at the beginning and end of the context than information buried in the middle \parencite{liu2023lostmiddle}. For practitioners, the implication is straightforward: pasting an entire agreement or filing into the prompt can still produce omission or hallucination if the relevant provision is located ``in the middle.''

\paragraph{Where to go deeper.} Chapter~3 (Reasoning and Conversations) provides the primary treatment of ``Lost in the Middle'' in the context of prompt assembly, memory strategies, and instruction placement.

\paragraph{Preview: Roles and Chat Formatting.} Most production systems structure interactions as role-tagged messages (system, user, assistant, tool). This is central to building stateful conversational systems and to understanding why prompt injection is a systems security problem, not a ``prompting trick.'' Chapter~3 covers role-based prompting and guardrails; this chapter's failure modes section (\Cref{sec:llm1-fail}) covers prompt injection risk and mitigations.

\subsection{Token Efficiency and Prompt Engineering}
\label{sec:llm1-tokens-efficiency}

Given the economics and constraints of tokenization, efficient prompt design matters:

\paragraph{Verbosity Costs.} Every unnecessary word consumes tokens and budget:
\begin{itemize}
  \item ``Please kindly summarize the following contract document for me'' $\rightarrow$ many tokens
  \item ``Summarize this contract:'' $\rightarrow$ fewer tokens, same effect
\end{itemize}

\paragraph{Format Choices.} Structured formats vary in token efficiency:
\begin{itemize}
  \item Markdown lists are relatively efficient
  \item JSON has overhead from quotes and brackets but enables structured parsing
  \item XML is typically the most verbose (many tokens for tags)
  \item YAML balances readability and efficiency
\end{itemize}

\paragraph{References vs. Repetition.} In multi-turn conversations or complex prompts:
\begin{itemize}
  \item Refer back to earlier content rather than repeating it
  \item Use numbered sections that can be referenced (e.g., ``See Section~2'')
  \item Summarize prior context rather than including full history
\end{itemize}

\begin{keybox}[title={Token Efficiency Checklist}]
\begin{enumerate}
  \item Remove unnecessary pleasantries and filler words
  \item Choose concise formats (Markdown or YAML over XML)
  \item Use retrieval to include only relevant document sections
  \item Summarize prior conversation context rather than including full history
  \item Set maximum token limits to bound output length
  \item Consider smaller models for simple tasks (fewer capabilities but lower cost)
\end{enumerate}
\end{keybox}

With the mechanics of tokenization and context established, we now turn to how models use these token sequences to generate outputs: the probabilistic sampling process that makes LLMs simultaneously powerful and unpredictable.
