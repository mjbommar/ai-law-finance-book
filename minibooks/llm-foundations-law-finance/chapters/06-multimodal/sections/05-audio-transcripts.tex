% =============================================================================
% Audio & Transcripts — Multimodal Fundamentals
% Purpose: Summaries with timestamps and speakers
% Label: sec:llmD2-audio
% =============================================================================

\section{Audio and Transcripts}
\label{sec:llmD2-audio}

Extending retrieval-augmented generation to temporal media (audio and video) introduces the dimension of time. A retrieved result is not just a ``document'' but a specific time span within a media file. For legal and financial practitioners, this means earnings calls, depositions, regulatory hearings, and training recordings become searchable and quotable with timestamp precision.

\subsection{Audio RAG Pipelines}
\label{sec:llmD2-audio-rag}

Audio RAG pipelines depend on the quality of automatic speech recognition (ASR) and the preservation of temporal metadata:

\paragraph{Transcription with Timestamps.} Models like \keyterm{Whisper} (OpenAI) and \keyterm{AssemblyAI} convert audio to text while preserving word-level or segment-level timestamps. When a relevant chunk is found during retrieval, the system maps the text back to the original timestamps, allowing the user to ``jump to'' the exact moment in the audio player.

\paragraph{Speaker Diarization.} Crucially, the transcription step must include \keyterm{speaker diarization}---identifying who is speaking. ``Speaker A said X'' is semantically different from ``Speaker B said X.'' In a deposition or earnings call, attributing statements to the correct speaker is essential for accurate analysis.

\begin{definitionbox}[title={Audio RAG Pipeline Components}]
\begin{enumerate}
  \item \textbf{Ingestion}: Audio files processed through ASR with diarization enabled.
  \item \textbf{Segmentation}: Text chunked by semantic breaks, speaker turns, or silence rather than arbitrary token counts.
  \item \textbf{Embedding}: Transcript segments embedded with speaker and timestamp metadata.
  \item \textbf{Retrieval}: Query matches return text plus temporal coordinates.
  \item \textbf{Synthesis}: Response includes citations with timestamps and optional audio playback links.
\end{enumerate}
\end{definitionbox}

\paragraph{Error Rates and Mitigation.} ASR is imperfect. Technical terminology, proper names, and accented speech increase word error rates (WER). For legal and financial applications:

\begin{itemize}
  \item Provide custom vocabularies (company names, legal terms) to the ASR system.
  \item Consider human review for high-stakes transcripts (depositions, regulatory testimony).
  \item Retain the original audio alongside transcripts for verification.
  \item Display confidence scores where available to flag uncertain passages.
\end{itemize}

\subsection{Video Understanding and Retrieval}
\label{sec:llmD2-video-rag}

Video RAG treats video as a sequence of visual frames synchronized with an audio track, enabling queries that span both modalities.

\paragraph{Dual-Stream Indexing.} A comprehensive video RAG system indexes both:
\begin{itemize}
  \item \textbf{Transcript vectors}: What was said (from ASR with diarization).
  \item \textbf{Visual frame descriptions}: What was shown (from keyframe extraction and VLM captioning).
\end{itemize}

A user query searches both streams, allowing questions like ``Find the scene where the speaker discusses quarterly revenue while showing the bar chart.''

\paragraph{Keyframe Extraction.} Keyframes are extracted at regular intervals (e.g., 1 frame per second) or at scene changes. Each frame is processed by a VLM to generate textual descriptions (``scene graphs'') or embedded directly using CLIP. For legal and financial video---training materials, recorded presentations, regulatory hearings---meaningful frames often coincide with slide transitions.

\paragraph{VideoRAG Architecture.} Advanced frameworks like \keyterm{VideoRAG} employ a dual-channel architecture with ``Graph-based Textual Knowledge Grounding'' to transform visual signals into structured text representations while preserving temporal dependencies. This allows complex queries that span both audio and visual content.

\begin{highlightbox}[title={Multimodal Video Query}]
\textit{Query}: ``Find where the CFO discusses the accounting change while the slide shows the impact table.''

\textit{System behavior}:
\begin{enumerate}
  \item Search transcript for ``accounting change'' + speaker ``CFO''
  \item Search visual index for ``table'' or ``impact''
  \item Intersect temporal windows to find overlapping segments
  \item Return video clips with start/end timestamps
\end{enumerate}
\end{highlightbox}

\subsection{Practical Considerations}
\label{sec:llmD2-audio-video-practical}

\paragraph{Storage and Streaming.} Video and audio files are large. Systems typically:
\begin{itemize}
  \item Store original media in object storage (S3, Azure Blob).
  \item Generate and index transcripts/descriptions separately.
  \item Stream relevant segments via FFMPEG or cloud media services.
  \item Return playback links with timestamp parameters rather than downloading entire files.
\end{itemize}

\paragraph{Privacy and Access Control.} Audio and video often contain sensitive content---voices are biometric identifiers, and recordings may capture privileged communications. Apply the privacy controls discussed in Section~\ref{sec:llmD2-privacy} before ingestion:
\begin{itemize}
  \item Redact or exclude segments containing privileged discussions.
  \item Apply speaker-level access controls where content is speaker-specific.
  \item Consider whether transcripts alone (without audio) suffice for the use case.
\end{itemize}

\begin{keybox}[title={A/V RAG Practices}]
\begin{itemize}
  \item Always preserve timestamp-to-text mappings for citation.
  \item Enable speaker diarization for multi-party recordings.
  \item Provide custom vocabularies for domain-specific terminology.
  \item Retain original media for verification of AI-generated transcripts.
  \item Apply access controls at the segment level where sensitivity varies.
\end{itemize}
\end{keybox}

\subsection{ASR Error Handling and Quality Control}
\label{sec:llmD2-asr-quality}

Automatic speech recognition is imperfect, and legal and financial applications require explicit quality management.

\paragraph{Word Error Rate Estimation.}
WER measures transcription quality but requires ground truth for calculation. In production:

\begin{itemize}
  \item Use confidence scores from the ASR engine as proxies.
  \item Flag low-confidence segments for review.
  \item Sample and manually review a percentage of transcripts to estimate system-wide WER.
  \item Track WER by speaker, audio quality, and domain to identify systematic issues.
\end{itemize}

\paragraph{Confidence Thresholds.}
Configure thresholds to balance automation and quality:

\begin{itemize}
  \item \textbf{High confidence ($>$0.9)}: Accept without review for routine use.
  \item \textbf{Medium confidence (0.7--0.9)}: Flag for optional review; acceptable for search but not citation.
  \item \textbf{Low confidence ($<$0.7)}: Require human review before reliance; mark uncertain passages visually.
\end{itemize}

\paragraph{Domain-Specific Vocabulary.}
Legal and financial terminology often confuses general ASR models:

\begin{itemize}
  \item \textbf{Custom vocabulary lists}: Company names, product names, legal terms, ticker symbols.
  \item \textbf{Pronunciation hints}: Guide recognition for unusual names or acronyms.
  \item \textbf{Boosted phrases}: Increase likelihood of domain-specific terms appearing.
  \item \textbf{Post-processing correction}: Rule-based substitution for common errors.
\end{itemize}

\begin{highlightbox}[title={Earnings Vocabulary}]
For an earnings call transcription system:
\begin{itemize}
  \item Boost company name, subsidiary names, and executive names.
  \item Include accounting terms: ``EBITDA,'' ``goodwill impairment,'' ``diluted EPS.''
  \item Add industry-specific terminology: ``basis points,'' ``comps,'' ``guidance.''
  \item Handle ticker symbols and numeric expressions.
\end{itemize}
\end{highlightbox}

\subsection{Video Frame Analysis}
\label{sec:llmD2-video-frames}

Beyond the audio track, video content carries visual information that may be critical for understanding.

\paragraph{Keyframe Selection Strategies.}
Not every frame needs processing. Selection strategies include:

\begin{itemize}
  \item \textbf{Fixed interval}: Sample every N seconds (simple but may miss important moments).
  \item \textbf{Scene change detection}: Identify visual transitions and sample at boundaries.
  \item \textbf{Motion analysis}: Capture frames when significant visual change occurs.
  \item \textbf{Audio-aligned}: Select frames when speakers change or key topics are mentioned.
\end{itemize}

\paragraph{Frame Understanding.}
Process selected keyframes through vision-language models to extract:

\begin{itemize}
  \item Scene descriptions (``Speaker at podium with presentation slide'').
  \item Text visible in frame (slide content, whiteboard text, on-screen graphics).
  \item Object detection (people, documents, equipment).
  \item Face identification (if authorized and relevant).
\end{itemize}

\paragraph{Exhibit Detection in Depositions.}
Video depositions often include exhibit presentations. Detection should:

\begin{itemize}
  \item Identify when documents are shown on camera.
  \item Capture exhibit images for separate processing.
  \item Link exhibit appearances to transcript timestamps.
  \item Note when witnesses are reviewing documents versus speaking.
\end{itemize}

\subsection{Slide Deck Extraction}
\label{sec:llmD2-slides}

Recorded presentations often feature slide decks that carry structured information separate from the spoken narrative.

\paragraph{Slide Detection and Extraction.}
\begin{itemize}
  \item Detect slide transitions in the video stream.
  \item Capture representative frame for each slide.
  \item OCR slide text content.
  \item Extract charts and diagrams for visual analysis.
\end{itemize}

\paragraph{Slide-Transcript Synchronization.}
Align extracted slides with the spoken transcript:

\begin{itemize}
  \item Map slide transitions to transcript timestamps.
  \item Associate spoken content with the visible slide.
  \item Enable queries like ``What did the speaker say about slide 5?''
\end{itemize}

\paragraph{Speaker Notes and Metadata.}
If the original presentation file is available (PowerPoint, Google Slides):

\begin{itemize}
  \item Extract speaker notes as additional context.
  \item Preserve slide titles and section structure.
  \item Capture embedded links and references.
  \item Index slide-level metadata alongside video segments.
\end{itemize}

\subsection{Real-Time vs. Batch Processing}
\label{sec:llmD2-realtime}

Audio and video processing can occur live or after recording, with different trade-offs.

\paragraph{Real-Time Transcription.}
Live transcription for hearings, meetings, or trading floors:

\begin{itemize}
  \item Lower latency requirements (seconds, not minutes).
  \item Streaming ASR models that process audio chunks incrementally.
  \item Immediate display for participants (accessibility, record-keeping).
  \item Reduced accuracy compared to batch processing (less context available).
\end{itemize}

\paragraph{Batch Processing for Discovery.}
Historical recordings processed in bulk:

\begin{itemize}
  \item Higher accuracy through multiple passes and post-processing.
  \item Full audio context available for each segment.
  \item Batch speaker diarization with cross-recording speaker linking.
  \item Parallelization across compute resources.
\end{itemize}

\paragraph{Latency-Accuracy Trade-offs.}
The fundamental trade-off:

\begin{itemize}
  \item Real-time: Lower latency, lower accuracy, immediate availability.
  \item Batch: Higher latency, higher accuracy, suitable for archival and search.
  \item Hybrid: Real-time draft followed by batch refinement for permanent record.
\end{itemize}

\begin{keybox}[title={Processing Mode Selection}]
\begin{description}
  \item[Real-time] Accessibility for live events, trading floor monitoring, meeting assistance.
  \item[Batch] E-discovery, historical research, compliance review, permanent transcripts.
  \item[Hybrid] Live captioning refined overnight for searchable archives.
\end{description}
\end{keybox}

